\documentclass[11pt, oneside, reqno]{amsart}

\usepackage{amsmath, amsthm, amssymb}
\usepackage[usenames,dvipsnames]{color}
\usepackage[all, cmtip]{xy}
\usepackage[pdftex, bookmarks=true, linkbordercolor={0 0 1}]{hyperref}
\usepackage[margin=1in]{geometry}
\usepackage[titletoc,title]{appendix}

\setlength{\parindent}{0pt}
\setlength{\parskip}{11pt}

\theoremstyle{definition} \newtheorem{definition}{Definition}[section]
\newtheorem{lemma}[definition]{Lemma}
\newtheorem{theorem}[definition]{Theorem}
\newtheorem{prop}[definition]{Proposition}
\newtheorem{conjecture}[definition]{Conjecture}
\newtheorem{corollary}[definition]{Corollary}
\newtheorem{construction}[definition]{Construction}
\newtheorem{observation}[definition]{Observation}
\newtheorem{assumption}[definition]{Assumption}
\newtheorem{claimnum}[definition]{Claim}
\newtheorem*{nonum}{Theorem}
\newtheorem*{lemma*}{Lemma}
\newtheorem*{claim}{Claim}
\newtheorem*{subclaim}{Subclaim}
\newtheorem*{fact}{Fact}
\newtheorem*{problem}{Problem}
\newtheorem*{ack}{Acknowledgements}

\theoremstyle{definition} \newtheorem{remark}[definition]{Remark}
\theoremstyle{definition} \newtheorem{remarks}[definition]{Remarks}
\theoremstyle{definition} \newtheorem{question}[definition]{Question}
\theoremstyle{definition} \newtheorem*{note}{Note}
\theoremstyle{definition} \newtheorem{example}[definition]{Example}
\theoremstyle{definition} \newtheorem{examples}[definition]{Examples}

\newtheorem{pseudoconj}[definition]{Pseudo-Conjecture}


\renewcommand{\gg}{\mathfrak{g}}

\newcommand{\bb}[1]{\mathbb{#1}}
\newcommand{\mr}[1]{\mathrm{#1}}
\newcommand{\mc}[1]{\mathcal{#1}}
\newcommand{\mf}[1]{\mathfrak{#1}}
\newcommand{\wt}[1]{\widetilde{#1}}
\newcommand{\bo}[1]{\boldsymbol{#1}}

\newcommand{\inj}{\hookrightarrow}
\newcommand{\bs}{\ \backslash \ }
\newcommand{\dd}{\partial}
\newcommand{\del}{\partial}


\newcommand{\ul}[1]{\underline{#1}}
\newcommand{\ol}[1]{\overline{#1}}

\newcommand{\CC}{\mathbb{C}}
\newcommand{\RR}{\mathbb{R}}
\newcommand{\OO}{\mathcal{O}}
\newcommand{\ZZ}{\mathbb{Z}}

\newcommand{\eps}{\varepsilon}

\newcommand{\SO}{\mathrm{SO}}
\newcommand{\SL}{\mathrm{SL}}
\newcommand{\GL}{\mathrm{GL}}
\newcommand{\SU}{\mathrm{SU}}
\newcommand{\PGL}{\mathrm{PGL}}
\newcommand{\spin}{\mathrm{Spin}}
\newcommand{\Spin}{\mathrm{Spin}}
\newcommand{\so}{\mathfrak{so}}
\renewcommand{\sl}{\mathfrak{sl}}
\renewcommand{\sp}{\mathfrak{sp}}
\newcommand{\gl}{\mathfrak{gl}}

\newcommand{\sfe}{\mathsf{e}}
\newcommand{\sff}{\mathsf{f}}
\newcommand{\sfh}{\mathsf{h}}
\newcommand{\sfs}{\mathsf{s}}

\newcommand{\frakq}{\mathfrak{q}}

\newcommand{\sub}{\subseteq}
\newcommand{\iso}{\cong}

\DeclareMathOperator{\tr}{tr}
\DeclareMathOperator{\rank}{rank}
\DeclareMathOperator{\coh}{Coh}
\DeclareMathOperator{\higgs}{Higgs}
\DeclareMathOperator{\bun}{Bun}
\DeclareMathOperator{\Gr}{Gr}
\DeclareMathOperator{\spec}{Spec}
\DeclareMathOperator{\res}{res}
\DeclareMathOperator{\EOM}{EOM}
\DeclareMathOperator{\id}{id}
\DeclareMathOperator{\dvol}{dvol}
\DeclareMathOperator{\aut}{Aut}
\DeclareMathOperator{\sym}{Sym}
\DeclareMathOperator{\Flat}{Flat}
\DeclareMathOperator{\mhiggs}{mHiggs}
\DeclareMathOperator{\mon}{Mon}
\DeclareMathOperator{\diff}{Diff}
\DeclareMathOperator{\Hol}{Hol}
\DeclareMathOperator{\mhitch}{mHitch}
\DeclareMathOperator{\colim}{colim}
\DeclareMathOperator{\im}{im}

\newcommand{\map}{\ul{\mr{Map}}}
\newcommand{\qconn}{q\text{-Conn}}
\newcommand{\conn}{\text{-Conn}}
\newcommand{\epsconn}{\varepsilon\text{-Conn}}
\renewcommand{\d}{\mathrm{d}}
\newcommand{\fr}{\mathrm{fr}}
\newcommand{\ad}{\mr{ad}}
\newcommand{\Ad}{\mr{Ad}}
\newcommand{\HT}{\mr{HT}}

\title{Multiplicative Hitchin Systems and Supersymmetric Gauge Theory}
\author{Chris Elliott \and Vasily Pestun}
\date{\today}

\newcommand{\chris}[1]{(\textcolor{red}{Chris: #1})}
\newcommand{\vasily}[1]{(\textcolor{blue}{Vasily: #1})}

\begin{document}

\section{Notes on Sklyanin bracket on Poisson - Lie group
  and symplectic leaves}

Consider a multiplicative Higgs field on $C = \mathbb{P}^{1}$
with fixed framing at infinity. Because the $C = \mathbb{P}^{1}$ the
gauge bundle is fixed. We choose a component where the gauge bundle is trivial.
Then all degrees of freedom are concentrated
in the framed meromorphic Higgs field $g: C \to G$ such that $g(c) = 1$ where $c = \infty$. We denote
the space of framed meromorphic Higgs fields by $\mhiggs^{\fr} = G_1(C)$. 

\subsection{Poisson structure on $G_1(C)$} 
Let $\kappa$ be a Killing form $\kappa: \gg \otimes \gg \to \mathbb{C}$.
Then $\kappa^{-1}: \gg^{*} \to \gg$. Also we'll assume  without loosing generality that $\kappa$ is choosen such that
there is a faithful $G$-representation $\rho$ such that 
$\tr d\rho(X) d\rho(X') = c_\rho \kappa(X, X') $ for $X , X' \in \gg$
where $d\rho$ denotes the associated representation of the Lie algebra of $G$,
and $c_{\rho}$ is a non-zero complex number.  \chris{In this expression $\rho(X)$ should presumably be $\d \rho(X)$, the representation of the Lie algebra.  It's true that a complex group $G$ is reductive iff it has a faithful representation, and therefore its Lie algebra has a pairing of this form.}
\vasily{Sure, I agree. I tried to improve the phrasing.
  Please feel free to edit the phrase when it feels that you can
  improve} 

Sklyanin has defined a Poisson bracket on evaluation functions on $G_1(C)$ for $C = (\mathbb{C}_z, dz)$ as follows. 

Let $\phi: G \to \mathbb{C}$ be a rational function on $G$.
For $z \in C$ denote $\phi_{z} =  \phi \circ ev_{z} : G_1(C) \to \mathbb{C}$ a composition of evaluation morphism  $ev_{z}: G_1(C) \to G$ in point $z$ and a function $\phi$. Notice that the domain of definition of
$\phi_{z}$ is not the whole space $G_{1}(C)$ but a subspace of
$G$-valued functions which are regular at $z$. 

Similarly $\psi_{w}$ is evaluation of $\psi: G \to \mathbb{C}$ at $w \in C$. 

\chris{I'm a bit confused here.  The group $G_1(C)$ is supposed to be the group of rational functions on $\bb{CP}^1$ with value 1 at $\infty$, correct?  In that case $\phi_z$ is not defined on all of $G_1(C)$, only the subspace of functions without a pole at $z$.  Are you defining a bracket on a larger space of partial functions on $G_1(C)$, or maybe on a smaller group than the group of rational functions?}
\vasily{Sure, I agree. Let's see if this restriction brings any problems downstairs}

Sklyanin defined a Poisson bracket on evaluation functions on $G_1(C)$ as follows
\begin{equation}
\label{eq:Sklyanin}
  \{ \psi_{w}, \phi_{z} \} = \frac{1}{w - z}(\langle \nabla_{L} \psi_w, \kappa^{-1} \nabla_{L} \phi_{z}
\rangle  - \langle \nabla_{R} \psi_{w},  \kappa^{-1} \nabla_{R} \phi_{z}\rangle)
\end{equation}
\chris{what is $\kappa$ here?  Below it's used for the Killing form?}
\vasily{yes, I corrected}

\chris{I think this coincides, under the inclusion $G_1(C) \inj G_1[[z^{-1}]]$, with the Poisson bracket defined on the group of formal power series in $G$, as in \cite[Theorem 3.10]{Williams}.  If we make sure of this then we can combine the current discussion involving $G_1[[z^{-1}]]$ with the discussion here.}

\vasily{It definitely should be true. Let's try to argue}

where $\nabla_{L}\phi , \nabla_{R}\phi \in \gg^{*} $ denote the left and right differentials
of a function $\phi$ on $G$. For any $X \in \gg$
\begin{equation}
  \langle X, \nabla_{L} \phi \rangle := \frac{d}{dt} \phi(e^{Xt} g),
  \qquad \langle X, \nabla_{R} \phi \rangle := \frac{d}{dt} \phi(g e^{Xt})
\end{equation}

Sklyanin's Poisson structure can be also understood as follows. Let $\phi$ be a function
on a Poisson manifold $M$ equipped with Poisson structure $\pi \in \Gamma(M,\Lambda^2 T_{M})$,
or equivalently with $\pi \in \Gamma(M, \hom (T_{M}^{*}, T_{M}))$. 

A vector field $X_{\phi} = \pi d \phi $ associated to a function $\phi$ and Poisson structure $\pi$ is called a Hamiltonian vector field. For a point $m \in M$, a tangent space to a symplectic
leaf through $m$ is spanned by Hamiltonian vector fields passing through $m$, 
that is by the image of $\pi$ at $m$.

Let us explore the structure of these vector fields. We write a vector field on $G$
as a sum  $(X^L, X^R)$  of the left action vector field $L_{X^L}$ and right action vector field $R_{X^R}$
for $X^L, X^R \in \gg$ modulo the equivalence relation (\ref{eq:equivalence})

In matrix notations, for $g \in G$ we have
\begin{equation}
  \delta g = X^L g + g X^R 
\end{equation}
and consequently there is an equivalence 
\begin{equation}
\label{eq:equivalence0}
  (X^L, X^R) \sim (X^L + g X g^{-1}, X^R -  X)
\end{equation}

Denote the adjoint operator $X \mapsto g X g^{-1}$ for $X \in \gg, g \in G$ by $A_{g}: \gg \to \gg$.
Then the equivalence is
\begin{equation}
\label{eq:equivalence}
    (X^L, X^R) \sim (X^L + A_{g} X , X^{R} -  X) \sim (X^{L} +  A_{g} X^{R}, 0) 
\end{equation}

\vasily{Do notations make sense in the above paragraph, and the
  terminology \emph{equivalence} is suitable? If you like, please improve}


Modulo such equivalence, Sklyanin's Poisson bracket (\ref{eq:Sklyanin}) implies
that a vector field associated to an evaluation Hamiltonian $\phi_z$ is represented by
left and right components as follows 
\begin{equation}
\label{eq:Xphiz}
X_{\phi_z} (w) =  \frac{1}{w - z} ( \kappa^{-1} \nabla_{L} \phi_z,  - \kappa^{-1}\nabla_{R} \phi_{z})
\end{equation}


\begin{lemma}\label{phitoX}
  Fix an element $g \in \mhiggs^{\fr}_{D}$ represented by a $G$-valued
  meromorphic function $g(z)$ with singularities at $D$.    If $\phi$ is a differentiable function and $z$ is away from $D$, then $X_{\phi_z}$
  belongs to the tangent space to $\mhiggs^{\fr}_{D}$, so we have $\mathrm{im} \, \pi \subset T_{g}\mhiggs^{\fr}_D$. 
\end{lemma}

\chris{So the argument below is assuming that the cotangent space to the big space $G_1(C)$ at $g$ is generated by the derivatives of evaluation functions in some sense.  In what sense exactly?  Well most likely, first we want to note that all 1-forms on $G_1(C)$ are exact, so it suffices to check that evaluation functions generate $\OO(G_1(C))$ as an algebra.  Now we maybe still need to be careful with functional analysis because this group is infinite-dimensional (so we need to understand what kind of functions we are talking about), we maybe want to say that $G_1(C)$ is the (filtered) colimit over $(D,\omega^\vee$ of the spaces of functions with poles on fixed subsets $D$ with degree bounded by coweights $\omega^\vee$ (i.e. our $\mhiggs_D$ spaces), so then $\OO(G_1(C))$ is the (cofiltered) limit of the algebras of polynomial functions on these finite-dimensional varieties. The evaluation functions here, on each finite-dimensional space, are algebraic as long as the $\phi$ are algebraic, and they do seem to generate (though with infinitely many relations).  But one should clarify that the functions in $\OO(G_1(C))$ are in fact formal sequences of linear combinations of equivalence classes (under the relations) of evaluation functions for each coloured divisor $(D, \omega^\vee)$.  Did you have something simpler in mind?}

\vasily{I agree, I didn't really pay 
attention to the distinction between the cotangent
  space and the derivatives of evaluations functions. For the
  sake of notations I used derivatives of the evaluation functions.
  But perhaps the proof below will not loose anything if instead
  of $(\nabla_{L} \phi_z)$ we'll write, say, $\alpha_z$, for an element
  of the cotangent space evaluated at $z$ dual to the left
  vector fields at $z$? Instead of $\nabla_{R} \phi_{z}$ we'll write
  $A_{g_{z}}^{-1} \alpha_z$. The issue that remains is then an issue
  of functional analysis: what is the dual space? Here we've considered
  a subspace in the dual space: this subspace
 is generated by the evaluation functionals.
  Should we consider the distributions to make it correct? If
  you have suggestions, I'll be happy if we can improve. 
} 

\begin{proof}
  Let us bring the vector field (\ref{eq:Xphiz}) into equivalence frame $(X^{L}, 0)$ using
  (\ref{eq:equivalence}). We get
  \begin{equation}
\label{eq:xlw0}
(X^L_{\phi_z} (w) , 0) =  \frac{1}{w - z} ( \kappa^{-1} \nabla_{L} \phi_z  - A_{g_{w}} (\kappa^{-1}\nabla_{R} \phi_{z}), 0 )
  \end{equation}
Here and below by a subscript $g_{z}$ we denote evaluation $g_{z}: = g(z)$. 
  We need to check two \chris{only two below?} \vasily{correct} points to ensure that $(X^L_{\phi_z} (w) , 0) $ is a deformation
  of $g$ in tangent direction to $\mhiggs^{\fr}_{D}$
  \begin{enumerate}
  \item in each chart there exists equivalence frame in which $X(w) \sim (\tilde X_{\phi_{z}}^{L}(w), \tilde X_{\phi_{z}}^{R}(w))$  are regular sections (as functions of $w$).
  \item $X(w = \infty) = 0$.
  \end{enumerate}
  \chris{As well as checking that the deformation doesn't create any new singularities outside of $D$, do we need to check something saying that the deformations don't increase the fixed degree given by $\omega^\vee$ of the singularities at $D$?}\vasily{Yes, I agree. And this actually follows from (1) where we demand
    that $X(w) \sim (\tilde X_{\phi_{z}}^{L}(w), \tilde X_{\phi_{z}}^{R}(w))$
  are regular in each chart, right?}


  To check (1) we need to look on the potential singularities as $w \to z$ or as $w \to z_i$.
  There is no singularity as $w \to z$ since (\ref{eq:xlw0}) can be rewritten
  as
  \begin{equation}
\label{eq:sklyanin-left}
(X^L_{\phi_z} (w) , 0)  =     \frac{1}{w - z} ( A_{g_z}( \kappa^{-1} \nabla_{R} \phi_z)  - A_{g_{w}} (\kappa^{-1}\nabla_{R} \phi_{z}),0)
  \end{equation}
  and since $g(w)$ is regular \chris{In notation, do $g_w$ and $g(w)$ mean the same thing?} \vasily{right, I inserted a sentence after (7)} in $w \to z$ the ratio is also regular at $w \to z$.
  There is also no singularity as $w \to z_i$ in the original frame (\ref{eq:Xphiz}) 
  (or equivalently, the singularity of (\ref{eq:xlw0}) near $w \to z_i$ 
  is in the image of $A_{g_{w}}$ of regular functions by (\ref{eq:Xphiz})).
 The (2) is clear since $g_{w}$ is regular at $w = \infty$. 
\end{proof}

We've just shown that $\mathrm{Im} \, \pi \subset T_{\mhiggs^{\fr}_{D}}$ where
$\pi$ is Sklyanin's Poisson structure. 

Now we'll try to compare the restriction of symplectic
structure on $T_{g}(\mhiggs^{\fr}_{D})$ given by the residue formula to the $\mathrm{im} \, \pi
\subset T_{g}(\mhiggs^{\fr}_D)$, and Sklyanin's Poisson structure $\pi$ on $G_1(C)$. 

 A compatibility of symplectic structure $\Omega$ on a symplectic leaf $S$ in a Poisson space $M$
 with Poisson structure on $M$  means that for any two Hamiltonian functions $\phi, \psi$ on $M$
 we've got
 \begin{equation}
\label{eq:compatibility}
   \Omega(X_{\psi}, X_{\phi}) = \pi (d\phi, d \psi)
 \end{equation}
 where $X_{\psi}, X_{\phi}$ are Hamiltonian vector fields in $T_{S}$ generated by $\phi, \psi$:
 \begin{equation}
   X_{\psi} = \pi d \psi, \qquad X_{\psi} = \pi d \phi 
 \end{equation}

 Equivalently, we want to find symplectic form $\Omega: T_{S} \to T_{S}^{*} $ such
 that for any hamiltonian function $\phi$ on $M$ and any vector field $X$ in $T_{S}$ we
 have
 \begin{equation}
   \Omega(X, \pi d \phi) = d_{X} \phi 
 \end{equation}

 Let $X, X'$ be vectors in the tangent space to $\mhiggs^{\fr}_D$ at a point $g$.

 Consider the bilinear form $\Omega(X, X')$ (motivated by the monopole
 computations in another section...) 
 \begin{equation}
\label{eq:Omega}
   \Omega(X, X')  = \frac{1}{2 \pi \imath} \sum_{z_i \in \tilde D} \oint_{\partial U_i}  dw (
\langle  X^{L}_{i} X^{L'}_{0} \rangle  - \langle X^{R}_{i}  X^{R'}_{0} \rangle )
\end{equation}
where we have choosen the equivalence frames (\ref{eq:equivalence0}) in $U_i$' and $U_0$ such that $X_i^{L}, X_i^{R}$ are regular in small disks $U_i$'s around $z_i$, and $X_0^{L}, X_{0}^{R}$ are regular in
the rest $U_0  = C \setminus \{\tilde D\}$. \chris{Do we need to say something to justify that this is well-defined when we choose different equivalence frames?  I also wonder whether this can be rephrased in terms of a \v Cech complex, like we had before?}

\vasily{I agree, let's justify}  

1. Let's check invariance under a change of equivalence frame 
as in (\ref{eq:equivalence0}). First we consider the change
of equivalence frame in the second argument. 
The parameter of a change is denoted $X_0'$ in the patch $U_0$. The section $X_{0}'$ is regular in $U_{0}$. 
\begin{multline}
   \Omega(X, \tilde X') - \Omega( X,X')  = \frac{1}{2 \pi \imath} \sum_{z_i \in \tilde D} \oint_{\partial U_i}  dw (
   \langle  X^{L}_{i} ,  X^{L'}_{0} + A_{g_{w}} X_{0}' \rangle  - \langle X^{R}_{i} ,  X^{R'}_{0} - X_{0}'\rangle ) - \\
   \frac{1}{2 \pi \imath} \sum_{z_i \in \tilde D} \oint_{\partial U_i}  dw (
   \langle  X^{L}_{i} ,  X^{L'}_{0} \rangle  - \langle X^{R}_{i} ,  X^{R'}_{0} \rangle )   =
\frac{1}{2 \pi \imath} \sum_{z_i \in \tilde D} \oint_{\partial U_i}  dw (
\langle   X_{i}^{L},   A_{g_{w}}X^{'}_{0} \rangle  + \langle  X_{i}^{R},  X^{'}_{0} \rangle ) = \\
\frac{1}{2 \pi \imath} \sum_{z_i \in \tilde D} \oint_{\partial U_i}  dw (
\langle   A_{g_{w}^{-1}}  X_{i}^{L} + X_{i}^{R}, X_{0}' \rangle )  =\\
\frac{1}{2 \pi \imath} \sum_{z_i \in \tilde D} \oint_{\partial U_i}  dw (
\langle   A_{g_{w}^{-1}}  X_{0}^{L} + X_{0}^{R}, X_{0}' \rangle )  =
- \frac{1}{2 \pi \imath} \oint_{\partial U_0} dw (
\langle   A_{g_{w}^{-1}}  X_{0}^{L} + X_{0}^{R}, X_{0}' \rangle )  = 0 \\
\end{multline}

We used that $A_{g_{w}^{-1}}  X_{i}^{L} + X_{i}^{R}$ (variation of $g$ in the right frame)
is invariant across the patches, $A_{g_{w}^{-1}}  X_{i}^{L} + X_{i}^{R} = A_{g_{w}^{-1}}  X_{0}^{L} + X_{0}^{R}$,
in the next to the final equality, and the fact that the integrand is regular on $U_0$ in the final equality
and the residue formula in the final equality. 

2. Since we've proven the invariance under the change of equivalence frame in the second argument,
let's set $X_{0}^{R'} = 0$ by a suitable change of equivalence frame,
and in this equivalence frame we have $X_{0}^{L'} = X_i^{L'} + A_{g} X_i^{R'}$
on overlaps of $U_i$ and $U_0$,  hence
the original formula becomes 
\begin{equation}
\Omega(X, X') =   \frac{1}{2 \pi \imath} \sum_{z_i \in \tilde D} \oint_{\partial U_i}  dw (
\langle  X^{L}_{i}, X_i^{L'} + A_{g} X_i^{R'} \rangle 
\end{equation}
Since $\langle X_i^{L}, X_{i}^{L'} \rangle $ is regular on $U_i$ this term is dropped and we get
\begin{equation}
\label{eq:drop-right}
  \Omega(X, X') =   \frac{1}{2 \pi \imath} \sum_{z_i \in \tilde D} \oint_{\partial U_i}  dw (
\langle  X^{L}_{i},  A_{g} X_i^{R'} \rangle  =  \frac{1}{2 \pi \imath} \sum_{z_i \in \tilde D} \oint_{\partial U_i}  dw (
\langle  A_{g^{-1}} X^{L}_{i},   X_i^{R'} \rangle 
\end{equation}

On the other hand, we can set in the original formula by a suitable change
of framing in the second argument $X_0^{L'} = 0$, and in this equivalence
frame we have $X_0^{R'} =  X_{i}^{R'}  + A_{g^{-1}} X_{i}^{L'}$ on the
overlaps of $U_i$ and $U_0$, and consequently, the original formula is transformed into
\begin{equation}
\label{eq:drop-left}
  \Omega(X, X') =   -\frac{1}{2 \pi \imath} \sum_{z_i \in \tilde D} \oint_{\partial U_i}  dw (
\langle  X^{R}_{i},  A_{g^{-1}} X_i^{L'} \rangle 
\end{equation}
Comparing (\ref{eq:drop-left}) with (\ref{eq:drop-right}) we see that we've proved that $\Omega(X,X')$ is anti-symmetric.


3. Since in (1.) we've proved that $\Omega(X,X')$ is invariant under
the change of equivalence frame in the second argument, and in (2.) we've proved
antisymmetry of $\Omega(X, X')$ (using (1.)), we obtain that $\Omega(X, X')$ is also
invariant under the change of equivalence frame in the first argument.

\vasily{I hope the paragraphs above justify that $\Omega$ is well-defined
  (does not depend on the choice of equivalence frames across the patches
  in each argument. We've also got antisymmetry on the way, and a couple
of equivalent presentation of the formula for $\Omega$} 

\vasily{I've got till here ----------------------  } 





Here $\tilde D = D \cup \{ c \}$, and $\langle, \rangle$ means Killing form $\langle, \rangle = \kappa: \gg \otimes \gg \to \mathbb{C}$.

We don't yet assume that $\Omega$ is non-degenerate.\vasily{adjective
  anti-symmetric is removed since now we've shown anti-symmetry
  and consistency of the definition of $\Omega$ on the previous page} We'll show
that these properties of $\Omega$  we'll be implied by the arguments below.


\begin{lemma}\label{lemma:OmegaPi}
  The bilinear form (\ref{eq:Omega}) on $T_{\mhiggs^{\fr}_{D}}$ 
restricted to the $\mathrm{im} \, \pi \subset T_{\mhiggs^{\fr}_{D}}$  is compatible
  with Sklyanin's Poisson structure $\pi$ on $\mhiggs^{\fr} = G_1(C)$.
\end{lemma}
\begin{proof}
  Given two hamiltonian functions $\phi_{z_1}, \psi_{z_2}$ where $z_1, z_2$ are regular
  points away from $\tilde D$ we'd like to check
  that Sklyanin's Poisson bracket (\ref{eq:Sklyanin})  
is compatible (\ref{eq:compatibility}) with the bilinear form  (\ref{eq:Omega})
\begin{equation}
  \Omega(X_{\psi_{z_1}}, X_{\phi_{z_2}})  = - \{\psi_{z_1}, \phi_{z_2}\}
\end{equation}
First we'll choose the equivalence frames such that $X_{\psi_{z_1}}$ is regular in $\tilde D$. 
while $X_{\phi_{z_2}}$ is regular in $C \setminus \tilde D$.
The equivalence frame regular in $\tilde D$ can be taken as in  (\ref{eq:Xphiz}),
and the equivalence frame regular in $C \setminus \tilde D$ can be taken
as in (\ref{eq:xlw0}). Thus we'll take (in this computation we'll omit writing inverse
Killing form $\kappa^{-1}: \gg^{*} \to \gg $ redefining gradient $\nabla$ to take values in $\gg$ instead of $\gg^{*}$
by compositition with $\kappa^{-1}$. )
\begin{equation}
  \begin{aligned}
    (X_{\psi_{z_1}})_{i} = ((X_{\psi_{z_1}}^{L}, X_{\psi_{z_1}}^{R}))_{i} =  \frac{1}{w - z_1} (A_{g_{z_1}} \nabla_R \psi_{z_1},   \nabla_{R} \psi_{z_1}) \\
    (X_{\phi_{z_2}})_{0} = ((X_{\phi_{z_2}}^{L}, X_{\phi_{z_2}}^{R}))_{0} = ( \frac{1}{w - z_2} ( A_{g_{z_2}} \nabla_R \phi_{z_2} - A_{g_{w}} \nabla_{R} \phi_{z_2}), 0) 
  \end{aligned}
\end{equation}
Then the definition of the bilinear form (\ref{eq:Omega}) gives
\begin{equation}
  \label{eq:pairing}
  \Omega(X_{\psi_{z_1}}, X_{\phi_{z_2}}) = \frac{1}{2 \pi \imath }
  \sum_{z_i \in \tilde D} \oint_{\partial U_i} \frac{dw }{(w - z_1)(w - z_2)} \langle A_{g_{z_1}} \nabla_{R} \psi_{z_1},
  A_{g_{z_2}} \nabla_{R} \phi_{z_2}  - A_{g_w} \nabla_{R} \phi_{z_2} \rangle
\end{equation}
The integrand is regular everywhere on $U_0 \setminus \{z_1, z_2\}$, and since
\begin{equation}
\label{eq:residue}
  \sum_{i \in \tilde D} \oint_{\partial U_i }  = - \oint_{\partial U_0} 
\end{equation}
the result (\ref{eq:pairing}) is a minus sum of residues in $w = z_1$ and $w = z_2$
which gives
  \begin{multline}
\label{eq:z1z2residue}
   \Omega(X_{\psi_{z_1}}, X_{\phi_{z_2}}) = - (\res_{w = z_1}  + \res_{w = z_2}) \frac{dw }{(w - z_1)(w - z_2)} \langle A_{g_{z_1}} \nabla_{R} \psi_{z_1},
   A_{g_{z_2}} \nabla_{R} \phi_{z_2}  - A_{g_w} \nabla_{R} \phi_{z_2} \rangle \\
   =   \frac{-1}{z_1 - z_2}  \langle A_{g_{z_1}} \nabla_{R} \psi_{z_1} , A_{g_{z_2}} \nabla_{R} \phi_{z_2}  - A_{g_{z_1}} \nabla_{R} \phi_{z_2} \rangle   - \frac{-1}{z_1 - z_2}  \langle A_{g_{z_1}} \nabla_{R} \psi_{z_1} , A_{g_{z_2}} \nabla_{R} \phi_{z_2}  - A_{g_{z_2}} \nabla_{R} \phi_{z_2}   \rangle =\\
      =   \frac{-1}{z_1 - z_2}  \langle A_{g_{z_1}} \nabla_{R} \psi_{z_1} ,  - A_{g_{z_1}} \nabla_{R} \phi_{z_2} \rangle   - \frac{-1}{z_1 - z_2}  \langle A_{g_{z_1}} \nabla_{R} \psi_{z_1} ,  - A_{g_{z_2}} \nabla_{R} \phi_{z_2}   \rangle =\\
   = \frac{1}{z_1 - z_2} (\langle \nabla_{R} \psi_{z_1} , \nabla_{R} \phi_{z_2}\rangle - 
   \langle  \nabla_{L} \psi_{z_1} , \nabla_{L} \phi_{z_2} \rangle ) =
   - \{\psi_{z_1}, \phi_{z_2} \} 
  \end{multline}
  \chris{I'm not following part of the argument here.  We're using a calculation of $\frac 1{w-z_2}(A_{g_{z_2}} \nabla_{R} \phi_{z_2}  - A_{g_w} \nabla_{R} \phi_{z_2})$ in the limit $w\to z_2$?}
  \vasily{I've added more intermediate steps. There are two terms in the second line in (\ref{eq:z1z2residue}).
    The first term came from  the residue at $w=z_1$.
    The second term came from the residue at $w=z_2$. 
In the third to the fourth line of the computation we used that Killing form is adjoint invariant
$\langle A_{g} X, A_{g} X' \rangle = \langle X, X' \rangle $,
and the relation between left and right gradient $\nabla_{L} \phi_{z_i} = A_{g_{z_i}} \nabla_{R} \phi_{z_i}$}
\end{proof}

So we've shown (pointwise) that restriction of the bilinear form $\Omega$ on $T_g\mhiggs^{\fr}_{D}$
  to the subspace $\mathrm{im} \pi \subset T_g\mhiggs^{\fr}$ is compatible with
Sklyanin's Poisson structure $\pi$.

To complete the argument  $(\mhiggs^{\fr}_{D}, \Omega)$ is
a symplectic leaf for $(G_1(C), \pi)$ we'd need to show that in fact
$\mathrm{im} \, \pi = T_{\mhiggs^{\fr}_{D}}$.
That means for any $X \in T_{\mhiggs^{\fr}_{D}} $ we'd like to find a Hamiltonian
function $\phi$ on $G_1(C)$ such that $X = X_{\phi} = \pi d \phi$.

We'll try to do that in the next section. Here we'll notice some observations
about vector fields generated by adjoint invariant function $\phi_z$ that are useful
to construct integrable system 
\begin{lemma}
 If $\phi_z$ is an adjoint invariant function on $G$, then
$\nabla_{L} \phi_z  = \nabla_{R} \phi_z$ and consequently $X_{\phi_z}(w)$ is
a vector field generating adjoint transformation of $g(w)$. 
\end{lemma}
\begin{proof}
  If $\phi$ is adjoint invariant then
  \begin{equation}
    \frac{d}{dt} \phi(e^{Xt} g e^{-Xt}) = 0
  \end{equation}
  and consequently $\nabla_{L} \phi_z - \nabla_{R} \phi_z  = 0$. Therefore $X_{\phi_z}(w)$ is
  of the form $(X(w), -X(w))$ and that is adjoint action on $g(w)$. 
\end{proof}

\begin{corollary}\label{cor:poisson-commuting}
  If $\psi_{w}$ and $\phi_{z}$ are adjoint invariant functions then they Poisson commute 
  \begin{equation}
    \{ \psi_{w}, \phi_{z} \} = 0
  \end{equation}
\end{corollary}

Consequently, this immediately implies that intersection
of the fibers of the multiplicative Hitchin
projection $G_1(C) \to T(C)/W$ where $T$ is the maximal torus and $W$ is the Weyl group (generated
by adjoint invariant functions) with symplectic leaves are isotropic subspaces of symplectic leaves.



\vasily{OK, I repsponded to comments until here (Mon Apr 29, 9:41 AM)}


\subsection{Symplectic structure on symplectic leaves in $G_1(C)$}

Now we consider a symplectic leaf in $G_1(C)$. 

In this subsection we'll try to prove
\begin{theorem}\label{theorem:symplectic_leaf}
  Let $G$ be a reductive group. We'll take the underlying $G$-principal bundle on $C$
  to be trivial bundle with a framing at $c \in C$. 
In the Poisson-Lie group $(\mhiggs^{fr}, \pi)$ with Sklyanin's bracket $\pi$
on the space of meromorphic functions $G_1(C)$ with fixed framing at infinity $c \in C$
consider a subspace  $\mhiggs^{fr}_{D}$
defined 
  by fixing singularity of $g$ at $D$ where $D$ is a divisor colored in dominant co-weights of $G$. Then $\mhiggs^{fr}_{D}$ is a symplectic leaf in $(\mhiggs^{fr}, \pi)$. 
\end{theorem}

We'll try to show this at the level of tangent spaces in a nice point $g \in \mhiggs^{fr}$.  For any deformation $\delta g $ of $g$ that
preserves the singularity divisor $D$ of $g$ we want to find a Hamiltonian
function $\phi$ for Sklyanin's bracket such that its vector field
$X_\phi = \delta g$.  (The other direction we've already shown in lemma \ref{phitoX}.)


First let's compute the tangent space of $\mhiggs_{D}^{\fr}$ at point $g$.
Assume that $g$ is regular semi-simple near singularities. That means assume
that there is a punctured neighborhood $U_i^{\times} = U_{i} \setminus \{z_i \}$ of each singularity $z_i \in D $ in which $g(z)$ is regular semi-simple for $z \in U_{i}^{\times}$. 
With this assumption, the operator $A_{g}(z): \gg \to \gg$ is diagonalizable in $U_{i}^{\times}$.
\newcommand{\hh}{\mathfrak{h}}
Pick ($z$-dependent) Cartan sublagebra $\hh_{z} \subset \gg$  for $z \in U_{i}^{\times}$
to be the centralizer of $g(z)$ in analytic way in $z$. 

Then we have ($z$-dependent) splitting of $\gg$ to the Cartan $\hh_z$ and the root spaces 
$\gg = \hh_{z} \oplus \sum_{\alpha} \gg_{\alpha,z}$. 
If $g(z)$ has Dirac singularity of co-weight $\omega^{\vee}$ in $z_i$ with $[g(z) ] \sim z^{-\omega_i^{\vee}}$, then  $A_{g}$ has eigenvalue $ (z - z_i)^{- (\alpha, \omega^{\vee}_{i})}$ on $\gg_{\alpha}$
in the leading order. Let $e_{\alpha, z}$ be a generator of $\gg_{\alpha, z}$.

Below we'll need a technical assumption on $g(z)$ in $U_{i}^{\times}$ that the splitting
$\hh_{z} \oplus \sum_{\alpha} \gg_{\alpha,z}$ extends analytically from $U_{i}^{\times}$ to $U_{i}$,
i.e. there is a limit 
 \[ e_{\alpha, z_i} = \lim_{z \to z_i} e_{\alpha, z}\] 
\vasily{The existence of this limit will be probably implied by the
  assumption that $g(z)$ is regular semi-simple in the punctured
  neighborhood $U_{i}^{\times}$ and has the form $ g_L(z) (z-z_i)^{-\omega_{i}^{\vee}} g_{R}(z)$
  where $g_{L}(z), g_{R}(z)$ are analytic in $U_i$. }


\begin{lemma}
  The tangent space of $\mhiggs^{\fr}_{D}$ to $g$ with semi-simple singularities 
  $D = \{(z_i, \omega^{\vee}_{i})\}$ in an equivalence frame with
  vanishing right vector field component $X = (X^{L}, 0)$ is isomorphic to the
  space of meromorphic $\gg$-valued functions $X^{L}$ of the form
  \begin{equation}
\label{eq:XL}
    X^{L} = \sum_{i} \sum_{\alpha: \langle \alpha, \omega_i^{\vee} \rangle > 0 }
    \sum_{k_{i, \alpha} = 1}^{ \langle \alpha , \omega_i^{\vee} \rangle} e_{\alpha,{z_i}} x_{i, \alpha, k_{i, \alpha}} (z - z_i)^{-\langle \alpha, \omega_i^{\vee}\rangle }, \qquad x_{i, \alpha, k_{i, \alpha}} \in \mathbb{C}
  \end{equation}
  and consequently at $g$ with regular semi-simple singularities we have smooth
  tangent space of $\mhiggs^{\fr}_{D}$ of dimension 
  \begin{equation}
\dim T_{\mhiggs^{\fr}_{D}} = \sum_{i} \sum_{\alpha: \langle \alpha, \omega_i^{\vee} \rangle > 0 } \langle \alpha, \omega_i^{\vee} \rangle = 2 \sum_{i} \langle \rho, \omega^{\vee}_i\rangle 
  \end{equation}
  where $\rho$ is the Weyl weight vector
  \begin{equation}
    \rho = \frac 1 2 \sum_{\alpha > 0} \alpha 
  \end{equation}
\end{lemma}
\begin{proof}
  In the framing $(X^{L}, X^{R}) \sim (X^{L} + A_{g} X^{R}, 0 )$ the singular
  part of $X^{L}$ in $U_{i}^{\times}$ in the subspace generated by $e_{\alpha, z_i}$
  is in the image of $A_{g}$ applied to a regular section generated by $e_{\alpha, z_i}$,
  and consequently, there is an equivalence frame $(X^{L}, X^{R}) \sim (\tilde X^{R}, \tilde X^{R})$   in which $\tilde X^{L}, \tilde X^{R}$ are both regular in $U_{i}$. 
\end{proof}

After we've described the tangent space  $T_{\mhiggs_{D}^{\fr}}$ (more precisely its fiber
a nice poly-stable base point $g_0$ )   we'd like to show that
for every vector $X$ in $T_{g_0}(\mhiggs_{D}^{\fr})$ there is a 
hamiltonian function $\phi$ on $G(C)$ such that $X$ is generated by $\phi$ at $g_0$ with respect Sklyanin's Poisson structure on $(G_1(C), \pi)$
\begin{equation}
  \forall X \in  T_{g_0}( \mhiggs_{D}^{\fr}) \qquad \exists \phi: X  = \pi d_{g}\phi(g) |_{g_0}
\end{equation}
This we'll show that $\mathrm{im} \pi = T_{\mhiggs_{D}^{\fr}}$
and complete the symplectic leaf theorem since equivalence of $\Omega$ restricted to
$\mathrm{im} \pi $ with $\pi$ is already shown in the lemma \ref{lemma:OmegaPi}.


Suppose that $X$ in $T_{g_0}( \mhiggs_{D}^{\fr}) $ is cast in the equivalence frame (\ref{eq:equivalence0}) in the format $X = ((X^{L}, X^{R}))_{i}$ where $(X^{L}, X^{R})_{i}$ are regular
in each patch $U_i$ around $z_i \in \tilde D$. Here is a construction of $\phi$ (of course not unique) such that $X = X_{\phi}$.


\begin{lemma}\label{xtophi}
  For any $X \in T_{g} \mhiggs^{\fr}_{D}$ there is a local potential (hamiltonian function)
  $\phi$ on $G_1(C)$ such that $X = \pi d \phi|_{g} $ and this function $\phi(\tilde g)$
  in a in a local neighborhood of the base point $g$
  can be constructed 
  concretely for example by the formula \footnote{recall
  that we assumed that we have a faithful representation $\rho$
  of reductive group  $G$ such that $\tr\rho(X) \rho(X') = c_\rho \kappa(X, X')$, hopefully
we don't loose any generality here}
\begin{equation}
  \phi(\tilde g) : = -\frac 1 2 \sum_{z_i \in \tilde D} \frac{1}{2 \pi \imath c_\rho} \oint_{\partial U_{i}} 
dz  \tr \rho( \tilde g_z  g_z^{-1}) \rho(X^{L}_i - A_{ g_z} X_{i}^{R}) 
\end{equation}
\end{lemma}
  
\begin{proof}
The left gradient of $\phi(\tilde g)$ under $ \delta \tilde g = Y^{L} \tilde g$ at $\tilde g = g$ is
\begin{equation}
  Y^{L} \nabla_{L} \phi|_{\tilde g = g}  = - \frac 1 2  \sum_{z_i \in \tilde D} \frac{1}{2 \pi \imath } \oint_{\partial U_{i}}  dz
 \langle  Y^{L} , (X^{L}_i - A_{g_z} X_i^R) ) 
\end{equation}
so 
\begin{equation}
  \nabla_{L} \phi|_{\tilde g = g}  = -\frac 1 2  \sum_{z_i \in \tilde D} \frac{1}{2 \pi \imath } \oint_{\partial U_{i}}  dz
 (X^{L}_i - A_{g_z} X_i^R) 
\end{equation}
where we've identified $\gg$ and $\gg^{*}$ using the Killing form $\langle , \rangle$. 

Now we'll compute Sklyanin's Hamiltonian vector field in the left frame
$(X_{\phi}^{L}, 0)$ by the equation (\ref{eq:xlw0}) (and using that $\nabla_{R} \phi  =
A_{g^{-1}} \nabla_{L} \phi$):
\begin{multline}
  (X^{L}_\phi, 0) = -\frac 1 2 \sum_{i \in \tilde D} \frac{1}{2 \pi \imath} \oint_{\partial U_i}
  \frac{dz}{w - z} ( (X_i^{L} - A_{g_z} X_i^{R})  - A_{g_w} A_{g_z}^{-1} (X_i^{L} - A_{g_z} X_i^{R}),0) =\\
=  -\frac 1 2 \sum_{i \in \tilde D} \frac{1}{2 \pi \imath} \oint_{\partial U_i}
\frac{dz}{w - z} ( (X_i^{L}  + A_{g_w} X_i^{R}) - (A_{g_z} X_i^{R} +  A_{g_{w}} A_{g_{z}}^{-1} X_i^{L}),0) \\
 = -\frac 1 2  \sum_{i \in \tilde D} \frac{1}{2 \pi \imath} \oint_{\partial U_i}
 \frac{dz}{w - z} ( (X_i^{L}  + A_{g_w} X_i^{R}) - (X_0^{L} + A_{g_z} X_0^{R}- X_{i}^{L} +  A_{g_{w}} ( A_{g_{z}}^{-1} X_0^{L} + X_0^{R} - X_{i}^{R})),0)= \\
  = -\frac 1 2  \sum_{i \in \tilde D} \frac{1}{2 \pi \imath} \oint_{\partial U_i}
    \frac{dz}{w - z} ( 2 (X_i^{L}  + A_{g_w} X_i^{R}) - (X_0^{L} + A_{g_z} X_0^{R} +  A_{g_{w}} ( A_{g_{z}}^{-1} X_0^{L} + X_0^{R} )),0)
  \end{multline}
  Now, as function of $z$, the first term of the numerator of the integrand is regular in each $U_i$, and the second term of the numerator is regular in $U_0$. Therefore,
  the first term  evaluates to the residue at $z = w$ while the second is zero if $w \in U_i$,
  and inversely,  the second term evaluates to a residue at $z = w$
 while the first is zero if $w \in U_0$. (Assume that $w$ is not on the integration
  contour, as we can always move it away.)
Either way, if $w \in U_i$ we have from the residue in the first term
  \begin{equation}
    (X^{L}_\phi, 0)(w) = \mathrm{res}_{z = w} \frac{dz}{ z - w} (X_{i}^{L} + A_{g_w} X_{i}^{R}) =
    (X_{i}^{L} + A_{g_w} X_{i}^{R})(w), \qquad w \in U_i
  \end{equation}
  and if $ w \in U_0$ we have from the residue in the second term by (\ref{eq:residue})
  \begin{equation}
    (X^{L}_\phi, 0)(w) = \frac 1 2 \mathrm{res}_{z = w} \frac{dz}{ z - w} (X_0^{L} + A_{g_z} X_0^{R} +  A_{g_{w}} ( A_{g_{z}}^{-1} X_0^{L} + X_0^{R} )) =
   ( X_{0}^{L} + A_{g_w} X_{0}^{R})(w), \qquad w \in U_0 
  \end{equation}
  and thus $X_\phi$ in the left equivalence
  frame $(X^{L}_{\phi}, 0)$ in each patch matches the vector field $X$ that we've started
  in the same left equivalence frame $(X^{L}, X^{R}) \sim (X^{L} + A_{g} X^{R}, 0 )$.
\end{proof}

\begin{proof}[proof of theorem \ref{theorem:symplectic_leaf}]
Now we've shown in lemma \ref{phitoX} that $\mathrm{im}\, \pi \subset T_{g} \mhiggs^{\fr}_{D}$
and in lemma \ref{xtophi} that $ T_{g} \mhiggs^{\fr}_{D} \subset \mathrm{im}\, \pi$,
hence $\im \pi =  T_{g} \mhiggs^{\fr}_{D}$. In combination with lemma \ref{lemma:OmegaPi}
on compatibility of $\pi$ and $\Omega$ this implies that $\mhiggs^{\fr}_{D}$ is
a symplectic leaf for Poisson-Lie group $(G_1(C), \pi)$ of framed rational $G$-valued functions
on $C$ equipped with  with Sklyanin's Poisson structure $\pi$. In particular,
since $\pi$ is Poisson, the bilinear form $\Omega$ is non-degenerate, closed, antisymmetric form
and so $\Omega$ is actually a  symplectic structure on $\mhiggs^{\fr}_{D}$. 
\end{proof}




 % Consider the variation of the one-form $g^{-1} dg$ and $dg g^{-1}$ with respect to
 % \begin{equation}
 %   \delta g  = X_L g  + g X_{R}
 % \end{equation}
 % we get
 % \begin{equation}
 %   \begin{aligned}
 %     \delta ( dg g^{-1}) =   g d X_{R} g^{-1} + [d - dg g^{-1}, X_L] \\
 %     \delta ( g^{-1} dg) =  [d + g^{-1} dg, X_{R}] + g^{-1} dX_L g 
 %   \end{aligned}
 % \end{equation}



 
\chris{Is there something wrong with the following formal argument that your Lemma \ref{lemma:OmegaPi} suffices?}
\vasily{Alternatively, we'll also give the following argument which
  would be applicable if we also show that $\Omega$
  is a non-degenerate  bilinear form on $T_{g} \mhiggs^{\fr}_{D}$. }


Let me write $i \colon \mhiggs^\fr_D \to G_1(C)$ for the inclusion map, and fix a point $g \in \mhiggs^\fr_D$.  So we have a diagram of linear maps
\[\xymatrix{
T^*_g \mhiggs^\fr_D \ar[d]^{\omega^\vee_g} &T^*_g {G_1(C)} \ar[d]^{\pi_g} \ar[l]_{\d i^\vee} \\
T_g \mhiggs^\fr_D \ar[r]^{\d i} &T_g {G_1(C)}. 
}\]
where the map $\omega^\vee_g$ induced from the symplectic structure is an isomorphism, the map $\d i$ is injective, and its dual $\d i^\vee$ is surjective.  Concretely -- just to make sure we're clear on functional analysis -- the tangent space $T_g {G_1(C)}$ is the subspace of the colimit $T_g G_1[[z^{-1}]] = \colim_n \gg_1[z^{-1}]/z^{-n}$ consisting of jets of rational functions, so the map $\d i^\vee$ is defined as the limit of the linear duals of the projected maps $\tau_n \circ \d i \colon T_g \mhiggs^\fr_D \to \gg_1[z^{-1}]/z^{-n}$.



\begin{lemma} \label{commuting_diagram_lemma}
This square of linear maps commutes, i.e. $\pi_g = \d i \omega^\vee_g \d i^\vee$.
\end{lemma}

\begin{proof}
Choose a covector $\alpha_\infty \in T^*_g {G_1(C)}$, and consider its image $\pi_g(\alpha_\infty)$ in $T_g G_1(C)$.  By Lemma \ref{phitoX}, there is an inclusion $\im(\pi_g) \sub \im(\d i)$, so we can find $X \in T_g \mhiggs^\fr_D$ with $\d i(X) = \pi_g(\alpha_\infty)$.  Because $\omega^\vee_g$ and $\d i^\vee$ are surjective, find a new covector $\alpha_\infty' \in T^*_g {G_1(C)}$ so that
\begin{align*}
\omega^\vee_g \d i^\vee(\alpha'_\infty) &= X \\
\text{so } \d i \omega^\vee_g \d i^\vee(\alpha'_\infty) &= \d i(X) \\
&= \pi_g(\alpha_\infty).
\end{align*}
We need, therefore, to show that $\alpha_\infty = \alpha'_\infty$ to show that the square commutes.  To do so, choose another covector $\beta_\infty \in T^*_g G_1(C)$.  By the above we have
\begin{align*}
\pi_g(\alpha_\infty)(\beta_\infty) &= \d i \omega^\vee_g \d i^\vee(\alpha'_\infty)(\beta_\infty) \\
&= \omega^\vee_g \d i^\vee(\alpha'_\infty)(\d i^\vee(\beta_\infty))\\
&= \omega_g(\omega^\vee_g \d i^\vee(\alpha'_\infty), \omega^\vee\d i^\vee(\beta_\infty))
\end{align*}
where $\omega_g$ is the non-degenerate pairing on $T_g \mhiggs^\fr_D$ given by the symplectic form.  But because any tangent vector in $T_g G_1(C)$ of the form $\d i \omega^\vee_g \d i^\vee(\gamma)$ for some $\gamma$ is necessarily in the image of $\pi_g$ (as argued above), we can apply Lemma \ref{lemma:OmegaPi}, which tells us that 
\begin{align*} 
\omega_g(\omega^\vee_g \d i^\vee(\alpha'_\infty), \omega^\vee\d i^\vee(\beta_\infty)) &= \pi_g(\alpha'_\infty, \beta_\infty)\\
&= \pi_g(\alpha'_\infty)(\beta_\infty).
\end{align*}
In particular, since this holds for every $\beta_\infty$, we have $\alpha_\infty = \alpha'_\infty$, and so the diagram commutes.
\end{proof}

\begin{corollary}
There is an inclusion $\im(\d i) \sub \im(\pi_g)$. 
\end{corollary}

\begin{proof}
Let $X_\infty = \d i X$ be a vector field in the image of $\d i$.  Because $\omega^\vee_g$ and $\d i^\vee$ are surjective, we can find a covector $\alpha_\infty \in T^*_g {G_1(C)}$ so that $\d i \omega^\vee_g \d i^\vee(\alpha_\infty) = X_\infty$.  By Lemma \ref{commuting_diagram_lemma}, $\d i \omega^\vee_g \d i^\vee(\alpha) = \pi_g(\alpha_\infty)$, so $X_\infty$ is in the image of $\pi_g$ as required.
\end{proof}

In particular, combining this with Lemma \ref{phitoX}, $\im(\d i) = \im(\pi_g)$.



\section{Symplectic reduction by residue at infinity}

Assume that we consider the moduli space of framed Higgs bundles $\mhiggs^{\fr}_{D, g_{\infty}}$ 
with a regular semi-simple framing value $g_{\infty}$ of the Higgs field at the framing point $c = \infty$.
By the previous section $\mhiggs^{\fr}_{D, g_{\infty}}$ is a symplectic leaf with respect
to Sklyanin's Poisson structure on the rational Poisson-Lie group $(G_{1}, \pi)$. (A multiplication
by a constant $g_\infty$ sending $G_{1}(C) \to G_{g_{\infty}}(C)$ preserves
 Sklyanin's Poisson bracket, and so we have isomorphic symplectic structures on $\mhiggs^{\fr}_{D, g_{\infty}}$ and $\mhiggs^{\fr}_{D, 1}$)



Let $H \subset G$ be a centralizer of the element $g_\infty \in G$, so that $h g_\infty h^{-1}  = g_\infty$ for all $h \in H$. By the assumption that $g_\infty$ is a regular semi-simple element,
we can consider $H$ to be a Cartan subalgebra of $G$. Notice that adjoint action $A_{h}$  by
a constant element $h \in H$ on Higgs field $g(z)$
\begin{equation}
A_{h} :  g(z) \mapsto h g(z) h^{-1} 
\end{equation}
preserves the type of singularities at the divisor $D$ and also
preserves  the framing $g_\infty$. Therefore, the differential of $A_{h}$ is
a tangent vector in $T_{g} \mhiggs^{\fr}_{D, g_\infty}$. 

Recall from corollary \ref{cor:poisson-commuting}
 that adjoint invariant functions of $g(z)$ descent to 
 Poisson commuting functions on $\mhiggs^{\fr}_{D, g_\infty}$.
 For example, let $\rho$ be a faithful representation of $G$.
 % $c_\rho$ is a constant
 % such that  $\tr \rho(X) \rho(Y)  = c_\rho \langle X, Y \rangle$. 
The
evaluation character  $\varphi_{\rho,z}: G(C) \to \mathbb{C}$  defined by
   \begin{equation}
     g \mapsto  tr_{\rho} g(z)
   \end{equation}
 is adjoint invariant function. 


 We'll show that $H$ action on $\mhiggs^{\fr}_{D, g_\infty}$ is Hamiltonian
 and that the vector fields of $H$-action on $\mhiggs^{\fr}_{D, g_\infty}$ are generated by the residues at infinity of adjoint
 invariant hamiltonian functions.

 First we'll proceed in reverse direction, we'll show
 thatr residue at infinity of an adjoint invariant function generates a
 vector field for an adjoint action by an element $h \in H$. 
 \begin{lemma} Let $\varphi_{\rho, z}$ be a character of a faithful $G$-representation evaluated at $z$. Define
   \begin{equation}
    \res  \varphi_{\rho, \infty} : = \frac{1}{2 \pi \imath} \oint_{\partial U_{\infty}}  dz \varphi_{\rho, z} 
   \end{equation}
   to be the residue of $\varphi_{\rho, \infty}$ at infinity. 
   Then the Hamiltonian vector field  $X_{\rho, \infty} = \pi d \res \varphi_{\rho, \infty}$
   generates a constant adjoint action by $h \in H$ on $\mhiggs^{\fr}_{D, g_\infty}$
   where $H \subset G$ is a centralizer of $g_\infty$. 
 \end{lemma}

 
 \begin{proof}
   From the left frame version of Sklyanin's formula (\ref{eq:sklyanin-left})  we have
   \begin{equation}
     (X^{L}_{\rho, \infty}(w),0)  =\frac{1}{2 \pi \imath} \oint_{\partial U_{\infty}} \frac{dz}{w - z}
     ((A_{g_z} \nabla_{R} \phi_{\rho, z} - A_{g_{w}} \nabla_{R} \phi_{\rho, z} ), 0)
   \end{equation}
We've identified $\gg$ and $\gg^{*}$ by the Killing form. 
   The numerator of the integrand is a regular function in $U_{\infty}$ away from $z = \infty$.
   Therefore the contour integral is evaluated as a sum of residues at $z=\infty$
   and $z = w$ if $w \in U_{\infty}$. The residue at $z = w$ vanishes anyway because
   numerator vanishes at $z = w$. Therefore
   \begin{multline}
     (X^{L}_{\rho, \infty}(w),0)  = \mathrm{res}_{z = \infty} (\frac{dz}{z - w}
     (A_{g_z} \nabla_{R} \phi_{\rho,z}  - A_{g_{w}} \nabla_{R} \phi_{\rho, z} ),0)= \\
    = ( (A_{g_\infty} - A_{g_w}) \nabla_{R} \phi_{\rho, \infty} ,0) \sim (\nabla_{L} \phi_{\rho, \infty},
    - \nabla_{R} \phi_{\rho, \infty})
\end{multline}
where the last operation $\sim$ is an equivalence (\ref{eq:equivalence0}). 
Consequently
we've obtained that $X_{\rho, \infty}$ is a constant vector field ($w$-independent)
\begin{equation}
  X_{\rho, \infty} = (\nabla_{L} \phi_{\rho, \infty},
    - \nabla_{R} \phi_{\rho, \infty})
\end{equation} 
Moreover, since $\phi$ is adjoint invariant it holds that $\nabla_{L} \phi_{\rho, \infty} = \nabla_{R} \phi_{\rho, \infty}$, and therefore $X_{\rho, \infty}$ generates a constant adjoint $G$-action on $\mhiggs^{\fr}_{D, g_\infty}$  by $\gg$ element $\nabla_{L} \phi_{\rho, \infty} $. It remains to see
that $\nabla_{L} \phi_{\rho, \infty} $ belongs to the Lie
algebra of the centralizer $H \subset G$ of $g_{\infty}$.  Recall
that $\phi_{\rho, \infty} = \tr \rho(g_\infty)$. By definition, using the Killing form,
$X = \nabla_{L} \phi_{\rho, \infty}$ is an element of $\gg$ such that\footnote{We are using
  short hand notation $\tr_{\rho} a b c \dots = \tr \rho(a) \rho(b) \rho(c) \dots$}
\begin{equation}
  \tr_{\rho} Y g_\infty  = \langle Y, X \rangle \qquad \forall Y \in \gg
\end{equation}

We want to show that $X$ is in the centralizer of $g_\infty$, that is $\tilde X = X$
where $\tilde X: = A_{g_\infty} X$. Then $\tilde X$ is defined by 
\begin{equation}
  \tr_{\rho} Y g_\infty  = \langle A_{g_\infty} Y, \tilde X \rangle \qquad \forall Y \in \gg
\end{equation}
Let $\tilde Y =  A_{g_\infty} Y$, and since $A_{g_\infty}: \gg \to \gg$ is isomorphism,
then $\tilde X$ is defined by 
\begin{equation}
  \tr_{\rho}  A_{g_\infty^{-1}} \tilde Y g_\infty  = \langle \tilde Y, \tilde X \rangle \qquad \forall \tilde Y \in \gg
\end{equation}
where the left hand side translates into
\begin{equation}
  \tr_{\rho} g_\infty^{-1} \tilde Y g_\infty g_{\infty} = \tr \tilde Y g_\infty 
\end{equation}
where the last equality is by cyclic trace invariant. Therefore, $\tilde X$ is
uniquely determined by the relation as $X$, and hence $\tilde X = X$, so $A_{g_\infty} X = X$,
and hence $X$ is in the centralizer $H \subset G$ of $g_\infty$. 
\end{proof}



Now we'd like to see that a collection of hamiltonians functions $\res \phi_{\rho_i, \infty}$
for a sufficient set of faithful $G$-representations $\{ \rho_i \}$ will
generate the action of the whole group $H$, the centralizer of $g_\infty$.
Since $g_\infty$ is regular semi-simple, choose $H$ to be a Cartan subgroup.
Then the vector field $X_\rho$ generated by $\res \rho_{\phi, \infty}$ as in the previous lemma is
defined by the relation 
\begin{equation}
    \tr_{\rho} Y g_\infty  = \langle Y, X \rangle \qquad \forall Y \in \hh
  \end{equation}
  and now we can assume that  $X, Y \in \hh$ and $g_\infty \in H$. Take $g_\infty$
  in the form\footnote{this form
    is motivated by the gauge theory construction in NP 2012 where $\tau$
  are coupling constants in a quiver theory} 
\newcommand{\qq}{\mathfrak{q}}
  $g_\infty =   \prod_{k} \qq_k ^{\omega_k^{\vee}}$
  where $\omega_k^{\vee}$ are fundamental co-weights $\omega_{k}^{\vee}: \mathbb{C}^{\times} \to H$
and their evaluation is  denoted as $\omega: \qq \mapsto \qq^{\omega^{\vee}}$. Assume $|\qq_k | < 1$. 
  That means $X_\rho \in \hh$ is uniquely defined by 

\begin{equation}
    \sum_{w \in \mathrm{weights}_{\rho}} w(Y) w(g_\infty)  = \langle Y, X \rangle 
  \end{equation}
 Identified with $\hh^{*}$ element, the $X_\rho$ is 
  \begin{equation}
\label{eq:Xrho}
    X_{\rho} = \sum_{w \in \mathrm{weights}_{\rho}} w   \prod_{k} \qq^{w(\omega_k^{\vee})}
  \end{equation}
  where the sum is over weights $w \in \hh^{*}$  of $\rho$. 

Assume $\rho$ is a finite-dimensional irreducible representation with highest weight $w_h$.
  We'll pick the highest weight term and notice that the other weights $w'$
  differ from the highest weight $w_{h}$ by $w' = w_{h} - \sum {n_i \alpha_i} $ where $n_i \geq 0$,
  $\sum n_i \geq 1$ and $\alpha_i$ are simple roots dual to coweights $\omega_i^{\vee}$.
  Therefore, the terms corresponding to the lower weights $w'$
  in the sum we'll be suppressed by the coefficient $ \prod_{k} \qq_k^{n_k}$.
  Take a collection of highest weight irreducible representations $\{\rho_{w}\}$
  with linearly independent highest weights $w_{\rho}$ (for example fundamental weights). Then in the leading order at $\qq_k \to 0$ the $X_{\rho}$ is determined by the first term
  with highest weight $w_{\rho}$ and the corrections are polynomial in $\qq_k$
  vanishing at $\qq \to 0$. For a collection of highest weight representations $\{ \rho_{w} \}$ 
  with linearly independent highest weights $w$ we obtain that $X_{\rho_{w}}$ are linearly
  independent in the limit $\qq \to 0$.  Since linear independence is
  an open condition, we get that $X_{\rho_w}$ are linearly independent
  for sufficiently small $\qq_k$.

  \vasily{Chris, perhaps with a small push of thought we could prove that $X_{\rho_w}$
    are linearly independent for a set of highest weight fundamental representations
    for all $|\qq_k| < 1$? }

\begin{remark}
  The limit of small $\qq_k$ is the perturbative limit of gauge theoreis in [NP2012]
  and as well is the perturbative limit of Foscolo's construction of monopoles
  corresponding to the large value of Higgs field at infinity. Again
  we see that it is easier to analyize the geometry first in this limit. 
\end{remark}

So we've proved the following lemma.
\begin{lemma}
  Suppose that $g_\infty$ is semi-simple and very regular element
  (we say very regular if $[g_\infty] = \prod_{k} \qq^{\omega_k^{\vee}}$ for small $\qq_k$).
  Then for a collection of highest weight representations $\rho_w$ with
  linearly independent highest weights $w$, the Hamiltonian functions
  $\res \phi_{\rho_w, \infty}$ generate the injective action of subgroup $H \subset G$
  that is a centralizer of $g_\infty$. 
\end{lemma}

\newcommand{\red}{\mathrm{red}}
\begin{corollary}
  For a semi-simple (very) regular element $g_\infty \in G$,
  there is a symplectic space $\mhiggs^{\fr, \red}_{D, g_\infty}$ defined as symplectic reduction
  by the centralizer $H$ of $g_\infty$ of the symplectic leaf $\mhiggs^{\fr}_{D, g_\infty}$
  \begin{equation}
 \mhiggs^{\fr, \red}_{D, g_\infty} = (\res \phi_{\rho_{w}, \infty} )^{-1} (c_{w}) // H
  \end{equation}
  where $\rho_{w}$ runs over a collection of irreducible
  highest weight representations with linearly independent highest weights $w$,
  and $c_{w} \in \mathbb{C}$. Then
  \begin{equation}
   \dim_{\mathbb{C}}  \mhiggs^{\fr, \red}_{D, g_\infty}  =    \dim_{\mathbb{C}}  \mhiggs^{\fr}_{D, g_\infty} - 2 \rank(G)
  \end{equation}
\end{corollary}
  
\begin{conjecture}
  For a semi-simple very regular $g_\infty$, the reduced space  $\mhiggs^{\fr, \red}_{D, g_\infty}$  supports  algebraic integrable system with generically compact smooth fibers.
\end{conjecture}
\vasily{For $G =GL(n)$ the proof is straighforward by counting the genus of the spectral curve
with Newton polygon}

\begin{conjecture}
  For a semi-simple very regular $g_\infty$, the reduced space  $\mhiggs^{\fr, \red}_{D, g_\infty}$
  supports hyperKahler metric isomorphic
  to the metric on the moduli space of monopoles on $\mathbb{R}^2 \times S^1$
  with fixed $z^{0}$ and $z^{-1}$ coefficients of the conjugacy class of the asymptotics
  of the Higgs field as constructed by Charboneau-Mohizuki's refinement of the heat flow
  construction of Donaldson and Simpsons, and constructed physically by Kapustin and Cherkis.
\end{conjecture}
\vasily{We've discussed the proof in the last few weeks. Mochizuki's construction of heat
  flow applies in this situation}. 


 





\bibliographystyle{alpha}
\bibliography{Mult_Hitchin}

\textsc{Institut des Hautes \'Etudes Scientifiques}\\
\textsc{35 Route de Chartres, Bures-sur-Yvette, 91440, France}\\
\texttt{celliott@ihes.fr}\\ 
\texttt{pestun@ihes.fr}
 
\end{document}

