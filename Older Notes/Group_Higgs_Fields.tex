\documentclass[12pt,psamsfonts,reqno]{amsart} 
\usepackage{ytableau}
\usepackage[margin=1in]{geometry}
\usepackage{amsmath}
\usepackage{amssymb}
\usepackage{amsxtra}
\usepackage{amscd}
\usepackage{color}
\usepackage{tikz-cd}
\usetikzlibrary{matrix,arrows,decorations.pathmorphing}
\usepackage[mathscr]{euscript}
\usepackage{amscd}
\usepackage{slashed}
\usepackage{listings}
\usepackage[right]{showlabels}
\usepackage{graphicx}
\usepackage{todonotes}
\usepackage[pagebackref=true]{hyperref}
\hypersetup{colorlinks = true,extension = notused,linkcolor = blue,anchorcolor = red,citecolor = blue,filecolor = red,pagecolor = red,urlcolor = blue}
\usepackage[numbers,sort&compress]{natbib}
\usepackage{hypernat}
\usepackage{iftex}
\ifxetex
        \usepackage{fontspec}
        \setmainfont[Ligatures=TeX,Extension=.otf,BoldFont=cmunbx,ItalicFont=cmunti,BoldItalicFont=cmunbi]{cmunrm}
\else
\fi
 %%% THEOREMS
\theoremstyle{definition}
\newtheorem{problem}{Problem}
\newtheorem{claim}{Claim}
\newtheorem{proposition}{Proposition}
\newtheorem{lemma}{Lemma}
\newtheorem{corollary}{Corollary}
\newtheorem{theorem}{Theorem}
\newtheorem{definition}{Definition}
\newtheorem{conjecture}{Conjecture}
\newtheorem{question}{Question}
\theoremstyle{remark}
\newtheorem{example}{Example}
\newtheorem{remark}{Remark}
\newtheorem{note}{Note}
\newtheorem{exercise}{Exercise}
        
%%% BRACKETS
\newcommand {\la} {\left \langle}
\newcommand {\ra} {\right \rangle}
\newcommand {\lb} {\left (}
\newcommand {\rb} {\right )}
\newcommand {\lfb} {\left \{}
\newcommand {\rfb} {\right \}}
\newcommand {\lsb} {\left [}
\newcommand {\rsb} {\right]}
\newcommand{\br}[1]{\left( #1 \right)}
\newcommand{\vev}[1]{\left\langle #1 \right\rangle}
\newcommand{\ket}[1]{\left |  #1 \right \rangle}
\newcommand{\bra}[1]{\left \langle  #1 \right |}
\newcommand{\vect}[1]{\overrightarrow{ #1 }}

%%% CAL LETTERS
\newcommand {\CalA} {\mathcal A}
\newcommand {\CalD} {\mathcal D}
\newcommand {\CalE} {\mathcal E}
\newcommand {\CalF} {\mathcal F}
\newcommand {\CalG} {\mathcal G}
\newcommand {\CalH} {\mathcal H}
\newcommand {\CalI} {\mathcal I}
\newcommand {\CalJ} {\mathcal J}
\newcommand {\CalO} {\mathcal O}
\newcommand {\CalZ} {\mathcal Z}
\newcommand {\CalN} {\mathcal N}
\newcommand {\CalC} {\mathcal C}
\newcommand {\CalL} {\mathcal L}
\newcommand {\CalK} {\mathcal K}
\newcommand {\CalM} {\mathcal M}
\newcommand {\CalP} {\mathcal P}
\newcommand {\CalR} {\mathcal R}
\newcommand {\CalS} {\mathcal S}
\newcommand {\CalT} {\mathcal T}
\newcommand {\CalU} {\mathcal U}
\newcommand {\CalV} {\mathcal V}
\newcommand {\CalX} {\mathcal X}
\newcommand {\CalW} {\mathcal W}

%%% BOLD LETTERS
\newcommand {\BB}   {\mathbb B}
\newcommand {\BD}   {\mathbb D}
\newcommand {\BS}  {\mathbb S}
\newcommand {\BI}   {\mathbb I}
\newcommand {\BT}   {\mathbb T}
\newcommand {\BR}   {\mathbb R}
\newcommand {\BZ}   {\mathbb Z}
\newcommand {\BC}   {\mathbb C}
\newcommand {\BN}   {\mathbb N}
\newcommand {\BO}   {\mathbb O}
\newcommand {\BP}   {\mathbb P}
\newcommand {\BQ}   {\mathbb Q}
\newcommand {\BW}   {\mathbb W}
\newcommand {\RP}   {\mathbb R \mathbb P}
\newcommand {\CP}   {\mathbb C \mathbb P}
\newcommand {\Unity}{\mathbf 1}

%%% GREEK LETTERS
\newcommand {\al} {\alpha}
\newcommand {\ald}{\dot \alpha}
\newcommand {\be} {\beta}
\newcommand {\bed} {\dot \beta}
\newcommand {\de} {\delta}
\newcommand {\ve}  {\varepsilon}
\newcommand {\ep}  {\epsilon}
\newcommand {\ka} {\kappa}
\newcommand {\lam}  {\lambda}
\newcommand {\si}   {\sigma}
\newcommand {\sib}  {\bar \sigma}
\newcommand {\thb}  {\bar \theta}
\newcommand {\vphi} {\varphi}
\newcommand {\ze} {\zeta}
\newcommand {\ro} {\rho}
\newcommand {\om} {\omega}

%%% FRAK LETTERS
\newcommand{\g}{\mathfrak{g}}
\newcommand{\h}{\mathfrak{h}}
\newcommand{\frakH}{\mathfrak{H}}
\newcommand{\frakd}{\mathfrak{d}}
\newcommand{\frakk}{\mathfrak{k}}
\newcommand{\frakl}{\mathfrak{l}}

%%% DERIVATIVES
\newcommand {\p} {\partial}
\newcommand{\Dslash}{\ensuremath \raisebox{0.025cm}{\slash}\hspace{-0.25cm} D}


%%% OPERATORS
\DeclareMathOperator{\ch}{ch}
\DeclareMathOperator{\td}{td}
\DeclareMathOperator{\eu}{eu}
\DeclareMathOperator{\supp}{supp}
\DeclareMathOperator{\spanlin}{span}
\DeclareMathOperator{\Map}{Map}
\DeclareMathOperator{\Ker}{Ker}
\DeclareMathOperator{\coker}{coker}
\DeclareMathOperator{\coKer}{coKer}
\DeclareMathOperator{\Img}{Img}
\DeclareMathOperator{\End}{End}
\DeclareMathOperator{\Hom}{Hom}
\DeclareMathOperator{\Div}{div}
\DeclareMathOperator{\tr} {tr}
\DeclareMathOperator{\Tr} {Tr}
\DeclareMathOperator{\str} {str}
\DeclareMathOperator{\rk} {rk}
\DeclareMathOperator{\vol}{vol}
\DeclareMathOperator{\HH} {HH}
\DeclareMathOperator{\Pexp} {Pexp}
\DeclareMathOperator{\ad} {ad}
\DeclareMathOperator{\ind} {ind}
\DeclareMathOperator{\Id} {Id}
\DeclareMathOperator{\diag}{diag}
\DeclareMathOperator{\res}{res}
\DeclareMathOperator{\codim}{codim}



\renewcommand{\Re}{\operatorname{Re}}
\renewcommand{\Im}{\operatorname{Im}}


\newcommand{\SU}{SU}
\newcommand{\SO}{SO}
\newcommand{\Spin}{Spin}
\newcommand{\Sp}{Sp}
\newcommand{\OSp}{OSp}
\newcommand{\GL}{GL}
\newcommand{\SL}{SL}
\newcommand{\PSL}{PSL}
\newcommand{\PSU}{PSU}
\newcommand{\Cl}{\mathrm Cl}
\newcommand{\spin}{\mathfrak {spin}}
\newcommand{\sog}{\mathfrak {so}}
\newcommand{\osp}{\mathfrak {osp}}
\newcommand{\spn}{\mathfrak {sp}}
\newcommand{\su}{\mathfrak {su}}
\newcommand{\un}{\mathfrak {u}}
\newcommand{\sln}{\mathfrak {sl}}

%%%OTHER
\newcommand{\const}{\mathrm{const}}


\newcommand{\arxiv}[1]{\href{http://arxiv.org/abs/#1}{http://arxiv.org/abs/#1}}
\newcommand{\libs}[1]{\href{file://localhost/Users/pestun/Dropbox/lib/spires/#1}{\nolinkurl{#1}}}
\newcommand{\libm}[1]{\href{file://localhost/Users/pestun/Dropbox/lib/math/#1}{\nolinkurl{#1}}}
\newcommand{\libt}[1]{\href{file://localhost/Users/pestun/Dropbox/lib/talks/#1}{\nolinkurl{#1}}}
\newcommand{\libb}[1]{\href{file://localhost/Users/pestun/Dropbox/lib/books/#1}{\nolinkurl{#1}}}
\newcommand{\libr}[1]{\href{file://localhost/Users/pestun/Dropbox/lib/research/#1}{\nolinkurl{#1}}}
\newcommand{\libn}[1]{\href{file://localhost/Users/pestun/Dropbox/lib/research/_notes/#1}{\nolinkurl{#1}}}
\newcommand{\libp}[1]{\href{file://localhost/Users/pestun/Dropbox/lib/pestun/#1}{\nolinkurl{#1}}}
\newcommand{\myref}[1]{[notes #1]}


\numberwithin{equation}{section}



 
\begin{document}


Notations. Let $G$ be a simple Lie group of rank $r$. Let $i = 1 \dots r$ label the nodes of Dynkin graph. Let $\lambda_i^{\vee}: \BC^{\times} \to T_{G}$ be fundamental coweights, and $\lambda_i:
T_{G} \to \BC^{\times}$ be fundamental weights. Let $\rho_i: G \to \mathrm{End}(V_i)$
be irreducible representation of $G$, i.e.  $V_i$ is irreducible $G$-module with highest weight $\lambda_i$. Let $\chi_i$ be the character of $\rho_i$, that is $\rho_i(g) = \tr_{V_{i}} \rho_i(g)$. 





Conjecture 1. The moduli space $M_{G, \mathbf{n}, g_{\infty}}$ of smooth $G$-monopoles on $\BC_{x} \times S^{1}$
with fixed charge $\mathbf{n} =(n_1, \dots, n_r) \in \BZ_{>=0}^{r}$ at infinity $x =\infty$ 
in the asymptotic sector $g_\infty \in T_{G}/W$  is the subspace of analytic maps $g: \BC_{x} \to G$ restricted by the conditions

1) The matrix elements of $\rho_i(g(x))$ are polynomials of $x$ of degree $n_i$

2) The coefficient at $x^n$ of $ \chi_{i}(g(x))$ is fixed to be $\chi_i(g_\infty)$.

\medskip

Conjecture 2. The $\dim_{\BC} M_{G, \mathbf{n}, g_{\infty}} = 2 \sum_{i=1}^{r} n_i$

\medskip

Conjecture 3. The coefficients $u_{i,k}$ for $k=1, \dots n_i$ in the character polynomials
\begin{equation}
  \chi_i(g(x)) = \chi_i(g_{\infty})( x^{n_i} + u_{i,1} x^{n_i-1} + \dots u_{i,n_i})
\end{equation}
 are algebraically independent Poisson commuting functions on $M_{G, \mathbf{n}, \g_{\infty}}$ which provide a non-degenerate map $u: M_{G, \mathbf{n}, \g_{\infty}} \to U$
where $\dim_{\BC} U =  \sum_{i=1}^{r} n_i$. 
  
 Hence $M_{G, \mathbf{n}, \g_{\infty}}$ is the phase space of algebraic completely integrable system with the base $U$.

 \medskip


 Conjecture 4. The $M_{G, \mathbf{n}, \g_{\infty}}$ is a holomorphic symplectic leaf of
 holomorphic Poisson-Lie group $G[\BC]$ with quasi-triangular rational $r$-matrix type Poisson bracket.

\medskip
 
 Conjecture 5. The holomorphic symplectic structure on $M$ is isomorphic to the holomorphic symplectic
  structure defined by the hyperKahler definition of $M$ as the moduli space of solutions of the Bogomolny equationson $\BR^2 \times S^1$ of the form
  \begin{equation}
     d_{A} \Phi = \star F_{A}
  \end{equation}
  with a certain fixed asymptotics of the solutions at $\infty$ (fix the topological sector $n$)
  and the asymptotics of the field $\Phi$ at $x \to \infty$ such that $\exp(\Phi) = g_{\infty}$. 



  Conjecture 6. (Example). For $G = SL(2)$ the complex variables $(p_i, \phi_i)|_{i=1,\dots, n}$ with $p_i \in \BC$ and $\phi_i \in \BC/(2\pi\sqrt{-1}\BZ)$ are Darboux coordinates on $M_{G,n,1}$ with the parametrization
  \begin{equation}
    g(x) = \prod_{i=1}^{n}
    \begin{pmatrix}
      p_i - x & e^{\phi_i} \\
      e^{-\phi_i} & 0 
    \end{pmatrix}
  \end{equation}


%\bibliographystyle{utphysurl} \bibliography{lib} 
\end{document} 


