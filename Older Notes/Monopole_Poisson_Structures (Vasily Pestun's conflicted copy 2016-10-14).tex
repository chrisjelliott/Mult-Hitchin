\documentclass[10pt, oneside, a4paper]{article}


\IfFileExists{./math_headers.sty}
  {\input ./math_headers.sty}
  {\IfFileExists{../math_headers.sty}
    {\input ../math_headers.sty}
    {\typeout{Header file not found}
    }
  }
  
\newcommand{\map}{\ul{\mr{Map}}}
\newcommand{\mapf}{\ul{\mr{Map}}^{\mr{fr}}}

\title{Poisson Structures on Moduli Spaces of Monopoles}
\author{Chris Elliott}
\date{\today}

\begin{document}
\maketitle

These are some sketchy notes about how to generalise the Poisson structures on Jarvis's moduli space of monopoles on $\RR^3$ to moduli spaces with Poisson structures that could model things like monopoles on $\RR^2 \times S^1$.  I got a lot of these ideas from e-mail conversations with Penghui Li and Pavel Safronov.

\section{General Story}
\begin{definition}
 Given a Manin triple $(\mf d, \gg_+, \gg_-)$ with associated Poisson Lie groups $(D, G_+, G_-)$, there is an associated 1-shifted symplectic stack $BG_+ \times_{BD} BG_-$.  This is 1-shifted symplectic because the pairing on $\mf d$ makes $BD$ into a 2-shifted symplectic stack, the fact that $\gg_\pm$ are Lagrangian provides a canonical Lagrangian structure to the morphisms $BG_\pm \to BD$, making the Lagrangian intersection 1-shifted symplectic.  Finally, the complementarity condition tells us that the square
 \[\xymatrix{
  \pt \ar[r] \ar[d] &\bb T_{BG_+} \ar[d] \\
  \bb T_{BG_-} \ar[r] &\bb T_{BD}
 }\]
is cocartesian in the category of chain complexes, therefore also cartesian, meaning that the tangent complex to the fiber product $BG_+ \times_{BD} BG_-$ has vanishing tangent complex at the canonical point that factors through $BG_+ \times BG_-$, which means that this map $\pt \to BG_+ \times_{BD} BG_-$ is canonically Lagrangian.
\end{definition}

Let's apply this to some examples of Manin triples associated to familiar quantum groups.  Throughout we'll fix a semisimple Lie algebra $\gg$, with associated complex semisimple Lie group $G$.
\begin{example}
Consider the Manin triple coming from a rational R-matrix.  That is, take $(\gg((t)), \gg[[t]], t^{-1} \gg[t^{-1}])$ where we use the residue pairing
\[( f, g ) = \res_0\left( \frac {\langle f,g \rangle}{t^{2}} dt\right).\]
The associated 1-shifted symplectic stack is $BG[[t]] \times_{BG((t))} B(t^{-1}G[t^{-1}]) \iso \bun_G^{\mr{fr}}(\bb P^1)$, the moduli stack of $G$-bundles on $\bb P^1$ with a framing at $\infty$.  The shifted symplectic structure coincides with the AKSZ symplectic structure on the framed mapping stack $\map(\bb P^1, p, BG)$ defined by Spaide \cite{Spaide}.  Equivalently, this is the moduli stack $\bun_G(E^{\mr{cusp}})$ of $G$-bundles on a cuspidal cubic curve.
\end{example}

\begin{example}\label{trig_Manin_triple_example}
Consider the Manin triple coming from a trigonometric R-matrix.  Fix $\mf p_\pm \sub \gg$ to be complementary parabolic subalgebras, intersecting in a Levi subalgebra $\mf l$.  Take $(\gg(t) \otimes_{\CC(t)} (\CC((t)) \oplus \CC((t^{-1}))), \gg(t)_{\mr{diag}}, \ol{\gg_-})$ using the sum of residues pairing
\[((f_1, f_2), (g_1, g_2)) = \res_0\left( \frac {\langle f_1, g_1\rangle}{t} dt\right) + \res_\infty\left( \frac {\langle f_2, g_2\rangle}{t} dt\right)\]
and where $\ol{\gg_-}$ is the closure of the subalgebra
\[\gg_-= \{(f_+, f_-) \colon f_+ \in \mf p_+ + t\gg[t], f_- \in \mf p_- + t^{-1}\gg[t^{-1}], \pi_{\mf l}(f_+(0) + f_-(\infty)) = 0\}\]
where $\pi_{\mf l}$ is the projection onto $\mf l$.  The associated 1-shifted symplectic stack represents $G$-bundles on $\bb P^1$ with a $P_+$ reduction at $0$, a $P_-$ reduction at $\infty$, and an isomorphism of the fiber of the associated $L$-bundle at $0$ with the inverse of the fiber of the associated $L$-bundle at $\infty$.

For a subexample, let $P_+ = P_- = G$.  Then the associated 1-shifted symplectic stack is exactly the moduli stack $\bun_G(E^{\mr{nod}})$ of $G$-bundles on a nodal cubic curve.
\end{example}

\begin{remark}
If $P_\pm$ are opposite Borels, this is the sort of Lie bialgebra used in the construction of Semenov-Tian-Shansky and Sevostyanov \cite{STSS}.  To more closely match their construction it's natural to slightly generalise the construction above, by replacing $\gg_-$ by a similar subalgebra, but with the condition $\pi_{\mf l}(f_+(0)) + \theta \pi_{\mf l}(f_-(\infty)) = 0$ where $\theta$ is an endomorphism of the Levi subalgebra $\mf l$.
\end{remark}

\begin{remark}
One should be able to do an elliptic example as well, starting from the classical limit of an elliptic quantum group and building the 1-shifted symplectic stack $\bun_G(E)$ (with its canonical shifted symplectic structure, using the Calabi-Yau structure on $E$).  I don't know how to work out the details of this example.
\end{remark}


\section{Rational and Trigonometric Monopoles}
Let's use these 1-shifted symplectic stacks, with the Lagrangian $\pt \to BG_+ \times_{BD} BG_-$, to construct some 0-symplectic and 0-Poisson stacks as coisotropic intersections.  The first example, the rational case, is due to Spaide in the derived setting, generalising the underived Poisson structures constructed by Finkelberg, Kuznetsov, Markarian and Mirkovi\'c \cite{FKMM}.

\begin{example}[{\cite[Section 4.2]{Spaide}}]
Consider the fiber product
\[\xymatrix{
 \mapf(\bb P^1, G/P) \ar[r]\ar[d] &\mapf(\bb P^1, BP) \ar[d] \\
 \pt \ar[r] &\mapf(\bb P^1, BG),}
\]
i.e. the fiber of the canonical map $\mapf(\bb P^1, BP) \to \mapf(\bb P^1, BG)$ over the trivial bundle.  Here $\mapf$ denotes the space of maps with a fixed framing at $\infty$.  The map $\pt \to \mapf(\bb P^1, BG)$ is Lagrangian, as discussed above (though it's easy to check directly -- the tangent complex at this point is equivalent to $C^\bullet(\bb P^1; \gg \otimes \OO(-1))[1]$, which has trivial cohomology), and the map $BP \to BG$ is coisotropic, therefore so is the map $\mapf(\bb P^1, BP) \to \mapf(\bb P^1, BG)$ (by \cite[Theorem 3.5]{Spaide}).  Therefore the fiber product $\mapf(\bb P^1, G/P)$ is 0-shifted Poisson.  If $P = B$ is a Borel, then the map $\mapf(\bb P^1, BB) \to \mapf(\bb P^1, BG)$ is actually Lagrangian, so $\mapf(\bb P^1, G/B)$ is 0-shifted symplectic.
\end{example}

\begin{example} \label{nodal_curve_example}
Let's tell the same story, but for the trigonometric Manin triple with $P_+ = P_- = G$.  We can make a completely analogous construction to the one we gave above, by replacing $E^{\mr{cusp}}$ with $E^{\mr{nod}}$.  We get the fiber product
\[\xymatrix{
 \map(E^{\mr{nod}}, G/P) \ar[r]\ar[d] &\map(E^{\mr{nod}}, BP) \ar[d] \\
 \pt \ar[r] &\map(E^{\mr{nod}}, BG).}
\]
Again, and for exactly the same reasons the map $\pt \to \map(E^{\mr{nod}}, BG)$ is Lagrangian, the map $\map(E^{\mr{nod}}, BP) \to \map(E^{\mr{nod}}, BG)$ is coisotropic, and Lagrangian if $P= B$, and therefore the fiber product is 0-shifted Poisson, or 0-shifted symplectic if $P= B$.  This is one possible trigonometric generalization of the moduli space of monopoles, and seems to be very close to the trigonometric Zastava space constructed by Finkelberg, Kuznetsov and Rybnikov \cite{FKR} (it might be exactly the same, I'm not sure right now).
\end{example}

\begin{example} \label{Borel_example}
If we really want to compare our Poisson structures to the Poisson structures on difference opers, we should consider more general pairs of opposite parabolics, so let's try to do that.  For simplicity, let $P_+= B_+$ be a Borel with opposite Borel $B_-$ and maximal torus $T$.  Write $\map^{B_+, B_-}(\bb P^1, BG)$ for the 1-shifted symplectic moduli stack constructed in example \ref{trig_Manin_triple_example}, write $\map^{T}(\bb P^1, BB_+)$ for the space of $B_+$-bundles on $\bb P^1$ with a $T$-reduction at $\infty$ and an identification of the fibers of the $T$-bundles at $0$ and $\infty$, and form the fiber product
\[\xymatrix{
 F \ar[r]\ar[d] & \map^{T}(\bb P^1, BB_+) \ar[d]\\
 \pt\ar[r] &\map^{B_+,B_-}(\bb P^1, BG).
}\]
We already know for abstract reasons that $\pt\to\map^{B_+,B_-}(\bb P^1, BG)$ is a Lagrangian point in a 1-shifted symplectic stack, and we'd like to check that the vertical map $\map^{T}(\bb P^1, BB_+) \to \map^{B_+,B_-}(\bb P^1, BG)$ is also Lagrangian.  I'll keep thinking about this.  To concretely describe the pullback, it represents $B$-structures on the trivial $G$ bundle on $\bb P^1$, with an additional $T$-reduction at $\infty$ and an identification of the fibers at $0$ and $\infty$ after restricting to $T$.
\end{example}

\section{Integrable System Structure}
In this section I'll follow the notation in \cite{FKR}.  Let's try to understand why these moduli stacks have the structure of a completely integrable system.  To do this, we should fix a topological sector. That is, each of our moduli spaces (when we use the parabolic $P = B$, a Borel) admits a forgetful map to the mapping stack $\map(\bb P^1, G/B)$, which a priori has components indexed by $\pi_2(G/B) \iso \ZZ^r$, but in fact only those components corresponding to $\ZZ^r_{\ge 0}$ (the cone of positive coroots) are non-empty.  Let's restrict to the preimage of a fixed $\alpha \in \ZZ^r_{\ge 0}$.  Write $\bb A^\alpha$ for the vector space $\CC^{\alpha_1 + \cdots + \alpha_r}$, thought of as the space of $r$-coloured divisors in $\bb A^1$.

\begin{remark}
In the rational case we can check that the components $\map^{\mr{fr},\alpha}(\bb P^1, G/B)$ have the right dimensions.  Let's do it for $G = \SL_2$.  In this case, $\bb T_{G/B}$ is the line bundle $\OO(2)$ on $\bb P^1$, so if $\phi$ has degree $\alpha$ then $\phi^*\bb T_{G/B} = \OO(2\alpha)$.  The tangent complex to the mapping space is quasi-isomorphic to 
\begin{align*}
\bb T_{\phi, \map^{\mr{fr},\alpha}(\bb P^1, G/B)} &\iso H^\bullet(\bb P^1; \phi^*\bb T_{G/B} \otimes \OO(-1)) \\
&\iso H^\bullet(\bb P^1; \OO(2\alpha-1)).
\end{align*}
so the cohomology vanishes outside of degree 0, and $H^0$ has dimension $2\alpha$.

One would like to do this calculation more generally.  It would suffice to check that the tangent bundle of $G/B$ (which is the associated bundle $G \times_B \gg/\mf b$) can be written as an iterated extension of the rank $r$ vector bundle $L_{\lambda_1} \oplus \cdots \oplus L_{\lambda_r}$ by trivial bundles, where $\lambda_i$ are fundamental weights.  Then the pullback of $L_{\lambda_i}$ under a map $\bb P^1 \to G/B$ of degree $\alpha$ is $\OO(2\alpha_i)$, and we end up with the tangent complex having dimension $\sum_i 2\alpha_i$.

We can do this in the case of example \ref{nodal_curve_example} too, and verify that formally locally the moduli spaces coincide.  Again, let's do this for the straightforward example of $\SL_2$.  Viewing the moduli space as the moduli of maps $\phi \colon \bb P^1 \to G/B$ of degree $\alpha$ with an identification of fibers $\phi(0) = \phi(\infty)^{-1}$, the tangent complex at $\phi$ is identified as the subcomplex of $C^\bullet(\bb P^1; \phi^*\bb T_{G/B}) \iso C^\bullet(\bb P^1; \OO(2\alpha))$ consisting (in degree 0) of sections $f$ such that $f(0)/\phi(0) + f(\infty)/\phi(\infty) = 0$.  This condition
\end{remark}


There's a canonical ($r$-coloured) divisor on $G/B$ obtained as the sum of the ($r$) codimension 1 Schubert varieties.  Denote this divisor by $D$. 

\begin{example}
For the usual (rational) monopole moduli space, there's a canonical map $\map^{\mr{fr},\alpha}(\bb P^1, G/B) \to \bb A^{\alpha}$ defined by sending a map $\phi$ to the pullback $\phi^*(D)$, which lands in $\bb A^1 \sub \bb P^1$ because the map was framed, so send $\infty$ to the base point.
\end{example}

\begin{example}
For the trigonometric monopole moduli space constructed in example \ref{nodal_curve_example}, there's a canonical map $\map^\alpha(E^{\mr{nod}}, G/B) \to (E^{\mr{nod}})^\alpha$, defined as above, by sending a map $\phi$ to the pullback $\phi^*(D)$.  One can promote this to a map to $\bb A^\alpha$ by choosing a map $E^{\mr{nod}}\to \bb A^1$ (there are two inverse ways of doing this).
\end{example}

\begin{example}
For the more general example that we constructed in example \ref{Borel_example}, there's again a canonical map $F \to (E^{\mr{nod}})^\alpha$ as above, and again we might promote it to a map to $\bb A^\alpha$.
\end{example}


\bibliographystyle{alpha}
\bibliography{Oper}

\textsc{Institut des Hautes \'Etudes Scientifiques,}\\
\textsc{35 Rue des Chartres, Bures-sur-Yvette 91440, France} \\
\texttt{celliott@ihes.fr}\\
\end{document}