\documentclass[10pt, oneside, a4paper]{article}


\IfFileExists{./math_headers.sty}
  {\input ./math_headers.sty}
  {\IfFileExists{../math_headers.sty}
    {\input ../math_headers.sty}
    {\typeout{Header file not found}
    }
  }
  
\newcommand{\Map}{\ul{\mr{Map}}}

\title{Symplectic Structures on Moduli of Regular Higgs Bundles on a Punctured Line}
\author{Chris Elliott}
\date{\today}

\begin{document}
\maketitle

The following are a set of rough notes describing moduli stacks of Higgs bundles on non-compact curves with regular singularities and prescribed residues, and in particular giving two equivalent descriptions of the 0-shifted symplectic structures on this moduli space in genus zero.  These notes are still quite preliminary, so be warned that the proofs will be sketchy.

\section{Regular Higgs Bundles on a Punctured Line}
Let $G$ be a semisimple complex algebraic group.  We'll give two equivalent descriptions of the moduli space of $G$-Higgs bundles on the punctured curve $\bb P^1 \bs \{z_1, \ldots, z_n\}$, with regular singularities and prescribed residues $\delta_1, \ldots, \delta_n$ at the punctures, up to gauge equivalence.  Each perspective will come with a natural 0-shifted symplectic structure (as defined by Pantev-To\"en-Vaqui\'e-Vezzosi \cite{PTVV}), and we'll provide an equivalence of derived symplectic stacks.

We'll write $\bb D$ and $\bb D^\times$ for the formal disk $\spec \CC[[z]]$ and the formal punctured disk $\spec \CC((z))$ respectively.  We denote by $G((z))$ the mapping stack $\ul{\mr{Map}}(\bb D^\times, G)$ (the \emph{formal loop group}), and similarly we denote by $\gg((z))$ the mapping stack $\ul{\mr{Map}}(\bb D^\times, \gg)$.  Throughout this note we'll freely use the invariant pairing coming from the Killing form to identify $\gg$ and $\gg^*$.

\begin{definition}
The moduli stack of \emph{algebraic $G$-bundles} on a scheme $X$ is the mapping stack $\Map(X, BG)$.  The moduli stack of \emph{$G$-Higgs bundles} on a scheme $X$ is the mapping stack $\Map(X_{\mr{Dol}}, BG)$, where $X_{\mr{Dol}}$ is the \emph{Dolbeault stack}: the formal completion $T_{\mr{form}}[1]X$ of the 1-shifted cotangent bundle at the zero section.
\end{definition}

\begin{lemma} \label{algebraic_bundles_vanish_lemma}
The moduli stack $\bun_G(\bb A^1 \bs \{z_1, \ldots, z_n\})$ of algebraic $G$-bundles on a punctured affine line is equivalent to the classifying stack $BG(\CC[z]_{z_1, \ldots, z_n})$.  That is, all algebraic $G$-bundles are trivializable.
\end{lemma}

\begin{proof}
The tangent complex to $\bun_G(X)$ at a classical point $P$ is abstractly isomorphic to the shifted Dolbeault complex $(\Omega^{0, \bullet}_{\mr{alg}}(X; \gg_P)[1], \ol \dd)$.  If $X$ is a punctured line, this complex is concentrated in degree 0, so the derived stack is equivalent to its classical truncation, the classical moduli stack of $G$-bundles.

By a foundational result in algebraic geometry, a normal Noetherian affine scheme $\spec R$ has vanishing class group if and only if it is a UFD.  The ring $\CC[z]$ is a UFD, and therefore so are all its localizations, so every line bundle on the punctured affine line is trivializable.  A theorem of Serre (\cite[Theorem 1]{Serreprojective}) tells us that every vector bundle on an affine curve splits as the sum of a line bundle and a trivial bundle, therefore every vector bundle on a punctured line is trivializable.  Finally, the category of principal $G$-bundles on a smooth variety $X$ admits a Tannakian description as exact faithful monoidal functors $\rep(G) \to \vect(X)$ (satisfying a fibre functor condition), where $\vect(X)$ is the category of vector bundles on $X$.  Therefore if every vector bundle is trivializable, so, necessarily is every $G$-bundle, as required.
\end{proof}

\begin{corollary}
The moduli stack $\bun_G(\bb D^\times)$ of $G$-bundles on the formal punctured disk is equivalent to $BG((z))$.
\end{corollary}

\begin{prop} \label{higgs_bundles_on_disk_prop}
There are natural equivalences of derived stacks 
\begin{align*}
\higgs_G(\bb A^1 \bs \{z_1, \ldots, z_n\}) &\iso \gg(\CC[z]_{z_1, \ldots, z_n})/G(\CC[z]_{z_1, \ldots, z_n}) \\
\text{and } \higgs_G(\bb D^\times) &\iso \gg((z)) / G((z)),
\end{align*} 
where $G(\CC[z]_{z_1, \ldots, z_n})$ and $G((z))$ act by the respective adjoint actions.
\end{prop}

\begin{proof}
The tangent complex to $\higgs_G(X)$ at a classical point $P$ is abstractly isomorphic to the shifted complex of $(p,q)$-forms $(\Omega^{\bullet, \bullet}_{\mr{alg}}(X; \gg_P)[1], \ol \dd)$.  If $X$ is a punctured line, this complex is concentrated in degrees 0 and 1, so the derived stack is equivalent to its classical truncation, the classical moduli stack of $G$-Higgs bundles. 

As such the fiber over $\pt \to BG(\CC[z]_{z_1, \ldots, z_n})$ in the moduli stack is just the space of Higgs fields on the trivial bundle.  This is clearly isomorphic to $\gg(\CC[z]_{z_1, \ldots, z_n})$, so the moduli stack in question is just the adjoint quotient $\gg(\CC[z]_{z_1, \ldots, z_n})/G(\CC[z]_{z_1, \ldots, z_n})$ as required.  The same argument holds for Higgs bundles on the formal punctured disk.
\end{proof}

The moduli stack $\higgs_G(\bb D^\times)$ fails to be shifted symplectic.  Indeed, we'd like to think of $\bb D^\times$ as 1-oriented with orientation given by the residue, and then obtain a 1-shifted symplectic structure by the AKSZ-PTVV formalism \cite[Theorem 2.5]{PTVV}.  On the tangent complex $\bb T_0 \higgs_G(\bb D^\times) \iso \gg((z))[1] \oplus \gg((z))$, the invariant pairing on $\gg$ combined with the residue gives a candidate pairing of degree 1, but it fails to be well-defined because the degree $-1$ part of the product may be a sum with infinitely many terms.  Let's describe a substack where this problem is resolved.

\begin{definition}
 The moduli space of \emph{regular} Higgs bundles on $\bb D^\times$ is the 1-shifted symplectic substack $t^{-1}\gg/G \inj \gg((z))/G((z))$.
\end{definition}

Here the 1-shifted symplectic structure is the canonical one, using the invariant pairing to identify $\gg/G$ with $\gg^*/G$.  Under the embedding $\gg/G \inj \gg((z))/G((z))$ placing $\gg$ in $z$ degree $-1$, this degree 1 pairing -- on the tangent complex -- is exactly given by the invariant pairing on $\gg$ combined with the residue pairing.

This motivates a definition of regular Higgs bundles on any punctured curve.
\begin{definition} \label{regular_higgs_def}
If $X = \ol X \bs \{z_1, \ldots, z_n\}$ is a connected punctured curve, the moduli stack $\higgs_G^{\mr{reg}}(X)$ of \emph{regular} Higgs bundles on $X$ is the pullback
\[\xymatrix{
 \higgs_G^{\mr{reg}}(X) \ar[r] \ar[d] &\higgs_G(X) \ar[d] \\
 \left(t^{-1}\gg\right)^n/G \ar[r] &\left(\gg((z))/G((z))\right)^n
}\]
where the vertical arrow $\higgs_G(X) \to \left(\gg((z))/G((z))\right)^n$ is given by restriction to a formal punctured neighbourhood of $\{z_1, \ldots, z_n\}$.
\end{definition}

\begin{prop} \label{higgs_bundles_on_punctured_line_prop}
The moduli space of regular Higgs bundles on a punctured line $\bb P^1 \bs \{z_1, \ldots, z_n\}$ can be described as
\[\higgs_G^{\mr{reg}}(\bb P^1 \bs \{z_1, \ldots, z_n\}) \iso \gg^n/G \times_{\gg/G} BG\]
where the map $\gg^n/G \to \gg/G$ comes from the sum $\gg^n \to \gg$.
\end{prop}

\begin{proof}
Without loss of generality we can assume that none of the $z_i = \infty$, and therefore write
\begin{align*}
\higgs_G^{\mr{reg}}(\bb P^1 \bs \{z_1, \ldots, z_n\}) &\iso \higgs_G^{\mr{reg}}(\bb A^1 \bs \{z_1, \ldots, z_n\}) \times_{\higgs_G^{\mr{reg}}(\bb D^\times)} \higgs_G^{\mr{reg}}(\bb D) \\
&\iso \higgs_G^{\mr{reg}}(\bb A^1 \bs \{z_1, \ldots, z_n\}) \times_{\gg/G} BG. 
\end{align*}
According to proposition \ref{higgs_bundles_on_disk_prop}, we can identify 
\[\higgs_G(\bb A^1 \bs \{z_1, \ldots, z_n\}) \iso \gg(\CC[z]_{z_1, \ldots, z_n})/G(\CC[z]_{z_1, \ldots, z_n}),\]
and when we form the pullback as in definition \ref{regular_higgs_def} we obtain 
\[\higgs_G^{\mr{reg}}(\bb A^1 \bs \{z_1, \ldots, z_n\}) \iso \left((t-z_1)^{-1}\gg \times \cdots \times (t-z_n)^{-1}\gg\right)/G\]
and therefore the desired equivalence.
\end{proof}

We can go one step further, and define moduli spaces of regular Higgs bundles on a punctured curve with prescribed residues at the punctures.
\begin{definition}
If $X = \ol X \bs \{z_1, \ldots, z_n\}$ is a punctured curve, the moduli stack $\higgs_G^{\mr{reg}}(X; \delta_1, \ldots, \delta_n)$ of \emph{regular} Higgs bundles on $X$ with residues conjugate to $\delta_1, \ldots, \delta_n \in \gg$ at the punctures, is the pullback
\[\xymatrix{
\higgs_G^{\mr{reg}}(X; \delta_1, \ldots, \delta_n) \ar[r] \ar[d] &\higgs_G^{\mr{reg}}(X) \ar[d] \\
[\delta_1]/G \times \cdots \times [\delta_n]/G \ar[r] &\left(t^{-1}\gg/G\right)^n
}\]
where $[\delta_i]/G$ is the image of $\delta_i$ under the projection $\gg \to \gg/G$.
\end{definition}

\begin{lemma} \label{higgs_lagrangian_structure_lemma}
The map $\higgs_G^{\mr{reg}}(\bb P^1 \bs \{z_1, \ldots, z_n\}) \to \left(\gg/G\right)^n$ admits a canonical Lagrangian structure.
\end{lemma}

\begin{proof} 
We'll use the description in proposition \ref{higgs_bundles_on_punctured_line_prop}.  The pullback of the symplectic form on $\left(\gg/G\right)^n$ to $\gg^n/G \times_{\gg/G} BG$ vanishes (is not merely equivalent to zero), which one can check directly by evaluating it as a pairing on the tangent complex: indeed, when we pull it back to $\gg^n/G$, we get
\[\omega((X, (Y_1, \ldots, Y^n)), (X', (Y_1', \ldots, Y_n'))) = \sum_{i=1}^n \left( \langle X, Y_i' \rangle + \langle X', Y_i \rangle \right)\]
as a pairing on the tangent complex $\gg[1] \oplus \gg^n$  When we pullback to the fiber product, this imposes the condition that $\sum Y_i = \sum Y_i' = 0$, so the pairing vanishes identically.  A Lagrangian structure is therefore an element of $\Omega^{2, \mr{cl}}(\gg^n/G \times_{\gg/G} BG)$ of degree 2 which is closed for the internal differential, and there's a canonical choice, namely the pullback of the invariant pairing on $\gg$ as an element of $\Omega^{2, \mr{cl}}(BG)$. 
\end{proof}

\begin{corollary} \label{residue_moduli_symplectic_cor}
The moduli stack $\higgs_G^{\mr{reg}}(\bb P^1 \bs \{z_1, \ldots, z_n\}; \delta_1 \ldots, \delta_n)$ is 0-shifted symplectic.
\end{corollary}

\begin{proof}
Since $[\delta_i]/G \to \gg/G$ is Lagrangian, so is the product $[\delta_1]/G \times \cdots \times [\delta_n]/G \to \left(\gg/G\right)^n$.  Therefore, lemma \ref{higgs_lagrangian_structure_lemma} tells us that the moduli stack $\higgs_G^{\mr{reg}}(\bb P^1 \bs \{z_1, \ldots, z_n\}; \delta_1 \ldots, \delta_n)$ arises as a Lagrangian intersection in $\gg/G$, and therefore by \cite[Theorem 2.9]{PTVV} inherits a canonical 0-shifted symplectic structure.
\end{proof}

\section{An Alternative Perspective}
Let's consider the following alternative way of understanding $G$-Higgs bundles on the punctured line, via Hamiltonian reduction.  Consider the adjoint orbit $[(\delta_1, \ldots, \delta_n)]$ of an element in $\gg^n$ with respect to the diagonal adjoint action of $G$. There is a summation map $\sigma [(\delta_1, \ldots, \delta_n)] \to \gg$.  This map is adjoint-equivariant and Hamiltonian, so one can form the Hamiltonian reduction $\sigma^{-1}(0)/G$, as usual using the invariant pairing to identify $\gg$ and $\gg^*$.

Concretely, we should view this as asking for a set of $n$ residues for a $G$-Higgs field on an $n$-punctured $\bb P^1$, such that the residues can be simultaneously conjugated to $(\delta_1, \ldots, \delta_n)$, and with the constraint that the residues sum to zero, all modulo the adjoint action.  This gives an alternative description of the moduli stack $\higgs_G^{\mr{reg}}(\bb P^1 \bs \{z_1, \ldots, z_n\}; \delta_1 \ldots, \delta_n)$.  We'll explain how to give is a 0-shifted symplectic structure in the derived setting, and then prove that these two symplectic stacks are equivalent.

Safronov \cite{Safronovquasi} explained how to make sense ot Hamiltonian reduction in the setting of derived symplectic geometry.
\begin{definition}[Safronov]
A \emph{derived Hamiltonian structure} on a $G$-equivariant morphism $\mu \colon X \to \gg^*$ of derived stacks is a Lagrangian structure on the associated morphism $X/G \to \gg^*/G$.  The \emph{derived Hamiltonian reduction} of $X$ along the morphism $\mu$ is the derived fiber product $X/G \times_{\gg^*/G} BG$.  Since $BG \to \gg^*/G$ is canonically Lagrangian, a Hamiltonian structure on $\mu$ makes the Hamiltonian reduction into a 0-shifted symplectic stack.
\end{definition}

\begin{example} \label{sum_coadjoint_example}
If $\delta_1, \ldots, \delta_n$ are elements of $\gg$, the map $\mu \colon [(\delta_1, \ldots, \delta_n)] \to \gg^n \to \gg$ obtained by composing the inclusion with the summation map is Hamiltonian.  Indeed, the quotient stack $ [(\delta_1, \ldots, \delta_n)]/G$ with respect to the diagonal adjoint action is canonically equivalent to $BG_{(\delta_1, \ldots, \delta_n)}$, the classifying space of the stabilizer.  The pullback of the symplectic form on $\gg/G$ to this stabilizer is equal to (not just equivalent to) zero.  We choose as isotropic data the invariant pairing on $\gg_{(\delta_1, \ldots, \delta_n)}$ viewed as a closed degree 2 element of $\Omega^{2, \mr{cl}}(BG_{(\delta_1, \ldots, \delta_n)}, 0)$.  This element is non-degenerate, so defines a Lagrangian structure.  Therefore the derived Hamiltonian intersection
\[[(\delta_1, \ldots, \delta_n)]/G \times_{\gg/G} BG\]
is 0-shifted symplectic.
\end{example}

\begin{theorem} \label{equivalent_hamiltonian_reduction_thm}
The derived stacks $\higgs_G^{\mr{reg}}(\bb P^1 \bs \{z_1, \ldots, z_n\}; \delta_1 \ldots, \delta_n)$ and $[(\delta_1, \ldots, \delta_n)]/G \times_{\gg/G} BG$ are equivalent as 0-shifted symplectic stacks.
\end{theorem}

\begin{proof}
We'll first observe that the two stacks are equivalent, then match up the two symplectic structures.  The equivalence as stacks is easy, indeed, we can expand the Higgs moduli stack as
\begin{align*}
\higgs_G^{\mr{reg}}(\bb P^1 \bs \{z_1, \ldots, z_n\}; \delta_1, \ldots, \delta_n) &\iso \left(\delta_1/G \times \cdots \times \delta_n/G\right) \times_{\left(\gg/G\right)^n} \higgs_G^{reg}(\bb P^1 \bs \{z_1, \ldots, z_n\}) \\
&\iso \left([\delta_1]/G \times \cdots \times [\delta_n]/G\right) \times_{\left(\gg/G\right)^n} \left(\gg^n/G \times_{\gg/G} BG\right) \\
&\iso  \left(\left([\delta_1]/G \times \cdots \times [\delta_n]/G\right) \times_{\left(\gg/G\right)^n} \gg^n/G\right) \times_{\gg/G} BG \\
&\iso [(\delta_1, \ldots, \delta_n)]/G \times_{\gg/G} BG
\end{align*}
as required, where on the last line we used the observation that the base change $- \times_{\left(\gg/G\right)^n} \gg^n/G$ sends a point in $(\gg/G)^n = \gg^n/G^n$ to the corresponding point in $\gg^n/G$ by forgetting the $G^n$ action down to the diagonal.  Now we have to match up the symplectic structures.  Our triple Lagrangian intersection can be made into a 0-shifted symplectic stack in two ways, coming from the two ways of bracketing the expression (i.e. the second and third lines above).

In order to understand this, let's consider a more general triple intersection of the form
\[L_1 \times_{X_1} Y \times_{X_2} L_2,\]
with symplectic structures $\omega_1$ and $\omega_2$ on $X_1$ and $X_2$, and Lagrangian structures on the morphisms $L_1 \to X_1$, $L_2 \to X_2$ as well as $L_1 \times_{X_1} Y \to X_2$ and $Y \times_{X_2} L_2 \to X_1$.  Suppose in addition that, if $g_i$ is the morphism $Y \to X_i$, there is an equivalence between $g_1^*\omega_1$ and $g_2^*\omega_2$.  Under this equivalence, the Lagrangian structure on $L_1 \times_{X_1} Y \to X_2$ becomes a Lagrangian structure on the projection $L_1 \times_{X_1} Y \to X_1$, and similarly the Lagrangian structure on $Y \times_{X_2} L_2 \to X_1$ becomes a Lagrangian structure on the projection $Y \times_{X_2} L_2 \to X_2$.  In order to prove that the two symplectic structures on the triple intersection are equivalent, we must find homotopies between the Lagrangian structure on $L_1 \times_{X_1} Y \to X_1$ we just described and the pullback of the Lagrangian structure on $L_1 \to X_1$ to the fiber product, and similarly on the other side.

In our example, the closed 2-forms on $\gg^n/G$ obtained by pulling back the symplectic structures on $\left(\gg/G\right)^n$ and $\gg/G$ are equal: on the tangent complex $\gg[1] \oplus \gg^n$ both are given by the formula 
\[\omega((X, (Y_1, \ldots, Y^n)), (X', (Y_1', \ldots, Y_n'))) = \sum_{i=1}^n \left( \langle X, Y_i' \rangle + \langle X', Y_i \rangle \right).\]
According to the discussion in the previous paragraph, we must therefore construct two homotopies between two pairs of Lagrangian structures.  First, there are two Lagrangian structures for the left-hand side, $[(\delta_1, \ldots, \delta_n)]/G \to \left(\gg/G\right)^n$.  As we discussed in example \ref{sum_coadjoint_example} above, one of these Lagrangian structures is the invariant pairing for the centralizer $\gg_{(\delta_1, \ldots, \delta_n)}$.  The other is the pullback of the invariant pairing for the product of centralizers $\gg_{\delta_1} \oplus \cdots \oplus \gg_{\delta_n}$ to the diagonal inside $(\gg^n)^{\otimes 2}$, and these are equal.

On the other side, we have a pair of Lagrangian structures on $\gg^n/G \times_{\gg/G} BG \to \gg/G$: the pullback of the invariant pairing on $\gg$ to a closed 2-form on the fiber product, and the Lagrangian structure on $\gg^n/G \times_{\gg/G} BG \to \left(\gg/G\right)^n$.  However, as we saw in lemma \ref{higgs_lagrangian_structure_lemma}, this is again equal to this pullback.  Therefore the Lagrangian structures coincide, and thus so do our two symplectic structures.
\end{proof}

\section{Quasi-Hamiltonian Reduction}
One advantage of this derived geometry approach is that the constructions can be immediately generalized to a \emph{group} version.  Safronov explains how, in his language, one can immediately define \emph{quasi-Hamiltonian reduction} analogously to Hamiltonian reduction.  One simply replaces the quotient $\gg^*/G$ by the \emph{group} adjoint quotient $\frac GG$.  This adjoint quotient also admits a 1-shifted symplectic structure: I like to understanding it by identifying $\frac GG$ with the derived loop space $\mc L G = \Map(S^1, BG)$, which has an AKSZ 1-shifted symplectic structure using the 2-shifted symplectic structure on $BG$ and the 1-orientation on the circle.
\begin{definition}[Safronov]
A \emph{derived quasi-Hamiltonian structure} on a $G$-equivariant morphism $\mu \colon X \to G$ (where $G$ acts on itself by the adjoint action) of derived stacks is a Lagrangian structure on the associated morphism $X/G \to \frac GG$.  The \emph{derived quasi-Hamiltonian reduction} of $X$ along the morphism $\mu$ is the derived fiber product $X/G \times_{\frac GG} BG$.  Since $BG \to \frac GG$ is canonically Lagrangian, a Hamiltonian structure on $\mu$ makes the Hamiltonian reduction into a 0-shifted symplectic stack. 
\end{definition}

From this point of view, we can think of \emph{group}-valued Higgs bundles -- where the Higgs field is group valued rather than Lie algebra valued.  I don't have a good description of these derived moduli spaces in general as mapping spaces, but by analogy with the previous section it's clear how we should define this for a punctured line.

\begin{definition}
The moduli space of regular \emph{group-valued} $G$-Higgs bundles on $\bb P^1 \bs \{z_1, \ldots, z_n\}$ is the fiber product
\[\higgs_G^{\text{grp,reg}}(\bb P^1 \bs \{z_1, \ldots, z_n\}) = \frac {G^n}G \times_{\frac GG} BG.\]
The moduli space of regular group valued $G$-Higgs bundles with prescribed residues $\Delta_1, \ldots, \Delta_n \in G$, up to conjugation, is the fiber product
\[\higgs_G^{\text{grp,reg}}(\bb P^1 \bs \{z_1, \ldots, z_n\}; \Delta_1, \ldots, \Delta_n) = \higgs_G^{\text{grp,reg}}(\bb P^1 \bs \{z_1, \ldots, z_n\}) \times_{\left(\frac GG \right)^n} \left( \frac {[\Delta_1]}G \times \cdots \times \frac {[\Delta_n]}G \right).\]
\end{definition}

\begin{lemma}
The moduli stack $\higgs_G^{\text{grp,reg}}(\bb P^1 \bs \{z_1, \ldots, z_n\}; \Delta_1, \ldots, \Delta_n)$ has a canonical 0-shifted symplectic structure.
\end{lemma}

\begin{proof}
Identical to the proof of lemma \ref{higgs_lagrangian_structure_lemma} and corollary \ref{residue_moduli_symplectic_cor}.
\end{proof}

As before, we can view this moduli stack equivalently as a derived quasi-Hamiltonian reduction, where now we use the \emph{multiplication map} $\mu \colon [(\Delta_1, \ldots, \Delta_n)] \to G$ instead of the summation map.  Otherwise, the argument is identical.

\begin{theorem}
There is an equivalence of 0-shifted symplectic stacks between $\higgs_G^{\text{grp,reg}}(\bb P^1 \bs \{z_1, \ldots, z_n\}; \Delta_1, \ldots, \Delta_n)$ and the derived quasi-Hamiltonian reduction $\frac{[(\Delta_1, \ldots, \Delta_n)]}G \times_{\frac GG} BG$.
\end{theorem}

\begin{proof}
Identical to the proof of theorem \ref{equivalent_hamiltonian_reduction_thm}.
\end{proof}

 
\bibliographystyle{alpha}
\bibliography{Oper}

\textsc{Institut des Hautes \'Etudes Scientifiques,}\\
\textsc{35 Rue des Chartres, Bures-sur-Yvette 91440, France} \\
\texttt{celliott@ihes.fr}\\
\end{document}
