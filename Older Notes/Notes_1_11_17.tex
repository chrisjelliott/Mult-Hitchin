\documentclass[10pt, oneside]{article}

\IfFileExists{./math_headers.sty}
  {\input ./math_headers.sty}
  {\IfFileExists{../math_headers.sty}
    {\input ../math_headers.sty}
    {\typeout{Header file not found}
    }
  }

\title{Notes on Higgs Moduli}
\author{Chris Elliott}

\DeclareMathOperator{\EOM}{EOM}
\newcommand{\PT}{\mathbb{PT}}
\newcommand{\vac}{\mc{V} \text{ac}}
\newcommand{\del}{\partial}
\def\d{{\rm d}}
\newcommand{\Obs}{\mathrm{Obs}}
\newcommand{\IC}{\mathrm{IndCoh}}
\renewcommand{\flat}{\mathrm{Flat}}
\newcommand{\arth}{\mathrm{Arth}}
\newcommand{\sing}{\mathrm{Sing}}
\newcommand{\HC}{\mathrm{HC}}
\newcommand{\map}{\underline{\mathrm{Map}}}

\begin{document}
\setcounter{section}{1}

\maketitle

Let me try to refine and correct the description I gave before of multiplicative Higgs bundles, which wasn't quite correct.  

\begin{definition}
The derived moduli stack $\mr{GpHiggs}_G^\mr{fr}(\bb {CP}^1; D)$ of \emph{multiplicative Higgs bundles} on $\bb {CP}^1$ with a framing at $\infty$ and singularities at the effective divisor $D = \{z_1, \ldots, z_k\}$ (where the $z_i$ are distinct points in $\bb A^1$) is the derived fiber product
\[\mr{GpHiggs}_G^\mr{fr}(\bb {CP}^1; D) = \map^{\mr{fr}}(\bb{CP}^1, BG) \times_{\map^{\mr{fr}}(\bb{CP}^1 \bs D, BG)} \map^{\mr{fr}}(\bb{CP}^1 \bs D, G/G),\]
modelling a framed $G$-bundle on $\bb{CP}^1$ with a section of the multiplicative adjoint bundle with singularities permitted at the divisor $D$.
\end{definition}

\begin{definition}
The derived moduli stack of multiplicative Higgs bundles with prescribed singularities $(\omega^\vee_{z_1}, \ldots, \omega^\vee_{z_k})$ at $D$ is the derived fiber product
\[\mr{GpHiggs}_G^\mr{fr}(\bb {CP}^1; D, \omega^\vee) = \mr{GpHiggs}_G^\mr{fr}(\bb {CP}^1; D) \times_{(LG/LG)^k} (BLL_1 \times \cdots \times BLL_k)\]
where $\mr{GpHiggs}_G^\mr{fr}(\bb {CP}^1; D)$ maps down to $(LG/LG)^k$ by restricting to a punctured neighbourhood of each singularity, and we take the fiber product with a point $\omega^\vee = (\omega^\vee_{z_1}, \ldots, \omega^\vee_{z_k})$ in $LG^k$ whose stabilizer is $(BLL_1 \times \cdots \times BLL_k)$.  If we assume the loops $\omega^\vee_{z_i}$ have simple poles with prescribed residues which we'll also denote by $\omega^\vee$ (for instance, as the notations suggests, elements given by coweights) then this can equivalently be described as the derived fiber product
\[\mr{GpHiggs}_G^\text{fr,simp}(\bb {CP}^1; D) \times_{(G/G)^k} (BL_1 \times \cdots \times BL_k)\]
where now we're restricting to the substack of multiplicative Higgs bundles with simple poles at $D$, mapping to $(G/G)^k$ via the residue map.  This is now a fiber product of stacks locally of finite type.
\end{definition}

We want to compute the graded dimension of these stacks.
\begin{prop}
There is an exponential map
\[\exp \colon \mr{Higgs}_G^\mr{fr}(\bb {CP}^1; D, \omega^\vee) \to \mr{GpHiggs}_G^\mr{fr}(\bb {CP}^1; D, \omega^\vee)\]
from the ordinary Higgs stack to the multiplicative Higgs stack.  This map is formally \'etale so preserves graded dimensions.  I'm abusing notation and using $\omega^\vee$ to denote the residues both in the group and its Lie algebra.
\end{prop}

\begin{remark}
This map doesn't see the whole multiplicative Higgs moduli space.  For instance, in the Lie algebra the stratification by the conjugacy class of the centralizer has strata corresponding to Levis, but in the Lie group outside of type A there are additional possibilities referred to as pseudo-Levis.  
\end{remark}

Let's therefore compute the tangent complex of the ordinary Higgs stack at a closed point $(P,\phi)$.  This is straightforward because taking the tangent complex commutes with fiber products, so
\[\bb T_{\mr{Higgs}_G^\mr{fr}(\bb {CP}^1; D, \omega^\vee),(P,\phi)} \iso \bb T_{\mr{Higgs}_G^\text{fr,simp}(\bb {CP}^1; D),(P,\phi)} \times_{\bb T_{(\gg/G)^k, \omega^\vee}} (\bb T_{BL_1} \oplus \cdots \oplus \bb T_{BL_k}).\]
The tangent complex to $\gg/G$ at the point $\omega^\vee_{z_i}$ is equivalent to $\mf l_i[1] \oplus \mf l_i$, where $\mf l_i$ is the centralizer of $\omega^\vee_{z_i}$ in $\gg$.  The tangent complex $\bb T_{BL_i}$ is likewise equivalent to $\mf l_i[1]$, which means we can compute the tangent complex to the fiber product as the two step complex
\[\bb T_{\mr{Higgs}_G^\text{fr,simp}(\bb {CP}^1; D),(P,\phi)} \to \bigoplus_{i=1}^k \mf l_i[-1].\]
It's easy to describe the fiber of the tangent complex to $\mr{Higgs}_G^\text{fr,simp}(\bb {CP}^1; D)$ at $(P,\phi)$ in terms of the definition, so the overall tangent complex we want is equivalent to
\[H^\bullet(\bb {CP}^1; \gg_P[1] \oplus \gg_P(D))/(\hh[1] \oplus \hh) \to \bigoplus_{i=1}^k \mf l_i[-1]\]
where the quotient by $\hh[1] \oplus \hh$ comes from fixing a regular semisimple framing at $\infty$.

\begin{remark}
In order to make this identification we need to justify replacing the two tangent complexes on the left and the right by their cohomology.  To see that we can do this, just note that the complex on the left is concentrated in degree $-1,0$ and 1, and the complex on the right is concentrated in degrees 0 and 1, and so there's no possible differential on the $E_2$-page of the spectral sequence.
\end{remark}

Now, of course we can compute the cohomology of our bundle on $\bb {CP}^1$.  It's given by 
\[H^\bullet(\bb {CP}^1; \gg_P[1] \oplus \gg_P(D)) \iso \Gamma(\bb {CP}^1; \gg_P[1] \oplus (\gg_P(-2) \oplus \gg_P(k)) \oplus \gg_P(-k-2)[-1])\]
where we've used the Killing form to identify $\gg_P$ with $\gg_P^*$.  In particular we've established the following.

\begin{prop}
The fiber of the moduli stack of Higgs bundles over the point $P \in \bun_G^{\mr{fr}}(\bb{CP}^1)$ is classical if the following two conditions are satisfied.
\begin{enumerate} 
 \item The vector bundle $\gg_P(-k-2)$ has no global sections.
 \item The map $\Gamma(\bb{CP}^1; \gg_P(k)) \to \bigoplus_{i=1}^k \mf l_i$ arising as the derivative of the residue map at $\omega^\vee$ is surjective.
\end{enumerate}
\end{prop}

The first condition is always satisfied if $k$ is large enough compared to the degree of $P$.  The second condition is a little trickier -- the obstruction to surjectivity is the condition that the deformation of the section of $\gg_P(k)$ corresponding to the prescribed deformation of the residues can be extended across $\infty$.  This seems to be satisfied only for elements $(\ell_1, \ldots, \ell_k)$ satisfying $\sum_{i=1}^k \ell_i = 0$, which would mean the moduli stack would essentially never be classical.

\begin{remark}
There's a separate condition involving the $\omega^\vee_{z_i}$ for a classical closed point $(P,\phi)$ with the appropriate residues to even exist, saying that the $\omega^\vee_{z_i}$ are conjugate to elements summing to zero.  This condition should essentially be that $\sum_{i=1}^k \tr(\omega^\vee_{z_i}) = 0$, but I'm not sure if it's literally that at the moment.
\end{remark}

Ok, now let's compute the virtual dimension at the point $(P,\phi)$ assuming condition 1) above is satisfied, or equivalently the virtual dimension to the multiplicative Hitchin system at the point $(P,\exp(\phi))$.  I can do this at least for $G = \GL_n$.

\begin{prop}
Assuming $\gg_P(-k-2)$ has no global sections, the virtual dimension of $\mr{GpHiggs}_{\GL_n}^\mr{fr}(\bb {CP}^1; D, \omega^\vee)$ at a point $(P,\exp(\phi))$ in the image of the exponential map is 
\[kn^2 - \sum_{i=1}^k \dim Z_{\gl_n}(\omega^\vee_{z_i}).\]
\end{prop}

\begin{remark}
I think that for general $G$ I just get the virtual dimension to be
\[k\dim(G) - \sum_{i=1}^k \dim Z_{\gl_n}(\omega^\vee_{z_i}) = \sum_{i=1}^k \codim(Z_{\gl_n}(\omega^\vee_{z_i}))\]
but I'm not 100\% sure of this.  One might also reasonably conjecture this to hold away from the image of the exponential map: that the virtual dimension near a point $(P,g)$ where $g$ is a general multiplicative Higgs field would be the sum of the codimensions of the stabilizers of the residues at each puncture (where these stabilizers can now be arbitrary pseudo-Levi subgroups).
\end{remark}

\begin{example}
If $G = \GL_n$ and all the residues $\omega^\vee_{z_i}$ are regular semisimple, then the virtual dimension of the moduli space is $k(n^2-n)$.
\end{example}

\begin{remark}
Note that the moduli stack of Higgs bundles (multiplicative or otherwise) in this language is not only always derived, but always stacky -- the tangent complex always has a non-trivial $H^{-1}$.  Furthermore the dimension of $H^0$ of the tangent complex depends on the choice of $G$-bundle $P$, as does the dimension of the automorphism group acting on it.  It's only in the quotient that these dependencies cancel out.  This cancelling also only happens when the condition on $\gg_P(k-2)$ is satisfied: if $k$ is too small then we'll once again see a dependence on the degree of the principal $G$-bundle.
\end{remark}

\begin{remark}
Note that there's no reason for the virtual dimension of the moduli space to always be even: the sign depends on the choice of residues $\omega^\vee_{z_i}$.
\end{remark}

\begin{question}
Up to the issue of stackiness and derivedness, does this answer seem to match the answer you were expecting?
\end{question}


\textsc{Institut des Hautes \'Etudes Scientifiques}\\
\textsc{35 Route de Chartres, Bures-sur-Yvette, 91440, France}\\
\texttt{celliott@ihes.fr}\\
\vspace{5pt}
\end{document}





