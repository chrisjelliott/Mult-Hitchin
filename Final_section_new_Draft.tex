\documentclass[11pt, oneside, reqno]{amsart}

\usepackage{amsmath, amsthm, amssymb}
\usepackage[usenames,dvipsnames]{color}
\usepackage[all, cmtip]{xy}
\usepackage[pdftex, bookmarks=true, linkbordercolor={0 0 1}]{hyperref}
\usepackage[margin=1in]{geometry}

\setlength{\parindent}{0pt}
\setlength{\parskip}{11pt}

\theoremstyle{definition} \newtheorem{definition}{Definition}[section]
\newtheorem{lemma}[definition]{Lemma}
\newtheorem{theorem}[definition]{Theorem}
\newtheorem{prop}[definition]{Proposition}
\newtheorem{conjecture}[definition]{Conjecture}
\newtheorem{corollary}[definition]{Corollary}
\newtheorem{construction}[definition]{Construction}
\newtheorem{observation}[definition]{Observation}
\newtheorem{assumption}[definition]{Assumption}
\newtheorem{claimnum}[definition]{Claim}
\newtheorem*{nonum}{Theorem}
\newtheorem*{lemma*}{Lemma}
\newtheorem*{claim}{Claim}
\newtheorem*{subclaim}{Subclaim}
\newtheorem*{fact}{Fact}
\newtheorem*{problem}{Problem}
\newtheorem*{ack}{Acknowledgements}

\theoremstyle{definition} \newtheorem{remark}[definition]{Remark}
\theoremstyle{definition} \newtheorem{remarks}[definition]{Remarks}
\theoremstyle{definition} \newtheorem{question}[definition]{Question}
\theoremstyle{definition} \newtheorem*{note}{Note}
\theoremstyle{definition} \newtheorem{example}[definition]{Example}
\theoremstyle{definition} \newtheorem{examples}[definition]{Examples}

\newtheorem{pseudoconj}[definition]{Pseudo-Conjecture}


\renewcommand{\gg}{\mathfrak{g}}

\newcommand{\bb}[1]{\mathbb{#1}}
\newcommand{\mr}[1]{\mathrm{#1}}
\newcommand{\mc}[1]{\mathcal{#1}}
\newcommand{\mf}[1]{\mathfrak{#1}}
\newcommand{\wt}[1]{\widetilde{#1}}
\newcommand{\bo}[1]{\boldsymbol{#1}}

\newcommand{\inj}{\hookrightarrow}
\newcommand{\bs}{\ \backslash \ }
\newcommand{\dd}{\partial}
\newcommand{\del}{\partial}

\newcommand{\ul}[1]{\underline{#1}}
\newcommand{\ol}[1]{\overline{#1}}

\newcommand{\CC}{\mathbb{C}}
\newcommand{\RR}{\mathbb{R}}
\newcommand{\OO}{\mathcal{O}}
\newcommand{\ZZ}{\mathbb{Z}}

\newcommand{\eps}{\varepsilon}

\newcommand{\SO}{\mathrm{SO}}
\newcommand{\SL}{\mathrm{SL}}
\newcommand{\GL}{\mathrm{GL}}
\newcommand{\SU}{\mathrm{SU}}
\newcommand{\spin}{\mathrm{Spin}}
\newcommand{\Spin}{\mathrm{Spin}}
\newcommand{\so}{\mathfrak{so}}
\renewcommand{\sl}{\mathfrak{sl}}
\renewcommand{\sp}{\mathfrak{sp}}
\newcommand{\gl}{\mathfrak{gl}}

\newcommand{\sfe}{\mathsf{e}}
\newcommand{\sff}{\mathsf{f}}
\newcommand{\sfh}{\mathsf{h}}
\newcommand{\sfs}{\mathsf{s}}

\newcommand{\frakq}{\mathfrak{q}}

\newcommand{\sub}{\subseteq}
\newcommand{\iso}{\cong}

\DeclareMathOperator{\coh}{Coh}
\DeclareMathOperator{\higgs}{Higgs}
\DeclareMathOperator{\bun}{Bun}
\DeclareMathOperator{\Gr}{Gr}
\DeclareMathOperator{\spec}{Spec}
\DeclareMathOperator{\res}{res} %lower case: residue.
\DeclareMathOperator{\EOM}{EOM}
\DeclareMathOperator{\id}{id}
\DeclareMathOperator{\dvol}{dvol}
\DeclareMathOperator{\aut}{Aut}
\DeclareMathOperator{\sym}{Sym}
\DeclareMathOperator{\Flat}{Flat}
\DeclareMathOperator{\mhiggs}{mHiggs}
\DeclareMathOperator{\mon}{Mon}
\DeclareMathOperator{\diff}{Diff}
\DeclareMathOperator{\Hol}{Hol}
\DeclareMathOperator{\mhitch}{mHitch}

\newcommand{\map}{\ul{\mr{Map}}}
\newcommand{\qconn}{q\text{-Conn}}
\newcommand{\conn}{\text{-Conn}}
\newcommand{\epsconn}{\varepsilon\text{-Conn}}
\renewcommand{\d}{\mathrm{d}}
\newcommand{\fr}{\mathrm{fr}}
\newcommand{\ad}{\mr{ad}}
\newcommand{\Ad}{\mr{Ad}}
\newcommand{\HT}{\mr{HT}}

\title{Multiplicative Hitchin Systems and Supersymmetric Gauge Theory}
\author{Chris Elliott \and Vasily Pestun}
\date{\today}

\newcommand{\chris}[1]{(\textcolor{red}{Chris: #1})}
\newcommand{\vasily}[1]{(\textcolor{blue}{Vasily: #1})}

\begin{document}
 
\section{$q$-Opers and $q$-Characters}

In this section we will discuss the space of $q$-Opers in more depth.  In particular we will connect the geometric setup described in this paper, in terms of multiplicative Higgs bundles, to the gauge theoretic story studied by the second author and collaborators \cite{NekrasovPestunShatashvili,Kimura:2015rgi,Nekrasov:2015wsu}.  The main goal of this subsection will be to describe and motivate a connection between $q$-Opers and the $q$-character maps from the theory of quantum groups.  In order to make our statements as concrete as possible it will be useful to first describe the Steinberg section of a semisimple group explicitly.

Throughout this section, assume that $G$ is a simple simply-laced Lie group of adjoint type with Lie algebra $\gg$.  Let $\Delta = \{\alpha_1, \ldots, \alpha_r\}$ be the set of simple roots of $\gg$.  In order to define the Steinberg section uniquely we'll fix a \emph{pinning} on $G$.  That is, choose a Borel subgroup $B \sub G$ with maximal torus $T$ and unipotent radical $U$, and choose a generator $e_i$ for each simple root space $\gg_{\alpha_i}$.

We'll also choose an element $\sigma_i \in N(T)$ in the normalizer of $T$ representing each element of the Weyl group $W = N(T)/T$.  The Steinberg section will be independent of this choice up to conjugation by a unique element of $T$, and independent of the ordering on the set of simple roots.

\begin{definition}
The \emph{Steinberg section} of $G$ associated to a choice of pinning is the image of the injective map $\sigma \colon T/W \to G$ defined by
\[\sigma(t_1, \ldots, t_r) = \prod_{i=1}^r \exp(t_i e_i) \sigma_i.\]
Steinberg proved \cite[Theorem 1.4]{Steinberg} that, after restriction to the regular locus in $G$, the map $\sigma$ defines a section of the Chevalley map $\chi \colon G \to T/W$.
\end{definition}

\begin{definition}
Fix a coloured divisor $(D,\omega^\vee)$ The \emph{multiplicative Hitchin section} of the map $\pi \colon \mhiggs^\fr_G(\bb{CP}^1,D,\omega^\vee) \to \mc B(D,\omega^\vee)$ is the image $\mhitch^\fr_G(\bb{CP}^1, D, \omega^\vee)$ of the map defined by post-composing a meromorphic $T/W$-valued function on $\bb{CP}^1$ with the Steinberg map $\sigma$.
\end{definition}

\begin{remark}
The multiplicative Hitchin section is indeed a section of the map $\pi$ after restricting to the connected component in $\mhiggs^\fr_G(\bb{CP}^1,D,\omega^\vee)$ corresponding to the trivial bundle, provided one chooses a value for the framing within the Steinberg section.  For example, if we choose the identity framing on the multiplicative Hitchin basis then the multiplicative Hitchin section lands in multiplicative Higgs bundles with framing $c = \sigma(1)$ at infinity, i.e. framing given by a Coxeter element.
\end{remark}

Now, let's introduce the key idea in this section: the notion of \emph{triangularization} for the multiplicative Hitchin section.
\begin{definition} \label{gen_evals_def}
Let $g(z)$ be an element of $\mhitch^\fr_G(\bb{CP}^1, D, \omega^\vee)$.  We'll abusively identify $g(z)$ with its image under the restriction map $r_\infty \colon \mhiggs_G^\fr(\bb{CP}^1,D,\omega^\vee) \to G_c[[z^{-1}]]$ to a formal neighbourhood of $\infty$.  Say that $g(z)$ has \emph{generalized eigenvalues} $y(z) \in T[[z^{-1}]]$ if there exists an element $u(z)$ of $U[[z^{-1}]]$ such that $u(z)g(z)u(z)^{-1}$ is an element of $B_-[[z^{-1}]]$, where $B_-$ is the opposite Borel subgroup to $B$, which maps to $y(z)$ under the canonical projection.

We say that $g(z)$ has \emph{$q$-generalized eigenvalues} $y(z) \in T[[z^{-1}]]$ if there exists an element $u$ of $U[[z^{-1}]]$ such that $u(q^{-1}z)g(z)u(z)^{-1}$ is an element of $B_-[[z^{-1}]]$ that maps to $y(z)$ under the canonical projection. \chris{should we be using additive notation $u(z-\eps)$ here instead?}
\end{definition}

This idea appeared previously in \cite{NekrasovPestun, NekrasovPestunShatashvili}.  For example, in the undeformed case, the generalized eigenvalues have a very geometric meaning: they are equivalent to a sequence of algebraic functions defining the cameral cover at a point $t(z)$ in the Hitchin base.  To see this, it's easiest to use a slightly different representation of the multiplicative Hitchin section, packaging the singularity datum in a more uniform way. 

Choose a point $b(z)$ in the multiplicative Hitchin base $B(D,\omega^\vee)$. By clearing denominators, we can identify $b(z)$ with a canonical polynomial $t(z)$ in $T[z]$ of fixed degree, with fixed top degree term. \chris{...}
 
\chris{further motivation, then definition of the $q$-character.  We're in the rational case, so I think the definition we want is the one first appearing due to Knight \cite{Knight}}
\begin{definition}
The \emph{$q$-character} associated to \chris{...} is the map
\[\chi_q \colon \]
\chris{... generated by the Weyl action?  Merge with the below:}

Then quiver gauge theory $q$-characters are generated by the iWeil reflections of the form
 \cite{NekrasovPestun,NekrasovPestunShatashvili,Nekrasov:2015wsu} (we follow conventions of \cite{Kimura:2015rgi} for shifts)
\begin{equation}
  y_{i,z} \mapsto    (y_{i, z q^{-1}}^{-1} \prod_{e: i \to j} y_{j, z} \prod_{e: j\to i} y_{j,q^{-1} z})  p_{i, q^{-1} z}
\end{equation}
\end{definition}

Generalizing this story, after $q$-deformation, we conjecture the following surprising relationship between the space of $q$-opers and the $q$-character.
 
\begin{conjecture} \label{qchar_conjecture}
For any $q$, every element of $g(z)$ of $\mhitch^\fr_G(\bb{CP}^1, D, \omega^\vee)$ has a unique $q$ generalized eigenvalue, and therefore there is a well-defined map
\[E \colon \mc B(D,\omega^\vee) \to T[[z^{-1}]]\]
given by applying the multiplicative Hitchin section then computing its generalized eigenvalues.  The image of this map is \chris{...}, and therefore there is a well-defined composite map $\chi_q \circ E \colon \mc B(D,\omega^\vee) \to \mc B(D,\omega^\vee)$.  In fact, this composite map is merely an affine automorphism of the multiplicative Hitchin base.
\end{conjecture}

\begin{examples}
\begin{enumerate}
 \item In type $A_1$ we can calculate everything very explicitly.  We've already described the multiplicative Hitchin section in Section \ref{GL2_example_section}: it consists of matrices of the form
\begin{equation*}
  g^{t} =
  \begin{pmatrix}
    t   & - 1 \\
    1 & 0
  \end{pmatrix}.
\end{equation*}
We would like to triangularize this to obtain a matrix of the form
\begin{equation*}
  g^{y} =
  \begin{pmatrix}
    y  & 0 \\
    1 & y^{-1} 
  \end{pmatrix}.
\end{equation*}
First, let $q=1$.  It's easy to solve the equation $g^t = ug^y u^{-1}$ explicitly.  One finds a solution with $t = y + y^{-1}$, after conjugation by the element $u = - y^{-1}$.  As the conjecture tells us to expect, $t$ is identified with a Weyl invariant polynomial in $\mathbb{C}[y, y^{-1}]$ which starts from the highest weight monomial $y$. 

For general $q$, \chris{todo, but need to decide whether to write additively or multiplicatively.}

 \item We can also make concrete calculations for type $A_2$.  For more direct comparison with the formulae in the literature we'll use the alternative formulation discussed above involving polynomials $p_i(z)$ encoding the singularity datum $(D, \omega^\vee)$.  We label positive roots as $\alpha_1, \alpha_2$ and $\alpha_3:=\alpha_1 + \alpha_2$ and parametrize a $U[[z^{-1}]]$-valued
  gauge transformation $u(z)$ the by collection of functions  
 $(u_{i}(z))_{\alpha_i  \in \Delta^{+}}$ 
  \begin{equation}
    u(z) = \prod_{3,2,1} \exp( u_{i}(z) e_{\alpha_i})
  \end{equation}
Then solving the equation 
\begin{equation}
g^t(z) =  u(q^{-1} z)^{-1} g^y(z) u(z)
\end{equation}
for $u_{1}(z), u_{2}(z), u_{3}(z)$ and $t_{1}(z), t_{2}(z)$  we find that
\begin{equation*}
\begin{aligned}
& u_{1}(z) = p_{1}(z) u_{2}(q^{-1} z)-p_{1}(z) y_{2}(z) y_{1}(z)^{-1} \\
& u_{2}(z) =  -p_{2}(z) y_{2}(z)^{-1} \\
& u_{3}(z) = -p_{1}(z) p_{2}(z) y_{1}(z)^{-1} \\
& t_{1}(z) = y_{1}(z)-u_{1}(q^{-1}z) \\
& t_{2}(z) = y_{2}(z) - y_{1}(z) u_{2}(q^{-1}z)-u_{3}(q^{-1}z) \\
\end{aligned}
\end{equation*}
which implies in turn that
\begin{equation*}
  \begin{aligned}
    t_{1}(z) =y_{1}(z)  +  \frac{p_{1}(q^{-1}z) y_{2}(q^{-1}z)}{y_{1}(q^{-1}z)} + \frac{p_{1}(q^{-1}z) p_{2}(q^{-2} z)}{ y_{2}(q^{-2} z)}\\
    t_{2}(z) = y_{2}(z)  +\frac{y_{1}(z) p_{2}(q^{-1}z)}{y_{2}(q^{-1}z)}+   \frac{p_{1}(q^{-1}z) p_{2}(q^{-1}z)}{y_{1}(q^{-1}z)}
  \end{aligned}
\end{equation*}
and that indeed coincides with the expression for the $q$-characters for the $A_2$ quiver appearing in
\cite{Nekrasov:2015wsu,NekrasovPestunShatashvili,NekrasovPestun,Kimura:2015rgi}.
\end{enumerate}
\end{examples}

One approach to proving Conjecture \ref{qchar_conjecture} would be to follow the following algorithm.
\begin{enumerate}
\chris{todo: edit for consistency}
 \item There exists $U_{+}$ gauge transformation 
where $u_{i,z}$ are Laurent polynomials in $y_{i,q^{\mathbb{Z}} z}$
and $p_{i, q^{\mathbb{Z}} z}$ to a $q$-triangularized element of $G[[z^{-1}]]$.

\item The $t_{i,z}$ are Laurent polynomials in $y_{i,q^{\mathbb{Z}} z}$
and $p_{i, q^{\mathbb{Z}} z}$.

\item The Laurent polynomial $t_{i,z}$  in $y_{i,q^{\mathbb{Z}} z}$
and $p_{i, q^{\mathbb{Z}} z}$ starts from the highest weight monomial $y_{i,z}$, i.e. 
\begin{equation*}
  t_{i,z} = y_{i,z} + \dots 
\end{equation*}

\item And the key property of being q-character:
the Laurent polynomial $t_{i,z}$ in  $y_{i,q^{\mathbb{Z}} z}$
and $p_{i, q^{\mathbb{Z}} z}$ is invariant under $q$-shifted $p$-shifted Weyl group action (iWeyl): for every monomial that contains as
a factor, say $y_{i,q^{a} z}$, there is also
another monomial obtained as image of the iWeyl action
\begin{equation*}
    y_{i,z} \mapsto    (y_{i, z q^{-1}}^{-1} \prod_{e: i \to j} y_{j, z} \prod_{e: j\to i} y_{j,q^{-1} z})  p_{i, q^{-1} z} 
  \end{equation*}
e.g. the second monomial in the $q$-character is always like 
  \begin{equation*}
  t_{i,z} = y_{i,z} + (y_{i, z q^{-1}}^{-1} \prod_{e: i \to j} y_{j, z} \prod_{e: j\to i} y_{j,q^{-1} z})  p_{i, q^{-1} z}  + \dots 
  \end{equation*}
and so on...
  
\end{enumerate}

To complete a constructive proof of this form one could try to follow the algorithm appearing in \cite[Page 11-12]{STSSevostyanov},   see also \cite{Sevostyanov1}.  One would need to modify this proof by a twist that introduces the required singularity datum.

\begin{remark}
\chris{todo, edit for compatibility}
The quasi-diagonalized $T$-form of the $q$-connection has apparent singularities (even away from the divisor $D$)  where the functions $y_{i}(z)$ have zeroes and poles.  However, these singularities are cancelled in pairs between the monomials
 related by the iWeyl reflections in $t_{i}(z)$.
 This property has been called regularity of the $q$-character in the quantum group representation theory literature\cite{FrenkelReshetikhin1,FrenkelReshetikhin2} and in the gauge theory construction 
 \cite{NekrasovPestunShatashvili,Nekrasov:2015wsu,NekrasovPestun,Kimura:2015rgi}.
 
 The meromorphic functions $y_{i,z}$ can be expressed as ratios of the form $y_{i,z} = Q_{i,z}/Q_{i,q^{-1}z}$ \vasily{check $q^{-1} z$ vs $q z$}, and the zeroes of the $Q$ functions are known as \emph{Bethe roots} in the context of the Bethe ansatz equations.  In this language, the Bethe ansatz equations are precisely the equations which ensures that  that poles in $t_{i,z}$ are cancelled. 
\end{remark}


 \begin{remark}
The paper \cite{KoroteevSageZeitlin} contains a proof of specialization of the main theorem of this section to the case of $G = \SL_n$ \chris{if we're in adjoint type shouldn't it be $\mr{PSL}_n$ instead?} with a special form of the coloured divisor $(D,\omega^\vee)$ where the $T$-valued polynomials $p(z)$ encoding their positions and orders can be effectively presented as the ratio of Drinfeld polynomials shifted by $\eps$, so that effectively $p_{i,z} = d_{i}(z)/d_{i}(z - \eps)$.  This special form for the singularity datum means that the Yangian module obtained by quantization of the symplectic leaf $\mhiggs_G(C,D)$ contains (as a quotient) the finite-dimensional Drinfeld module specified by the Drinfeld polynomials $d_{i,z}$. In the language of quiver gauge theory, this specialization is known as 4d to 2d specialization \cite{ChenDoreyHollowoodLee, DoreyHollowoodLee}. This specialization leads to the Bethe ansatz equations with finite dimensional representations of Yangians and finite number of Bethe roots. 
\end{remark}

\begin{remark}
\chris{todo: edit for compatibility} One can now try to test the consistency
  of the conjecture \ref{eq:classical-q-langlands} as follows.
  Take $G = \SL_2$ (simply-connected form)
  and $G^\vee = \mr{PSL}_2$ (adjoint-form).

  In particular, we claim that under the equivalent of \ref{eq:classical-q-langlands} :

( the spectrum $B_{G}$  of the ring of commuting Hamiltonians
  of the integrable system $\mhiggs_{G}$ (this commuting ring
  is generated by the  fundamental character $\mathrm{tr}_{R_{\omega_1}} g(z)$ for $g \in G$  where $R_{\omega_1}$ is 2-dimensional $SL_2$ representation,
  the spectrum $B_{G}$ is the base of the integrable system $\mhiggs_{G}$) ) 
  

{  \center{ is isomorphic to}}

(  the oper for $\check G = PSL_2$  (adjoint form) which is also nicely
parametrized 
by $t(z) = y + y^{-1}$ ) 
\end{remark}
 
\bibliographystyle{alpha}
\bibliography{Mult_Hitchin}
\end{document}
