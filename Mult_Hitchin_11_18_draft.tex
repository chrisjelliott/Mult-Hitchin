\documentclass[11pt, oneside, reqno]{amsart}

\usepackage{amsmath, amsthm, amssymb}
\usepackage[usenames,dvipsnames]{color}
\usepackage[all, cmtip]{xy}
\usepackage[pdftex, bookmarks=true, linkbordercolor={0 0 1}]{hyperref}
\usepackage[margin=1in]{geometry}
\usepackage{comment}

\setlength{\parindent}{0pt}
\setlength{\parskip}{11pt}

\theoremstyle{definition} \newtheorem{definition}{Definition}[section]
\newtheorem{lemma}[definition]{Lemma}
\newtheorem{theorem}[definition]{Theorem}
\newtheorem{prop}[definition]{Proposition}
\newtheorem{conjecture}[definition]{Conjecture}
\newtheorem{corollary}[definition]{Corollary}
\newtheorem{construction}[definition]{Construction}
\newtheorem{observation}[definition]{Observation}
\newtheorem{assumption}[definition]{Assumption}
\newtheorem{claimnum}[definition]{Claim}
\newtheorem*{nonum}{Theorem}
\newtheorem*{lemma*}{Lemma}
\newtheorem*{claim}{Claim}
\newtheorem*{subclaim}{Subclaim}
\newtheorem*{fact}{Fact}
\newtheorem*{problem}{Problem}
\newtheorem*{ack}{Acknowledgements}

\theoremstyle{definition} \newtheorem{remark}[definition]{Remark}
\theoremstyle{definition} \newtheorem{remarks}[definition]{Remarks}
\theoremstyle{definition} \newtheorem{question}[definition]{Question}
\theoremstyle{definition} \newtheorem*{note}{Note}
\theoremstyle{definition} \newtheorem{example}[definition]{Example}
\theoremstyle{definition} \newtheorem{examples}[definition]{Examples}

\newtheorem{pseudoconj}[definition]{Pseudo-Conjecture}


\renewcommand{\gg}{\mathfrak{g}}

\newcommand{\bb}[1]{\mathbb{#1}}
\newcommand{\mr}[1]{\mathrm{#1}}
\newcommand{\mc}[1]{\mathcal{#1}}
\newcommand{\mf}[1]{\mathfrak{#1}}
\newcommand{\wt}[1]{\widetilde{#1}}
\newcommand{\bo}[1]{\boldsymbol{#1}}

\newcommand{\inj}{\hookrightarrow}
\newcommand{\bs}{\ \backslash \ }
\newcommand{\dd}{\partial}
\newcommand{\del}{\partial}

\newcommand{\ul}[1]{\underline{#1}}
\newcommand{\ol}[1]{\overline{#1}}

\newcommand{\CC}{\mathbb{C}}
\newcommand{\RR}{\mathbb{R}}
\newcommand{\OO}{\mathcal{O}}
\newcommand{\ZZ}{\mathbb{Z}}

\newcommand{\eps}{\varepsilon}

\newcommand{\SO}{\mathrm{SO}}
\newcommand{\SL}{\mathrm{SL}}
\newcommand{\GL}{\mathrm{GL}}
\newcommand{\SU}{\mathrm{SU}}
\newcommand{\PGL}{\mathrm{PGL}}
\newcommand{\spin}{\mathrm{Spin}}
\newcommand{\Spin}{\mathrm{Spin}}
\newcommand{\so}{\mathfrak{so}}
\renewcommand{\sl}{\mathfrak{sl}}
\renewcommand{\sp}{\mathfrak{sp}}
\newcommand{\gl}{\mathfrak{gl}}

\newcommand{\sfe}{\mathsf{e}}
\newcommand{\sff}{\mathsf{f}}
\newcommand{\sfh}{\mathsf{h}}
\newcommand{\sfs}{\mathsf{s}}

\newcommand{\frakq}{\mathfrak{q}}

\newcommand{\sub}{\subseteq}
\newcommand{\iso}{\cong}

\DeclareMathOperator{\coh}{Coh}
\DeclareMathOperator{\higgs}{Higgs}
\DeclareMathOperator{\bun}{Bun}
\DeclareMathOperator{\Gr}{Gr}
\DeclareMathOperator{\spec}{Spec}
\DeclareMathOperator{\res}{res}
\DeclareMathOperator{\EOM}{EOM}
\DeclareMathOperator{\id}{id}
\DeclareMathOperator{\dvol}{dvol}
\DeclareMathOperator{\aut}{Aut}
\DeclareMathOperator{\sym}{Sym}
\DeclareMathOperator{\Flat}{Flat}
\DeclareMathOperator{\mhiggs}{mHiggs}
\DeclareMathOperator{\mon}{Mon}
\DeclareMathOperator{\diff}{Diff}
\DeclareMathOperator{\Hol}{Hol}
\DeclareMathOperator{\mhitch}{mHitch}

\newcommand{\map}{\ul{\mr{Map}}}
\newcommand{\qconn}{q\text{-Conn}}
\newcommand{\conn}{\text{-Conn}}
\newcommand{\epsconn}{\varepsilon\text{-Conn}}
\renewcommand{\d}{\mathrm{d}}
\newcommand{\fr}{\mathrm{fr}}
\newcommand{\ad}{\mr{ad}}
\newcommand{\Ad}{\mr{Ad}}
\newcommand{\HT}{\mr{HT}}

\title{Multiplicative Hitchin Systems and Supersymmetric Gauge Theory}
\author{Chris Elliott \and Vasily Pestun}
\date{\today}

\newcommand{\chris}[1]{(\textcolor{red}{Chris: #1})}
\newcommand{\vasily}[1]{(\textcolor{blue}{Vasily: #1})}

\begin{document}

\maketitle 
\begin{abstract}
Multiplicative Hitchin systems are analogues of Hitchin's integrable system based on moduli spaces of $G$-Higgs bundles on a curve $C$ where the Higgs field is group-valued, rather than Lie algebra valued.  We discuss the relationship between several occurences of these moduli spaces in geometry and supersymmetric gauge theory with a particular focus on the case where $C = \bb{CP}^1$ with a fixed framing at infinity.  In this case we prove that the identification between multiplicative Higgs bundles and periodic monopoles proved by Charbonneau and Hurtubise can be promoted to an equivalence of hyperk\"ahler spaces, and analyze the twistor rotation for the multiplicative Hitchin system.  We also discuss quantization of these moduli spaces, yielding the modules for the Yangian $Y(\gg)$ discovered by Gerasimov, Kharchev, Lebedev and Oblezin.
\end{abstract}

\section{Introduction}

In this paper, we will compare five different perspectives an a single moduli space: three coming from geometry and representation theory, and two coming from supersymmetric gauge theory.  By leveraging these multiple perspectives we'll equip our common moduli space with the structure of a completely integrable system which we call the \emph{multiplicative Hitchin system}.  Many of the structures associated with the ordinary Hitchin system, such as the brane of opers, have parallels in the multiplicative setting.  

Let us begin by presenting these five perspectives, before explaining their inter-relationships.  Let $C$ be a Riemann surface, and let $G$ be a reductive complex Lie group.


%\begin{enumerate}

% \item 
\subsection{Multiplicative Higgs Bundles}
We'll first describe the moduli space that motivates the
``multiplicative Hitchin system'' terminology.  Say that a
\emph{multiplicative Higgs bundle} on $C$ is a principal $G$-bundle
$P$ on $C$ with a meromorphic automorphism $g \colon P \to P$, or
equivalently, a meromorphic section of the group-valued adjoint bundle
$\mr{Ad}_P$.  We refer to these by analogy with Higgs bundles, which
consist of a meromorphic section of the Lie algebra (co)adjoint
bundle, rather than its Lie group valued analogue \footnote{Usually
  Higgs bundles on a curve are defined to be sections of the coadjoint
  bundle twisted by the canonical bundle.  There isn't an obvious
  replacement for this twist in the multiplicative context, but we'll
  mostly be interested in the case where $C$ is Calabi-Yau, and
  therefore this twist is trivial.}.  Just like for ordinary Higgs
bundles, the space of multiplicative meromorphic Higgs bundles with
arbitrary poles is infinite-dimensional, but we can cut it down to a
finite-dimensional space by fixing the locations of the singularities,
and their local behaviour at each singular point.  For example, for $G = \GL_1$ this local behaviour is given by a degree at each puncture: we restrict to multiplicative Higgs fields $g$ such that, near each singular point $z_i$, $g$ is given by the product of a local holomorphic function and $(z-z_i)^{n_i}$ for some integer $n_i$.  More generally the local ``degrees'' are described by orbits in the
affine Grassmannian $\Gr_G$: restrict the multiplicative Higgs field
to a formal neighborhood of the puncture, to obtain an element of the
algebraic loop group $LG = G(\!(z)\!)$.  This element is well-defined
up to the action of $L^+G = G[[z]]$ on the left and right, and so
determines a well-defined double coset in
$L^+G \!\!\bs\!\! LG/L^+G = L^+G \!\!\bs\!\! \Gr_G$.  Orbits in the
affine Grassmannian are in canonical bijection with dominant coweights
of $\gg$, so fixing a ``degree'' at the punctures means fixing a dominant
coweight $\omega^\vee_{z_i}$ at each puncture $z_i$.
 
 Fix a curve $C$ and consider the \emph{moduli space of multiplicative Higgs bundles} on $C$ with prescribed singularities lying at the points $D = \{z_1, \ldots, z_k\}$, with degrees $\omega_{z_1}^\vee, \ldots, \omega_{z_k}^\vee$ respectively.  We denote this moduli space by $\mhiggs_G(C,D,\omega^\vee)$.
 
 Moduli spaces of multiplicative Higgs bundles have been studied in the literature previously, though they've more often been referred to simply as ``moduli spaces of $G$-pairs''.  The earliest descriptions of these moduli spaces that we're aware of occur in the work of Hurtubise and Markman \cite{HurtubiseMarkman, HurtubiseMarkman1}.  Their work was motivated by the desire to understand the elliptic Skylanin Poisson brackets on loop groups, and to that end they mainly studied the case where the underlying curve $C$ is elliptic -- we'll refer back to their work throughout the paper.  From a different perspective, the moduli space of multiplicative Higgs bundles was also studied by Frenkel and Ng\^o \cite[Section 4]{FrenkelNgo} as part of their geometrization of the trace formula.  Further work was done following their definition by Bouthier \cite{Bouthier2, Bouthier1}, who gave an alternative construction of the moduli space as the space of sections of a vector bundle associated to a fixed singuarity datum defined using the Vinberg semigroup (see Remark \ref{Bouthier_remark}).  
 
 The moduli space of multiplicative Higgs bundles admits a version of the Hitchin fibration, just like the ordinary moduli space of Higgs bundles.  Intuitively, this map is defined using the Chevalley map $\chi \colon G \to T/W$ (for instance if $G = \GL_n$, the map sending a matrix to its characteristic polynomial): the multiplicative Higgs field is, in particular, a meromorphic $G$-valued function, and the multiplicative Hitchin fibration post-composes this function with $\chi$.  See Section \ref{Hitchin_system_section} for a more detailed definition.
 
 While the structure of the multiplicative fibration makes sense for any curve, it only defines a completely integrable system -- in particular there is only a natural symplectic structure on the total space -- in the very special situation where the curve $C$ is Calabi-Yau.  That is, $C$ must be either $\CC$, $\CC^\times$ or an elliptic curve.  In this paper we'll be most interested in the ``rational'' case, where $C = \CC$.  We use a specific boundary condition at infinity: we consider multiplicative Higgs bundles on $\bb{CP}^1$ with a fixed framing at infinity.  In other words we study the moduli space $\mhiggs^{\text{fr}}_G(\bb{CP}^1,D,\omega^\vee)$ of multiplicative Higgs bundles on $\bb{CP}^1$ with prescribed singularities lying at the points $D = \{z_1, \ldots, z_k\}$, with local data at the singuarities encoded by $\omega_{z_1}^\vee, \ldots, \omega_{z_k}^\vee$ respectively, and with a fixed framing at $\infty$.  This moduli space is a smooth algebraic variety with dimension depending on the coweights we chose
 
 In the case where $C$ is Calabi-Yau the moduli space of multiplicative Higgs bundles admits a one-parameter deformation; from the above point of view, the moduli space is hyperk\"ahler so we can rotate the complex structure within the twistor sphere.  The result of this deformation also has a natural interpretation in terms of \emph{$\eps$-connections}.  That is, $G$-bundles $P$ on $C$ equipped with an isomorphism $P \to \eps^*P$ between $P$ and its translate by $\eps \in \CC$.  We can summarize this identification as follows.
 
 \begin{theorem} [See Theorem \ref{HK_rotation_thm}]
 The moduli space $\mhiggs^{\text{fr}}_G(\bb{CP}^1,D,\omega^\vee)$, given the complex structure at $\eps \in \CC \sub \bb{CP}^1$ in the twistor sphere can be identified (in the limit where an auxiliary parameter is taken to $\infty$) with the moduli space $\epsconn^{\text{fr}}_G(\bb{CP}^1,D,\omega^\vee)$ of framed $\eps$-connections on $\bb{CP}^1$.
 \end{theorem}

\subsection{Moduli Space of Monopoles}
%\item Consider now another, a priori different, representation-theoretic moduli space.  Let $M$ be a Riemannian 3-manifold, and let $G_\RR$ be a compact connected Lie group.  The moduli space of \emph{$G_\RR$-monopoles} on $M$ is the moduli space of triples $(P,A,\Phi)$, consisting of a smooth principal $G_\RR$-bundle $P$ equipped with a connection $A$ and a section $\Phi$ of the associated bundle $\gg_{P}$, together satisfying the \emph{Bogomolny equation}: $\ast F_A = \d_A \Phi$. 

If we focus on the example where real three-dimensional Riemannian
manifold $M = C \times S^1$ splits as the product of a compact Riemann
surface and a circle, for the rational case $C = \CC$ the moduli space
of monopoles on $M$ was studied in by Cherkis and Kapustin
\cite{CherkisKapustin1,CherkisKapustin2,CherkisKapustin3} from the
perspective of the Coulomb branch of vacua of 4d $\mathcal{N}=2$
supersymmetric gauge theory. For general $C$ the moduli space of
monopoles on $M = C \times S^{1}$ was studied by
Charbonneau--Hurtubise \cite{CharbonneauHurtubise}, for
$G_\RR = \mr U(n)$, and by Smith \cite{Smith} for general $G$.  These
moduli spaces were also studied in the rational case where $C = \CC$
with more general boundary conditions than the framing at infinity
which we consider.  This example has also been studied in the
mathematics literature by Foscolo \cite{FoscoloDef, FoscoloThesis} and
by Mochizuki \cite{Mochizuki} -- note that considering weaker boundary
conditions at infinity means they need to use far more sophisticated
analysis in order to work with hyperk\"ahler structures on their
moduli spaces than we'll consider in this paper.

The connection between periodic monopoles and multiplicative Higgs bundles is provided by the following theorem.
\begin{theorem}[Charbonneau--Hurtubise, Smith]
  There is an analytic isomorphism 
  \begin{equation*}
    H: \mon_G^\fr(C \times S^1,D \times\{0\},\omega^\vee) \to \mhiggs_G^{\text{fr}}(C,D,\omega^\vee)
  \end{equation*}
between the moduli space of (polystable) multiplicative $G$-Higgs bundles on a compact curve $C$ with singularities at points $z_1, \ldots, z_k$ and residues $\omega^\vee_{z_1}, \ldots, \omega^\vee_{z_k}$ and the moduli space of periodic monopoles on $C \times S^1$ with Dirac singularities at $(z_1,0), \ldots, (z_k,0)$ with charges $\omega^\vee_{z_1}, \ldots, \omega^\vee_{z_k}$
\end{theorem}

The morphism $H$,  discussed first by Cherkis and Kapustin in \cite{CherkisKapustin2} and later in
\cite{CharbonneauHurtubise}, \cite{Smith}, \cite{NekrasovPestun}
defines the value of the multiplicative Higgs field at $z \in C$ to be equal to
the holonomy of the monopole connection (complexified by the scalar field) along the fiber circle $\{z\} \times S^{1} \subset M$.






One of our main goals in this paper is to compare -- in the rational case -- the symplectic structure on the moduli space of multiplicative Higgs fields analogous to the symplectic structure defined by Hurtubise and Markman in the elliptic case, and the hyperk\"ahler structure on the moduli space of periodic monopoles, defined by realizing the moduli space of periodic monopoles as a hyperk\"ahler quotient.  We prove that they coincide.

\begin{theorem}[See Theorem \ref{symplectic_comparison_thm}]
  The holomorphic symplectic structure on
  $\mon_G^\fr(\bb{CP}^1 \times S^1,D \times\{0\},\omega^\vee)$ (at 0
  in the twistor sphere of holomorphic symplectic structures) 
  is isomorphic to the pullback of the natural symplectic structure on
  $\mhiggs_G^{\text{fr}}(\bb{CP}^1,D,\omega^\vee)$ under the
  holonomy morphism $H: \mon_G^\fr(\bb{CP}^1 \times S^1,D \times\{0\},\omega^\vee) \to \mhiggs_G^{\text{fr}}(\bb{CP}^1,D,\omega^\vee)$  
\end{theorem}

In particular, the symplectic structure on the multiplicative Higgs moduli space extends to a hyperk\"ahler structure, as we discussed in point 1 above.

%\item 
\subsection{Poisson-Lie Groups}
The third perspective that we'll consider connects our moduli spaces directly to the theory of Poisson Lie groups and Lie bialgebras, and leads to an interesting connection to quantum groups upon quantization.  By the \emph{rational Poisson Lie group} we'll mean the group $G_1[[z^{-1}]]$ consisting of $G$-valued power series in $z^{-1}$ with constant term $1$.  More precisely this is an ind-algebraic group: it is expressed as a direct limit of algebraic varieties $G_1[[z^{-1}]]_n$ with an associative multiplication $G_1[[z^{-1}]]_m \times G_1[[z^{-1}]]_n \to G_1[[z^{-1}]]_{m+n}$.  The Poisson structure on this group can be thought of as coming from a Manin triple: specifically the triple $(G(\!(z^{-1})\!), G_1[[z^{-1}]], G[z])$, where $G(\!(z)\!)$ is equipped with the residue pairing (see Section \ref{quantization_section} for more details).  

We claim that our moduli spaces $\mhiggs_G^{\text{fr}}(\bb{CP}^1,D,\omega^\vee)$ correspond to \emph{symplectic leaves} in the rational Poisson Lie group, extending a classification result of Shapiro \cite{Shapiro} for the groups $G = \SL_n$ and $G = \GL_n$.  More specifically, we prove the following.
\begin{theorem}[See Theorem \ref{Poisson_Lie_Comparison_thm}]
The map $\mhiggs^\fr_G(\bb{CP}^1,D,\omega^\vee) \to G_1[[z^{-1}]]$, defined by restricting a multiplicative Higgs field to a formal neighborhood of $\infty$, is Poisson.  That is, the pullback of the Poisson structure on $\OO(G_1[[z^{-1}]])$ coincides with the Poisson bracket on $\OO(\mhiggs^\fr_G(\bb{CP}^1,D,\omega^\vee))$. 
\end{theorem}

From this point of view it's natural to try to understand how our moduli spaces behave under deformation quantization.  The quantization of the rational Poisson Lie group is well known: it's modelled by the \emph{Yangian} quantum group $Y(\gg)$.  When we quantize our symplectic leaves, we obtain $Y(\gg)$-modules.  This follows from the work of Gerasimov, Kharchev, Lebedev and Oblezin \cite{GKLO} who constructed the $Y(\gg)$-modules in question and analyzed their classical limits.

\begin{remark}
The article \cite{GKLO}, as well as its generalizations such as the work of Kamnitzer, Webster, Weekes and Yacobi \cite{KWWY}, view these $Y(\gg)$-modules as quantizing certain moduli spaces of monopoles on $\RR^3$, and not of monopoles on $\RR^2 \times S^1$.  These two points of view are expected to be related in the limit where the radius of $S^1$ is sent to infinity, with the positions of the singularities in $S^1$ kept fixed (note that while the holomorphic symplectic structure on the moduli space is independent of this radius, the hyperk\"ahler structure \emph{will} be sensitive to it).  The moduli space of periodic monopoles in the rational case would in this limit be related to the moduli space of monopoles on $\RR^3$ (perhaps with certain restrictions as in the work of Finkelberg, Kuznetsov and Rybnikov on trigonometric Zastava spaces \cite{FKR}).
\end{remark}

\subsection{Moduli Spaces in Supersymmetric Gauge Theory} \label{intro_gauge_section}
Having discussed three points of view on our moduli space in geometric representation theory, let us move on to explain some of the ways in which multiplicative Higgs bundles / periodic monopoles arise naturally in the world of supersymmetric gauge theory.  Firstly, these moduli spaces appeared in work of the second author, Nekrasov and Shatashvili \cite{NekrasovPestun, NekrasovPestunShatashvili} concerning the Seiberg-Witten integrable systems of 4d $\mc N=2$ ADE quiver gauge theories.  Let $\gg$ be a simple Lie algebra of ADE type.  One can define an $\mc N=2$ superconformal 4d gauge theory modelled on the Dynkin diagram of $\gg$.  To define this theory one must specify masses $m_{i,j}$ for fundamental hypermultiplets associated to the vertices of the Dynkin diagram, so $i = 1,\ldots,r$ varies over the vertices of the Dynkin diagram and $j=1, \ldots,w_i$ is an index parameterizing the number of hypermultiplets at each vertex.  Let us assume we're in the generic situation, where the masses are all distinct.

The paper \cite{NekrasovPestun} studied the Seiberg-Witten integrable system associated to this theory, so in particular the hyperk\"ahler moduli space $\mf P$ occuring as the target of the $\mc N=4$ supersymmetric sigma model obtained by compactifying the 4d theory on a circle.  In particular, it proved the following.

\begin{theorem}[{\cite[Section 8.1]{NekrasovPestun}}]
Let $G$ be a simple Lie group with ADE Lie algebra $\gg$.  The phase space $\mf P$ is isomorphic, as a hyperk\"ahler manifold, to the moduli space $\mon_G^\fr(\bb{CP}^1 \times S^1,D \times\{0\},\omega^\vee)$ of periodic monopoles, where $D = \{m_{i,j}\}$, and where the residue $\omega^\vee_{m_{i,j}}$ is the fundamental coweight corresponding to the vertex $i$ of the Dynkin diagram of $\gg$.
\end{theorem}

\begin{remark}
If one considers gauge theories that are not necessarily conformal, but only asymptotically free, the phase space is modelled by periodic monopoles that are permitted to have a singularity at infinity.  If $G_\RR = \SU(2)$ then this moduli space fits into the framework studied by Cherkis--Kapustin and by Foscolo.  We won't discuss this more complicated setting further in this paper.
\end{remark}

\begin{remark}[Elliptic fiber version]
  The analysis of \cite{NekrasovPestun} suggests an elliptic fiber
  generalization of our multiplicative story, corresponding to the
  phase space of a theory modelled on an \emph{affine} ADE quiver.
  The phase space of these theories is identified with a moduli space
  of doubly periodic instantons.  In our setting, the idea is the
  following.  We can think of the moduli space of multiplicative Higgs
  bundles as a moduli space of meromorphic functions into the adjoint
  quotient stack $G/G$, just like we can think of the moduli space of
  ordinary Higgs bundles as a moduli space of meromorphic functions
  into the adjoint quotient stack $\gg/G$ for the Lie algebra.  There
  is a rational/trigonometric/elliptic trichotomy extending these two
  examples: one studies the moduli stack of semistable $G$-bundles on
  a cuspidal, nodal or smooth elliptic curve.  That is:
\begin{align*}
\bun^{\mr{ss}}_G(E^{\mr{cusp}}) &\iso \gg/G \\
\bun^{\mr{ss}}_G(E^{\mr{nod}}) &\iso G/G \\
\bun^{\mr{ss}}_G(E_q) &\iso LG/_q LG.
\end{align*}
The last statement is a theorem of Looijenga (see e.g. \cite{Laszlo}): $LG/_q LG$ denotes the $q$-twisted adjoint quotient of the loop group $LG$: so $g(z)$ acts by $h(z) \mapsto g(qz)^{-1}h(z)g(z)$.  

Without singularities, therefore, the elliptic analogue of our
multiplicative Hitchin system should be the space of maps from $C$
into $\bun^{\mr{ss}}_G(E_q)$: this moduli space is closely related to
the moduli space of instantons on $C \times E_q$, i.e. doubly periodic
instantons.  It would be interesting to study this elliptic Hitchin
system in the full, singular setting.
\end{remark}

% \item
\subsection{Multiplicative $q$-Geometric Langlands Correspondence}
The final perspective we'll consider is perhaps the most interesting from our point of view, because it suggests that the categorical geometric Langlands conjecture of Beilinson and Drinfeld might admit a multiplicative analogue, built from the multiplicative Hitchin system instead of the ordinary Hitchin system.  Again, we'll describe a moduli space coming from supersymmetric gauge theory, but in a quite different setting to that of perspective 4 above.

We consider now a five-dimensional $\mc N=2$ supersymmetric gauge theory.  In order to extract interesting moduli spaces for geometric representation theory, we won't study its moduli space of vacua, but instead we'll \emph{twist} the theory, then look at the entire moduli space of solutions to the equations of motion in the twisted theory.  We'll review the idea behind twisting at the beginning of Section \ref{twist_section}, but very informally, we'll choose a supersymmetry $Q_\HT$ that squares to zero, and study the derived space of $Q_\HT$-invariant solutions to the equations of motion.  In Section \ref{twist_section} we'll sketch the following, which also follows from forthcoming work of Butson \cite{Butson}.

\begin{claimnum}
The twist by $Q_\HT$ of $\mc N=2$ super Yang-Mills theory with gauge group $G$, on the 5-manifold $C \times S^1 \times \RR^2$ where $C$ is a Calabi-Yau curve,and  with monopole surface operators placed at the points $(z_1,0), \ldots, (z_k,0)$ in $C \times S^1$ with charges $\omega^\vee_{z_1}, \ldots, \omega^\vee_{z_k}$ respectively, has the following stack of solutions to the equations of motion:
\[\mr{EOM}(C \times S^1 \times \RR^2) = T^*[1]\mhiggs(C,D,\omega^\vee)\]
where $T^*[1]X$ denotes the ``1-shifted cotangent space'' of $X$.
\end{claimnum}

This story leads us to a ``multiplicative'' analogue of the approach to geometric Langlands introduced by Kapustin and Witten \cite{KapustinWitten}.  When we take the radius of $S^1$ to zero we recover Kapustin's partially topological twist \cite{KapustinHolo} of 4d $\mc N=4$ super Yang-Mills, which degenerates to Kapustin-Witten's A- and B-topologically twisted theories.  By ``multiplicative geometric Langlands'' we mean a version of the geometric Langlands conjecture modelled on the multiplicative Hitchin system instead of the ordinary Hitchin system, where the radius of this circle is kept positive.  For more details on what this means, see Section \ref{Langlands_section}.
%\end{enumerate}

\begin{remark}[Derived geometry]
  We'll use the language of derived algebraic geometry in several places in this paper, mainly in Sections \ref{mhiggs_def_section} and \ref{twist_section}.  While we believe this perspective provides a clear way of thinking about the moduli spaces we're interested in studying, the derived language is not necessary when we state and prove the main results of this paper.  We hope that the majority of the paper is understandable without a derived geometry background.  There are several clear and references explaining the point of view of derived algebraic geometry, for instance the survey articles of To\"en \cite{ToenOverview,ToenSurvey}.  A comprehensive account of the theory of derived geometry can be found in the book of Gaitsgory and Rozenblyum \cite{GRvol1, GRvol2}.  We'll also refer a few times to the theory of shifted symplectic structures.  This theory was developed by Pantev, To\"en, Vaqui\'e and Vezzosi \cite{PTVV}, and we also recommend the explanations in \cite{Calaque}.  From the physical point of view, the ideas of derived geometry appear in classical field theory through the Batalin-Vilkovisky formalism, and through Pantev, To\"en, Vaqui\'e and Vezzosi's derived version of the AKSZ construction of symplectic structures on mapping spaces \cite{AKSZ}. 
\end{remark}

\subsection{Outline of the Paper}
We begin in Section \ref{mhiggs_def_section} by introducing and defining the moduli spaces we'll be studying in this paper: the moduli spaces of multiplicative Higgs bundles.  We discuss the trichotomy of rational, trigonometric and elliptic examples that are most relevant to us: where the moduli space has the structure of an integrable system.  We include a discussion of how the symplectic structures we're studying in this paper are expected to arise from the theory of derived symplectic geometry.

The following section, Section \ref{twist_section}, stands alone from the rest of the paper.  In this section we explain how the moduli space of multiplicative Higgs bundles appears when one studies a certain partially topological twist of 5d $\mc N=2$ supersymmetric gauge theory.  As a consequence, we can speculate on the existence of a multiplicative version of the geometric Langlands conjecture, using the work of Kapustin and Witten.

In Section \ref{periodic_monopole_section} we discuss the moduli space of periodic monopoles and its identification with the moduli space of multiplicative Higgs bundles.  We describe the tangent space to the moduli space from both points of view, which we'll then use in the following section, Section \ref{symp_section}.  Here we prove our main result, identifying the symplectic structures on the moduli spaces of periodic monopoles and on multiplicative Higgs bundles.

Having done this, in Section \ref{hyperkahler_section} we can transfer the hyperk\"ahler structure on the moduli space of periodic monopoles across to the multiplicative Hitchin system.  The main thing we can do is to identify the twistor rotation on this moduli space: after rotating to a point $\eps$ in the twistor sphere we identify the holomorphic symplectic moduli space with the moduli space of $\eps$-difference connections, i.e. $G$-bundles $P$ with an isomorphism between $P$ and its translate by $\eps$.

In Section \ref{quantization_section} we discuss the quantization of our moduli spaces.  We identify the moduli spaces of multiplicative Higgs bundles with symplectic leaves in the rational Poisson Lie group, and then after quantization we identify the quantized algebra of functions on our moduli spaces with modules for the Yangian as first constructed by Gerasimov, Kharchev, Lebedev and Oblezin. 

Finally, in Section \ref{qchar_section} we discuss the relationship between the multiplicative analogue of the space of opers and the $q$-character, motivated by the origin of the multiplicative Hitchin system in quiver gauge theory.

The preliminary results of this work were announced by the second author at String-Math 2017 \cite{PestunStringMath}.

\subsection{Acknowledgements}
We would like to thank David Jordan, Davide Gaiotto, Dennis Gaitsgory, Nikita Nekrasov, Kevin Costello and especially Pavel Safronov for helpful conversations about this work. The calculation of twists of 5d and 6d supersymmetric gauge theories was performed independently by Dylan Butson, and we would like to thank him for sharing his forthcoming manuscript. CE would like to thank the Perimeter Institute for Theoretical Physics for supporting research on this project. Research at Perimeter Institute is supported by the Government of Canada through Industry Canada and by the Province of Ontario through the Ministry of Economic Development \& Innovation. We acknowledge the support of IH\'ES.  This project has received funding from the European Research Council (ERC) under the European Union's Horizon 2020 research and innovation programme (QUASIFT grant agreement 677368).



\section{Multiplicative Higgs Bundles and $q$-Connections} \label{mhiggs_def_section}
We'll begin with an abstract definition of moduli spaces of multiplicative Higgs bundles using the language of derived algebraic geometry.  We note, however, that once we specialize to our main, rational, example, the moduli spaces we'll end up studying are actually smooth algebraic varieties, not derived stacks.  However, the derived point of view gives us a useful and concise definition of these moduli spaces in full generality.

Let $G$ be a reductive complex algebraic group, let $C$ be a smooth complex algebraic curve and fix a finite set $D = \{z_i, \ldots, z_k\}$ of closed points in $C$.  We write $\bun_G(C)$ for the moduli stack of $G$-bundles on $C$, which we view as a mapping stack $\map(C, BG)$ into the classifying stack of $G$.

\begin{definition}
The moduli stack of \emph{multiplicative $G$-Higgs fields} on $C$ with singularities at $D$ is the fiber product
\[\mhiggs_G(C,D) = \bun_G(C) \times_{\bun_G(C \! \bs \! D)} \map(C \! \bs \! D, G/G)\]
where $G/G$ is the adjoint quotient stack.
\end{definition}

\begin{remark}
A closed point of $\mhiggs_G(C,D)$ consists of a principal $G$-bundle $P$ on $C$ along with an automorphism of the restriction $P|_{C \! \bs \! D}$, i.e. a section of $\Ad_P$ away from $D$.
\end{remark}

The adjoint quotient stack can also be described as the derived loop space $\map(S^1_B, BG)$ of the classifying stack, where $S^1_B$ is the ``Betti stack'' of $S^1$, i.e. the constant derived stack at the simplicial set $S^1$.  We can therefore view $\map(C \! \bs \! D, G/G)$ instead as the mapping stack $\map((C \! \bs \! D) \times S^1_B, BG)$, and the moduli stack of multiplicative Higgs bundles instead as
\[\mhiggs_G(C,D) = \map((C \times S^1_B) \bs (D \times \{0\}), BG).\]  
The source of this mapping stack can be $q$-deformed.  Indeed, let $q$ denote an automorphism of the curve $C$.  Write $C \times_q S^1_B$ for the \emph{mapping torus} of $q$, i.e the derived fiber product
\[C \times_q S^1_B = C \times_{C \times C} C\]
where the two maps $C \to C \times C$ are given by the diagonal and the $q$-twisted diagonal $x \mapsto (x,q(x))$ respectively.

\begin{definition}
The moduli stack of \emph{$q$-difference connections} for the group $G$ on $C$ with singularities at $D$ is the mapping space
\[\qconn_G(C,D) = \map((C \times_q S^1_B) \bs (D \times \{0\}), BG).\] 
In particular when $q=1$ this recovers the moduli stack of multiplicative Higgs bundles.
\end{definition}

\begin{remark}
  A closed point of $\qconn_G(C,D)$ consists of a principal $G$-bundle
  $P$ on $C$ along with a \emph{$q$ difference connection}: an
  isomorphism of $G$-bundles
  $P|_{C \! \bs \! D} \to q^*P|_{C \! \bs \! D}$ away from the divisor
  $D$.  For an introduction and review of the classical theory of $q$-difference
  connections we refer the reader to \cite{STSSevostyanov} and \cite{Sauloy}.
\end{remark}

\subsection{Local Conditions at the Singurities}
These moduli stacks are typically of infinite type.  In order to obtain finite type stacks, and later in order to define symplectic rather than only Poisson structures, we can fix the behaviour of our multiplicative Higgs fields and $q$-difference connections near the punctures $D \sub C$.

We'll write $\bb D$ to denote the \emph{formal disk} $\spec \CC[[z]]$.  Likewise we'll write $\bb D^\times$ for the \emph{formal punctured disk} $\spec \CC(\!(z)\!)$.  We'll then write $\bb B$ for the derived pushout $\bb D \sqcup_{\bb D^\times} \bb D$.  Let $LG = \map(\bb D^\times, G)$ and let $L^+G = \map(\bb D, G)$.

There is a canonical inclusion $\bb B^{\sqcup k} \to (C \times_q S^1_B) \!\!\bs\!\! (D \times \{0\})$, the inclusion of the formal punctured neighborhood of $D \times \{0\}$.  This induces a restriction map on mapping spaces
\[\res_D \colon \qconn_G(C, D) \to \bun_G(\bb B)^k.\]

One can identify $\bun_G(\bb B)$ with the double quotient stack $L^+G \!\bs\! LG / L^+G$, or equivalently with the quotient $L^+G \!\bs\! \Gr_G$ of the affine Grassmannian.  The following is well-known (see e.g. the expository article \cite{Zhu}).

\begin{lemma}
The set of closed points of $\bun_G(\bb B)$ is in canonical bijection with the set of dominant coweights of $G$.
\end{lemma}

\begin{definition}
Choose a map from $D$ to the set of dominant coweights and denote it by $\omega^\vee \colon z_i \mapsto \omega^\vee_{z_i}$.  Write $\Lambda_i$ for the isotropy group of the point $\omega^\vee_{z_i}$ in $\bun_G(\bb B)$. The moduli stack of $q$ difference connections on $C$ with singularities at $D$ and fixed local data given by $\omega^\vee$ is defined to be the fiber product
\[\qconn_G(C,D, \omega^\vee) = \qconn_G(C,D) \times_{\bun_G(\bb B)^k} (B\Lambda_1 \times \cdots \times B\Lambda_k).\]
\end{definition}

\begin{remark}\label{ind_structure_remark}
  In a similar way we can define a filtration on the moduli stack of
  $q$-connections.  Recall that the affine Grassmannian $\Gr_G$ is
  stratified by dominant coweights $\omega^\vee$, and there is an
  inclusion $\Gr_G^{\omega^\vee_1}\sub \ol{\Gr_G^{\omega^\vee_2}}$ of
  one stratum into the closure of another stratum if and only if
  $\omega^\vee_1 \preceq \omega^\vee_2$ with respect to the standard
  partial order on dominant coweights (again, this is explained in \cite{Zhu}).  We can then define
\[\qconn_G(C,D, \preceq \omega^\vee) = \qconn_G(C,D) \times_{\bun_G(\bb B)^k} \left(L^+G \!\!\bs\!\! \ol{\Gr_G^{\omega^\vee_1}} \times \cdots \times L^+G \!\!\bs\!\! \ol{\Gr_G^{\omega^\vee_k}}\right).\]
The full moduli stack $\qconn_G(C,D)$ is the filtered colimit of these moduli spaces.  One can additionally take the filtered colimit over all finite subsets $D$ in order to define a moduli stack $\qconn^\mr{sing}_G(C)$ of $q$-connections on $C$ with arbitrary singularities.
\end{remark}

\begin{examples}
The most important examples for our purposes are given by the following rational/trigonometric/elliptic trichotomy.
 \begin{itemize}
  \item \textbf{Rational:} We can enhance the definition of our moduli space by including a framing at a point $c \in C$ not contained in $D$.  We always assume that such framed points are fixed by the automorphism $q$.
    \begin{definition}
      \label{def:framing}
    The moduli space of $q$-difference connections on $C$ with a framing at $c$ is defined to be the relative mapping space 
    \[\qconn_G^\fr(C) = \map(C \times_q S^1_B, BG; f)\]
    where $f \colon \{c\} \times S^1_B \to BG$ (or equivalently $f \colon \{c\} \to G/G$) is a choice of adjoint orbit.  We can define the framed mapping space with singularities and fixed local data in exactly the same way as above.  
  \end{definition}
    
  In this paper we'll be most interested in the following example.  Let $C = \bb{CP}^1$ with framing point $c = \infty$ and framing given by a fixed element in $G/G$, and consider automorphisms of the form $z \mapsto z + \eps$ for $\eps \in \CC$.  Choose a finite subset $D \sub \bb A^1$ and label the points $z_i \in D$ by dominant coweights $\omega^\vee_{z_i}$.  We can then study the moduli space $\epsconn^\fr_G(\bb{CP}^1,D, \omega^\vee)$.  The main object of study in this paper will be the holomorphic symplectic structure on this moduli space. Note that the motivation for this definition comes in part from
    Spaide's formalism \cite{Spaide} of AKSZ-type derived symplectic structures (in the sense of \cite{AKSZ,PTVV}) on
    relative mapping spaces -- in this formalism $\bb{CP}^1$ with a
    single framing point is relatively 1-oriented, so mapping spaces
    out of it with 1-shifted symplectic targets have AKSZ 0-shifted
    symplectic structures.
  
  \item \textbf{Trigonometric:} Alternatively, we can enhance our definition by including a reduction of structure group at a point $c \in C$ not contained in $D$, again fixed by the automorphism $q$.
  \begin{definition}
   The moduli space of $q$-difference connections on $C$ with an $H$-reduction at $c$ for a subgroup $H \sub G$ is defined to be the fiber product
   \[\qconn_G^{H,c}(C) = \map(C \times_q S^1_B,BG) \times_{G/G} H/H\]
   associated to the evaluation at $c$ map $\map(C \times_q S^1_B,BG) \to G/G$.  We can define the moduli space with $H$-reduction with singularities and fixed local data in the same way as above.
  \end{definition}
  
  Again let $C = \bb{CP}^1$.  Fix a pair of opposite Borel subgroups $B_+$ and $B_- \sub G$ with unipotent radicals $N_\pm$ and consider the moduli space of $q$-connections with $B_+$-reduction at $0$ and $N_-$-reduction at $\infty$.  We'll now take $q$ to be an automorphism of the form $z \mapsto qz$ for $q \in \CC^\times$.  We'll defer in depth analysis of this example to future work.
  
  \item \textbf{Elliptic:} Finally, let $C = E$ be a smooth curve of genus one.  In this case we won't fix any additional boundary data, but just consider the moduli space $\qconn_G(E,D, \omega^\vee)$.  In the case $q = 1$ this space -- or rather its polystable locus -- was studied by Hurtubise and Markman \cite{HurtubiseMarkman}, who proved that it can be given the structure of an algebraic integrable system with symplectic structure related to the elliptic R-matrix of Etingof and Varchenko \cite{EtingofVarchenko}.
 \end{itemize}
\end{examples}

\begin{remark}
In the rational case, the moduli space of framed $q$-difference connections now depends on a new parameter: the value of the framing $g_\infty \in G/G$.  From the point of view of the ADE quiver gauge theory, as in Section \ref{intro_gauge_section}, this value -- or rather its image in $H/W$ -- corresponds to the value of the gauge coupling constants in the quiver gauge theory.
\end{remark}

\begin{remark} \label{Elliptic_AKSZ_remark}
In the elliptic case it's natural to ask to what extent Hurtubise and Markman's integrable system structure can be extended from the variety of polystable multiplicative Higgs bundles to the full moduli stack.  If $D$ is empty then it's easy to see that we have a symplectic structure given by the AKSZ construction of Pantev-To\"en-Vaqui\'e-Vezzosi \cite{PTVV}.  Indeed, $E$ is compact 1-oriented and the quotient stack $G/G$ is 1-shifted symplectic, so the mapping stack $\map(E, G/G) = \mhiggs_G(E)$ is equipped with a 0-shifted symplectic structure by \cite[Theorem 2.5]{PTVV}.  The role of the Hitchin fibration is played by the Chevalley map $\chi \colon G/G \to T/W$, and therefore
\[\map(E,G/G) \to \map(E,T/W).\]
The fibers of this map over regular points in $T/W$ are given by moduli stacks of the form $\bun_T(\wt E)^W$ where $\wt E$ is a $W$-fold cover of $E$ (the cameral cover).  Note that in this unramified case the curve $\wt E$ also has genus 1; counting dimensions we see that the base has dimension $r = \mr{rk}(G)$ and the generic fibers are $r$-dimensional (Lagrangian) tori.
\end{remark}

\begin{remark}
While the moduli space of multiplicative Higgs bundles makes sense on a general curve it's only after restricting attention to this trichotomy of examples that we'll expect the existence of a Poisson structure.  In the non-singular case, such a structure arises by the AKSZ construction, i.e. by transgressing the 1-shifted symplectic structure on $G/G$ to the mapping space using a fixed section of the canonical bundle on $C$ (possibly with a boundary condition).  
\end{remark}

\subsection{The Multiplicative Hitchin System} \label{Hitchin_system_section}
We can define the global Chevalley map as in Remark \ref{Elliptic_AKSZ_remark} in the case of non-empty $D$ as well.  We'll show that in the rational case this defines a completely integrable system structure.  Let $T \sub G$ be a maximal torus, and write $W$ for the Weyl group of $G$.

\begin{definition} \label{mult_Hitchin_system_def}
Fix a curve $C$, a divisor $D$ and a dominant coweight $\omega_{z_i}^\vee$ at each point $z_i$ in $D$.  The \emph{multiplicative Hitchin base} is the stack
\[\mc B(C,D,\omega^\vee) = \mr{Sect}(C, X(D,\omega^\vee)/W)\]
of sections of $X(D,\omega^\vee)/W$, the $T/W$-bundle on $C$ where $X(D,\omega^\vee)$ is the $T$-bundle characterized by the condition that the associated line bundle $X(D,\omega^\vee) \times_T {\lambda}$ corresponding to a weight $\lambda$ is given by $\OO(\sum \omega_{z_i}^\vee(\lambda) \cdot z_i)$ (c.f. \cite[Section 3.3]{HurtubiseMarkman}).

The \emph{multiplicative Hitchin fibration} is the map
\[\pi \colon \mhiggs_G(C,D,\omega^\vee) \to \mc B(C,D,\omega^\vee)\]
given by post-composing a map $C \bs D \to G/G$ with the Chevalley map $\chi \colon G/G \to T/W$.  
\end{definition}

\begin{prop}
The multiplicative Hitchin fibration described above is well-defined.
\end{prop}

\begin{proof}
We need to verify that the image of a point in $\mhiggs_G(C,D,\omega^\vee)$ under $\pi$, viewed as a section of the trivial $T/W$-bundle on $C \bs D$, extends to a section of $X(D,\omega^\vee)/W$.  We look locally near a singularity $z_i \in D$.  Let $\phi \in G((z_i))$ be a local representative for a multiplicative Higgs field in $\mhiggs_G(C,D,\omega^\vee)$, i.e. an element of the associated $G[[z_i]]$-adjoint orbit.  Since $\phi$ is equivalent to $z_i^{-\omega^\vee_{z_i}}$ under the action of $G[[z_i]]^2$ by left and right multiplication, without loss of generality we can say that $\phi = z_i^{-\omega^\vee_{z_i}} \phi_0$ for some $\phi_0 \in G[[z_i]]$.  Consider $\chi(\phi) \in T((z_i))$ -- the singular part of this element is the same as $\chi(z_i^{-\omega^\vee_{z_i}} g)$ for some $g \in G$, which implies the section extends to a meromorphic $T/W$-valued function of the required type.
\end{proof}

\begin{remark}
Note that the base stack that we're defining here is not exactly the same as the base of the integrable system defined in \cite{HurtubiseMarkman}.  The difference comes from the way in which we treat the Weyl group quotient; the Hurtubise-Markman base is an algebraic variety defined essentially by taking the part of our space of sections where $W$ acts freely as an open subset, then constructing a compactification using techniques from toric geometry.
\end{remark}

We'll show in Section \ref{general_symplectic_sec} that this fibration defines a completely integrable system in the rational case where $C = \bb{CP}^1$ with a framing at $\infty$.  More specifically, we'll describe a non-degenerate pairing on the tangent space of the multiplicative Higgs moduli space, and in Proposition \ref{Hitchin_isotropic_prop} we'll show that the multiplicative Hitchin fibers are isotropic for this pairing.  Then, by comparison to the symplectic structure on periodic monopoles, we'll see that the pairing was symplectic.

Here, we'll first observe that the generic fibers are half-dimensional tori, which will ultimately imply that the multiplicative Hitchin system is generically a Lagrangian fibration.  Computing the fibers of the Hitchin fibration works similarly to the non-singular case: a point in the base is, in particular, a map $C \bs D \to T/W$.  Suppose this map lands in the regular locus $T^{\mr{reg}}/W$, then an element of the fiber over this point defines, in particular, a map $C \bs D \to BT/W$.  We would like to argue that the fiber consists of $T$-bundles on the cameral cover $\wt C$: a $W$-fold cover of $C$ ramified at the divisor $D$.

In order to say this a bit more precisely this we'll compare our moduli space with the space of abstract Higgs bundles introduced by Donagi and Gaitsgory \cite{DonagiGaitsgory} (see also \cite{DonagiLectures}, where Donagi proposed the applicability of this abstract Higgs theory to the multiplicative situation and asked for a geometric interpretation).  Our argument will follow the same ideas as the arguments of \cite[Section 6]{HurtubiseMarkman}.

\begin{definition}
An \emph{abstract $G$-Higgs bundle} on a curve $C$ is a principal $G$-bundle $P$ along with a sub-bundle $\mf c$ of $\gg_P$ of \emph{regular centralizers}, meaning that the fibers are subalgebras of $\gg$ which arise as the centralizer of a regular element of $\gg$.  Write $\higgs_G^{\mr{abs}}(C)$ for the moduli stack of abstract $G$-Higgs bundles on $C$.
\end{definition}

There's an algebraic map from the regular part of our moduli space $\mhiggs_G(C,D,\omega^\vee)_{\mr{reg}}$ (where the Higgs field is required to take regular values) into $\higgs_G^{\mr{abs}}(C)$ that sends a multiplicative Higgs bundle $(P,g)$ to the abstract Higgs bundle $(P, \mf c_g)$, where $\mf c_g$ is the sub-bundle of $\gg_P$ fixed by the adjoint action of the multiplicative Higgs field $g$.  What's more, there is a commutative square relating the Hitchin fibration for the multiplicative moduli space with a related projection for the abstract moduli stack:
\[\xymatrix{
\mhiggs_G(C,D,\omega^\vee)_{\mr{reg}} \ar[r] \ar[d] &\higgs_G^{\mr{abs}}(C) \ar[d] \\
\mc B(C,D,\omega^\vee)_{\mr{reg}} \ar[r] &\mr{Cam}_G(C),
}\]
where $\mr{Cam}_G(C)$ is the stack of cameral covers of $C$, as defined in \cite[Section 2.8]{DonagiGaitsgory}.  The map $\mc B(C,D,\omega^\vee)_{\mr{reg}} \to \mr{Cam}_G(C)$ is defined by sending a meromorphic function $f \colon C \to T/W$ to the $D$-ramified cameral cover $\wt C = C \times_{T/W} T$.  In particular there's a map from the multiplicative Hitchin fiber to the corresponding Donagi-Gaitsgory fiber: the moduli space of abstract $G$-Higgs bundles with fixed cameral cover.  This map is surjective: having fixed the cameral cover, and therefore the ramification data, every sub-bundle $\mf c \sub \gg_P$ of regular centralizers arises as the centralizer of some regular multiplicative Higgs field.  Likewise once one restricts to a single generic multiplicative Hitchin fiber the map is an unramified $W$-fold cover.

To conclude this discussion we'll discuss dimensions and the geometry of the multiplicative Hitchin fibers.  Firstly, we can compute the dimension of a regular multiplicative Hitchin fiber by computing the dimension of the base and the dimension of the total space.  The dimension of the base is given by computing the number of linearly independent sections of the $T$-bundle $X(D,\omega^\vee)$ on $\bb{CP}^1$ vanishing at $\infty$.  This is given by 
\[\dim \mc B(C,D,\omega^\vee) = \sum_{z_i \in D} \langle \rho, \omega^\vee_{z_i} \rangle,\]
where $\rho$ is the Weyl vector.  On the other hand the dimension of the total space is calculated in Corollary \ref{dim_of_moduli_space_cor} to be $2 \sum_{z_i \in D} \langle \rho, \omega^\vee_{z_i} \rangle$.  The base is indeed half-dimensional, therefore so is the fiber.

The Donagi-Gaitsgory fiber is, according to the main theorem of \cite{DonagiGaitsgory}, equivalent to the moduli space of $W$-equivariant $T$-bundles on the cameral curve $\wt C$ up to a discrete correction involving the root datum of $G$.  In particular it is generically an abelian variety.  The multiplicative Hitchin fiber is isogenous to this abelian variety, since the map from the multiplicative Hitchin fiber to the Donagi-Gaitsgory fiber is surjective and \'etale.

\begin{remark} \label{q_opers_remark}
Like in the case of the ordinary Hitchin system, in good examples the multiplicative Hitchin system admits a canonical Hitchin section.  One can construct this section using the Steinberg section (the multiplicative analogue of the Kostant section) \cite{Steinberg}.  This is a section of the map $G/G \to T/W$, canonical after choosing a Borel subgroup $B$ with maximal torus $T$ and a basis vector for each simple root space, and well defined as long as $G$ is simply connected \footnote{By a theorem of Popov \cite{Popov} this condition is necessary for semisimple $G$.  A section also exists for $G = \GL_n$, but we aren't aware of a necessary and sufficient condition for general reductive groups.}.  The \emph{multiplicative Hitchin section} is the map $\sigma \colon \mc B(C,D,\omega^\vee) \to \mhiggs_G(C,D,\omega^\vee)$ defined by post-composition with the Steinberg section.  One can use this section to define the moduli space of \emph{$q$-opers} for the group $G$ and the curve $\bb{CP}^1$ with its framing at infinity.  We'll discuss the hyperk\"ahler structure on the moduli space of multiplicative Higgs bundles in Section \ref{hyperkahler_section}.  In particular we'll show that when one rotates to $q$ in the twistor sphere one obtains the moduli space of $q$-connections on $\bb{CP}^1$.  The moduli space of $q$-opers is defined to be the Hitchin section, but viewed as a subspace of $\qconn^{\fr}_G(\bb{CP}^1,D,\omega^\vee)$.  For a more detailed discussion of the multiplicative Hitchin section and $q$-opers see Section \ref{qchar_section}. 
\end{remark}

\subsection{Stability Conditions}
For comparison to results in the literature it is important that we briefly discuss the role of stability conditions for difference connections.  In our main example of interest -- the rational case -- these conditions won't play a role, but they do appear in the comparison results between $q$-connections and monopoles in the literature for more general curves.  For definitions for general $G$ we refer to \cite{Smith}, although see also \cite{AnchoucheBiswas} on polystable $G$-bundles.  In what follows we fix a choice of $0 < t_0 < 2\pi R$.

\begin{definition}
Let $(P,g)$ be a $q$-connection on a curve $C$, and let $\chi$ be a character of $G$.  The \emph{$\chi$-degree} of $(P,g)$ is defined to be 
\[\deg_\chi(P,g) = \deg(P \times_\chi \CC) - \frac {t_0}{2\pi R} \sum_{i=1}^k \deg(\chi \circ \omega^\vee_{z_i}).\]

A $q$-connection $(P,g)$ on $C$ is \emph{stable} if for every maximal parabolic subgroup $H \sub G$ with Levi decomposition $H = LN$ and every reduction of structure group $(P_H, g)$ to $H$, we have
\[\deg_\chi(P_H, g) < 0\]
for the character $\chi = \det(\mr{Ad}_L^{\mf n})$ defined to be the determinant of the adjoint representation of $L$ on $\mf n$.

The $q$-connection $(P,g)$ is \emph{polystable} if there exists a (not necessarily maximal) parabolic subgroup $H$ with Levi factor $L$ and a reduction of structure group $(P_L, g)$ to $L$ so that $(P_L,g)$ is a stable $q$-connection and so that the associated $H$-bundle is admissible, meaning that for every character $\chi$ of $H$ which is trivial on $Z(G)$ the associated line bundle $P_H \times_\chi \CC$ has degree zero. 
\end{definition}

Below we'll write $\qconn_G^{\text{ps}}(C, D, \omega^\vee)$ for the moduli space of polystable $q$-connections.  This moduli space is a smooth algebraic variety of finite type.  If $q$ is the identity this is a theorem of Charbonneau, Hurtubise and Smith.  For more general $q$, as a smooth manifold these moduli spaces are actually independent when one varies $q$ -- we'll see this when we describe an equivalent description of these moduli spaces as spaces of periodic monopoles on diffeomorphic manifolds.  

When $C = \bb{CP}^1$ every principal $G$-bundle on $C$ admits an essentially unique (up to the action of the Weyl group) holomorphic reduction of structure group to a maximal torus \cite{GrothendieckSphere}.  Since $q$-connections for an abelian group are automatically stable, polystability on the sphere is equivalent to admissibility of the torus reduction.  As a consequence, for our main example of interest -- the rational case -- the moduli space of polystable $q$-connections is equivalent to the moduli space of all $q$-connections of admissible degree.  For instance for $G=\SL_n$ the moduli space of polystable $q$-connection is equivalent to the moduli space of $q$-connections on the trivial bundle.

\subsection{Poisson Structures from Derived Geometry}
As we mentioned above in Remark \ref{Elliptic_AKSZ_remark}, in the case where $C$ is an elliptic curve and there are no punctures there is a symplectic structure on $\mhiggs_G(C)$ given by the AKSZ formalism.  More generally, when we do have punctures, we expect the moduli space $\mhiggs_G(C,D)$ to have a Poisson structure with a clear origin story coming from the theory of derived Poisson geometry.  In this section we'll explain what this story looks like.  However, we emphasise that there are technical obstructions to making this story precise with current technology: this section should be viewed as motivation for the structures we'll discuss in the rest of the paper.  On the other hand, readers who aren't familiar with derived symplectic geometry can freely skip this section.  We refer the reader to \cite{CPTVV} for the theory of derived Poisson structures and to \cite{MelaniSafronov1, MelaniSafronov2, Spaide} for that of derived coisotropic structures.  We would like to thank Pavel Safronov for explaining many of the ideas discussed in this section to us.

Here's the idea.  Recall that we can identify the moduli space of singular $q$-connections on a curve $C$ as a fiber product: $\qconn_G(C, D) \iso \bun_G^\fr(C) \times_{\bun_G(C \! \bs \! D)^2} \bun_G(C \! \bs \! D)$ where the map $g \colon \bun_G^\fr(C) \to \bun_G(C \! \bs \! D)$ is given by $P \mapsto (P|_{C \! \bs \! D}, q^*P|_{C \! \bs \! D})$.  Consider the following commutative cube:

\[\xymatrix@C-30pt@R-8pt{
& \qconn_G(C, D) \ar[rr]^{f_1} \ar'[d][dd]^(.25){\mr{res}} & & \bun_G(C) \ar[dd]
\\
\bun_G(C \bs D) \ar@{<-}[ur]^{f_2} \ar[rr]^(.6){g_2} \ar[dd] & & \bun_G(C \bs D)^2 \ar@{<-}[ur]^{g_1} \ar[dd]^(.4)r
\\
& \bun_G(\bb B)^k \ar'[r][rr] & & BL^+G^{2k}
\\
BLG^k \ar[rr]\ar@{<-}[ur] & & BLG^{2k}. \ar@{<-}[ur]
}\]
Here the top and bottom faces are homotopy Cartesian squares.  What does this setup buy us?  We'll first answer informally.

\begin{claim}
First consider the bottom face of the cube.  The stack $BLG$ is 2-shifted symplectic because the Lie algebra $L\gg$ has a non-degenerate invariant pairing: the residue pairing.  The Lie subalgebra $L^+\gg$ forms part of a Manin triple $(L\gg, L^+\gg, L^-+0\gg)$ which means that $BL^+G \to BLG$ is 2-shifted Lagrangian.  Therefore the bottom face of the cube defines a 2-shifted Lagrangian intersection, which means that the pullback $\bun_G(\bb B)^k$ is 1-shifted symplectic.

Now consider the top face of the cube.  If either $C$ is an elliptic curve, or $C=\bb{CP}^1$ and we fix a framing at $\infty$, then the map $\bun_G(C \bs D) \to BLG^k$ is also 2-shifted Lagrangian.  In particular $\bun_G(C \bs D)$ is 1-shifted Poisson.  Finally, the map $\bun_G(C) \to \bun_G(C \bs D)$ is 1-shifted coisotropic, or equivalently the canonical map $\bun_G(C) \to \bun_G(C \bs D) \times_{BLG^k} BL^+G^k$ is 1-shifted Lagrangian.  That means that the top face of the cube defines a 1-shifted coisotropic intersection, which means that the pullback $\qconn_G(C,D)$ is 0-shifted Poisson.

The restriction map $\qconn_G(C,D) \to \bun_G(\bb B)^k$ is 1-shifted Lagrangian, which means that if we form the intersection with a $k$-tuple of Lagrangians in $\bun_G(\bb B)$ then we obtain a 0-shifted symplectic stack.  For example, if $\omega_i^\vee$ is a point in $\bun_G(\bb B)$ corresponding to a dominant coweight with stabilizer $\Lambda_i$ then $B \Lambda_i \to \bun_G(\bb B)$ is 1-shifted Lagrangian, so the moduli stack $\qconn_G(C,D, \omega^\vee)$ obtained by taking the derived intersection is ind 0-shifted symplectic.
\end{claim}

Now, let us make this claim more precise.  The main technical condition that makes this claim subtle comes from the fact that most of the derived stacks appearing in this cube, for instance the stack $BLG$, are not Artin.  As such we need to be careful when we try to, for instance, talk about the tangent complex to such stacks.  One can make careful statements using the formalism of ``Tate stacks'' developed by Hennion \cite{Hennion}.  We can therefore make our claim into a more formal conjecture.

\begin{conjecture}
Suppose $C$ is either an elliptic curve or $\bb{CP}^1$ with a fixed framing at $\infty$.
\begin{enumerate}
\item The stack $BLG$ is Tate 2-shifted symplectic, and both $BL^+G \to BLG$ and $\bun_G(C \bs D) \to BLG^k$ are Tate 2-shifted Lagrangian.  
\item The stack $\bun_G(C \bs D)$ is ind 1-shifted Poisson, and the map $\bun_G(C) \to \bun_G(C \bs D)$ is ind 1-shifted coisotropic witnessed by the 2-shifted Lagrangian map $BL^+G^k \to BLG^k$.
\item The Lagrangian intersection $\bun_G(\bb B)$ is Tate 1-shifted symplectic, and the map $B\Lambda_i \to \bun_G(\bb B)$ associated to the inclusion of the stabilizer of a closed point is 1-shifted Lagrangian.
\end{enumerate}
As a consequence, the moduli stack $\qconn_G(C,D)$ is ind 0-shifted Poisson and the moduli stack $\qconn_G(C,D, \omega^\vee)$ is 0-shifted symplectic.
\end{conjecture}

\begin{remark}
We should explain heuristically why the Calabi-Yau condition on $C$ is necessary.  This is a consequence of the AKSZ formalism in the case where $D$ is empty: for the mapping stack $\map(C, G/G)$ to be 0-shifted symplectic, or for the mapping stack $\map(C,BG)$ to be 1-shifted symplectic, we need $C$ to be compact and 1-oriented.  A $d$-orientation on a smooth complex variety of dimension $d$ is exactly the same as a Calabi-Yau structure.

More generally we can say the following.  Let us consider the rational case where $C = \bb{CP}^1$.  Consider the inclusion $\d r \colon \gg_- = \bb T_{\bun_G^\fr(\bb{CP}^1 \! \bs \! D)}[-1] \to r^*\bb T_{BLG^k}[-1] = \gg(\!(z)\!)^k$: a map of ind-pro Lie algebras concentrated in degree zero.  The residue pairing vanishes after pulling back along $r$ since elements of $\gg_-$ are  $\gg$-valued functions on $\bb{CP}^1$ with at least a simple pole at every puncture in $D$.  So the map $r$ is isotropic with zero isotropic structure; this structure is unique for degree reasons.  We must check that this structure is non-degenerate.  It suffices to check that the sequence
\[\bb T_{\bun_G^\fr(\bb{CP}^1 \! \bs \! D)}[-1] \to r^*\bb T_{BLG^k}[-1] \to (\bb T_{\bun_G^\fr(\bb{CP}^1 \! \bs \! D)}[-1])^\vee\]
is an exact sequence of ind-pro vector spaces, and therefore an exact sequence of quasi-coherent sheaves on the stack $\bun_G^\fr(\bb{CP}^1 \! \bs \! D)$.  To do this we identify the pair $(\gg_-, \gg(\!(z)\!)^k)$ as part of a Manin triple, where a complementary isotropic subalgebra to $\gg_-$ is given by $\gg_+ = \gg[[z]]^k$.  Using the residue pairing we can identify $\gg_+$ with $(\gg_-)^\vee$ and therefore identify our sequence with the split exact sequence
\[0 \to \gg_- \to \gg(\!(z)\!)^k \to \gg_+ \to 0.\]
\end{remark}

\begin{remark}
We will conclude this section with some comments on the multiplicative Hitchin system described above in Section \ref{Hitchin_system_section}.  In particular, the derived point of view suggests that both the multiplicative Hitchin fibers and the multiplicative Hitchin section of Remark \ref{q_opers_remark} will be -- at least generically -- canonically Lagrangian.  We can motivate this directly from the Chevalley map $\chi \colon G/G \to T/W$, by studying the non-singular example.  Generically, i.e. after restricting to the regular semisimple locus, we can identify $G^{\mr{rss}}/G$ with $(T^{\mr{reg}} \times BT)/W$, so that the fibers of the tangent space to $G/G$ are generically equivalent to $\mf t[1] \oplus \mf t$.  The directions tangent to the generic fibers of $\chi$ are concentrated in degree $-1$, meaning that the generic fibers of $\chi$ are canonically 1-shifted Lagrangian for degree reasons.  Likewise, the directions tangent to the Steinberg section $\sigma \colon T/W \to G/G$, at regular semisimple points, are concentrated in degree 0, meaning that after restriction to the regular semisimple locus is also canonically 1-shifted Lagrangian for degree reasons.  Now, if $L \to X$ is $n$-shifted Lagrangian and $M$ is $k$-oriented then there is an AKSZ $n-k$-shifted Lagrangian structure on the mapping space $\map(M,L) \to \map(M,X)$ \cite[Theorem 2.10]{Calaque}, which establishes our claim in the non-singular case. 

We should compare this discussion to Proposition \ref{Hitchin_isotropic_prop} in the rational case where $C = \bb{CP}^1$, where we prove that, in the rational case, the generic multiplicative Hitchin fibers are, indeed, Lagrangian.  
\end{remark}

\begin{remark} \label{Bouthier_remark}
There's yet another perspective that one might hope to pursue in order to define our symplectic structures in the language of derived symplectic geometry.  Bouthier \cite{Bouthier2} gave a description of the multiplicative Hitchin system along the following lines.  The Vinberg semigroup $V_G$ of $G$ is a family of affine schemes over $\CC^r$ whose generic fiber is isomorphic to $G$ but whose fibers lying on coordinate hyperplanes correspond to various degenerations of $G$.  Bouthier showed that the moduli space of multiplicative Higgs bundles is equivalent to the moduli space of $G$-bundles, along with a section of an associated bundle $V_G^{\omega^\vee}$ built from the Vinberg semigroup and the data of the coloured divisor $(D,\omega^\vee)$.  It's reasonable to ask whether one can construct an AKSZ shifted symplectic structure on the moduli space from this point of view, using the oriented structure on $(\bb{CP}^1,\infty)$ and the results of Ginzburg and Rozenblyum on shifted symplectic structures on spaces of sections \cite{GinzburgRozenblyum}.  
\end{remark}

\section{Twisted Gauge Theory} \label{twist_section}
Before we move on to the main mathematical content of the paper, we'll digress a little to talk about one situation where multiplicative Hitchin systems appear in gauge theory.  This story was our original motivation for studying the objects appearing in this paper, but we should emphasise that it is quite independent from the rest of the paper.  The reader who is only interested in algebraic and symplectic geometry, and not in gauge theory, can safely skip this section.

We'll describe our multiplicative Hitchin systems as the moduli spaces of solutions to the classical equations of motion in certain twisted five-dimensional supersymmetric gauge theories.  This story is distinct from the appearance of the moduli space in \cite{NekrasovPestun} as the Seiberg-Witten integral system of a 4d $\mc N=2$ theory.  Instead the moduli space will appear as the moduli space of solutions to the equations of motion in a twisted 5d $\mc N=2$ supersymmetric gauge theory, compactified on a circle.  This story will be directly analogous to the occurence of the ordinary moduli stack of Higgs bundles in a holomorphic twist of 4d $\mc N=4$ theory (see \cite{CostelloSH,ElliottYoo1} for a discussion of this story); we'll recover that example in the limit where the radius of the circle shrinks to zero.

\subsection{Background on Supersymmetry and Twisting}

We should begin by briefly recalling the idea behind twisting for supersymmetric field theories.  This idea goes back to Witten \cite{WittenTQFT}.  Supersymmetric field theories have odd symmetries coming from odd elements of the supersymmetry algebra.  Choose such an odd element $Q$ with the property that $[Q,Q]=0$.  Then $Q$ defines an odd endomorphism $\nu(Q)$ of the algebra of observables of the supersymmetric field theory with the property that $\nu(Q)^2 = 0$.  The \emph{twisted algebra of observables} associated to $Q$ is the cohomology of the operator $\nu(Q)$.  If $Q$ is chosen appropriately -- if the stress-energy tensor of the theory is $\nu(Q)$-exact -- then the $Q$-twisted field theory becomes topological.

\begin{remark}
From a modern perspective, using the language of factorization algebras, the first author and P. Safronov discussed the formalism behind topological twisting in \cite{ElliottSafronov}, and gave criteria for twisted quantum field theories to genuinely be topological.  The supersymmetry algebras and their loci of square zero elements are discussed in all cases in dimensions up to 10.  This classification is also performed in a paper of Eager, Saberi and Walcher \cite{EagerSaberiWalcher}.
\end{remark}

With the basic idea in hand, we'll focus in on the example we're interested in.  We'll be interested in twists by square-zero supercharges $Q$ that are not fully topological.  We'll begin by describing the supercharges we'll be interested in in dimensions 5 and 6.

Recall that there is an exceptional isomorphism identifying the groups $\Spin(5)$ and $\mr{USp}(4)$.  The Dirac spinor representation $S$ of $\Spin(5)$ is four dimensional: under the above exceptional isomorphism it is identified with the defining representation of $\mr{USp}(4)$.  Likewise there is an exceptional isomorphism identifying the groups $\Spin(6)$ and $\SU(4)$.  The two Weyl spinor representations $S_\pm$ are four dimensional: under the exceptional isomorphism they are identified with the defining representation of $\SU(4)$ and its dual.

\begin{definition}
The complexified $\mc N=k$ supersymmetry algebra in dimension 5 is the super Lie algebra
\[\mf A^5_k = (\sp(4;\CC) \oplus \sp(2k;\CC)_R \oplus V) \oplus \Pi(S \otimes W),\]
where $V$ is the five-dimensional defining representation of $\so(5;\CC) \iso \sp(4;\CC)$, $W$ is the $2k$-dimensional defining representation of $\sp(2k;\CC)_R$, and where there's a unique non-trivial way of defining an additional bracket $\Gamma \colon \sym^2(S \otimes W) \to V$.

Likewise, the complexified $\mc N=(k_+,k_1)$ supersymmetry algebra in dimension 6 is the super Lie algebra
\[\mf A^6_{(k_+,k_-)} = (\sl(4;\CC) \oplus \sp(2k_+;\CC)_{R} \oplus \sp(2k_-;\CC)_R \oplus V) \oplus \Pi(S_+ \otimes W_+ \oplus S_- \otimes W_-),\]
where $V$ is the six-dimensional defining representation of $\so(6;\CC) \iso \sl(4;\CC)$, $W_\pm$ is the $2k_\pm$-dimensional defining representation of $\sp(2k_\pm;\CC)_R$, and where there's a unique non-trivial way of defining an additional bracket $\Gamma_\pm \colon \sym^2(S_\pm \otimes W_\pm) \to V$.  Choosing a hyperplane in $V$ defines a super Lie algebra map $\mf A^5_{k_+ + k_-} \to \mf A^6_{(k_+,k_-)}$ which is an isomorphism on the odd summands.
\end{definition}

Let us fix some notation for the square-zero supercharges -- odd elements $Q$ where $\Gamma(Q,Q)=0$ -- that we will refer to in the discussion below.  Fix once and for all an embedding $\CC^5 \inj \CC^6$ and a symplectic embedding $W_+ \inj W$ (the twisted theories that we'll define don't depend on these choices).  Compare to the discussion in \cite[4.2.5--4.2.6]{ElliottSafronov}.
\begin{itemize}
 \item Let $Q_{\mr{min}} \in \mf A^5_1$ be any non-zero element such that $\Gamma(Q_{\mr{min}},Q_{\mr{min}})=0$ -- here ``min'' stands for ``minimal''.  All such elements have rank one, and lie in a single orbit for the action of $\so(5;\CC) \oplus \sp(2;\CC)$.  The image $\Gamma(Q_{\mr{min}}, -) \sub \CC^5$ is 3-dimensional.  Using the given embeddings we can view $Q_{\mr{min}}$ equally as an element of the larger supersymmetry algebras $\mf A^5_2, \mf A^6_{(1,0)}$ and $\mf A^6_{(1,1)}$.
 \item Let $Q_{\mr{HT}} = Q_{\mr{min}} + Q' \in \mf A^5_2$ be a non-zero element that squares to zero, and where the image $\Gamma(Q_{\mr{min}}, -) \sub \CC^5$ is 4-dimensional -- here ``HT'' stands for ``holomorphic-topological''.  All such elements have rank two and lie in a single orbit under the action of $\so(5;\CC) \oplus \sp(4;\CC)$.  Using the given embedding $\CC^5 \inj \CC^6$ we can view $Q_{\HT}$ equially as an element of $\mf A^6_{(1,1)}$.
\end{itemize}

\subsection{Description of the Twist}
Having set up our notation, we can state one the way in which the moduli space of multiplicative Higgs bundles arises from twisted supersymmetric gauge theory.  We will first state the relationship, then explain what exactly the statement is supposed to mean.

\begin{claimnum} \label{twist_claim}
If we take the twist of $\mc N=2$ super Yang-Mills theory with gauge group $G$ by $Q_\HT$, on the 5-manifold $C \times S^1 \times \RR^2$ where $C$ is a Calabi-Yau curve, with monopole surface operators placed at the points $(z_1,0), \ldots, (z_k,0)$ in $C \times S^1$ with charges $\omega^\vee_{z_1}, \ldots, \omega^\vee_{z_k}$ respectively, its moduli space of solutions to the equations of motion can be identified with the shifted cotangent space
\[\mr{EOM}_\HT(C \times S^1 \times \RR^2) \iso T^*[1]\mhiggs(C,D,\omega^\vee).\]
\end{claimnum}

\begin{remark}
The shifted cotangent space of a stack is a natural construction in the world of derived geometry.  In brief, the $k$-shifted cotangent space of a stack $\mc X$ has a canonical projection $\pi \colon T^*[k]\mc X \to \mc X$, and the fiber over a point $x \in \mc X$ is a derived affine space (i.e. a cochain complex) -- the tangent space to $\mc X$ at $x$ shifted down in cohomological degree by $k$.
\end{remark}

So, let us try to unpack the meaning of Claim \ref{twist_claim}.  Firstly, what does it mean to identify ``the moduli stack of solutions to the equations of motion'' of our twisted theory?  We have the following idea in mind (discussed in more detail, for instance, in \cite{CostelloSUSY, ElliottYoo1}).

\begin{construction} \label{twist_construction}
We'll use the Lagrangian description for the twisted field theory.  On the one hand, we can describe a twisted space of fields and twisted action functional.  This is a fairly standard construction: one factors the action functional into the sum $S_{\mr{tw}}(\phi) + Q\Lambda(\phi)$ of a ``twisted'' action functional and a $Q$-exact functional.  Note that the action of the group $\Spin(n) \times G_R$, where $G_R$ is the group of R-symmetries, is broken to a subgroup.  Only the subgroup of this product stabilizing $Q$ still acts on the twisted theory. 

Having written down the twisted action functional, we'll consider its critical locus: the moduli space of solutions to the equations of motion.  However, we can consider a finer invariant than only the ordinary critical locus: we can form the critical locus in the setting of derived geometry.  In classical field theory this idea is captured by the classical \emph{BV formalism}: for a modern formulation see e.g. \cite{CostelloBook}.  In particular, for each classical point in the critical locus, we can calculate the classical BV-BRST complex of the twisted theory -- concretely, we calculate the BV-BRST complex of the untwisted theory, and add the operator $Q$ to the differential.  The twisted BV-BRST complex models the tangent complex to the derived critical locus of the twisted theory.

So, to summarize, the BV formalism gives us a space: the critical locus of the twisted action functional, equipped with a sheaf of cochain complexes, whose fiber at a solution is the twisted BV-BRST complex around that solution.  This structure is weaker than a derived stack, but when we say a derived stack $\mc X$ ``can be identified with'' the moduli space $\EOM$ of solutions to the equations of motion, we mean that we can identify the space of $\CC$-points of $\mc X$ in the analytic topology with the underlying space of $\EOM$, and we can coherently identify the tangent complex of $\mc X$ at each $\CC$-point $x$ with the twisted BV-BRST complex at the corresponding classical solution.
\end{construction}

\begin{remark}
There's another subtlety that we glossed over in our discussion of the BV-BRST complex above: a priori when you add $Q$ to the differential of the untwisted BV-BRST complex the result is only $\ZZ/2\ZZ$-graded, since $Q$ has ghost number 0 and superdegree 1, whereas the BV-BRST differential has ghost number 1 and superdegree 0.  In order to promote it to a $\ZZ$-graded complex we need to modify the degrees of our fields using an action of $\mr U(1)$ inside the R-symmetry group, so that $Q$ has weight 1.  This is possible in all the examples we'll discuss below, and in our sketch argument we'll use this $\mr U(1)$ action to collapse the $\ZZ/2\ZZ \times \ZZ$-grading down to a single $\ZZ$-grading everywhere without further comment.
\end{remark}

Now that we know what Claim \ref{twist_claim} means, why is it true?  We'll only outline an argument here, since the whole twisted gauge theory story is somewhat orthogonal to the emphasis of the rest of this paper.  The outline we'll give goes via the minimal twist of 6d $\mc N=(1,1)$ super Yang-Mills theory.  This calculation was done independently by Dylan Butson -- a detailed version will appear in his forthcoming article \cite{Butson}.

\begin{enumerate}
 \item First, compute the twist of $\mc N=(1,0)$ super Yang-Mills theory in 6-dimensions by $Q_{\mr{min}}$.  This twisted theory is defined on a compact Calabi-Yau 3-fold $X^3$, and can be identified, in the sense of Construction \ref{twist_construction}, with the shifted cotangent space $T^*[-1]\bun_G(X^3)$.  Note: there are at least two ways of doing this calculation.  One can calculate the twist directly using a method very similar to Baulieu's calculation of the minimal twist of 10-dimensional $\mc N=1$ super Yang-Mills theory in \cite{Baulieu}.  Alternatively one can work in a version of the first order formalism where one introduces an auxiliary 2-form field: this is the approach followed by Butson. 
 
 \item One can extend this calculation to include a hypermultiplet valued in a representation $W$ of $G$.  One finds that the inclusion of the twisted hypermultiplet couples the moduli space of holomorphic $G$-bundles to sections in the associated bundle corresponding to $W$.  That is, the moduli space of solutions of the twisted theory can be identified with the shifted cotangent space $T^*[-1]\map(X^3, W/G)$, where $W/G$ is the quotient stack.  This tells us the twist of $\mc N=(1,1)$ super Yang-Mills: this is the case where $W = \gg$.
 
 \item The twist by the holomorphic-topological supercharge $Q_{\mr{HT}}$ corresponds to a deformation of this moduli space.  Specifically, let us set $X^3 = S \times C$ where $S$ is a Calabi-Yau surface and $\Sigma$ is a Calabi-Yau curve.  In this case we can identify the $Q_{\mr{min}}$-twisted $\mc N=(1,1)$ theory with
 \begin{align*}
 T^*[-1]\map(S \times \Sigma, \gg/G) &\iso T^*[-1]\map(S \times \Sigma, T[-1]BG) \\
 &\iso T^*[-1]\map(S \times T[1]\Sigma, BG).  
 \end{align*}
 The deformation to the twist by $Q_\HT$ corresponds to the Hodge deformation, deforming $T[1]\Sigma$ (the Dolbeault stack of $\Sigma$) to $\Sigma_{\mr{dR}}$ (the de Rham stack of $\Sigma$: this has the property that $G$-bundles on $\Sigma_{\mr{dR}}$ are the same as $G$-bundles on $\Sigma$ with a flat connection).  
 
 \item Now, all of these theories can be dimensionally reduced down to 5 dimensions.  Specifically, let us split $S$ as $C \times E_q$, and send the radius of one of the $S^1$ factors to zero, or equivalently degenerate the smooth elliptic curve $E_q$ to a nodal curve.  We're left with the identification, for the example we're interested in:
 \[\EOM_\HT(C \times S^1 \times \Sigma) \iso T^*[-1]\map(C \times \Sigma_{\mr{dR}}, G/G).\]
 Here we have identified $\map(E^{\mr{nod}}, BG)$ with the adjoint quotient $G/G$ \footnote{Strictly speaking this is only the semistable part of the moduli space.  We get $G/G$, for instance, by asking for only those bundles which lift to the trivial bundle on the normalization.}.  So far, tn this discussion $\Sigma$ was a compact curve.  If we want to set $\Sigma = \CC$ instead the only change is that the $-1$-shifted tangent space becomes the 1-shifted tangent complex.  There's a unique flat $G$-bundle on $\CC$, so we can identify
 \[\EOM_\HT(C \times S^1 \times \CC) \iso T^*[1]\map(C, G/G).\]
 
 \item Finally, we need to include surface operators.  In order to see how to do this, let us go back to the Lagrangian description of our classical field theory.  The inclusion of monopole operators is usually thought of as modifying the space of fields in our supersymmetric gauge theory, so that the gauge field is permitted to be singular along the prescribed surface, with specified residues.  If we track the above calculation where the fields are allowed those prescribed singularities, the result is that we should replace $\map(C \times E^{\mr{nod}}, BG)$ with the moduli space of singular multiplicative Higgs fields with local singularity data corresponding to the choice of charges of the monopole operators.  That is we can identify,
 \[\mr{EOM}_\HT(C \times S^1 \times \RR^2) = T^*[1]\mhiggs(C,D,\omega^\vee).\]
\end{enumerate}

\begin{remark}[Reduction to 4-dimensions]
If we reduce further, to four dimensions, by sending the radius of the remaining $S^1$ to zero, then we recover a more familiar twist first defined by Kapustin \cite{KapustinHolo}.  We can interpret this further degeneration as degenerating the factor $E_q$ not to a nodal, but to a cuspidal curve.  The semistable part of the stack $\map(E^{\mr{cusp}},BG)$ of $G$-bundles on a cuspidal curve is equivalent the Lie algebra adjoint quotient stack $\gg/G$, so we can identify the $Q_\HT$-twisted 4d $\mc N=4$ theory with
\begin{align*}
\EOM_\HT(C \times \Sigma) &\iso T^*[-1]\map(C \times \Sigma_{\mr{dR}}, \gg/G) \\
&\iso T^*[-1]\map(T[1]C \times \Sigma_{\mr{dR}}, BG).
\end{align*}
This is the shifted cotangent space whose base is the moduli space of $G$-bundles on $C \times \Sigma$ with a flat connection on $\Sigma$ and a Higgs field on $C$ (note that this twisted theory is defined where $C$ and $\Sigma$ are any curves, not necessarily Calabi-Yau).  This agrees with the calculation performed in \cite{ElliottYoo1}.  In particular, if we set $\Sigma = \CC$ the result is
\[\EOM_\HT(C \times \CC) \iso T^*[1]\higgs_G(C).\]
\end{remark}

\begin{remark} \label{Kapustin_twist_remark}
This supersymmetric gauge theory story should be compared to several recent twisting calculations in the literature.  Firstly, the minimal twist of 5d $\mc N=2$ gauge theory was discussed by Qiu and Zabzine \cite{QiuZabzine} on quite general contact five-manifolds.  They described the twisted equations of motion in terms of the Haydys-Witten equations.

A recent article of Costello and Yagi \cite{CostelloYagi} also described a relationship between multiplicative Higgs moduli spaces and twisted supersymmetric gauge theory in yet another context.  They calculated the twist of 6d $\mc N=(1,1)$ gauge theory, but with respect to yet another twisting supercharge $Q$ with the property that the image of $[Q,-]$ in $\CC^6$ is five-dimensional.  They then consider this theory on $\RR^2 \times C \times \Sigma$, and place it in the $\Omega$-background in the $\RR^2$ directions.  They argue that the resulting theory is Costello's four-dimensional Chern-Simons theory \cite{CostelloYangian}, whose moduli stack of derived classical solutions can be identified with $\map(C \times \Sigma_{\mr{dR}}, BG)$.  Costello and Yagi go on to argue via a sequence of string dualities that this theory is dual to the ADE quiver gauge theories discussed in \cite{NekrasovPestun, NekrasovPestunShatashvili}.  From another point of view, dualities of this form were also discussed by Ashwinkumar, Tan and Zhao \cite{AshwinkumarTanZhao}.
\end{remark}

\subsection{Langlands Duality} \label{Langlands_section}
Our main motivation for describing the multiplicative Hitchin system in terms of a twist of 5d $\mc N=2$ super Yang-Mills theory is to make the first steps towards a ``multiplicative'' version of the geometric Langlands conjecture.  The description in terms of supersymmetric gauge theory allows us to use the description of the geometric Langlands conjecture in terms of S-duality for topological twists of 4d $\mc N=4$ super Yang-Mills theory given by Kapustin and Witten \cite{KapustinWitten}.  Better yet, Witten described S-duality in terms of the 6d $\mc N=(2,0)$ superconformal field theory compactified on a torus: one obtains dual theories by compactifying on the two circles in either of the two possible orders \cite{Witten6d}.  We can leverage this story in order to describe a conjectural duality for twisted 5d gauge theories. 

The A- and B-twists of 4d $\mc N=4$ super Yang-Mills can both be defined by deforming an intermediate twist sometimes referred to as the Kapustin twist \cite{KapustinHolo} (see \cite{CostelloSUSY, ElliottYoo1} for more details).  The moduli stack of solutions to the equations of motion on $C \times \RR^2$ in the Kapustin twist is the 1-shifted cotangent space of $\higgs_G(C)$: the moduli stack of $G$-Higgs bundles on $C$.  The A- and B-twists modify this in two different ways.  In the B-twist $\higgs_G(C)$ is deformed by a hyperk\"ahler rotation, and it becomes $\Flat_G(C)$: the moduli stack of flat $G$-bundles on $C$.  In the A-twist the shifted cotangent bundle $T^*[1]\higgs_G(C)$ is deformed to the de Rham stack $\higgs_G(C)_{\mr{dR}}$ -- the de Rham stack of $\mc X$ has the property that quasi-coherent sheaves on $\mc X_{\mr{dR}}$ are the same as $D$-modules on $\mc X$.  

To summarize Kapustin and Witten's argument therefore, S-duality interchanges the A- and B-twists of 4d $\mc N=4$ super Yang-Mills thory,  and provides an equivalence between their respective categories of boundary conditions on the curve $C$.  These categories of boundary conditions can be identified on the B-side as the category of coherent sheaves on $\Flat_G(C)$, and on the A-side as the category of D-modules on $\bun_G(C)$, or equivalently coherent sheaves on $\bun_G(C)_{\mr{dR}}$.

Now, we observe that this family of twists arises as the limit of a family of twisted of 5d $\mc N=2$ gauge theory compactified on a circle, where the radius of the circle shrinks to zero.  This motivates an analogous five-dimensional story, which we summarize with the following ``pseudo-conjecture'', by which we mean a physical statement whose correct mathematical formulation remains to be determined.

\begin{pseudoconj}[Multiplicative Geometric Langlands] \label{multLanglands}
Let $G$ be a Langlands self-dual group.  There is an equivalence of categories
\[\text{A-Branes}_{q^{-1}}(\mhiggs_G(C,D,\omega^\vee)) \iso \text{B-Branes}(\qconn_G(C, D, \omega^\vee))\]
where the category on the right-hand side depends on the value $q$.
\end{pseudoconj}

What does this mean, and are there situations in which we can make it precise?  We'll discuss a few examples where we can say something more concrete.  In each case, by ``B-branes'' we'll just mean the category $\coh(\qconn_G(C, D, \omega^\vee))$ of coherent sheaves.  By ``A-branes'' we'll mean some version of \emph{$q^{-1}$-difference modules} on the stack $\bun_G(C)$. 

\begin{remark}
This equivalence is supposed to interchange objects corresponding to branes of opers on the two sides, and introduce an analogue of the Feigin-Frenkel isomorphism between deformed W-algebras (see \cite{FrenkelReshetikhinSTS, STSSevostyanov}.  This isomorphism only holds for self-dual groups, which motivates the restriction to the self-dual case here.
\end{remark}

\begin{remark}
Even for the ordinary geometric Langlands conjecture there are additional complications that we aren't addressing here.  For example, the most natural categorical version of the geometric Langlands conjecture is false for all non-abelian groups.  Arinkin and Gaitsgory \cite{ArinkinGaitsgory} explained how to correct the statement to obtain a believable conjecture: one has to consider not all coherent sheaves on the B-side, but only those sheaves satisfying a ``singular support'' condition.  In \cite{ElliottYoo2} it was argued that, from the point of view of twisted $\mc N=4$ super Yang-Mills theory these singular support conditions arise when one restricts to only those boundary conditions compatible with a choice of vacuum.  The same sorts of subtleties should equally occur in the multiplicative setting.
\end{remark}

\subsubsection{The Abelian Case}
Suppose $G = \GL(1)$ (more generally we could consider a higher rank abelian gauge group).  In general for an abelian group the moduli spaces we have defined are trivial -- for instance the rational and trigonometric spaces are always discrete.  However there is one interesting non-trivial example: the elliptic case.  For simplicity let us consider the abelian situation with $D = \emptyset$: the case with no punctures.

\begin{definition}
A \emph{$q$-difference module} on a variety $X$ with automorphism $q$ is a module for the sheaf $\Delta_{q,X}$ of non-commutative rings generated by $\OO_X$ and an invertible generator $\Phi$ with the relation $\Phi \cdot f = q^*(f) \cdot \Phi$.  Write $\diff_q(X)$ for the category of $q$-difference modules on $X$.
\end{definition}

In the abelian case the space $\qconn_{\GL(1)}(E)$ is actually a stack, but one can split off the stacky part to define difference modules on it.  Indeed, for any $q$ one can write
\[\bun_{\GL(1)}(E) \iso B\GL(1) \times \ZZ \times E^\vee\]
and so
\[\qconn_{\GL(1)}(E) \iso B\GL(1) \times \ZZ \times (E^\vee \times_q \CC^\times)\]
which means one can define difference modules on these stacks associated to an automorphism of $E^\vee$ or $E^\vee \times_q \CC^\times$ respectively.

\begin{conjecture}
There is an equivalence of categories for any $q \in \bb{CP}^1$
\[\diff_q(\bun_{\GL(1)}(E)) \iso \coh(q^{-1}\conn_{\GL(1)}(E)).\]
\end{conjecture}

In this abelian case we can go even farther and make a more sensitive 2-parameter version of the conjecture.

\begin{conjecture}
There is an equivalence of categories for any $q_1, q_2 \in \bb{CP}^1$
\[\diff_{q_1}(q_2\conn_{\GL(1)}(E)) \iso \diff_{q_2^{-1}}(q_1^{-1}\conn_{GL(1)}(E)),\]
where $q_1$ is the automorphism of $E^\vee \times_{q_2} \CC^\times$ acting fiberwise over each point of $\CC^\times$.
\end{conjecture}

This conjecture should be provable using the same techniques as the ordinary geometric Langlands correspondence in the abelian case, i.e. by a (quantum) twisted Fourier-Mukai transform (as constructed by Polishchuk and Rothstein \cite{PolishchukRothstein}).

\subsubsection{The Classical Case}
Now, let us consider the limit $q \to 0$.  This will give a conjectural statement involving coherent sheaves on both sides analogous to the classical limit of the geometric Langlands conjecture as conjectured by Donagi and Pantev \cite{DonagiPantev}.  The existence of an equivalence isn't so interesting in the self-dual case (where both sides are the same), but the classical multiplicative Langlands functor should be an \emph{interesting} non-trivial equivalence.  For example we can make the following conjecture

\begin{conjecture}
Let $G$ be a Langlands self-dual group and let $E$ be an elliptic curve.  There is an automorphism of categories (for the rational, trigonometric and elliptic moduli spaces)
\[F \colon \coh(\mhiggs_G(E)) \iso \coh(\mhiggs_{G}(E))\]
so that the following square commutes:
\[\xymatrix{
\coh(\mhiggs_T(E)) \ar[r]^{\mathrm{FM}} \ar[d]_{p_*q^!} &\coh(\mhiggs_T(E)) \ar[d]^{p_*q^!} \\
\coh(\mhiggs_G(E)) \ar[r]^{F} &\coh(\mhiggs_G(E)).
}\]
Here we're using the natural morphisms $p \colon \mhiggs_B(E) \to \mhiggs_G(E)$ and $q \colon \mhiggs_B(E) \to \mhiggs_T(E)$, and $\mr{FM}$ is the Fourier-Mukai transform.
\end{conjecture}

Alternatively, we can say something about classical geometric Langlands duality in the non-simply-laced case.  Recall in the usual geometric Langlands story that non-simply-laced gauge theories in dimension 4 arise by ``folding'' the Dynkin diagram.  In other words, one identifies a non-simply laced simple Lie group as the invariants of a simply laced self-dual Lie group $\wt G$ with respect to its finite group of outer automorphisms (either $\ZZ/2\ZZ$, or $S_3$ in the case of the exceptional group $G_2$).  One obtains a 5d $\mc N=2$ gauge theory with gauge group $G$ by taking the invariants of the 6d $\mc N=(2,0)$ theory reduced on a circle, where $\mr{Out}(\wt G)$ acts simultaneously on $\wt G$ and on the circle we reduce along.

Putting this together yields the following conjecture.
\begin{conjecture}
\label{eq:classical-q-langlands}
Let $G$ and $G^\vee$ be Langlands dual simple Lie groups, and say that $G$ arose by folding the Dynkin diagram of a self-dual group $\wt G$.  Then there is an equivalence of categories
\[\coh(\mhiggs_{G^\vee}(E) \iso \coh(\mhiggs_{\wt G}(E)^{\mr{Out}(\wt G)}),\]
where $\mr{Out}(\wt G)$ acts simultaneously on $\wt G$ and on the circle $S^1_B$, under the identification $\mhiggs_{\wt G}(E) = \bun_{\wt G}(E \times S^1_B)$.  As above, this equivalence should be compatible with the Fourier-Mukai transform relating the categories for the maximal tori.
\end{conjecture}


\section{Periodic Monopoles} \label{periodic_monopole_section}
Moduli spaces of $q$-connections on a Riemann surface $C$ are closely related to moduli spaces of periodic monopoles, i.e. monopoles on 3-manifolds that fiber over the circle (more specifically, with fiber $C$ and monodromy determined by $q$).  Let $G_\RR$ be a compact Lie group whose complexification is $G$.  The discussion in this section will mostly follow that of \cite{CharbonneauHurtubise, Smith}.

Write $M = C\times_q S^1_R$ for the $C$-bundle over $S^1$ with monodromy given by the automorphism $q$.  More precisely, $M$ is the Riemannian 3-manifold obtained by gluing the ends of the product $C \times [0,2\pi R]$ of Riemannian manifolds by the isometry $(x,2\pi R) \sim (q(x), 0)$.

\begin{definition}
A \emph{monopole} on the Riemannian 3-manifold $M = C \times_q S^1_R$ is a smooth principal $G_\RR$-bundle $\bo P$ equipped with a connection $A$ and a section $\Phi$ of the associated bundle $\gg_{\bo P}$ satisfying the Bogomolny equation 
\[\ast F_A = \d_A \Phi.\]
\end{definition}

\begin{remark}
We should emphasise the difference between the Riemannian 3-manifold $M = C \times_q S^1_R$ appearing in this section and the derived stack $C \times_q S^1_B$ (the mapping torus) appearing in the previous section.  These should be thought of as smooth and algebraic realizations of the same object (justified by the comparison Theorem \ref{monopole_qconn_comparison_thm}) but they are a priori defined in different mathematical contexts.
\end{remark}

We can rephrase the data of a monopole on $M$ as follows.  Let $C_0 = C \times \{0\}$ be the fiber over $0$ in $S^1$, viewed as a Riemann surface.  Let $P$ be the restriction of the complexified bundle $\bo P_\CC$ to $C_0$.  Consider first the restriction of the complexification of $A$ to a connection $A_0$ on $P$ over $C_0$.  The $(0,1)$ part of $A_0$ automatically defines a holomorphic structure on $P$.  We can introduce an additional piece of structure on this holomorphic $G$-bundle.  In order to do so we can decompose the Bogomolny equation into one real and one complex equation as follows.
\begin{align*}
F_{A_0} - \nabla_t \Phi &= 0 \\
[\ol{\del}_{A_0}, \nabla_t - i\Phi \d t] &= 0 
\end{align*}
where $\nabla_t$ is the component of the covariant derivative $\d_A$ normal to $C_0$.  

\begin{definition} 
From now on we'll use the notation $\mc A$ for the combination $\nabla_t - i\Phi \d t$: an element of the space $\Omega^0(C_0, \gg_P)\d t$ of sections of the complex vector bundle $\gg_P$ on the complex curve $C_0$. 
\end{definition}

Let us now introduce singularities into the story.  We'll keep the description brief, referring the reader to \cite{CharbonneauHurtubise, Smith} for details.
\begin{definition}
Let $D \sub M$ be a finite subset.  Let $\omega^\vee$ be a choice of coweight for $G$.  A monopole on $M \bs D$ has \emph{Dirac singularity} at $z \in D$ with charge $\omega^\vee$ if locally on a neighborhood of $z$ in $M$ it is obtained by pulling back under $\omega^\vee$ the standard Dirac monopole solution to the Bogomolny equation, where $\Phi$ is spherically symmetric with a simple pole at $z$, and the restriction of a connection $A$ to a two-sphere $S^2$ enclosing the singularity defines a $U(1)$ bundle on this $S^2$ of degree $1$ so that
    \[\frac{1}{2\pi} \int_{S^2} F = 1 .\]
  See e.g. \cite[Section 2.2]{CharbonneauHurtubise} for a more detailed description.
\end{definition}

We can also introduce a framing (or a reduction of structure group as in the trigonometric example, though we won't consider the latter in this paper).  As usual let $c \in C$ be a point fixed by the automorphism $q$.
\begin{definition}
  A monopole on $M$ with \emph{framing} at the point $c \in C$ is a monopole $(\bo P,A,\Phi)$ on $M$ (possibly with Dirac singularities at $D$) along with a trivialization of the restriction of $\bo P$ to the circle $\{c\} \times S^1_R$, with the condition that the holonomy of $\mc A$ around this circle lies in a fixed conjugacy class $f \in G/G$.
\end{definition}

The moduli theory of monopoles on general compact 3-manifolds was described by Pauly \cite{Pauly}.  In this paper we'll be interested in moduli spaces of monopoles on 3-manifolds of the form $C \times S^1$, possibly with a fixed framing at a point in $C$.  

\begin{remark}
The moduli space of periodic monopoles on $\RR^2 \times S^1$ specifically has been studied in the mathematics literature by Foscolo \cite{FoscoloDef} , applying the analytic techniques of deformation theory to earlier work on periodic monopoles by Cherkis and Kapustin \cite{CherkisKapustin1, CherkisKapustin2}. This analysis considers a less restrictive boundary condition at infinity in $\RR^2$ than a framing, and therefore requires more sophisticated analysis than we'll need to consider in the present paper.
\end{remark}

In the cases of interest to us the moduli space of periodic monopoles can be obtained as a hyperk\"ahler quotient.  Let us focus initially on the rational case, meaning monopoles on $M = \bb{CP}^1 \times_\eps S^1_R$ with Dirac singularities at $D \times \{t_0\}$ and a framing at $\infty$.  Consider the infinite-dimensional vector space $\mc V$ consisting of pairs $(A,\Phi)$ where $A$ is a connection on a fixed principal $G_\RR$-bundle $\bo P$ on $M$, $\Phi$ is a section of $\gg_{\bo P}$, and $(A,\Phi)$ have a Dirac singularity with charge $\omega^\vee_{z_i}$ at each $(z_i,t_0)$ in $D \times \{t_0\}$.  Let $\mc G$ be the group of gauge transformations of the bundle $\bo P$.

The hyperk\"ahler moment map is given by the Bogomolny functional, namely
\begin{align*}
\mu \colon \mc V &\to \Omega^1(M; (\gg_\RR)_{\bo P}) \\
(A,\Phi) &\mapsto \ast F_A - \d_A \Phi.
\end{align*}

\begin{definition} \label{monopole_moduli_def}
Let $D$ be a finite subset $\{(z_1,t_1), \ldots, (z_k, t_k)\}$ of points in $M = \bb{CP}^1 \times_\eps S^1_R$, and let $\omega^\vee_{i}$ be a choice of coweight for each point in $D$. The moduli space $\mon_G(M, D, \omega^\vee)$ is the hyperk\"ahler quotient
\[\mon_G(M, D, \omega^\vee) = \mu^{-1}(0) / \mc G.\]
\end{definition}

Now let us address the relationship between periodic monopoles and
$q$-connections.  Suppose from now on that $q$ is in the identity
component of the group of automorphisms of $C$ (fixing the framing
point $c$ if present).  The following theorem is a slightly
generalized version of the comparison theorem between multiplicative
Higgs bundles and periodic monopoles proved by Charbonneau--Hurtubise
\cite{CharbonneauHurtubise} for $\GL_n$, and Smith \cite{Smith} for
general $G$. This approach was first suggested by Kapustin-Cherkis
\cite{CherkisKapustin2} under the name `spectral data'. 

\begin{theorem} \label{monopole_qconn_comparison_thm}
There is an analytic isomorphism between the moduli space of polystable monopoles on $C \times_q S^1$ with Dirac singularities at $D \times \{t_0\}$ (and a possible framing on $\{c\} \times S^1$) and the moduli space of $q$-connections on $C$ with singularities at $D$ and framing at $\{c\}$.  More precisely there is an analytic isomorphism
\[H \colon \mon^{(\fr)}_G(C \times_q S^1, D \times \{t_0\}, \omega^\vee) \to \qconn_G^{\text{ps,(fr)}}(C, D, \omega^\vee)\]
given by the holonomy map around $S^1$, i.e. sending a monopole $(\bo P, \mc A)$ to the holomorphic bundle $P = (\bo P_\CC)|_{C_0}$ with $q$-connection $g = \Hol_{S^1}(\mc A) \colon P \to q^*(P)$.
\end{theorem}

\begin{remark}
Note that in this statement we assumed that all the singularities occur in the same location in $S^1$, i.e. in the same slice $C \times {t_0}$.  This assumption is not necessary, but there is a constraint on the possible locations of the singularities as explained in \cite[Proposition 3.5]{CharbonneauHurtubise}.  
\end{remark}

\begin{proof}
This follows by the same argument as that given by Charbonneau--Hurtubise and Smith.  More explicitly, first let us think about injectivity, so let $(\bo P, \mc A)$ and $(\bo P', \mc A')$ be a pair of periodic monopoles on $C \times_q S^1$ with images $(P,g)$ and $(P', g')$ respectively, and choose a bundle isomorphism $\tau \colon P \to P'$ intertwining the $q$-connections $g$ and $g'$.  One observes first that $\bo P$ and $\bo P'$ are also isomorphic $G$-bundles since, by intertwining with the $q$-connections, we have an isomorphism $\bo P|_{C \times \{t\}} \to \bo P'|_{C \times \{t\}}$ for every $t \in S^1$.  That the monopole structures also match up follows by the same argument as in \cite[Proposition 4.7]{CharbonneauHurtubise}.

For surjectivity, again we'll match the argument in the case where $q=\id$.  We begin by extending a holomorphic $G$-bundle $P$ on $C_0$ with $q$-connection $g$ to a $G$-bundle on $M \bs (D \times \{t_0\}) = (C \times_q S^1_R) \bs (D \times \{t_0\})$  Let $\gamma \colon [-2\pi R,2\pi R] \to \aut(C)$ be a geodesic with $\gamma(-2\pi R) = q^{-1}$, $\gamma(0)=1$ and $\gamma(2\pi R) = q$.  Let $\wt M$ be the 3-manifold
\[\wt M = ((-2\pi R, 2\pi R) \times C) \bs \bigcup_{j=1}^k (A^+_j \cup A^-_j)\]
where $A^+_j$ is the arc $\{(t+ t_0,\gamma(t)(z_j)) \colon t \in (0, 2\pi R - t_0]\}$ and $A^-_j$ is the arc $\{(t + t_0 - 4 \pi R,\gamma(t)(z_j)) \colon t \in [2\pi R-t_0, 2 \pi R)\}$.

Let $\pi \colon \wt M \to C$ be the projection sending $(t,z)$ to $\gamma(t)(z)$.  The bundle $P$ pulls back to a bundle $\pi^*(P)$ on $\wt M$.  We obtain a bundle on $M \bs (D \times t_0)$ by applying the identification $(t,z) \sim (t - 2 \pi R, q(z))$.  This bundle extends to an $S^1$-invariant holomorphic $G$-bundle on $M \times S^1$.  The remainder of the proof -- verifying the existence of the monopole structure associated to an appropriate choice of hermitian structure -- consists of local analysis which is independent of the value of the parameter $q$. 

It remains to remark on the compatibility of framing data on the two sides.  A trivialization of the bundle $\bo P$ along the circle $\{c\} \times S^1$ yields a trivialization of the fiber of the bundle $P$ at $c$.  The condition that the holonomy around the circle at $c$ is $f$ fixes the value of the $q$-connection at $c$ to be in the conjugacy class $f$.
\end{proof}

\begin{remark}
Mochizuki \cite{Mochizuki} proved a stronger result in the rational case for the group $G = \GL_n$.  He allows not just a framing at infinity in $\bb{CP}^1$ but also a singularity encoded in terms of a $B$-reduction of the bundle.  Again, we won't work in this generality in the present paper.
\end{remark}

\subsection{Deformation Theory} \label{def_section}
In the next section we'll compare symplectic forms on these moduli spaces.  In order to do so it will be important to understand the tangent spaces at a point of the source and target.  There's a natural description of these tangent spaces in terms of the hypercohomology of certain cochain complexes.

From now on we'll focus on the example we're most interested in.  That is, we'll exclusively study the rational situation where $C = \bb{CP}^1$ and we fix a framing point $c = \infty$.  In this case we can use the description as a hyperk\"ahler reduction.  For more general deformation theory calculations we refer to Foscolo \cite{FoscoloDef} (though the results in that paper are more broadly applicable).  Recall that we can write
\[\mon^{(\fr)}_G(C \times_q S^1, D \times \{t_0\}, \omega^\vee) \iso \mu^{-1}(0)/ \mc G\]
where $\mc G$ is the group of gauge transformations of $\bo P$ and $\mu \colon \mc V \to \Omega^1(M; (\gg_R)_{\bo P})$ is the Bogomolny functional $(A,\Phi) \mapsto \ast F_A - \d_A \Phi$.  The tangent complex to this hyperk\"ahler quotient at a point $(\bo P, \mc A)$ can be written as $\Omega^0(M \!\bs\! D; (\gg_\RR)_{\bo P})[1] \to \bb T_{\mu^{-1}(0)}$ where $\bb T_{\mu^{-1}(0)}$ is the tangent complex to the zero locus of the moment map, concentrated in non-negative degrees. Roughly speaking $\bb T_{\mu^{-1}(0)} = \bb T_{\mc V} \overset {\d\mu} \to \Omega^1(M; (\gg_R)_{\bo P})[-1]$.  

More explicitly, following \cite{FoscoloDef} define $\mc F^{\mr{mon}}_{\bo P, \mc A}$ to be the sheaf of cochain complexes
\[\left(\xymatrix{
\Omega^0(M \!\bs\! D; (\gg_\RR)_{\bo P}) \ar[r]^(.36){\d_1} &\Omega^1(M \!\bs\! D; (\gg_\RR)_{\bo P}) \oplus \Omega^0(M \!\bs\! D; (\gg_\RR)_{\bo P}) \ar[r]^(.64){\d_2} &\Omega^1(M \!\bs\! D; (\gg_\RR)_{\bo P})
}\right) \otimes_\RR \CC\]
placed in degrees $-1$, 0 and 1 where $\d_1(g) = -(\d_A(g),[\Phi, g])$ and $\d_2(a,\psi) = \ast \d_A(a) - \d_A(\psi) + [\Phi,a]$.  Write $\d_{\mr{mon}}$ for the total differential.

\begin{remark}
Here we have chosen a point in the twistor sphere, forgetting the hyperk\"ahler structure and retaining a holomorphic symplectic structure.  In other words we have identified the target of the hyperk\"ahler moment map -- the space of imaginary quaternions -- with $\RR \oplus \CC$, which is equivalent to choosing a point in the unit sphere of the imaginary quaternions: the twistor sphere.  
\end{remark}

\begin{remark} \label{monopole_holo_restriction_rmk}
If we restrict $\mc F^{\mr{mon}}_{\bo P, \mc A}$ to a slice $C_t = C \times \{t\}$ in the $t$-direction we can identify it with a complex of the form
\[\Omega^\bullet(C_t; \gg_P)[1] \overset {[\Phi,-]} \to \Omega^\bullet(C_t; \gg_P)\]
with total differential given by $\d_A$ on each of the two factors along with the differential $[\Phi,-]$ mixing the two factors.  These two summands each split up into the sum of a Dolbeault complex on $C$ with its dual.  That is, there's a natural subcomplex of the form
\[\Omega^{0,\bullet}(C_t; \gg_P)[1] \overset {[\Phi,-]} \to i \Omega^{0,\bullet}(C_t; \gg_P) \d t\]
where the internal differentials on the two factors are now given by $\ol \dd_{A_0}$.  This complex is in turn quasi-isomorphic to the complex
\[\Omega^\bullet(S^1; \Omega^{0,\bullet}(C_t;\gg_P))[1]\]
with total differential $\ol \dd_{A_0} + \d_{\mc A}$.
\end{remark}

\begin{remark}
If one introduces a framing at a point $c \in C$ then we must correspondingly twist the complex $\mc F^{\mr{mon}}$ above by the line bundle $\OO(c)$ on $C$ -- i.e. we restrict to sections that vanish at the framing point.  So in that case we define
\[\mc F^{\text{mon,fr}}_{(\bo P,\mc A)} = \mc F^{\mr{mon}}_{\bo P, \mc A} \otimes (\CC_{S^1} \boxtimes \OO(c)).\]
\end{remark}

The following is proved in \cite{FoscoloDef}.
 
\begin{prop}
The tangent space of $\mon_G(S^1 \times C, D \times \{t_0\}, \omega^\vee)$ at the point $(\bo P,\mc A)$ is quasi-isomorphic to the hypercohomology $\bb H^0(C \times S^1; \mc F'_{(\bo P,\mc A)})$ of a subsheaf $\mc F' \sub \mc F^{\mr{mon}}$ where growth conditions are imposed on the degree 0 part of $\mc F^{\mr{mon}}$ near the singularities.
\end{prop}

Now let us consider the tangent complex to the moduli space of $q$-connections.  For the arguments in this article we'll only need to carefully consider the case $q=\id$ of multiplicative Higgs bundles, but we'll include some remarks regarding the more general case.  In this case the calculation was performed by Bottacin \cite{Bottacin}, see also \cite[Section 4]{HurtubiseMarkman}. Fix a multiplicative Higgs bundle $(P,g)$ on $C$.  We consider the sheaf of cochain complexes on $C$
\[\mc F_{(P,g)} = (\gg_P[1] \overset {\Ad_g} {\to} \gg_P(-D))\]
in degrees $-1$ and 0 with differential given by the adjoint action of $g$.  More precisely let $L_g$ and $R_g$ be the bundle maps $\gg_P \to \gg_P$ obtained as the derivative of left- and right-multiplication.  Then $\Ad_g = L_g - R_g$.  We can alternatively phrase this, as in \cite[Section 4]{HurtubiseMarkman}, as follows.  Define $\ad(g)$ to be the vector bundle
\[\ad(g) = (\gg_P \oplus \gg_P)/\{(X, -g X g^{-1}): X \in \gg_P\}.\]
Then we can write $\mc F$ as the sheaf of complexes
\[\mc F_{(P,g)} = (\gg_P[1] \overset {\Ad_g} {\to} \ad(g))\]
where now $\Ad_g$ is just the map $X \mapsto [(X,-X)]$.

\begin{remark}
  If one introduces a framing at a point $c \in C$ then we must correspondingly twist the complex $\mc F$ above by the line bundle $\OO(-c)$ on $C$, i.e. we restrict to deformations that preserve the framing and therefore are zero at the point $c$.  So in that case we define
\[\mc F^\fr_{(P,g)} = (\gg_P[1] \overset {\Ad_g} {\to} \gg_P(-D)) \otimes \OO(-c).\]
\end{remark}

\begin{remark}
For more general $q$ we should modify this description by replacing $g$ by a $q$-connection.  Note that one can still define the ($q$-twisted) adjoint action $X \mapsto g X g^{-1}$ using a $q$-connection, and so we can still define the complex
\[\mc F_{(P,g)} = (\gg_P[1] \overset {\Ad_g} {\to} \ad(g))\]
just as in the untwisted case.
\end{remark}

This complex defines the deformation theory of the moduli space of multiplicative Higgs bundles.

\begin{prop}[{\cite[Proposition 3.1.3]{Bottacin}}]
The tangent space of $\mhiggs_G(C, D, \omega^\vee)$ at the point $(P,g)$ is quasi-isomorphic to the hypercohomology $\bb H^0(C; \mc F_{(P,g)})$ of the sheaf $\mc F$.
\end{prop}

\begin{remark}
The remaining hypercohomology of the sheaf $\mc F_{(P,g)}$ generically has dimension $\dim \mf z_{\gg}$ (or 0 if we fix a framing at $c \in C$) in degree $-1$, and dimension $\mr{genus}(C) \cdot \dim \mf z_{\gg}$ in degree $1$.  However the moduli space $\mhiggs_G(C, D,\omega^\vee)$ is in fact a smooth algebraic variety.  This follows from a result of Hurtubise and Markman \cite[Theorem 4.13]{HurtubiseMarkman}, noting that their argument does not rely on the curve $C$ being of genus 1.
\end{remark}

\begin{corollary} \label{dim_of_moduli_space_cor}
In the rational case, the moduli space $\mhiggs^\fr_G(\bb{CP}^1, D, \omega^\vee)$ has dimension 
\[2 \sum_{z_i \in D} \langle \rho, \omega^\vee_{z_i} \rangle.\]
\end{corollary}

\begin{proof}
One can use the same argument as \cite[]{HurtubiseMarkman} (see also \cite[Proposition 5.6]{CharbonneauHurtubise}), with the additional observation that tensoring by the line bundle $\OO(-\infty)$ kills the outer cohomology groups ($\bb H^{-1}$ and $\bb H^1$ with our degree conventions, which differ from the conventions of loc. cit. by one).  Indeed, $\bb H^{-1}$ consists of sections of $\gg_P$ that are annihilated by $\Ad_g$ (given for generic $g$ by constant sections valued in $\mf z_{\gg}$) while vanish at $\infty$, which are necessarily 0.  Likewise we can use the equivalence between the sheaf $\mc F_{(P,g)}$ and its Serre dual to see that $\bb H^1$ also vanishes.  Finally the Euler characteristic of the two step complex is unchanged by tensoring by $\OO(-\infty)$. 
\end{proof}

In order to calculate with this hypercohomology group we'll use a \v Cech resolution.  This will be straightforward for the multiplicative Higgs moduli space, and we'll use the isomorphism of Theorem \ref{monopole_qconn_comparison_thm} to give an analogous description on the monopole side.    We define a cover $\mc U = \{U_0, U_1, \ldots, U_k, U_\infty\}$ of $C$ as follows.  Let $U_i$ be a contractible open neighborhood of the point $z_i$ and let $U_\infty$ be a contractible analytic open neighborhood of $c \in C$, all chosen to be pairwise disjoint.  Let $U_0$.  Finally let $U_0 = C \bs (D \cup \{c\})$.  Since the $U_i$ are contractible and the remaining subset $C \bs (D \cup \{c\})$ is an affine algebraic curve, which means that for any quasi-coherent sheaf of cochain complexes the higher cohomology groups vanish.  Likewise for the intersections: the punctured open sets $U_i^\times$ are analytic open sets of an affine curve.  

Specify a representative 0-cocycle $(\alpha_\infty, \{\alpha_i\}, \alpha_0, \beta_\infty, \{\beta_i\})$ for the \v Cech cohomology group with respect to our chosen cover $\mc U$.  Explicitly a 0-cochain is given by the following data:
\begin{align*}
 \alpha_\infty &\in \ad(g)(-1)(U_\infty) \\
 \alpha_i &\in \ad(g)(1)(U_i) \text{ for } i = 1,\ldots,k \\
 \alpha_0 &\in \ad(g)(C \bs (D \cup \{\infty\})) \\
 \beta_\infty &\in \gg_P(U^\times_\infty) \\
 \beta_i &\in \gg_P(U^\times_i) \text{ for } i=1,\ldots,k
\end{align*}
where the notation $(\pm 1)$ indicates tensoring by the line bundle $\OO(\pm 1)$.

Being a 0-cocycle means that $(\alpha_\infty - \alpha_0)|_{U^\times_\infty} = \mr{Ad}_g(\beta_\infty)$ and $(\alpha_i - \alpha_0)|_{U^\times_i} = \mr{Ad}_g(\beta_i)$ for each $i$.  We consider 0-cocycles modulo 0-coboundaries of the form $(\mr{Ad}_g(f_\infty), \{\mr{Ad}_g(f_i)\}, \mr{Ad}_g(f_0), (f_\infty -  f_0)|_{U_\infty^\times}, \{(f_i - f_0)|_{U_i^\times}\})$.  In fact $\mr{Ad}_g$ is an isomorphism on $U_0$ for the sections $\alpha_0$ of $\gg_P$ that occur: those with no poles or zeroes in $U_0$.  That means that we can add a coboundary to force $\alpha_0=0$. 

Now, rather than describing the tangent space to the moduli space of monopoles we'll define a complex that maps to it which we can define locally with respect to the cover $\mc U$.  Consider the open cover $\mc U \times S^1 = \{U_i \times S^1\}$ of $M = C \times S^1$.  For each element $U_i$ of the cover we can define a map
\begin{align*}
\mc F'(U_i \times S^1) &\to (\Omega^0(S^1; \Omega^{0,\bullet}(U_i; \gg_P)) \to \Omega^1(S^1; \Omega^{0,\bullet}(U_i; \gg_P)(D_{U_i})))[1] \\ 
&\iso \gg_P(U_i)[1] \to \gg_P(U_i)(D|_{U_i}) \\
&\iso \mc F(U_i)
\end{align*}
where we restrict the sheaf $\mc F'$ whose hypercohomology calculated the monopole tangent complex to the holomorphic part of the slice at $\{t\} \in S^1$.  Altogether this defines a map from the \v Cech cohomology with respect to this cover to the tangent complex of the moduli space of monopoles.  That is, we have an explicit map
\[{\mr {\check H}}^\bullet(M, \mc U \times S^1, \mc F') \to \bb H^\bullet(M; \mc F')\]
that factors through the (isomorphic) tangent complex of the moduli space of multiplicative Higgs bundles.  To verify this we need only note that these maps commute with the differentials in the \v Cech complex, i.e. with the restriction to the intersection of a pair of open sets. 

Explicitly a 0-cochain in this \v Cech complex is given by $(\alpha_\infty, \{\alpha_i\}, \alpha_0, \beta_\infty, \{\beta_i\})$ where now 
\begin{align*}
 \alpha_\infty &\in \mc F^{\mr{mon}}_0(U_\infty \times S^1) \\
 \alpha_i &\in \mc F^{\mr{mon}}_0(U_i\times S^1) \text{ for } i = 1,\ldots,k \\
 \alpha_0 &\in \mc F^{\mr{mon}}_0(U_0\times S^1) \\
 \beta_\infty &\in \Omega^0_\CC(U_\infty^\times \times S^1; \gg_{\bo P_\CC}) \\
 \beta_i &\in \Omega^0_\CC(U_i^\times \times S^1; \gg_{\bo P_\CC}) \text{ for } i=1,\ldots,k.
\end{align*}
Here we write $\mc F^{\mr{mon}}_0$ to indicate the degree 0 term in the cochain complex.  There's a similar condition for being a 0-cocycle involving the differential $\d_{\mr{mon}}$. 

To conclude this subsection it will also be important to have an explicit description of the derivative of the holonomy map $H$ as a map between tangent spaces.  We can describe this map using our \v Cech resolutions on each contractible open set $U_i$ individually.

\begin{prop} \label{local_derivative_description_prop}
The derivative $\d H \colon \bb H^\bullet(U_i \times S^1 ; \mc F'_{(\bo P,\mc A)}) \to \bb H^\bullet(\bb D_i; \mc F_{(P,g)})$ is given on an open patch $U_i \times (0,2\pi)$ by the formula
\[\d H(\alpha_i) = \d H(\d_{\mr{mon}} b_i) = b_i(2\pi)H(\mc A) - H(\mc A)b_i(0)\]
where $i = 1, \ldots, k$ or $\infty$.  More precisely by $b_i(2\pi)$ and $b_i(0)$ we mean the limit of $b_i(t)$ as $t \to 2\pi$ or 0 respectively.
\end{prop}

\begin{proof}
Note that the right-hand side is the derivative at $\mc A$ of the map $B_i \mapsto B_i(2\pi)H(\mc A)B_i(0)^{-1}$ where $B_i \in \Omega^0((U_i \times (0,2\pi)) \bs \{(z_i, t_0)\}; \gg_P)$.  This is the definition of the action of the group of gauge transformations on the holonomy $H(\mc A)$ from $t=0$ to $2\pi$.
\end{proof}

\section{Symplectic Structures} \label{symp_section}
\subsection{An Example with $G=\GL_2$} \label{GL2_example_section}

Let us discuss in detail the geometry of the moduli space of multiplicative Higgs bundles in some of the simplest examples, for the group $G = \GL_2$ and minimal singularity data.  The symplectic structures, and the Hitchin integrable system, which we'll analyze for the more general moduli spaces can be described very concretely in this simple situation. 

Fix $G=\GL_2$.  We'll work in the rational case, so we'll work over the curve $C = \bb{CP}^1 = \CC \cup \{\infty\}$ where $\CC$ has coordinate $z$, with fixed framing point $z_\infty = \infty$, and fixed value $g_\infty \in \GL_2$ for the framing. 

Consider the connected component of $\mhiggs^\fr_{G}(\bb{CP}^1, D, \omega^{\vee})$ consisting of those multiplicative Higgs bundles whose underlying $G$-bundle is trivializable.  Fixing the framing fixes a trivialization of the underlying $G$-bundle, which means this connected component of the moduli space can be identified with the space of $G$-valued
rational functions $g(z)$ on $\CC$ with certain singularity conditions that we can write down explicitly.

As our first explicit example we will consider the moduli space with two singularities at distinct points $z_1$ and $z_2$ in $\CC$.  Fix the corresponding coweights to be the generators $\omega^\vee_{z_1} = (1,0)$ and $\omega^\vee_{z_2} = (0,-1)$ in the defining basis of the coweight lattice of $\GL_2$.  We'll denote the zero connected component in the moduli space of multiplicative Higgs bundles by $\mc M(z_1,z_2)$.
 
The zero connected component of our moduli space  $\mhiggs_{G}^\fr(C, D, \omega^{\vee})$ can then be identified with the space of functions $g(z)$ valued in $2 \times 2$ matrices 
   \begin{equation*}
     g(z) =
     \begin{pmatrix}
       a(z) & b(z) \\
       c(z) & d(z)
     \end{pmatrix}
   \end{equation*}
   where $a(z), b(z), c(z), d(z)$ are rational functions on $\bb{CP}^1$ satisfying the following conditions. 
   \begin{enumerate}
   \item The functions $a(z), b(z), c(z), d(z)$ are regular everywhere on $\bb{CP}^1 \setminus \{z_2\}$,
     in particular they are regular at $\infty$ and $z_1$.
   \item When evaluated at the point at infinity, $g(\infty) = g_\infty$ where $g_\infty \in \GL_2$ is a representative of the conjugacy class we fixed by our choice of framing
     \begin{equation*}
       g_\infty =
       \begin{pmatrix}
         a_\infty & b_\infty \\
         c_\infty & d_\infty 
       \end{pmatrix}, \qquad a_\infty d_\infty - c_\infty b_\infty \neq 0 
     \end{equation*}
where $a_\infty, b_\infty, c_\infty, d_\infty \in \CC$  satisfy the condition
   \item
     \begin{equation*}
       \det g(z) =  \frac{ z- z_1}{ z - z_2}  \det g_\infty.
     \end{equation*}
   \end{enumerate}
 
   The conditions (1), (2) and (3) together imply that the functions $a(z), b(z), c(z), d(z)$ have the form
   \begin{equation*}
     a(z) =  \frac{a_\infty z  - a_0}{z - z_2},\quad
     b(z) = \frac{ b_\infty z  - b_0}{z - z_2},\quad
     c(z) = \frac{ c_\infty  z  - c_0}{z - z_2},\quad
     d(z) = \frac{d_\infty  z  - d_0}{z - z_2},
   \end{equation*}
for an element $(a_0, b_0, c_0, d_0) \in \CC^4$ such that
   \begin{equation*}
 (a_\infty z - a_0)( d_\infty z - d_0) -   (b_\infty z - b_0)(c_\infty z -c_0) =
     (z - z_1)(z- z_2) (a_\infty d_\infty - b_\infty c_\infty).
   \end{equation*}

   The above equation translates into the system of a linear
   equation and a quadric equation on $(a_0, b_0, c_0, d_0) \in \CC^4$, so our moduli space is encoded by the following complex affine variety:
   \begin{multline*}
\mc M(z_1,z_2)=
\big \{  (a_0, b_0, c_0, d_0) \in \CC^4 | \\
       -a_0 d_\infty - a_\infty d_0 + b_0 c_\infty + b_\infty c_0 
       = ( - z_1 - z_2) (a_\infty d_\infty - b_\infty c_\infty), \\
      a_0 d_0 - b_0 c_0 = 
       z_1 z_2 (a_\infty d_\infty - b_\infty c_\infty) \big \}
     \end{multline*}
     We conclude that, in this example, our moduli space $\mc M(z_1,z_2)$ is described by the complete intersection of a hyperplane and a  quadric in $\CC^4$, therefore by a quadric on $\CC^3$. For example,
     say $(a_\infty, b_\infty, c_\infty, d_\infty) = (1,0,0,1)$,
     and $z_1 = -m, z_2 = m$ for some $m \in \CC^\times$, then the linear equation implies that $d_0 = -a_0$,
     and the quadratic equation gives a canonical form of the smooth affine quadric
     surface 
     \begin{equation*}
       \label{eq:quadric}
         a_0^2 + b_0 c_0  = m^2 
     \end{equation*}
on  $\CC^3 = (a_0, b_0, c_0)$.

\begin{remark}
  In the limit when the singularities $z_1$ and $z_2$ collide, that is where $m \to 0$,
  the quadric becomes singular: $a_0^2 + b_0 c_0 = 0$. The resolved
  singularity on a quadric obtained by blowing up the singularity is identified with $T^{*} \bb{CP}^1$.
  The $m$-deformed quadric $a_0^2 + b_0 c_0 = m^2$ can be identified
  with the total space of an affine line bundle over $\bb{CP}^1$. This base $\bb{CP}^1$ is the orbit
  of the fundamental miniscule weight in the affine Grassmanian of the group $\GL_2$.
  We see that the moduli space $\mc M(z_1,z_2)$ of multiplicative Higgs bundles in the case of two miniscule co-weight singularities for $\GL_2$ is an affine line bundle over the flag variety $\bb{CP}^1$, 
where, locally, the 1-dimensional base arises 
from the insertion of one singularity,
and 1-dimensional fiber comes from the insertion of the other.
\end{remark}

\begin{remark}
  The canonical coordinates $a \in \CC, b \in \CC^{\times}$ on the quadric (\ref{eq:quadric}) are given by
  \begin{equation*}
    a_0 = a, \qquad b_0 = b(m - a), \qquad c_0 = b^{-1}(m+a)
  \end{equation*}
  with Poisson brackets
  \begin{equation*}
    {a,b} = b
  \end{equation*}
  and symplectic form $\d a \wedge \frac{\d b} b$.  
\end{remark}

\begin{remark}
We can likewise calculate the multiplicative Hitchin section as in Remark \ref{q_opers_remark} in this example.  The Steinberg section for the group $G = \GL_2$ sends a pair of eigenvalues $(s,t) \in T/W$ to the element
\[\begin{pmatrix}-s-t & st \\ 1 & 0\end{pmatrix} \in G.\]
In order to describe the multiplicative Hitchin section, let us use a framing that lands in the Steinberg section, say $(a_\infty, b_\infty,c_\infty, d_\infty) = (0,-1,1,0)$.  Then the moduli space $\mc M(z_1,z_2)$ as above with this framing can be identified with the smooth affine quadric surface 
    \begin{equation*}
        a_0d_0 + b_0^2  = m^2 .
    \end{equation*}
The multiplicative Hitchin section lies within the locus where $d_0=0$, i.e. to the pair of lines $b_0 = \pm m$ inside our quadric surface.  The fixed framing at infinity picks out the line $b_0 = - m$, or the locus of matrices of the form 
\[g(z) = \begin{pmatrix} \frac{a_0}{z-m} & - \frac{z+m}{z-m} \\ 1 & 0\end{pmatrix}.\]
In Poisson coordinates $a,b$ as above, the multiplicative Hitchin section is equivalent to the Lagrangian line $a = m$: a section of the projection map onto the space $\CC^\times$ on which $b$ is a coordinate.    

For a more general singularity datum, but for the group $G=\GL_2$ the multiplicative Hitchin section admits a similar description, where now the datum $(D,\omega^\vee)$ is encoded by a rational function $p_1(z)/p_2(z)$ where $p_1$ and $p_2$ are monic polynomials of the same degree, say $d$.  The multiplicative Hitchin section then can be described as the locus of all matrices of the form
\[g(z) = \begin{pmatrix} \frac{q(z)}{p_2(z)} & - \frac{p_1(z)}{p_2(z)} \\ 1 & 0\end{pmatrix},\]
where $q(z)$ is a polynomial of degree less than $d$ (so the section is $d$-dimensional).  The sections which are in fact $\SL_2$-valued are those with $p_1 = p_2$.
\end{remark}

More general explicit cases of $\mhiggs^\fr_G(C)$ with parametrization by Darboux coordinates for $C = \bb{CP}^1$ and $G=\GL_n$ case are discussed in \cite{FrassekPestun}.
  
\subsection{Symplectic Structures for General $G$} \label{general_symplectic_sec}
We begin analysis of the general case by briefly discussing the holomorphic symplectic structure on the moduli space of periodic monopoles on $\bb{CP}^1$ following the work of Cherkis and Kapustin \cite{CherkisKapustin1, CherkisKapustin3}.  This structure arises from the description we gave as a hyperk\"ahler quotient.  To describe it specifically, let $\delta^{(1)} \mc A$ and $\delta^{(2)} \mc A$ be two tangent vectors at $(\bo P, \mc A)$ to the moduli space of monopoles.  Recall that $\mc A$ denotes the combination $\nabla_t - i\Phi \d t$.  Choose representatives for these two tangent vectors of the form $\alpha_i$ and $\alpha'_i$ respectively  in the \v Cech resolution we described above.  Then we can write the holomorphic symplectic form coming from the hyperk\"ahler reduction in terms of the symplectic pairing on the infinite-dimensional vector space $\mc V$, which is given by the Killing form $\kappa$ on $\gg$ along with the wedge pairing of differential forms.  So summing over the local patches in our \v Cech resolution we can write it as
\begin{align*}
\omega_{\mr{mon}}(\delta^{(1)} \mc A, \delta^{(2)} \mc A) &= \int_{M} \kappa(\delta^{(1)} \mc A, \delta^{(2)} \mc A) \d \ol{z} \\
&= \sum_{i=1}^k \int_{U_i \times S^1} \kappa(\alpha_i, \alpha'_i) \d z \d \ol{z} \d t
\end{align*}
where the contributions to the integral away from the $U_i$ vanish.  We'll discuss this further in Section \ref{hyperkahler_section}

Our goal in this section will be describe a symplectic structure on the moduli space of multiplicative Higgs bundles -- the rational analogue of Hurtubise and Markman's symplectic structure -- and then prove that it's equivalent to this symplectic form under the equivalence between multiplicative Higgs bundles and periodic monopoles.

\begin{remark}
From now on we'll write $\langle - , - \rangle$ to denote the residue pairing between elements of $L\gg$.  That is, 
\[\langle g_1, g_2 \rangle = \oint_{\bb D^\times} \kappa(g_1, g_2).\]
\end{remark}

\begin{lemma} \label{sym_nondegeneracy_lemma}
There is a natural non-degenerate anti-symmetric pairing on the moduli space $\mhiggs_G^{\fr}(\bb{CP}^1,D,\omega^\vee)$.  In terms of our \v Cech description it is described by the formula 
\[\omega((\{\alpha_i\}, \{\beta_i\}), (\{\alpha'_i\},\{\beta_i'\})) = \frac 12 \sum_i \langle \rho_g^*(\alpha'_i + \alpha'_0)|_{U^\times_i}, \rho_g^*(\Ad_g^*)^{-1}(\beta_i) \rangle - \langle \rho_g^*(\alpha_i + \alpha_0)|_{U^\times_i}, \rho_g^*(\Ad_g^*)^{-1}(\beta'_i) \rangle,\]
where we use the Killing form to identify $\beta'_i$ with a $\gg^*$-valued form, and where $\rho_g^*$ is the pullback along the right multiplication by $g$.  In this expression the sum is over $i=1,\ldots,k$ and $i=\infty$.
\end{lemma}

\begin{remark}
How should we think about this structure?  There's an intuitive description, just as one sees in Hurtubise and Markman's elliptic moduli space.  The pairing is induced from the natural equivalence between the tangent and cotangent spaces of the moduli space of multiplicative Higgs bundles as described in Section \ref{def_section}.  That is, there's a map of complexes of sheaves
\[\xymatrix{
(\mc F^\fr_{(P,g)})^*[2] \ar@{=}[r] \ar[d] &\Big(\gg^*_P(-D)[1] \otimes \OO(-c) \ar[d]^{\kappa \circ \Ad_g^*} \ar[r]^{\Ad_g^*} &\gg^*_P \otimes \OO(-c)\Big)\ar[d]^{\Ad_g \circ \kappa^{-1}} \\
\mc F^\fr_{(P,g)} \ar@{=}[r] &\Big(\gg_P[1] \otimes \OO(-c) \ar[r]^{\Ad_g} &\gg_P(D) \otimes \OO(-c)\Big)
}\]
where here $\kappa$ denotes the isomorphism from $\gg_P \to \gg^*_P$ induced by the Killing form.  The top line is the Serre dual complex to the bottom line; note that the incorporation of the framing was necessary for this to be the case (that is, we're using the relative Calabi-Yau structure on the pair $(\bb{CP}^1, c)$).  Taking 0th hypercohomology we obtain a map from the cotangent space to the tangent space of our moduli space of multiplicative Higgs bundles.
\end{remark}

\begin{remark}
At this point we will not prove that the pairing is a symplectic structure: we will not verify the Jacobi identity.  While we expect that it is possible to prove this directly using the techniques of \cite[Section 5]{HurtubiseMarkman}, instead we'll see this below by proving that the pairing coincides with the symplectic pairing on the moduli space of periodic monopoles.
\end{remark}

\begin{proof}
We need only verify that the pairing described is non-degenerate.  It suffices to check non-degeneracy for each individual summand, i.e. that for fixed $i$, given $\alpha_i$ if the expression $\langle \rho_g^*(\alpha_i)|_{U^\times_i}, \rho_g^*(\Ad_g^*)^{-1}(\beta'_i) \rangle$ vanishes for all $\beta'_i$ then $\alpha_i=0$ (and the same with $\alpha$ and $\beta$ interchanged).  This is immediate because the residue pairing $\langle - , - \rangle$ is non-degenerate and the map $(\Ad_g^*)^{-1}$ is an isomorphism on the open set $U^\times_i$.
\end{proof}

\begin{remark}
If we do not fix local data at the punctures, the full moduli space $\mhiggs_G^\fr(\bb{CP}^1,D)$, or indeed the moduli space $\mhiggs_G^{\text{fr,sing}}(\bb{CP}^1)$ with arbitrary singularities, will have an ind-Poisson structure, analogous to the ind-Poisson structure in the elliptic moduli space of Hurtubise and Markman.  One uses the filtration discussed in Remark \ref{ind_structure_remark} in order to define this structure: one needs to check that the moduli space  $\mhiggs_G^\fr(\bb{CP}^1,D, \preceq \omega^\vee)$ is Poisson, and that the inclusion of the symplectic moduli space $\mhiggs_G^\fr(\bb{CP}^1,D, \omega^\vee)$ is a Poisson map.  We won't discuss this in any more detail here, except to remark that this should follow using the techniques of \cite[Section 7]{HurtubiseMarkman}.  We will, however, discuss the close relationship between this Poisson structure and the rational Poisson Lie group, and the consequences for quantization, in Section \ref{quantization_section}.
\end{remark}

In fact the moduli space is not only symplectic, but the total space of a completely integrable system -- the multiplicative Hitchin system described in Definition \ref{mult_Hitchin_system_def}.  We already know that the generic fibers are half-dimensional tori, it remains to verify that they're isotropic.

\begin{prop} \label{Hitchin_isotropic_prop}
Generic fibers of the multiplicative Hitchin fibration, in the rational case, are isotropic.
\end{prop}

\begin{proof}
In order to check this we need to describe the subspace of the tangent space to the moduli space $\mhiggs^\fr_G(C,D,\omega^\vee)$ at a point $(P,g)$ tangent to the Hitchin fiber in our \v Cech description.  Recall that the tangent space to $\mhiggs^\fr_G(C,D,\omega^\vee)$ can be identified as the space of \v Cech cocycles $(\alpha_0, \{\alpha_i\}, \alpha_\infty, \{\beta_i\}, \beta_\infty)$, where in particular the $\beta_i$ and $\beta_\infty$ are local sections of the adjoint bundle $\gg_P$.  We will show that, for a \v Cech cocycle lying tangent to the Hitchin fiber, the elements $\beta_i, \beta_\infty$ must be valued in the associated bundle of centralizers $\mf c_g$ of $g$.  Having established this the claim follows because the operator $\Ad_g$, and therefore the operator $(\Ad_g^*)^{-1}$, acts trivially on the centralizer of $g$.  Replacing this operator with the identity in our expression for the symplectic form from Lemma \ref{sym_nondegeneracy_lemma} results in a symmetric expression.  Since it's also antisymmetric, it must vanish, which implies the desired isotropic condition.

So it remains to establish this restriction on tangent vectors to the Hitchin fiber.  Note that $\beta_i$ lies in $\mf c_g$ if and only if $\alpha_i - \alpha_0 = 0$.  If the derivative $\d \pi$ of the multiplicative Hitchin map annihilates a cocycle, then in particular the elements $\alpha_i - \alpha_0$ are functions without poles at $z_i$.  Indeed, the cocycle condition implies that $\alpha_i - \alpha_0$ has the same residue as $\Ad_g$ applied to a regular function, in particular, by the genericity assumption, its residue is regular semisimple.  On the other hand, being in the kernel of $\d \pi$ means vanishing under the map $\gg(\!(z)\!) \to \mf t(\!(z)\!)/W$ -- the two conditions together implies that the simple pole of $\alpha_i - \alpha_0$ has zero residue.  That means that the tangent vectors to the Hitchin fiber -- which are annihilated by $\d \pi$ -- are cohomologous to cocycles with $\alpha_i - \alpha_0 = 0$ as required.
\end{proof}

We'll now begin the comparison between the pairing described above and the symplectic structure on the moduli space of periodic monopoles.

\begin{prop} \label{qconn_symp_description}
The non-degenerate pairing on the moduli space of multiplicative Higgs bundles can be written in the form
\begin{equation}
\label{eq:resid}
\omega(\delta g, \delta g') = \sum_{i=1}^d \langle b_i^L g^{-1}, b^{'R}_i g^{-1} \rangle - (b \leftrightarrow b')
\end{equation}
where $\delta g$ is represented on $U_i$ by a pair $(b^L, b^R) \in \ad(g)$.  Here and throughout the notation ``$- (b \leftrightarrow b')$'' denotes antisymmetrization.
\end{prop}

\begin{proof}
 Specify a representative cocycle $(\alpha_\infty, \{\alpha_i\}, \alpha_0, \beta_\infty, \{\beta_i\})$ for an element of the first \v Cech cohomology group. By addition of a coboundary we can assume that $\alpha_0=0$, and to force $\alpha_\infty$ to land in $z\gg[[z]]$ and each $\alpha_i$ to land in $z^{-1}\gg[[z]]$.  We have now fixed a representative cocycle -- there's no further gauge freedom.  Ignoring the factor at infinity -- since after we add these coboundaries the residue pairing there vanishes -- the pairing we end up with looks like 
\begin{align*}
\omega(\delta g, \delta g') &= \frac 12 \sum_{i=1}^d \langle \rho_g^*(\alpha'_i + \alpha'_0)|_{U^\times_i},\rho_g^* (\Ad_g^*)^{-1}( \beta_i) \rangle - ((\alpha,\beta) \leftrightarrow (\alpha',\beta')\\ 
&= \frac 12 \sum_{i=1}^d \langle \rho_g^*(\alpha'_i + \alpha'_0)|_{U^\times_i}, \rho_g^*(\Ad_g^*)^{-1}(\Ad_g^{-1}(\alpha_i - \alpha_0)) \rangle - (\alpha \leftrightarrow \alpha')\\
&= \frac 12 \sum_{i=1}^d \langle \rho_g^*(\alpha'_i)|_{U^\times_i}, \rho_g^*(\Ad_g^*)^{-1}(\Ad_g^{-1}(\alpha_i)|_{U^\times_i}) \rangle - (\alpha \leftrightarrow \alpha').
\end{align*}

We can compute the composite operator $(\Ad_g^*)^{-1}\Ad_g^{-1}$ on $\mr{ad}(P)(U^\times_i)$.  It's given by the inverse of the operator $\Ad_g\Ad_g^*$ which sends a pair $(b^L, b^R) \in \mr{ad}(P)(U^\times_i)$ to $(b^L - b^R, b^R - b^L) \sim (b^L + g^{-1} b^L g, b^R - g b^R g^{-1})$.  Denote this by $((1+A_g)b^L,(1-A_{g^{-1}})b^R)$.  We can describe the inverse using the expansion $(1+A_g)^{-1} = 1 - A_g + A_g^2 + \cdots$.  After applying our gauge transformation above the remaining degree of freedom in $\alpha_i$ is its $z^{-1}$ term.  Each time we apply $A$ it raises the order in $z$ of this term by one, so only the linear summand $-A_g$ of $(1+A_g)^{-1}$ contributes to the residue pairing.  In the pairing we need to use the invariant pairing on $\gg((z)) \oplus \gg((z))$ that vanishes on the subalgebra spanned by $(X, A_g(X))$, so we take the difference of the residue / Killing form pairings on the two summands.
\begin{align*}
\omega(\delta g, \delta g') &= \frac 12 \sum_{i=1}^d  - \langle \rho_g^*A_g b^{'L}_i, \rho_g^*b^{R}_i \rangle - \langle \rho_g^*b^{'L}_i, \rho_g^*A_{g^{-1}}b^{R}_i \rangle- (b \leftrightarrow b') \\ 
&= \sum_{i=1}^d \langle b^L_ig^{-1}, b^{'R}_i g^{-1} \rangle  - (b \leftrightarrow b')
\end{align*}
using cyclic invariant to identify the two terms. 
\end{proof}

We'll use this description of the pairing to compare our symplectic structure with the Poisson structure of the rational Poisson Lie group in Section \ref{quantization_section}.

We can now establish our main result.
\begin{theorem} \label{symplectic_comparison_thm}
The symplectic structure on $\mon_G^\fr(\bb{CP}^1 \times S^1,D \times\{0\},\omega^\vee)$ and the pullback of the non-degenerate pairing on $\mhiggs_G^{\text{ps,fr}}(\bb{CP}^1,D,\omega^\vee)$ under $H$ coincide.
\end{theorem}

\begin{proof}
This is straightforward now, by combining the two local descriptions of the symplectic structure on monopoles and multiplicative Higgs bundles on a neighborhood of a puncture with the description in Proposition \ref{local_derivative_description_prop}.  So, let us begin by taking the symplectic form $\omega_{\mr{mHiggs}}$ on the moduli space $\mhiggs_G^{\text{ps,fr}}(\bb{CP}^1,D,\omega^\vee)$ and evaluating it at the image under $\d H$ of two elements $(\alpha_i, \alpha'_i) = (\d_{\mr{mon}}(b_i), \d_{\mr{mon}}(b_i'))$.  Let us write $b_i(0) = b_i^L$ and $b_i(2\pi) = b_i^R$ for brevity.  Likewise for consistency with the calculations above let us denote the image $H(\mc A)$ under the holonomy map by $g$.

According to Proposition \ref{local_derivative_description_prop} we have
\begin{align*}
\omega_{\mr{mHiggs}}(\d H(\alpha_i), \d H(\alpha_i')) &= \omega_{\mr{mHiggs}}(\d H(\d_{\mr{mon}}(b_i)), \d H(\d_{\mr{mon}}(b_i'))) \\
&= \omega_{\mr{mHiggs}}(b_i^Rg - gb_i^L,b_i^{'R}g - gb_i^{'L}).
\end{align*}
We can write this in terms of a pairing on the bundle $\ad(g)$.  Represent the class $b_i^Rg - gb_i^L$ by the pair $(b_i^Rg, -gb_i^L)$.  Then applying the description of $\omega_{\mr{mHiggs}}$ provided by Proposition \ref{qconn_symp_description} we have
\begin{align*}
\omega_{\mr{mHiggs}}(\d H(\alpha_i), \d H(\alpha_i')) &= \omega_{\mr{mHiggs}}((b_i^Rg, -gb_i^L),(b_i^{'R}g, -gb_i^{'L})) \\ 
&= - \langle  b_i^R , g b_i'^L g^{-1}\rangle - (b \leftrightarrow b')
\end{align*}
where the remaining terms are killed by the antisymmetrization.

On the other hand, we can evaluate the pairing $\omega_{\mr{Mon}}(\alpha_i,\alpha'_i) = \omega_{\mr{Mon}}(\d_{\mr{mon}}(b_i),\d_{\mr{mon}}(b_i'))$ via integration by parts. We'll compute the symplectic pairing for the full \v Cech complex, but it will split into a sum over open sets $U_i$.  The result is that 
\begin{align*}
\omega_{\mr{Mon}}(\{\alpha_i\},\{\alpha'_i\}) &= \sum_{i=0,\ldots,k,\infty} \int_{U_i} \kappa(\d_{\mr{mon}}(b_i) \wedge \d_{\mr{mon}}(b_i')) \d z - (b \leftrightarrow b')\\
&= \sum_{i=1,\ldots,k,\infty} \int_{\dd \bb D_i \times [0,2\pi]} \kappa(b_i - b_0, \d_{\mr{mon}}(b'_i)) \d z - (b \leftrightarrow b')
\end{align*}
using here the fact that $\d_{\mr{mon}}b_i = \d_{\mr{mon}} b_0$ on $U_i \cap U_0$ and that $\d_{\mr{mon}} b_i(t) = 0$ when $t = 0$ or $2\pi$.  By Stokes' theorem we then have
\begin{align*}
 \omega_{\mr{Mon}}(\{\alpha_i\},\{\alpha'_i\})&= \sum_{i=1,\ldots,k,\infty} \oint_{\dd \bb D_i} \kappa(b_i^R - b_0^R, b_i^{'R}) - \kappa(b_i^L - b_0^L, b_i^{'L}) - (b \leftrightarrow b') \\
 &= \sum_{i=1,\ldots,k,\infty} \oint_{\dd \bb D_i} - \kappa(b_0^R, b_i^{'R}) + \kappa(b_0^L, b_i^{'L}) - (b \leftrightarrow b').
\end{align*}
Pick out the summand corresponding to $U_i$.  Choose our potentials so that on the boundary $\dd \bb D_i$ we have $b_0^L = b_0^{'L} = 0$.  We can do this by setting $b^{R} = \delta g g^{-1}$ since $g$ is non-singular on $U_0$.  This choice means that on $U_0 \cap U_i$, since $\delta g  = b_{i} ^R g - g b_i^L = b_0^{R} g $ we can make the identification $ b_0^{R} = b_{i}^{R} - g b_{i}^{L} g^{-1}$.  Therefore
\[\omega_{\mr{Mon}}(\{\alpha_i\},\{\alpha'_i\}) = \sum_{i=1,\ldots,k,\infty} \langle(g b_i^{L} g^{-1}, b_i^{'R}\rangle \quad - \quad  (b \leftrightarrow b')\]
agreeing with the expression coming from $\mr{mHiggs}$. 
\end{proof}

In particular the non-degenerate pairing on the multiplicative Higgs moduli space is determined by a closed 2-form, and the multiplicative Hitchin fibers are Lagrangian.

\begin{corollary}
The moduli space $\mhiggs^\fr_G(\bb{CP}^1,D,\omega^\vee)$ has the structure of a completely integrable system.
\end{corollary}

\section{Hyperk\"ahler Structures} \label{hyperkahler_section}
The results of the previous section imply that the symplectic structure on $\mhiggs_G^{\text{ps,fr}}(\bb{CP}^1,D,\omega^\vee)$ extends canonically to a hyperk\"ahler structure.  In this section we'll compare the twistor rotation with the deformation to the moduli space of $\eps$-connections. We'll show that in an appropriate limit passing to a point $\eps$ in the twistor sphere we obtain a complex manifold equivalent to the moduli space of $\eps$-connections as a deformation of the moduli space of multiplicative Higgs bundles.

Let us begin by describing the hyperk\"ahler structure on the moduli space of periodic monopoles and how the holomorphic symplectic structure varies under rotation in the twistor sphere.  We'll begin by recalling that the moduli space of periodic monopoles can be described as a hyperk\"ahler quotient as in the work of Atiyah and Hitchin \cite{AtiyahHitchin}.  In a context with more general boundary data at $\infty$ this was demonstrated by Cherkis and Kapustin \cite{CherkisKapustin3} for the group $\SU(2)$, see also Foscolo \cite{FoscoloDef}[Theorem 7.12].  In the case of $\bb{CP}^1$ with a fixed framing the analysis is much easier.

Recall from Definition \ref{monopole_moduli_def} that the moduli space of periodic monopoles on $\bb{CP}^1$ with a fixed framing at $\infty$ and fixed Dirac singularities can be described as the quotient 
\[\mon_G^{\fr}(\bb{CP}^1 \times_\eps S^1_R, D, \omega^\vee) = \mu^{-1}(0)/\mc G,\]
where 
\begin{itemize}
\item The space $\mc V$ of fields is the space of pairs $(A,\Phi)$ consisting of framed connections $A$ on the trivial principal $G_\RR$-bundle $\bo P$ on $\bb{CP}^1 \times_\eps S^1_R$, $\Phi$ is a section of $\gg_{\bo P}$ vanishing at $\{\infty\} \times S^1_R$, and the pair $(A,\Phi)$ has a Dirac singularity with charge $\omega^\vee_{z_i}$ at each point $(z_i,t_0)$ in $D \times \{t_0\}$.
\item $\mu \colon \mc V \to \Omega^1(\bb{CP}^1 \times_\eps S^1_R; (\gg_\RR)_{\bo P})/\gg_\RR$ is the map sending a monopole $(A, \Phi)$ to the $\gg_\RR$-valued 1-form $\ast F_A - \d_A \Phi$ vanishing at $\{\infty\} \times S^1_R$.  
\item The group $\mc G$ of gauge transformations consists of automorphisms of the trivial bundle $\bo P$ fixing the framing.
\end{itemize}

This is an instance of hyperk\"ahler reduction.  Indeed the infinite-dimensional space $\mc V$ admits a hyperk\"ahler structure with symplectic forms defined by
\[\omega_i((A,\Phi),(A',\Phi')) = \int_{\bb{CP}^1 \times_\eps S^1_R} \left(\kappa(A_i,\Phi') - \kappa(\Phi, A_i') + \sum_{j,k=1,2,3} \eps_{ijk} \kappa(A_j,A'_k)\right) \d x \d y \d t, \]
where $i=1,2,3$, and where $A$ can be written as $A_1 \d t + A_2 \d x + A_3 \d y$ with respect to a choice $\d z = \d (x + iy)$ of meromorphic volume form on $\bb{CP}^1$ with a second order pole at $\infty$.  This descends to a hyperk\"ahler structure on the quotient; see \cite[Section 1.4.2]{FoscoloThesis}.

Let us restrict to the case where $\eps=0$ for now.  Consider the holomorphic symplectic structure $\omega_2 + i \omega_3$.  We can alternatively write the formula above for this choice of holomorphic symplectic structure in the form
\[\omega((A_0, \mc A), (A_0', \mc A')) = \int_{\bb{CP}^1 \times S^1_R} \kappa(A_0, \mc A') - \kappa(\mc A, A_0') \d x \d y \d t,\]
where $A_0 = A_2 + i A_3$ and $\mc A = A_1 - i \Phi$.  

\begin{remark}
We can compare this to the expression described at the beginning of Section \ref{general_symplectic_sec}. Indeed, up to a factor of $i$ we can rewrite this expression as
\[\omega((A_0, \mc A), (A_0', \mc A')) = \int_{\bb{CP}^1 \times S^1_R} \kappa(A_0\d z + \mc A \d t, A_0' \d z + \mc A' \d t) \d \ol{z}.\]
This expression agrees with the symplectic pairing on monopoles in Section \ref{general_symplectic_sec} as a pairing of $\gg$-valued 1-forms on $\CC \times \RR$, which are holomorphic in the $\CC$-direction, where we set $\delta \mc A = A_0\d z + \mc A \d t$.
\end{remark}

This description for the hyperk\"ahler structure also tells us clearly what happens when we perform a rotation in the twistor sphere.  The following statement follows by identifying the holomorphic symplectic structure at $\eps$ in the twistor sphere by applying the corresponding rotation in $\SO(3)$ to the coordinates $x$, $y$ and $t$ on $\RR^3$ in the expression for the holomorphic symplectic structure at $0$.  Note that when we calculate the symplectic form as an integral as above, it's enough to take the integral over just $\RR^2 \times S^1_R$ instead of $\bb{CP}^1 \times S^1_R$.

Notationally, from now on we'll write the holomorphic symplectic pairing in the abbreviated form $\kappa(A_0, \mc A') - \kappa(\mc A, A_0') = \kappa(\delta^{(1)} \mc A, \delta^{(2)} \mc A)$, and we'll write $\dvol$ for the volume form $\d x \wedge \d y \wedge \d t$ on $\CC \times \RR \sub \bb{CP}^1 \times S^1_R$.

\begin{prop}
Let $\zeta$ be a point in the twistor sphere.  The holomorphic symplectic form on the moduli space $\mon_G^{\fr}(\bb{CP}^1 \times S^1_R, D, \omega^\vee)$ of periodic monopoles is given by the formula 
\[\omega_\eps(\delta^{(1)}\mc A, \delta^{(2)}\mc A) = \int_{\RR^2 \times S^1_R} \kappa(\delta^{(1)} \mc A, \delta^{(2)} \mc A) \zeta(\dvol),\]
where we identify $\zeta$ with an element of $\SO(3)$ and apply this rotation to the volume form $\dvol$.  Equivalently we can identify $\RR^2 \times \bb{CP}^1$ with the quotient $\RR^3 \times L$ where $L$ is the rank one lattice $\{0\}^2 \times 2\pi R\ZZ$ and identify the rotated holomorphic symplectic form with 
\[\omega_\zeta(\delta^{(1)}\mc A, \delta^{(2)}\mc A) = \int_{\RR^3/(\zeta(L))} \kappa(\delta^{(1)} \mc A, \delta^{(2)} \mc A) \dvol.\]
\end{prop}

Now, we'll perform our identification of the twistor deformation on the multiplicative Higgs side by rewriting this twistor rotation in a somewhat different way in the large $R$ limit.  In this limit we can identify the twistor rotation with the deformation obtained by replacing the product $\bb{CP}^1 \times S^1_R$ with the product twisted by an automorphism of $\RR^2$.  More precisely, the rotated symplectic structure can then be described in the following way.

\begin{theorem} \label{HK_rotation_thm}
Let $\left(\mhiggs^{\text{ps,fr}}_G(\bb{CP}^1,D,\omega^\vee)\right)_{\zeta,R}$ be the family of complex manifolds obtained by pulling back the holomorphic symplectic structure $\omega_{\zeta,R}$ on $\mon_G^{\mr{ps},\fr}(\bb{CP}^1 \times S^1_R,D,\omega^\vee)$ at $\zeta$ in the twistor sphere.  In the limit where $R \to \infty$ with $2 \pi \zeta R = \eps$ fixed this complex structure coincides with the complex structure on $\epsconn^{\text{ps,fr}}_G(\bb{CP}^1,D,\omega^\vee)$, in the sense that the complex structures converge pointwise over the tangent bundle to the moduli space of monopoles. 
\end{theorem}

\begin{proof}
First let us note that the 3-manifolds $\bb{CP}^1 \times_\eps S^1_R$ for varying values of $R$ and $\eps$ are all diffeomorphic (via a diffeomorphism fixing the circle at $\infty$.  This means that we can identify the moduli spaces in question as smooth manifolds by choosing such a diffeomorphism.  Then, according to Theorem \ref{monopole_qconn_comparison_thm} it's enough to show that the complex structure on $\mon_G^{\text{ps,fr}}(\bb{CP}^1 \times S^1_R,D,\omega^\vee)$ at the point $\zeta$ converges to the complex structure on the $\eps$-deformation $\mon_G^{\mr{ps},\fr}(\bb{CP}^1 \times_\eps S^1_R,D,\omega^\vee)$.  The key observation that we'll use is that, fixing $2 \pi \zeta R = \eps$ we can identify the rotated lattice with a \emph{sheared} lattice.  Namely, when we rotate the rank one lattice $L_{R,0} = (0,0,2\pi R)\ZZ$ by $\zeta$ we obtain the lattice
\[\zeta(L_{R,0}) = \frac {1}{1+|\zeta|^2} (\mr{Re}(\eps), \mr{Im}(\eps), 2\pi R (1-|\zeta|^2) )\ZZ.\]
We can view the quotient of $\RR^3$ by the rotated lattice as a sheared lattice with different radius and shear.  Specifically, we identify
\[\RR^3 / \zeta(L_{R,0}) \iso \RR^2 \times_{\frac{\eps}{1+|\zeta|^2}} S^1_{R \frac {1-|\zeta|^2}{1+|\zeta|^2}}.\]

In order to identify the desired complex structures it's enough to identify the full family of symplectic structures on the $\eps$-deformation with the appropriately rotated family of symplectic structures without deforming.  So, if we apply the twistor rotation by $\zeta$ to the holomorphic symplectic structure $\omega_2 + i\omega_3$ on $\mon_G^{\text{ps,fr}}(\bb{CP}^1 \times S^1_R,D,\omega^\vee)$ (the argument will be identical for the other points in the $\bb{CP}^1$ family of holomorphic symplectic structures) we obtain the pairing
\[\omega_\zeta(\delta^{(1)}\mc A, \delta^{(2)}\mc A) = \int_{\RR^3/(\zeta(L_{R,0}))} \kappa(\delta^{(1)} \mc A, \delta^{(2)} \mc A) \dvol.\]
On the other hand we can identify the holomorphic symplectic structure on $\mon_G^{\mr{ps},\fr}(\bb{CP}^1 \times_\eps S^1_R,D,\omega^\vee)$ as
\[\omega_\zeta(\delta^{(1)}\mc A, \delta^{(2)}\mc A) = \int_{\RR^3/(L_{R,\eps})} \kappa(\delta^{(1)} \mc A, \delta^{(2)} \mc A) \dvol.\]
In particular, by tuning the radius and the shear parameter the above calculation means there's a hyperk\"ahler equivalence between the moduli spaces $\mon_G^{\text{ps,fr}}(\bb{CP}^1 \times S^1_R,D,\omega^\vee)$ -- after rotating the twistor sphere by $\zeta$ -- and $\mon_G^{\mr{ps},\fr}(\bb{CP}^1 \times_{\frac{\eps}{1+|\zeta|^2}} S^1_{R \frac {1-|\zeta|^2}{1+|\zeta|^2}},D,\omega^\vee)$.  

Now, let us consider the large $R$ limit.  We'd like to note that the following difference of symplectic pairings converges to zero pointwise as $R\to \infty$.
\[ \left\lvert \int_{\RR^3/(L_{R \frac {1-|\zeta|^2}{1+|\zeta|^2},\frac{\eps}{1+|\zeta|^2}})} \kappa(\delta^{(1)} \mc A, \delta^{(2)} \mc A) \dvol - \int_{\RR^3/(L_{R,\eps})} \kappa(\delta^{(1)} \mc A, \delta^{(2)} \mc A) \dvol \right \rvert.\]
Rescaling the radii of the circles only rescales the pairing by an overall constant, so we can rewrite this as 
\[ R\left\lvert \frac {1-|\zeta|^2}{1+|\zeta|^2} \int_{\RR^3/(L_{1,\frac{\eps}{1+|\zeta|^2}})} \kappa(\delta^{(1)} \mc A, \delta^{(2)} \mc A) \dvol - \int_{\RR^3/(L_{1,\eps})} \kappa(\delta^{(1)} \mc A, \delta^{(2)} \mc A) \dvol \right \rvert\]
in which we recall that $\zeta = \frac \eps {2\pi R}$.  Write $\delta^{(i)}\mc A_\zeta$ for the image of the deformation $\delta^{(i)}\mc A$ under the diffeomorphism induced by identifying the $\frac{\eps}{1+|\zeta|^2}$-twisted product with the $\eps$-twisted product, so $\delta^{(i)}\mc A_\zeta(z,t) = \delta^{(i)}\mc A(z-\frac{\eps}{1+|\zeta|^2},t)$.  Note that when we perform the integral the pairing between $\delta^{(1)}\mc A$ and $\delta^{(2)}\mc A$ and the pairing between $\delta^{(1)}\mc A_\zeta$ and $\delta^{(2)}\mc A_\zeta$ coincide.  In other words, our difference of pairings becomes
\begin{align*} 
&R\left\lvert \int_{\bb{CP}^1 \times_\eps S^1_1} \frac {1-|\zeta|^2}{1+|\zeta|^2} \kappa(\delta^{(1)} \mc A_\zeta, \delta^{(2)} \mc A_\zeta)  -  \kappa(\delta^{(1)} \mc A, \delta^{(2)} \mc A) \dvol \right \rvert \\
&\quad = R \left\lvert \frac {1-|\zeta|^2}{1+|\zeta|^2} - 1 \right\rvert \left\lvert \int_{\bb{CP}^1 \times_\eps S^1_1} \kappa(\delta^{(1)} \mc A, \delta^{(2)} \mc A) \dvol \right \rvert \\
&\quad = \left\lvert\frac {2R |\eps|^2}{4 \pi^2 R^2 + |\eps|^2} \right \rvert \left\lvert \int_{\bb{CP}^1 \times_\eps S^1_1} \kappa(\delta^{(1)} \mc A, \delta^{(2)} \mc A) \dvol \right \rvert.
\end{align*}
This converges to zero pointwise as $R \to \infty$.  The same calculation proves convergence for the other symplectic structures in the hyperk\"ahler family, and therefore convergence for the difference of the two complex structures.
\end{proof}

\begin{example} \label{Nahm_example}
In the case where the group $G=\GL_n$ it's possible to describe the multiplicative Hitchin system on $\bb{CP}^1$ in terms of an \emph{ordinary} Hitchin system on $\CC^\times$, but for a different group $\GL_k$ and with different singularity data.  This relationship is given by the \emph{Nahm transform}.  

Let us give an informal explanation.  From a gauge-theoretic point of view, the Nahm transform arises from the Fourier-Mukai transform for anti-self-dual connections on a 4-torus $T$, relating an of ASD connection on a vector bundle to an ASD connection on the bundle of kernels for the associated Dirac operator, a vector bundle on the dual torus $T^\vee$ (see the book of Donaldson--Kronheimer \cite[Section 3.2]{DonaldsonKronheimer}).  There is a Nahm transform relating the ordinary Hitchin system to the moduli space of periodic monopoles obtained in an appropriate limit: consider the torus $T = S^1_{r_1} \times S^1_{r_2} \times S^1_{r_3} \times S^1_{R}$.  The moduli space of periodic monopoles on $\CC$ arises in the limit where $r_1$ and $r_2$ go to $\infty$ and $r_3$ goes to zero.  On the other hand, the Hitchin system on $\CC^\times$ arises in the dual limit where $r_1$ and $r_2$ go to zero and $r_3$ goes to $\infty$.  The Nahm transform in this setting -- i.e. for periodic monopoles -- was developed by Cherkis and Kapustin \cite{CherkisKapustin2}, and worked out in detail in the case $n=2$ (they also give a brane description of the transform: see \cite[Section 2]{CherkisKapustin2}).

One then has to match up the singularity data.  In brief, the multiplicative Hitchin system for the group $\GL_n$ with $k$ singularities at $z_1, \ldots, z_k$ is equivalent to the ordinary Hitchin system on $\CC^\times$ for the group $\GL_k$ with $n$ regular singularities, which residues determined by the original positions $z_i$ and positions determined by the original local data $\omega^\vee_{z_i}$.  We refer to \cite[Section 7.1]{NekrasovPestun} for more details.

In the example where we have the Nahm transform available, we conjecture that the multiplicative geometric Langlands equivalence recovers the ordinary geometric Langlands equivalence with tame ramification.  That is, we expect the following.

\begin{claim}
Under the Nahm transform, Pseudo-Conjecture \ref{multLanglands} in the rational case for the group $\GL(n)$ becomes the ordinary geometric Langlands conjecture on $\bb{CP}^1$ with tame ramification.
\end{claim}
\end{example}

\begin{remark}
  The limit $R \to \infty$ with $2 \pi \zeta R = \eps$ appearing in the hyperk\"ahler rotation
  theorem \ref{HK_rotation_thm} was first suggested by Gaiotto \cite{GaiottoTBA} on the other side of the Nahm transform.
\end{remark}

\section{The Rational Poisson Lie Group and the Yangian} \label{quantization_section}
Let us proceed by discussing the connection between the symplectic structure on the moduli space of multiplicative Higgs bundles in the rational case and the rational Poisson Lie group.  This connection should be compared to the connection between the elliptic moduli space and the elliptic quantum group in the work of Hurtubise and Markman \cite[Theorem 9.1]{HurtubiseMarkman}.  

Consider the infinite-type moduli space $\mhiggs^{\text{fr,sing}}(\bb{CP}^1)$ of multiplicative Higgs bundles with arbitrary singularities.  There is a map
\[r_\infty \colon \mhiggs^{\text{fr,sing}}(\bb{CP}^1) \to G_1[[z^{-1}]],\]
defined by restricting a multiplicative Higgs bundle to a formal neighborhood of infinity.  In particular we can restrict $r_\infty$ to any of the finite-dimensional subspaces $\mhiggs^\fr_G(\bb{CP}^1,D,\omega^\vee)$ (the symplectic leaves).  Here the notation $G_1[[z^{-1}]]$ denotes Taylor series in $G$ with constant term 1.  Recall that $G_1[[z^{-1}]]$ is a Poisson Lie group, with Poisson structure defined by the Manin triple $(G(\!(z)\!), G[[z]], G_1[[z^{-1}]])$ (see \cite{Shapiro, Williams}).

The Poisson Lie group $G_1[[z^{-1}]]$ has the structure of an ind-scheme using the filtration where $G_1[[z^{-1}]]_n$ consists of $G$-valued polynomials in $z^{-1}$ with identity constant term and where the matrix elements in a fixed faithful representation have degree at most $n$.  When we talk about the Poisson algebra $\OO(G_1[[z^{-1}]])$ of regular functions on $G_1[[z^{-1}]]$, we are therefore referring to the limit over $n \in \ZZ_{\ge 0}$ of the finitely generated algebras $\OO(G_1[[z^{-1}]]_n)$.  Due to the condition on the constant term the group $G_1[[z^{-1}]]$ is nilpotent, which means we can identify its algebra of functions with the algebra of functions on the Lie algebra.  That is:
\begin{align*}
\OO(G_1[[z^{-1}]]) &= \underset n \lim \OO(G_1[[z^{-1}]]_n) \\
&\iso \underset n \lim \sym(\gg_1^*[[z^{-1}]]_n).
\end{align*}
The Poisson bracket is compatible with this filtration, in the sense that it defines a map 
\[\{,\} \colon \OO(G_1[[z^{-1}]]_m) \otimes \OO(G_1[[z^{-1}]]_n) \to \OO(G_1[[z^{-1}]]_{m+n})\] 
for each $m$ and $n$.

\begin{theorem} \label{Poisson_Lie_Comparison_thm}
The map $r_\infty \colon \mhiggs^\fr_G(\bb{CP}^1,D,\omega^\vee) \to G_1[[z^{-1}]]$ is Poisson.  That is, the pullback of the Poisson structure on $\OO(G_1[[z^{-1}]])$ coincides with the Poisson bracket on the space of regular functions on the moduli space $\OO(\mhiggs^\fr_G(\bb{CP}^1,D,\omega^\vee))$ of multiplicative Higgs bundles. 
\end{theorem}

\begin{proof}
We just need to compute the pullback of the Poisson bracket on $\OO(G_1[[z^{-1}]])$ under $r_\infty$.  This Poisson bracket is most easily described in terms of the rational $r$-matrix: that is, we have an element $r$ in $\gg_1[[z^{-1}]] \otimes \gg_1[[w^{-1}]]$ defined by the formula $\frac \Omega {z-w}$ where $\Omega$ is the quadratic Casimir element of $\gg$.  The pairing defining the Poisson structure is given by contracting two cotangent vectors with the bivector field whose value at $g \in G_1[[z^{-1}]]$ is $(1 \otimes \Ad_g - \Ad_g \otimes 1)(r)$.  Concretely, given a pair of elements $X$ and $X'$ in the cotangent space $T^*_{G_1[[z^{-1}]], g}$, this computes the pairing
\[(X, X') \mapsto \int \kappa(X, g X' g^{-1}) - \kappa(g X g^{-1},X')\d z,\]
where $X$, for example, is identified with a $\gg^*$-valued meromorphic function on $\CC$, vanishing at $\infty$, via pullback under the right multiplication by $g^{-1}$ map.  Now, suppose the element $g$ is in the image of the restriction map $r_\infty$ for fixed local data $(D,\omega^\vee)$.  Then we note that the map $r_\infty^*$ sends a cotangent vector $X$ to the meromorphic global section $(b^L, b^R) = (gX, -Xg)$ of the bundle $\ad(g) \iso \ad(g)^*$ -- identifying the bundle with its dual using the invariant pairing  -- and so with this identification, our expression yields the formula
\[(X,X') \mapsto \int \kappa(b^Lg^{-1}, b^{'R}g^{-1}) - \kappa(b^{'L}g^{-1}, b^{R}g^{-1})\d z.\]
This recovers the expression given for the symplectic form in Proposition \ref{qconn_symp_description} by applying the residue formula.
\end{proof}

\begin{remark} \label{shapiro_leaves_remark}
It's instructive to compare this calculation with the work of Shapiro \cite{Shapiro} on symplectic leaves for the rational Poisson Lie group.  According to Shapiro, and for the group $G = \SL_n$, there are symplectic leaves in $G_1[[z^{-1}]]$ indexed by Smith normal forms, i.e. by a sequence $d_1, \ldots, d_n$ of polynomials where $d_i | d_{i+1}$ for each $i$.  This data is equivalent to a dominant coweight coloured divisor $(D,\omega^\vee)$ in the following way.  The data of the sequence of polynomials is equivalent to the data of an increasing sequence $(D_1, \ldots, D_n)$ of $n$ effective divisors in $\bb{CP}^1$ disjoint from $\infty$: the tuples of roots of the polynomials $d_i$.  Let $D = \{z_1, \ldots, z_k\}$ be the support of the largest divisor $D_n$, and for each $z_j$ let $\omega^\vee_{z_j} = (m_1, \ldots, m_n)$ be the dominant coweight where $m_i$ is the order of the root $z_j$ in the polynomial $d_i$.  The dominant coweight Shapiro refers to as the ``type'' of a leaf is, therefore, the sum of all these coweights.

To be a little more precise, Shapiro proves that for $G = \SL_n$ these symplectic leaves span the Poisson subgroup $\mc G \sub G_1[[z^{-1}]]$ of elements that can be factorized as the product of a polynomial element in $G_1[z^{-1}]$ and a monic $\CC$-valued power series.  In fact, the map $r_\infty$ (for any group, not necessarily $G = \SL_n$) factors through the subgroup $\mc G$: the Taylor expansion of all \emph{rational} $G$ valued functions can be factorized in this way.  As a consequence, Theorem \ref{Poisson_Lie_Comparison_thm} implies that our symplectic leaves for the group $\SL_n$ agree with Shapiro's symplectic leaves.
\end{remark}

This comparison has interesting consequences upon quantization.  The Poisson algebra $\OO(G_1[[z^{-1}]])$ has a well-studied quantization to the \emph{Yangian} $Y(\gg)$.  This is the unique Hopf algebra quantizing the algebra $U(\gg[z])$ with first order correction determined by the Lie bialgebra structure on $\gg[z]$.  Recall that $(\gg(\!(z)\!), \gg[z], \gg_1[[z^{-1}]])$ is a Manin triple, so the residue pairing on $\gg(\!(z)\!)$ induces an isomorphism between $\gg_1[[z^{-1}]]$ and the dual to $\gg[z]$, and therefore a Lie cobracket on $\gg[z]$.  The uniqueness of this quantization is a theorem due to Drinfeld {\cite[Theorem 2]{DrinfeldQuantum1}}.

\begin{definition}
The \emph{Yangian} of the Lie algebra $\gg$ is the unique graded topological Hopf algebra $Y(\gg)$ which is a topologically free module over the graded ring $\CC[[\hbar]]$ (where $\hbar$ has degree 1), so that when we set $\hbar=0$ we recover $Y(\gg) \otimes_{\CC[[\hbar]]} \CC \iso U(\gg[z])$ as graded rings (where $z$ has degree 1), and where the first order term in the comultiplication $\Delta$ is determined by the cobracket $\delta$ in the sense that
\[\hbar^{-1}(\Delta(f) - \sigma(\Delta(f))) \text{ mod } \hbar = \delta(f \text{ mod } \hbar),\]
for all elements $f \in Y(\gg)$.  Here $\sigma$ is the braiding automorphism of $Y(\gg)^{\otimes 2}$ (i.e. $\sigma(f \otimes g) = g \otimes f$).
\end{definition}

There is a very extensive literature on the Yangian and related quantum groups.  The general theory for quantization of Poisson Lie groups, including the Yangian, was developed by Etingof and Kazhdan \cite{EtingofKazhdanIII}.  For more information we refer the reader to Chari and Pressley \cite{ChariPressley}, or for the Yangian specifically to the concise introduction in \cite[Section 9]{CostelloYangian}.

Likewise, we can study deformation quantization for the algebra $\OO(\mhiggs^\fr_G(\bb{CP}^1,D,\omega^\vee))$ of functions on our symplectic moduli space.  This moduli space is, in particular, a smooth finite-dimensional Poisson manifold, so its algebra of functions can be quantized, for instance using Kontsevich's results on formality \cite{KontsevichQuantization}.

\begin{definition}
The quantum algebra of functions $\OO_\hbar(\mhiggs^\fr_G(\bb{CP}^1,D,\omega^\vee))$ is a choice of deformation quantization of the Poisson algebra $\OO(\mhiggs^\fr_G(\bb{CP}^1,D,\omega^\vee))$.  That is, an associative $\CC[[\hbar]]$-algebra where the antisymmetrization of the first order term in $\hbar$ of the product recovers the Poisson bracket. 
\end{definition}

While we have a Poisson morphism $\mhiggs^\fr_G(\bb{CP}^1,D,\omega^\vee) \to G_1[[z^{-1}]]$ by Theorem \ref{Poisson_Lie_Comparison_thm}, and therefore a map of Poisson algebras $\OO(G_1[[z^{-1}]]) \to \OO(\mhiggs^\fr_G(\bb{CP}^1,D,\omega^\vee))$ there's no automatic guarantee that we can choose a quantization of the target admitting an algebra map from the Yangian quantizing this Poisson map -- one would need to verify the absence of an anomaly obstructing quantization.  There is however a natural model for this quantization, to a $Y(\gg)$-module,  constructed by Gerasimov, Kharchev, Lebedev and Oblezin \cite{GKLO} (extending an earlier calculation \cite{GKL} for $\gg = \gl_n$).

\begin{theorem}
For any reductive group $G$, and any choice of local singularity data, the Poisson map $\OO(G_1[[z^{-1}]]) \to \OO(\mhiggs^\fr_G(\bb{CP}^1,D,\omega^\vee))$ quantizes to a $Y(\gg)$-module structure on an algebra $\OO_\hbar(\mhiggs^\fr_G(\bb{CP}^1,D,\omega^\vee))$ quantizing the algebra of functions on the multiplicative Higgs moduli space.  
\end{theorem}

This theorem is a direct consequence of the Gerasimov, Kharchev, Lebedev and Oblezin (GKLO) construction of $Y(\gg)$-modules whose classical limits are symplectic leaves in the rational Poisson Lie group.  They constructed these representations for any semisimple Lie algebra $\gg$; they are indexed by the data of a dominant coweight-coloured divisor $(D, \omega^\vee)$.  As in Shapiro, this data is encoded in the form of a set $\nu_{i,k}$ of complex numbers, where $i$ varies over simple roots of $\gg$: the coweight at a point $z \in D$ is encoded by the vector $(m_1, \ldots, m_r)$ where $m_i = \lvert\{\nu_{i,k} = z\}\rvert$.  Let us denote this module by $M_{D,\omega^\vee}$.

GKLO discussed the classical limits of these modules $M_{D,\omega^\vee}$ as symplectic subvarieties of the Poisson Lie group.  This is discussed with further detail in \cite[Section 4]{Shapiro}.  In particular, one can compare the Poisson Lie group with the Zastava space for the group $G$.  Recall that  moduli spaces of (non-periodic) monopoles on $\bb{CP}^1$ can be modelled by spaces of framed maps from $\bb{CP}^1$ to the flag variety $G/B$, sending $\infty$ to the point $[B]$ (this is Jarvis's description of monopoles via scattering data \cite{Jarvis}).  The space $\mr{Map}_{\alpha^\vee}^\fr(\bb{CP}^1, G/B)$ of monopoles of degree $\alpha^\vee$ (where $\alpha^\vee$ is a positive coroot, identified as a class in $\mr H_2(G/B;\ZZ)$) admits a compactification known as \emph{Zastava space}.  Concretely, this is a stratified algebraic variety, with stratification given by
\[Z_{\alpha^\vee}(G) = \coprod_{\beta^\vee \preceq \alpha^\vee} \mr{Map}_{\alpha^\vee}^\fr(\bb{CP}^1, G/B) \times \sym^{\alpha^\vee - \beta^\vee}(\CC),\]
where the factor $\sym^{\alpha^\vee - \beta^\vee}(\CC)$ represents the space of coloured divisors $(D, \omega^\vee)$ with $\sum_{z_i \in D} \omega^\vee_{z_i} = \alpha^\vee - \beta^\vee$.  Zastava spaces were introduced by Bezrukavnikov and Finkelberg in \cite{BezrukavnikovFinkelbergI}, see also \cite{BravermanICM}.

Using a birational map to this Zastava space, GKLO showed that the union of symplectic leaves corresponding to representations of a given type (meaning a given sum $\sum \omega^\vee_{z_i}$) is equivalent to the union of the classical leaves of that given type.  It is, therefore, reasonable to conjecture that the decompositions of these two varieties into their individual symplectic leaves (i.e. into decompositions of $\sum \omega^\vee_{z_i}$ labelled by points in $\CC$) should coincide.  We should note that this doesn't seem to follow directly from the GKLO calculations.  They analyzed the symplectic leaves corresponding to the classical limits of $Y(\gg)$-modules via a rational map, denoted by $e_i(z) = b_i(z)/a_i(z)$, defined on an open set of this symplectic leaf.  This rational map is, by construction, not sensitive to the positions of the singularities, only of the total type of a representation.

Nevertheless, the symplectic leaves obtained as classical limits of the $Y(\gg)$-modules $M_{D,\omega^\vee}$ \emph{are} symplectic subvarieties of the Poisson Lie group sweeping out the union of symplectic leaves of a given type.  In particular every symplectic leaf $\mhiggs^\fr_G(\bb{CP}^1,D,\omega^\vee)$ quantizes to some module within the GKLO classification, as claimed.

\begin{example}
One example is given by the case where the only pole lies at $0 \in \CC$, so the map to the Poisson Lie group $G_1[[z^{-1}]]$ factors through the polynomial group $G_1[z^{-1}]$.  This example is included in the work of Kamnitzer, Webster, Weekes and Yacobi \cite{KWWY}.  They calculate the quantization of slices in the thick affine Grassmannian $G(\!(t^{-1})\!)/G[t]$.  The thick affine Grassmannian has an open cell isomorphic to $G_1[[z^{-1}]]$.  For each dominant coweight $\omega^\vee$ there is a slice defining a symplectic leaf in this open cell: it's exactly the leaf corresponding to multiplicative Higgs fields with a single pole at 0 with degree $\omega^\vee$.  Kamnitzer et al quantize this slice (in particular; they also quantize slices through to the other cells in the thick affine Grassmannian) to a $Y(\gg)$-module of GKLO type. 
\end{example}

\begin{remark}
This result should be compared to the conjecture made in \cite[Chapter 8.1]{NekrasovPestun}.  The Poisson Lie group $G_1[[z^{-1}]]$ receives a Poisson map from the full moduli space $\mhiggs^{\text{fr,sing}}(\bb{CP}^1)$ of multiplicative Higgs bundles with arbitrary singularities.  Upon deformation quantization therefore, the quantized algebra of functions on this moduli space is closely related to the Yangian.
\end{remark}

\begin{remark}[$q$-Opers and Quantization]
Finally, let us refer back to Remark \ref{q_opers_remark} and discuss the brane of $q$-opers, and the associated structures that should arise after quantization.  Recall that the multiplicative Hitchin system has a natural section defined by post-composition with the Steinberg section $T/W \to G/G$.  The moduli space of $q$-opers is the subspace of $\qconn_G(C,D,\omega^\vee)$ defined to be the multiplicative Hitchin section after rotating to the point $q$ in the twistor sphere of complex structures.

Now, we expect that the space of $q$-opers will, in the rational case, be Lagrangian, and therefore will admit a canonical $\bb P_0$-structure -- a $-1$-shifted Poisson structure -- on its derived algebra of functions.  It would be interesting to study the quantization of this $\bb P_0$-algebra, and to investigate its relationship with the $q$-W algebras of Sevostyanov \cite{Knight,STSSevostyanov, Sevostyanov}, of Aganagic-Frenkel-Okounkov \cite{AFO}, and of Avan-Frappat-Ragoucy \cite{AvanFrappatRagoucy}.  We'll now proceed to discuss some first steps in this direction.  Preliminary results were announced by the second author at String-Math 2017 \cite{PestunStringMath}.  
\end{remark}

\section{$q$-Opers and $q$-Characters} \label{qchar_section}

In this final section we will discuss the space of $q$-Opers in more depth.  In particular we will connect the geometric setup described in this paper, in terms of multiplicative Higgs bundles, to the gauge theoretic story studied by the second author and collaborators \cite{NekrasovPestunShatashvili,Kimura:2015rgi,Nekrasov:2015wsu}.  The main goal of this subsection will be to describe and motivate a connection between $q$-Opers and the $q$-character maps from the theory of quantum groups \cite{FrenkelReshetikhinSTS,FrenkelReshetikhin2,FrenkelReshetikhin1,Sevostyanov,STSSevostyanov,Sevostyanov1}.  In order to make our statements as concrete as possible it will be useful to first describe the Steinberg section \cite{Steinberg} of a semisimple group explicitly.

Throughout this section, assume that $G$ is a simple, simply-laced and simply-connected Lie group with Lie algebra $\gg$.  Let $\Delta = \{\alpha_1, \ldots, \alpha_r\}$ be the set of simple roots of $\gg$.  In order to define the Steinberg section uniquely we'll fix a \emph{pinning} on $G$.  That is, choose a Borel subgroup $B \sub G$ with maximal torus $T$ and unipotent radical $U$, and choose a generator $e_i$ for each simple root space $\gg_{\alpha_i}$.

We'll also choose an element $\sigma_i \in N(T)$ in the normalizer of $T$ representing each element of the Weyl group $W = N(T)/T$.  The Steinberg section will be independent of this choice up to conjugation by a unique element of $T$, and independent of the ordering on the set of simple roots.

\begin{definition} \label{Steinberg_section_def}
The \emph{Steinberg section} of $G$ associated to a choice of pinning is the image of the injective map $\sigma \colon T/W \to G$ defined by
\[\sigma(t_1, \ldots, t_r) = \prod_{i=1}^r \exp(t_i e_i) \sigma_i.\]
Steinberg proved \cite[Theorem 1.4]{Steinberg} that, after restriction to the regular locus in $G$, the map $\sigma$ defines a section of the Chevalley map $\chi \colon G \to T/W$.
\end{definition}

\begin{definition} \label{mhitch_section_def}
Fix a coloured divisor $(D,\omega^\vee)$ The \emph{multiplicative Hitchin section} of the map $\pi \colon \mhiggs^\fr_G(\bb{CP}^1,D,\omega^\vee) \to \mc B(D,\omega^\vee)$ is the image $\mhitch^\fr_G(\bb{CP}^1, D, \omega^\vee)$ of the map defined by post-composing a meromorphic $T/W$-valued function on $\bb{CP}^1$ with the Steinberg map $\sigma$.
\end{definition}

\begin{remark}
The multiplicative Hitchin section is indeed a section of the map $\pi$ after restricting to the connected component in $\mhiggs^\fr_G(\bb{CP}^1,D,\omega^\vee)$ corresponding to the trivial bundle, provided one chooses a value for the framing within the Steinberg section.  For example, if we choose the identity framing on the multiplicative Hitchin basis then the multiplicative Hitchin section lands in multiplicative Higgs bundles with framing $c = \sigma(1)$ at infinity, i.e. framing given by a Coxeter element.
\end{remark}

Now, let's introduce the key idea in this section: the notion of \emph{triangularization} for the multiplicative Hitchin section.
\begin{definition} \label{gen_evals_def}
Let $g(z)$ be an element of $\mhitch^\fr_G(\bb{CP}^1, D, \omega^\vee)$.  We'll abusively identify $g(z)$ with its image under the restriction map $r_\infty \colon \mhiggs_G^\fr(\bb{CP}^1,D,\omega^\vee) \to G_c[[z^{-1}]]$ to a formal neighbourhood of $\infty$.  Say that $g(z)$ has \emph{generalized eigenvalues} $y(z) \in T[[z^{-1}]]$ if there exists an element $u(z)$ of $U[[z^{-1}]]$ such that $u(z)g(z)u(z)^{-1}$ is an element of $B_-[[z^{-1}]]$, where $B_-$ is the opposite Borel subgroup to $B$, which maps to $y(z)$ under the canonical projection.

We say that $g(z)$ has \emph{$q$-generalized eigenvalues} $y(z) \in T[[z^{-1}]]$ if there exists an element $u$ of $U[[z^{-1}]]$ such that $u(q^{-1}z)g(z)u(z)^{-1}$ is an element of $B_-[[z^{-1}]]$ that maps to $y(z)$ under the canonical projection. 
\end{definition}

\begin{remark}
In this setup $q$ is the automorphism of the formal disk obtained by restricting an automorphism of $\CC$ or $\CC^\times$ to its formal punctured neighbourhood.  In particular we'll use this notation for the additive deformation we denoted by $\eps$ in previous sections, even though we'll use the multiplicative notation.  We make this notational choice in order to allow more direct comparison with the previous gauge theory literature.
\end{remark}

It's sometimes useful to use a slightly different representation of the multiplicative Hitchin section, packaging the singularity datum in a more uniform way. 

Choose a point $b(z)$ in the multiplicative Hitchin base $B(D,\omega^\vee)$. By clearing denominators, we can identify $b(z)$ with a canonical polynomial $t(z)$ in $T[z]/W$ of fixed degree $d$, with fixed top degree term.  Specifically, one finds
\[d_i = \sum_{z_j \in D} \omega^\vee_{z_j}(\lambda_i)\]
where $\lambda_i$ is the $i^\text{th}$ fundamental weight.  We'll denote the locus of polynomials of this form by $T[z]_d$.

We define an embedding from $T[z]_d/W$ into $G[[z^{-1}]]$ as follows.  First, we encode the data of the coloured divisor $(D, \omega^\vee)$ in terms of a $T$-valued polynomial.  That is, we set 
\[p_i(z) = \prod_{z_j \in D} (z-z_j)^{\omega^\vee_{z_j}(\omega_j)}\]
where $\omega_j$ is the $j^{\text{th}}$ fundamental weight.

\vasily{should it be $\omega_j \to \omega_{i}$ in the exponent?}

\begin{definition}
The \emph{$p$-twisted multiplicative Hitchin section} is the image of the map $\sigma' \colon T[z]_d/W \to G(z)$ defined by
\begin{equation}
\label{eq:steinberg}
\sigma'(t(z)) = \prod_{i=1}^r \exp\left(t_{i}(z) e_i\right) \sigma_i p_i(z)^{-\omega^\vee_i}
\end{equation}
where $\omega^\vee_i$ is the $i^\text{th}$ fundamental coweight.
\end{definition}

This $p$-twisted section is obtained from the ordinary multiplicative Hitchin section after conjugation by an element of $G[z]$.  The result is no longer directly related to the multiplicative Hitchin moduli space because this conjugation breaks the framing at $\infty$.  It is, however, sometimes useful for computation. One can define the $q$-triangularization of a point in this $p$-twisted section exactly as in the ordinary case.

Now, let's introduce the other main object we'll be discussing in this section.
 
\begin{definition}
The \emph{$q$-character} is an algebra homomorphism 
\[\chi_q \colon \mr{Rep}(Y(\gg)) \to \OO(T_1[[z^{-1}]])\]
from the ring of finite-dimensional representations of the Yangian to the ring of functions on $T_1[[z^{-1}]]$, first defined by Knight \cite{Knight}.  Given a choice of singularity datum $(D,\omega^\vee)$ one can define a linear map $\mc B(D, \omega^\vee)^* \to \mr{Rep}(Y(\gg))$ generated by the map identifying a coordinate functional on the multiplicative Hitchin base space with the irreducible representation with the given highest weight.  We'll abusively also denote the composite map $\mc B(D, \omega^\vee)^* \to \OO(T_1[[z^{-1}]])$ by $\chi_q$.
\end{definition}

The  $p$-twisted fundamental $q$-characters $t_i(z)$ are obtained by recursive Weyl reflections of the form \cite{NekrasovPestun,NekrasovPestunShatashvili,Nekrasov:2015wsu} (following the conventions from \cite{Kimura:2015rgi} for shifts)
\begin{equation}
  w \cdot_q y_i(z)) = y_{i}(z q^{-1})^{-1} \Bigg(\prod_{e: i \to j} y_j(z) \Bigg) \Bigg(\prod_{e: j\to i} y_j(q^{-1} z)\Bigg)  p_{i}( q^{-1} z).
  \label{iWeyl_action}
\end{equation}
starting from the highest weight monomial $t_i(z) = y_i(z) +  w \cdot_q y_i(z)) + \dots $. 

These formulae appeared previously in \cite{NekrasovPestun, NekrasovPestunShatashvili} and
should be interpreted as $p$-twisted version of the $q$-Weyl reflections of \cite{FrenkelReshetikhin2,FrenkelReshetikhinSTS,FrenkelReshetikhin1,Frenkel2001}.

In the undeformed case, the $y(z)$-functions have a very geometric meaning: they are equivalent to a sequence of algebraic functions defining the cameral cover at a point $b(z)$ in the Hitchin base.  The cameral cover associated to a point $b(z)$ in the multiplicative Hitchin base is, to put it briefly, obtained away from the singular locus as the fiber product $\mc C_b^\circ = (\bb{CP}^1_z \bs D) \times_{T/W} T$, where the map $\bb{CP}^1_z \bs D \to T/W$ is the meromorphic map corresponding to $b(z)$.  In other words, the closed points of $\mc C^\circ_b$ are simply pairs $(z, y) \in (\bb{CP}^1 \bs D) \times T$ such that $b(z) = [y]$ in $T/W$.  One can obtain the cameral curve as the graph of a $|W|$-valued
meromorphic function $\bb{CP}^1 \to T$ by
taking all images $y$ under the projection $T \to T/W$, $y \mapsto [y]$ where elements of $T$ are
coordinatized as  $\prod_{i = 1}^{r} y_i^{\alpha_i^{\vee}}$ where $\alpha_i^{\vee}$ are simple coroots.
Then, on a contractible local patch in the complement of the ramification locus define $t_i(z)$ to be the sum of fundamental weight $\omega_i: T \to \mathbb{C}^{\times}$ evaluated on $W$-orbit of $b(z)$ lift
to $T$-valued functions. Because of $W$-invariance such sum $t_i(z)$
is a global $\mathbb{C}$-valued function on the curve $\bb{CP}^1_z \bs D$.

\vasily{I have updated the above paragraph, but now we need to fix the below paragraph}


The result is, in particular, and after restricting to a formal neighbourhood of $\infty$, a map from $T[z]_d/W$ to $T[[z^{-1}]]$.  We claim this map -- constructing the defining equations for the cameral curve from a point in the multiplicative Hitchin base -- is inverse (potentially up to an affine transformation) to the map sending the multiplicative Hitchin section to its set of generalized eigenvalues.  More generally, the averaging procedure still makes sense after $q$-deformation, yielding a ``$q$-cameral curve''.  We conjecture the following surprising relationship between the space of $q$-opers (or, rather, the multiplicative Hitchin section with the $q$-deformed adjoint action) and the $q$-character.
 
\begin{conjecture} \label{qchar_conjecture}
\begin{enumerate}
\item For any $q$, every element of $g(z)$ of $\mhitch^\fr_G(\bb{CP}^1, D, \omega^\vee)$ has a unique $q$ generalized eigenvalue, and therefore there is a well-defined map
\[E_q \colon \mc B(D,\omega^\vee) \to T[[z^{-1}]] \sub G[[z^{-1}]]\]
given by applying the multiplicative Hitchin section then computing its $q$generalized eigenvalues.  

\item The composite $E_q^* \circ \chi_q \colon \mc B(D,\omega^\vee)^* \to \OO(\mc B(D,\omega^\vee))$ is an affine isomorphism onto the space of linear functionals $\mc B(D,\omega^\vee)^* \sub \OO(\mc B(D,\omega^\vee))$.
\end{enumerate}
\end{conjecture}

\begin{examples}
\begin{enumerate}
 \item In type $A_1$ we can calculate everything very explicitly.  We've already described the multiplicative Hitchin section in Section \ref{GL2_example_section}: it consists of matrices of the form
\begin{equation*}
  g^{t} =
  \begin{pmatrix}
    t   & - 1 \\
    1 & 0
  \end{pmatrix}.
\end{equation*}
We would like to triangularize this to obtain a matrix of the form
\begin{equation*}
  g^{y} =
  \begin{pmatrix}
    y  & 0 \\
    1 & y^{-1} 
  \end{pmatrix}.
\end{equation*}
It's easy to solve the equation $g^t = u(q^{-1}z)g^y(z) u^{-1}(z)$ explicitly.  One finds a solution with $t(z) = y(z) + y(q^{-1}z)^{-1}$, after conjugation by the element $u(z) = - y(z)^{-1}$.  As the conjecture tells us to expect, $t(z)$ is identified with a $q$-twisted Weyl invariant polynomial in $\mathbb{C}[y, y^{-1}]$ which starts from the highest weight monomial $y(z)$. 

 \item We can also make concrete calculations for type $A_2$.  For more direct comparison with the formulae in the literature we'll use the $p$-twisted formulation discussed above involving polynomials $p_i(z)$ encoding the singularity datum $(D, \omega^\vee)$.  We label positive roots as $\alpha_1, \alpha_2$ and $\alpha_3:=\alpha_1 + \alpha_2$ and parametrize a $U[[z^{-1}]]$-valued
  gauge transformation $u(z)$ by the collection of functions  
 $(u_{i}(z))_{\alpha_i  \in \Delta^{+}}$ 
  \begin{equation}
    u(z) = \prod_{3,2,1} \exp( u_{i}(z) e_{\alpha_i})
  \end{equation}
Then solving the equation 
\begin{equation}
g^t(z) =  u(q^{-1} z)^{-1} g^y(z) u(z)
\end{equation}
for $u_{1}(z), u_{2}(z), u_{3}(z)$ and $t_{1}(z), t_{2}(z)$  we find that
\begin{equation*}
\begin{aligned}
& u_{1}(z) = p_{1}(z) u_{2}(q^{-1} z)-p_{1}(z) y_{2}(z) y_{1}(z)^{-1} \\
& u_{2}(z) =  -p_{2}(z) y_{2}(z)^{-1} \\
& u_{3}(z) = -p_{1}(z) p_{2}(z) y_{1}(z)^{-1} \\
& t_{1}(z) = y_{1}(z)-u_{1}(q^{-1}z) \\
& t_{2}(z) = y_{2}(z) - y_{1}(z) u_{2}(q^{-1}z)-u_{3}(q^{-1}z) \\
\end{aligned}
\end{equation*}
which implies in turn that
\begin{equation*}
  \begin{aligned}
    t_{1}(z) =y_{1}(z)  +  \frac{p_{1}(q^{-1}z) y_{2}(q^{-1}z)}{y_{1}(q^{-1}z)} + \frac{p_{1}(q^{-1}z) p_{2}(q^{-2} z)}{ y_{2}(q^{-2} z)}\\
    t_{2}(z) = y_{2}(z)  +\frac{y_{1}(z) p_{2}(q^{-1}z)}{y_{2}(q^{-1}z)}+   \frac{p_{1}(q^{-1}z) p_{2}(q^{-1}z)}{y_{1}(q^{-1}z)}
  \end{aligned}
\end{equation*}
and that indeed coincides with the expression for the $q$-characters for the $A_2$ quiver appearing in
\cite{Nekrasov:2015wsu,NekrasovPestunShatashvili,NekrasovPestun,Kimura:2015rgi}.
\end{enumerate}
\end{examples}

\begin{remark}
After triangularizing, an element of the multiplicative Hitchin section transforms into a function on $\bb{CP}^1$ with apparent singularities (even away from the divisor $D$) where the functions $y_{i}(z)$ have zeroes and poles.  However, when we apply the $q$-character map these singularities are cancelled in pairs between the monomial summands of Equation \ref{iWeyl_action}, as we sum over the transformation by each element of the Weyl group.  This cancellation property has been called ``regularity of the $q$-character'' in the quantum group representation theory literature \cite{FrenkelReshetikhin1,FrenkelReshetikhin2} as well as in the gauge theoretic construction of \cite{NekrasovPestunShatashvili,Nekrasov:2015wsu,NekrasovPestun,Kimura:2015rgi}.
 
The meromorphic functions $y_{i}(z)$ can be expressed as ratios of the form $y_{i}(z) = Q_{i}(z)/Q_i(q^{-1}z)$, and the zeroes of the functions denoted $Q_i$ are known as \emph{Bethe roots} in the context of the Bethe ansatz equations.  In this language, the Bethe ansatz equations are precisely the equations which ensures that poles in $t_{i}(z)$ are cancelled. 
\end{remark}

 \begin{remark}
The paper \cite{KoroteevSageZeitlin} studies the specialization of the main conjecture of this section to the case where $G = \SL_n$ and one fixes a special form of the coloured divisor $(D,\omega^\vee)$ where the $T$-valued polynomials $p(z)$ encoding their positions and orders can be effectively presented as the ratio of Drinfeld polynomials rescaled by $q$, so that effectively $p_i(z) = d_{i}(z)/d_{i}(q^{-1}z)$.  This special form for the singularity datum means that the Yangian module obtained by quantization of the symplectic leaf $\mhiggs_G(C,D)$ contains (as a quotient) the finite-dimensional Drinfeld module specified by the Drinfeld polynomials $d_{i}(z)$. In the language of quiver gauge theory, this specialization is known as 4d to 2d specialization \cite{ChenDoreyHollowoodLee, DoreyHollowoodLee}. This specialization leads to the Bethe ansatz equations with finite dimensional representations of Yangians and finite number of Bethe roots. 
\end{remark}

\bibliographystyle{alpha}
\bibliography{Mult_Hitchin}
%\printbibliography

\textsc{Institut des Hautes \'Etudes Scientifiques}\\
\textsc{35 Route de Chartres, Bures-sur-Yvette, 91440, France}\\
\texttt{celliott@ihes.fr}\\ 
\texttt{pestun@ihes.fr}
 
\end{document}

