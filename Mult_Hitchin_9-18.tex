\documentclass[10pt, oneside]{article}
\input math_headers.sty

\title{Multiplicative Hitchin Systems and Supersymmetric Gauge Theory}
\author{Chris Elliott \and Vasily Pestun}
\date{\today}

\DeclareMathOperator{\mhiggs}{mHiggs}
\DeclareMathOperator{\mon}{Mon}
\newcommand{\map}{\ul{\mr{Map}}}
\newcommand{\qconn}{q\text{-Conn}}
\renewcommand{\conn}{\text{-Conn}}
\newcommand{\epsconn}{\varepsilon\text{-Conn}}
\DeclareMathOperator{\diff}{Diff}
\renewcommand{\d}{\mathrm{d}}
\newcommand{\fr}{\mathrm{fr}}
\DeclareMathOperator{\Hol}{Hol}
\renewcommand{\ad}{\mr{ad}}
\newcommand{\Ad}{\mr{Ad}}

\newtheorem{pseudoconj}[definition]{Pseudo-Conjecture}

\begin{document}

\maketitle 
\begin{abstract}
 
\end{abstract}

\section{Introduction}
\chris{
Preliminary notes: phrase it as comparing a number of different descriptions of the same moduli space and natural structures thereon.  The descriptions are:
\begin{enumerate}
\item Multiplicative Higgs
\item Periodic monopoles
\item A symplectic leaf in the Poisson Lie group
\item The Seiberg-Witten system associated to an ADE $\mc N=2$ quiver gauge theory
\item The phase space of a twisted 5d $\mc N=2$ super Yang-Mills theory.  
\end{enumerate}
The natural structures are: 
\begin{enumerate}[label=\alph*.]
\item Holomorphic symplectic structure (equivalent descriptions from each of the points of view 1-4).
\item Promotion to a hyperk\"ahler structure (naturally comes from 2, but also 4). 
\item Even more so, algebraic integrable system structure (can describe from point of view 1, but also 4).
\item Deformation to q-connections (from point of view 1, 2 or 5). 
\item The Hitchin section and the moduli space of q-Opers (most naturally from 1, though there are physical implications).  
\item Multiplicative Langlands duality (point of view 5).
\end{enumerate}}


\section{Multiplicative Higgs Bundles and $q$-Connections}
We'll begin with an abstract definition of the moduli spaces we'll investigate in this paper using the language of derived algebraic geometry.

Let $G$ be a reductive complex algebraic group, let $C$ be a smooth complex algebraic curve and fix a finite set $D = \{z_i, \ldots, z_k\}$ of closed points in $C$.  We write $\bun_G(C)$ for the moduli stack of $G$-bundles on $C$, which we view as a mapping stack $\map(C, BG)$ into the classifying stack of $G$.

\begin{definition}
The moduli stack of \emph{multiplicative $G$-Higgs fields} on $C$ with singularities at $D$ is the fiber product
\[\mhiggs_G(C,D) = \bun_G(C) \times_{\bun_G(C \! \bs \! D)} \map(C \! \bs \! D, G/G)\]
where $G/G$ is the adjoint quotient stack.
\end{definition}

\begin{remark}
A closed point of $\mhiggs_G(C,D)$ consists of a principal $G$-bundle $P$ on $C$ along with an automorphism of the restriction $P|_{C \! \bs \! D}$, i.e. a section of $\ad(P)$ away from $D$.
\end{remark}

The adjoint quotient stack can also be described as the derived loop space $\map(S^1_B, BG)$ of the classifying stack, where $S^1_B$ is the ``Betti stack'' of $S^1$, i.e. the constant derived stack at the simplicial set $S^1$.  We can therefore view $\map(C \! \bs \! D, G/G)$ instead as the mapping stack $\map((C \! \bs \! D) \times S^1_B, BG)$, and the moduli stack of multiplicative Higgs bundles instead as
\[\mhiggs_G(C,D) = \map((C \times S^1_B) \bs (D \times \{0\}), BG).\]  
The source of this mapping stack can be $q$-deformed.  Indeed, let $q$ denote an automorphism of the curve $C$.  Write $C \times_q S^1_B$ for the \emph{mapping torus} of $q$, i.e the derived fiber product
\[C \times_q S^1_B = C \times_{C \times C} C\]
where the two maps $C \to C \times C$ are given by the diagonal and the $q$-twisted diagonal $(1,q)$ respectively.

\begin{definition}
The moduli stack of \emph{$q$ difference connections} for the group $G$ on $C$ with singularities at $D$ is the mapping space
\[\qconn_G(C,D) = \map((C \times_q S^1_B) \bs (D \times \{0\}), BG).\] 
In particular when $q=1$ this recovers the moduli stack of multiplicative Higgs bundles.
\end{definition}

\begin{remark}
A closed point of $\qconn_G(C,D)$ consists of a principal $G$-bundle $P$ on $C$ along with a \emph{$q$ difference connection}: an isomorphism of $G$-bundles $P|_{C \! \bs \! D} \to q^*P|_{C \! \bs \! D}$ away from the divisor $D$.  For an introduction to the classical theory of $q$-difference connections we refer the reader to \cite{STSSevostyanov}. \chris{I'm not sure what to say about $q$-difference equations.  You suggested citing Sauloy but I'm not sure exactly what.  He has articles on $q$-difference modules which don't seem so relevant here?}
\end{remark}

\subsection{Residue Conditions}
These moduli stacks are typically of infinite type.  In order to obtain finite type stacks, and later in order to define symplectic rather than only Poisson structure, we can fix the behaviour of our multiplicative Higgs fields and $q$ difference connections near the punctures $D \sub C$.

We'll write $\bb D$ to denote the \emph{formal disk} $\spec \CC[\![z]\!]$.  Likewise we'll write $\bb D^\times$ for the \emph{formal punctured disk} $\spec \CC(\!(z)\!)$.  We'll then write $\bb B$ for the derived pushout $\bb D \sqcup_{\bb D^\times} \bb D$.  Let $LG = \map(\bb D^\times, G)$ and let $L^+G = \map(\bb D, G)$.

There is a canonical inclusion $\bb B^{\sqcup k} \to (C \times_q S^1_B) \!\!\bs\!\! (D \times \{0\})$, the inclusion of the formal punctured neighbourhood of $D \times \{0\}$.  This induces a restriction map on mapping spaces
\[\res_D \colon \qconn_G(C, D) \to \bun_G(\bb B)^k.\]

One can identify $\bun_G(\bb B)$ with the double quotient stack $L^+G \!\bs\! LG / L^+G$, or equivalently with the quotient $L^+G \!\bs\! \Gr_G$ of the affine Grassmannian.  The following is well-known (see e.g. the expository article \cite{Zhu}).

\begin{lemma}
The set of closed points of $\bun_G(\bb B)$ is in canonical bijection with the set of dominant coweights of $G$.
\end{lemma}

\begin{definition}
Choose a map from $D$ to the set of dominant coweights and denote it by $\omega^\vee \colon z_i \mapsto \omega^\vee_{z_i}$.  Write $\Lambda_i$ for the isotropy group of the point $\omega^\vee_{z_i}$ in $\bun_G(\bb B)$. The moduli stack of $q$ difference connections on $C$ with singularities at $D$ and fixed residues given by $\omega^\vee$ is defined to be the fiber product
\[\qconn_G(C,D, \omega^\vee) = \qconn_G(C,D) \times_{\bun_G(\bb B)^k} (B\Lambda_1 \times \cdots \times B\Lambda_k).\]
\end{definition}

\begin{remark}\label{ind_structure_remark}
In a similar way we can define a filtration on the moduli stack of $q$-connections.  Recall that the affine Grassmannian $\Gr_G$ is stratified by dominant coweights $\omega^\vee$, and there is an inclusion $\Gr_G^{\omega^\vee_1}\sub \ol{\Gr_G^{\omega^\vee_2}}$ of one stratum into the closure of another stratum if and only if $\omega^\vee_1 \preceq \omega^\vee_2$ with respect to the standard partial order on dominant coweights.  We can then define
\[\qconn_G(C,D, \preceq \omega^\vee) = \qconn_G(C,D) \times_{\bun_G(\bb B)^k} \left(L^+G \!\!\bs\!\! \ol{\Gr_G^{\omega^\vee_1}} \times \cdots \times L^+G \!\!\bs\!\! \ol{\Gr_G^{\omega^\vee_k}}\right).\]
The full moduli stack $\qconn_G(C,D)$ is the filtered colimit of these moduli spaces.  One can additionally take the filtered colimit over all finite subsets $D$ in order to define a moduli stack $\qconn^\mr{sing}_G(C)$ of $q$-connections on $C$ with arbitrary singularities.
\end{remark}

\begin{examples}
The most important examples for our purposes are given by the following rational/trigonometric/elliptic trichotomy.
 \begin{itemize}
  \item \textbf{Rational:} We can enhance the definition of our moduli space by including a framing at a point $c \in C$ not contained in $D$.  We always assume that such framed points are fixed by the automorphism $q$.
    \begin{definition}
      \label{def:framing}
    The moduli space of $q$-difference connections on $C$ with a framing at $c$ is defined to be the relative mapping space 
    \[\qconn_G^\fr(C) = \map(C \times_q S^1_B, BG; f)\]
    where $f \colon \{c\} \times S^1_B \to BG$ (or equivalently $f \colon \{c\} \to G/G$) is a choice of adjoint orbit.  We can define the framed mapping space with singularities and fixed residues in exactly the same way as above.  
  \end{definition}
    
    In this paper we'll be most interested in the following example.  Let $C = \bb{CP}^1$ with framing point $c = \infty$, and consider automorphisms of the form $z \mapsto z + \eps$ for $\eps \in \CC$.  Choose a finite subset $D \sub \bb A^1$ and label the points $z_i \in D$ by dominant coweights $\omega^\vee_{z_i}$.  We can then study the moduli space $\epsconn^\fr_G(\bb{CP}^1,D, \omega^\vee)$.  The main object of study in this paper will be the holomorphic symplectic structure on this moduli space.  
    
    Note that the motivation for this definition comes in part from Spaide's formalism \cite{Spaide} of AKSZ symplectic structures on relative mapping spaces -- in this formalism $\bb{CP}^1$ with a single framing point is relatively 1-oriented, so mapping spaces out of it with 1-shifted symplectic targets have AKSZ 0-shifted symplectic structures.
  
  \item \textbf{Trigonometric:} Alternatively, we can enhance our definition by including a reduction of structure group at a point $c \in C$ not contained in $D$, again fixed by the automorphism $q$.
  \begin{definition}
   The moduli space of $q$-difference connections on $C$ with an $H$-reduction at $c$ for a subgroup $H \sub G$ is defined to be the fiber product
   \[\qconn_G^{H,c}(C) = \map(C \times_q S^1_B,BG) \times_{G/G} H/H\]
   associated to the evaluation at $c$ map $\map(C \times_q S^1_B,BG) \to G/G$.  We can define the moduli space with $H$-reduction with singularities and fixed residues in the same way as above.
  \end{definition}
  
  Again let $C = \bb{CP}^1$.  Fix a pair of opposite Borel subgroups $B_+$ and $B_- \sub G$ with unipotent radicals $N_\pm$ and consider the moduli space of $q$-connections with $B_+$-reduction at $0$ and $N_-$-reduction at $\infty$.  We'll now take $q$ to be an automorphism of the form $z \mapsto qz$ for $q \in \CC^\times$.  We'll defer in depth analysis of this example to future work.
  
  \item \textbf{Elliptic:} Finally, let $C = E$ be a smooth curve of genus one.  In this case we won't fix any additional boundary data, but just consider the moduli space $\qconn_G(E,D, \omega^\vee)$.  In the case $q = 1$ this space -- or rather its polystable locus -- was studied by Hurtubise and Markman \cite{HurtubiseMarkman}, who proved that it can be given the structure of an algebraic integrable system with symplectic structure related to the elliptic R-matrix of Etingof and Varchenko \cite{EtingofVarchenko}.
 \end{itemize}
\end{examples}

\begin{remark} \label{Elliptic_AKSZ_remark}
In the elliptic case it's natural to ask to what extent Hurtubise and Markman's integrable system structure can be extended from the variety of polystable multiplicative Higgs bundles to the full moduli stack.  If $D$ is empty then it's easy to see that we have a symplectic structure given by the AKSZ construction of Pantev-To\"en-Vaqui\'e-Vezzosi \cite{PTVV}.  Indeed, $E$ is compact 1-oriented and the quotient stack $G/G$ is 1-shifted symplectic, so the mapping stack $\map(E, G/G) = \mhiggs_G(E)$ is equipped with a 0-shifted symplectic structure by \cite[Theorem 2.5]{PTVV}.  The role of the Hitchin fibration is played by the Chevalley map $\chi \colon G/G \to T/W$, and therefore
\[\map(E,G/G) \to \map(E,T/W).\]
The fibers of this map over regular points in $T/W$ are given by moduli stacks of the form $\bun_T(\wt E)^W$ where $\wt E$ is a $W$-fold cover of $E$ (the cameral cover).  Note that in this unramified case the curve $\wt E$ also has genus 1; counting dimensions we see that the base has dimension $r = \mr{rk}(G)$ and the generic fibers are $r$-dimensional (Lagrangian) tori.
\end{remark}

\begin{remark}
While the moduli space of multiplicative Higgs bundles makes sense on a general curve it's only after restricting identity to this trichotomy of examples that we'll expect the existence of a Poisson structure.  Such a structure arises by the AKSZ construction, i.e. by transgressing the 1-shifted symplectic structure on $G/G$ to the mapping space using a fixed section of the canonical bundle on $C$ (possibly with a boundary condition).  
\end{remark}

\subsection{The Multiplicative Hitchin System}
We can define the global Chevalley map as in Remark \ref{Elliptic_AKSZ_remark} in the case of non-empty $D$ as well.  We'll show that in the rational case this defines a completely integrable system structure.

\begin{definition} \label{mult_Hitchin_system_def}
Fix a curve $C$, a divisor $D$ and a dominant coweight $\omega_{z_i}^\vee$ at each point $z_i$ in $D$.  The \emph{multiplicative Hitchin base} is the stack
\[\mc B(C,D,\omega^\vee) = \mr{Sect}(C, X(D,\omega^\vee)/W)\]
of sections of $X(D,\omega^\vee)/W$, the $T/W$-bundle on $C$ where $X(D,\omega^\vee)$ is the $T$-bundle characterized by the condition that the associated line bundle $X(D,\omega^\vee) \times_T {\lambda}$ corresponding to a weight $\lambda$ is given by $\OO(\sum \omega_{z_i}^\vee(\lambda) \cdot z_i)$ (c.f. \cite[Section 3.3]{HurtubiseMarkman}).

The \emph{multiplicative Hitchin fibration} is the map
\[\pi \colon \mhiggs_G(C,D,\omega^\vee) \to \mc B(C,D,\omega^\vee)\]
given by post-composing a map $C \bs D \to G/G$ with the Chevalley map $\chi \colon G/G \to T/W$.  
\end{definition}

\begin{prop}
The multiplicative Hitchin fibration described above is well-defined.
\end{prop}

\begin{proof}
We need to verify that the image of a point in $\mhiggs_G(C,D,\omega^\vee)$ under $\pi$, viewed as a section of the trivial $T/W$-bundle on $C \bs D$, extends to a section of $X(D,\omega^\vee)/W$, look locally near a singularity $z_i \in D$.  Let $\phi \in G((z_i))$ be a local representative for a multiplicative Higgs field in $\mhiggs_G(C,D,\omega^\vee)$, i.e. an element of the associated $G[[z_i]]$-adjoint orbit.  Since $\phi$ is equivalent to $z_i^{-\omega^\vee_{z_i}}$ under the action of $G[[z_i]]^2$ by left and right multiplication, without loss of generality we can say that $\phi = z_i^{-\omega^\vee_{z_i}} \phi_0$ for some $\phi_0 \in G[[z_i]]$.  Consider $\chi(\phi) \in T((z_i))$ -- the singular part of this element is the same as $\chi(z_i^{-\omega^\vee_{z_i}} g)$ for some $g \in G$, which implies the section extends to a meromorphic $T/W$-valued function of the required type.
\end{proof}

\begin{remark}
\chris{Compare to the base in \cite{HurtubiseMarkman}.  Variety versus stack?  Difference in processing the $W$-action?  They are birational.}
\end{remark}

\chris{I don't want this section to have to be too rigorous, so make a comment to that effect.}
We'll show that this fibration defines a completely integrable system in the rational case where $C = \bb{CP}^1$ with a framing at $\infty$.  We should first observe that the generic fibers are half-dimensional tori.  Computing the fibers of the Hitchin fibration works similarly to the non-singular case: a point in the base is, in particular, a map $C \bs D \to T/W$.  Suppose this map lands in the regular locus $T^{\mr{reg}}/W$, then an element of the fiber over this point defines, in particular, a map $C \bs D \to BT/W$.  We would like to argue that the fiber consists of $T$-bundles on the cameral cover $\wt C$: a $W$-fold cover of $C$ ramified at the divisor $D$ \chris{am I making an error here: is $D$ really the same as the ramification divisor?  Maybe they should be different, and in fact it should be ramified at points whose value is a point in $T/W$ with stabilizer (but these aren't regular!)}.

In order to say this a bit more precisely this we'll compare our moduli space with the space of abstract Higgs bundles introduced by Donagi and Gaitsgory \cite{DonagiGaitsgory} (see also \cite{DonagiLectures}, where Donagi proposed the applicability of this abstract Higgs theory to the multiplicative situation and asked for a geometric interpretation).  Our argument will follow the same ideas as the arguments of \cite[Section 6]{HurtubiseMarkman}.

\begin{definition}
An \emph{abstract $G$-Higgs bundle} on a curve $C$ is a principal $G$-bundle $P$ along with a sub-bundle $\mf c$ of $\gg_P$ of \emph{regular centralizers}, meaning that the fibers are subalgebras of $\gg$ which arise as the centralizer of a regular element of $\gg$.  Write $\higgs_G^{\mr{abs}}(C)$ for the moduli stack of abstract $G$-Higgs bundles on $C$.
\end{definition}

There's an algebraic map from the regular part of our moduli space $\mhiggs_G(C,D,\omega^\vee)_{\mr{reg}}$ (where the Higgs field is required to take regular values) into $\higgs_G^{\mr{abs}}(C)$ that sends a multiplicative Higgs bundle $(P,g)$ to the abstract Higgs bundle $(P, \mf c_g)$, where $\mf c_g$ is the sub-bundle of $\gg_P$ fixed by the adjoint action of the multiplicative Higgs field $g$  \chris{check}.  What's more, there is a commutative square relating the Hitchin fibration for the multiplicative moduli space with a related projection for the abstract moduli stack:
\[\xymatrix{
\mhiggs_G(C,D,\omega^\vee)_{\mr{reg}} \ar[r] \ar[d] &\higgs_G^{\mr{abs}}(C) \ar[d] \\
\mc B(C,D,\omega^\vee)_{\mr{reg}} \ar[r] &\mr{Cam}_G(C),
}\]
where $\mr{Cam}_G(C)$ is the stack of cameral covers of $C$, as defined in \cite[Section 2.8]{DonagiGaitsgory}.  The map $\mc B(C,D,\omega^\vee)_{\mr{reg}} \to \mr{Cam}_G(C)$ is defined by sending a meromorphic function $f \colon C \to T/W$ to the $D$-ramified cameral cover $\wt C = C \times_{T/W} T$.  In particular there's a map from the multiplicative Hitchin fiber to the corresponding Donagi-Gaitsgory fiber: the moduli space of abstract $G$-Higgs bundles with fixed cameral cover.  This map is surjective: having fixed the cameral cover, and therefore the ramification data, every sub-bundle $\mf c \sub \gg_P$ of regular centralizers arises as the centralizer of some regular multiplicative Higgs field.  Likewise once one restricts to a single generic multiplicative Hitchin fiber the map is an unramified $W$-fold cover.

To conclude this discussion we'll discuss dimensions and the geometry of the multiplicative Hitchin fibers.  Firstly, we can compute the dimension of a regular multiplicative Hitchin fiber by computing the dimension of the base and the dimension of the total space.  The dimension of the base is given by computing the number of linearly independent sections of the $T$-bundle $X(D,\omega^\vee)$ on $\bb{CP}^1$ vanishing at $\infty$.  This is given by 
\[\dim \mc B(C,D,\omega^\vee) = \sum_{z_i \in D} \langle \rho, \omega^\vee_{z_i} \rangle,\]
where $\rho$ is the Weyl vector.  On the other hand the dimension of the total space is calculated in Corollary \ref{dim_of_moduli_space_cor} to be $2 \sum_{z_i \in D} \langle \rho, \omega^\vee_{z_i} \rangle$.  The base is indeed half-dimensional, therefore so is the fiber.

The Donagi-Gaitsgory fiber is, according to the main theorem of \cite{DonagiGaitsgory}, equivalent to the moduli space of $W$-equivariant $T$-bundles on the cameral curve $\wt C$ up to a discrete correction involving the root datum of $G$.  In particular it is generically an abelian variety.  The multiplicative Hitchin fiber is isogenous to this abelian variety, since the map from the multiplicative Hitchin fiber to the Donagi-Gaitsgory fiber is surjective and \'etale.

\begin{remark}
Like in the case of the ordinary Hitchin system, the multiplicative Hitchin system admits a canonical Hitchin section.  One can construct this section using the Steinberg section (the multiplicative analogue of the Kostant section).  This is a section of the map $G/G \to T/W$, canonical after choosing a Borel subgroup $B$ with maximal torus $T$ and a basis vector for each simple root space.  The \emph{multiplicative Hitchin section} is the map $\sigma \colon \mc B(C,D,\omega^\vee) \to \mhiggs_G(C,D,\omega^\vee)$ defined by post-composition with the Steinberg section.  One can use this section to define the moduli space of \emph{$q$-opers} for the group $G$ and the curve $\bb{CP}^1$ with its framing at infinity.  We'll discuss the hyperk\"ahler structure on the moduli space of multiplicative Higgs bundles in Section \ref{hyperkahler_section}.  In particular we'll show that when one rotates to $q$ in the twistor sphere one obtains the moduli space of $q$-connections on $\bb{CP}^1$.  The moduli space of $q$-opers is defined to be the Hitchin section, but viewed as a subspace of $\qconn^{\fr}_G(\bb{CP}^1,D,\omega^\vee)$.  We'll make some brief remarks about the role played by this moduli space at the end of Section \ref{hyperkahler_section}
\end{remark}

\subsection{Stability Conditions}
For comparison to results in the literature it is important that we briefly discuss the role of stability conditions for difference connections.  In our main example of interest -- the rational case -- these conditions won't play a role, but they do appear in the comparison results between $q$-connections and monopoles in the literature for more general curves.  For definitions for general $G$ we refer to \cite{Smith}, although see also \cite{AnchoucheBiswas} on polystable $G$-bundles.  In what follows we fix a choice of $0 < t_0 < 2\pi R$.

\begin{definition}
Let $(P,g)$ be a $q$-connection on a curve $C$, and let $\chi$ be a character of $G$.  The \emph{$\chi$-degree} of $(P,g)$ is defined to be 
\[\deg_\chi(P,g) = \deg(P \times_\chi \CC) - \frac {t_0}{2\pi R} \sum_{i=1}^k \deg(\chi \circ \omega^\vee_{z_i}).\]

A $q$-connection $(P,g)$ on $C$ is \emph{stable} if for every maximal parabolic subgroup $H \sub G$ with Levi decomposition $H = LN$ and every reduction of structure group $(P_H, g)$ to $H$, we have
\[\deg_\chi(P_H, g) < 0\]
for the character $\chi = \det(\mr{Ad}_L^{\mf n})$ defined to be the determinant of the adjoint representation of $L$ on $\mf n$.

The $q$-connection $(P,g)$ is \emph{polystable} if there exists a (not necessarily maximal) parabolic subgroup $H$ with Levi factor $L$ and a reduction of structure group $(P_L, g)$ to $L$ so that $(P_L,g)$ is a stable $q$-connection and so that the associated $H$-bundle is admissible, meaning that for every character $\chi$ of $H$ which is trivial on $Z(G)$ the associated line bundle $P_H \times_\chi \CC$ has degree zero. 
\end{definition}

Below we'll write $\qconn_G^{\text{ps}}(C, D, \omega^\vee)$ for the moduli space of polystable $q$-connections.  This moduli space is a smooth algebraic variety of finite type.  If $q$ is the identity this is a theorem of Charbonneau, Hurtubise and Smith.  For more general $q$, as a smooth manifold these moduli spaces are actually independent when one varies $q$ -- we'll see this when we describe an equivalent description of these moduli spaces as spaces of periodic monopoles on diffeomorphic manifolds.  

When $C = \bb{CP}^1$ every principal $G$-bundle on $C$ admits an essentially unique (up to the action of the Weyl group) holomorphic reduction of structure group to a maximal torus \cite{GrothendieckSphere}.  Since $q$-connections for an abelian group are automatically stable, polystability on the sphere is equivalent to admissibility of the torus reduction.  As a consequence, for our main example of interest -- the rational case -- the moduli space of polystable $q$-connections is equivalent to the moduli space of all $q$-connections of admissible degree.  For instance for $G=\SL_n$ the moduli space of polystable $q$-connection is equivalent to the moduli space of $q$-connections on the trivial bundle.

\subsection{Poisson Structures from Derived Geometry}
As we mentioned above in Remark \ref{Elliptic_AKSZ_remark}, in the case where $C$ is an elliptic curve and there are no punctures there's a symplectic structure on $\mhiggs_G(C)$ given by the AKSZ formalism.  More generally, when we do have punctures, we expect the moduli space $\mhiggs_G(C,D)$ to have a Poisson structure with a clear origin story coming from the theory of derived Poisson geometry.  In this section we'll explain what this story looks like.  However, we emphasise that there are technical obstructions to making this story precise with current technology: this section should be viewed as motivation for the structures we'll discuss in the rest of the paper.  We refer the reader to \cite{CPTVV} for the theory of derived Poisson structures and to \cite{MelaniSafronov1, MelaniSafronov2, Spaide} for that of derived coisotropic structures.

Here's the idea.  Recall that we can identify the moduli space of singular $q$-connections on a curve $C$ as a fiber product: $\qconn_G(C, D) \iso \bun_G^\fr(C) \times_{\bun_G(C \! \bs \! D)^2} \bun_G(C \! \bs \! D)$ where the map $g \colon \bun_G^\fr(C) \to \bun_G(C \! \bs \! D)$ is given by $P \mapsto (P|_{C \! \bs \! D}, q^*P|_{C \! \bs \! D})$.  Consider the following commutative cube:

\[\xymatrix@C-30pt@R-8pt{
& \qconn_G(C, D) \ar[rr]^{f_1} \ar'[d][dd]^(.25){\mr{res}} & & \bun_G(C) \ar[dd]
\\
\bun_G(C \bs D) \ar@{<-}[ur]^{f_2} \ar[rr]^(.6){g_2} \ar[dd] & & \bun_G(C \bs D)^2 \ar@{<-}[ur]^{g_1} \ar[dd]^(.4)r
\\
& \bun_G(\bb B)^k \ar'[r][rr] & & BL^+G^{2k}
\\
BLG^k \ar[rr]\ar@{<-}[ur] & & BLG^{2k}. \ar@{<-}[ur]
}\]
Here the top and bottom faces are homotopy Cartesian squares.  What does this setup buy us?  We'll first answer informally.

\begin{claim}
First consider the bottom face of the cube.  The stack $BLG$ is 2-shifted symplectic because the Lie algebra $L\gg$ has a non-degenerate invariant pairing: the residue pairing.  The Lie subalgebra $L^+\gg$ forms part of a Manin triple $(L\gg, L^+\gg, L^-+0\gg)$ which means that $BL^+G \to BLG$ is 2-shifted Lagrangian.  Therefore the bottom face of the cube defines a 2-shifted Lagrangian intersection, which means that the pullback $\bun_G(\bb B)^k$ is 1-shifted symplectic.

Now consider the top face of the cube.  If either $C$ is an elliptic curve, or $C=\bb{CP}^1$ and we fix a framing at $\infty$, then the map $\bun_G(C \bs D) \to BLG^k$ is also 2-shifted Lagrangian.  In particular $\bun_G(C \bs D)$ is 1-shifted Poisson.  Finally, the map $\bun_G(C) \to \bun_G(C \bs D)$ is 1-shifted coisotropic, or equivalently the canonical map $\bun_G(C) \to \bun_G(C \bs D) \times_{BLG^k} BL^+G^k$ is 1-shifted Lagrangian.  That means that the top face of the cube defines a 1-shifted coisotropic intersection, which means that the pullback $\qconn_G(C,D)$ is 0-shifted Poisson.

The restriction map $\qconn_G(C,D) \to \bun_G(\bb B)^k$ is 1-shifted Lagrangian, which means that if we form the intersection with a $k$-tuple of Lagrangians in $\bun_G(\bb B)$ then we obtain a 0-shifted symplectic stack.  For example, if $\omega_i^\vee$ is a point in $\bun_G(\bb B)$ corresponding to a dominant coweight with stabilizer $\Lambda_i$ then $B \Lambda_i \to \bun_G(\bb B)$ is 1-shifted Lagrangian, so the moduli stack $\qconn_G(C,D, \omega^\vee)$ obtained by taking the derived intersection is ind 0-shifted symplectic.
\end{claim}

Now, let us make this claim more precise.  The main technical condition that makes this claim subtle comes from the fact that most of the derived stacks appearing in this cube, for instance the stack $BLG$, are not Artin.  As such we need to be careful when we try to, for instance, talk about the tangent complex to such stacks.  One can make careful statements using the formalism of ``Tate stacks'' developed by Hennion \cite{Hennion}.  We can therefore make our claim into a more formal conjecture.

\begin{conjecture}
Suppose $C$ is either an elliptic curve or $\bb{CP}^1$ with a fixed framing at $\infty$.
\vspace{-10pt}
\begin{enumerate}
\item The stack $BLG$ is Tate 2-shifted symplectic, and both $BL^+G \to BLG$ and $\bun_G(C \bs D) \to BLG^k$ are Tate 2-shifted Lagrangian.  
\item The stack $\bun_G(C \bs D)$ is ind 1-shifted Poisson, and the map $\bun_G(C) \to \bun_G(C \bs D)$ is ind 1-shifted coisotropic witnessed by the 2-shifted Lagrangian map $BL^+G^k \to BLG^k$.
\item The Lagrangian intersection $\bun_G(\bb B)$ is Tate 1-shifted symplectic, and the map $B\Lambda_i \to \bun_G(\bb B)$ associated to the inclusion of the stabilizer of a closed point is 1-shifted Lagrangian.
\end{enumerate}
\vspace{-10pt}
As a consequence, the moduli stack $\qconn_G(C,D)$ is ind 0-shifted Poisson and the moduli stack $\qconn_G(C,D, \omega^\vee)$ is 0-shifted symplectic.
\end{conjecture}

\begin{remark}
We should explain heuristically why the Calabi-Yau condition on $C$ is necessary.  This is a consequence of the AKSZ formalism in the case where $D$ is empty: for the mapping stack $\map(C, G/G)$ to be 0-shifted symplectic, or for the mapping stack $\map(C,BG)$ to be 1-shifted symplectic, we need $C$ to be compact and 1-oriented.  A $d$-orientation on a smooth complex variety of dimension $d$ is exactly the same as a Calabi-Yau structure.

More generally we can say the following.  Let us consider the rational case where $C = \bb{CP}^1$.  Consider the inclusion $\d r \colon \gg_- = \bb T_{\bun_G^\fr(\bb{CP}^1 \! \bs \! D)}[-1] \to r^*\bb T_{BLG^k}[-1] = \gg(\!(z)\!)^k$: a map of ind-pro Lie algebras concentrated in degree zero.  The residue pairing vanishes after pulling back along $r$ since elements of $\gg_-$ are  $\gg$-valued functions on $\bb{CP}^1$ with at least a simple pole at every puncture in $D$.  So the map $r$ is isotropic with 0 isotropic structure; this structure is unique for degree reasons.  We must check that this structure is non-degenerate.  It suffices to check that the sequence
\[\bb T_{\bun_G^\fr(\bb{CP}^1 \! \bs \! D)}[-1] \to r^*\bb T_{BLG^k}[-1] \to (\bb T_{\bun_G^\fr(\bb{CP}^1 \! \bs \! D)}[-1])^\vee\]
is an exact sequence of ind-pro vector spaces, and therefore an exact sequence of quasi-coherent sheaves on the stack $\bun_G^\fr(\bb{CP}^1 \! \bs \! D)$.  To do this we identify the pair $(\gg_-, \gg(\!(z)\!)^k)$ as part of a Manin triple, where a complementary isotropic subalgebra to $\gg_-$ is given by $\gg_+ = \gg[[z]]^k$.  Using the residue pairing we can identify $\gg_+$ with $(\gg_-)^\vee$ and therefore identify our sequence with the split exact sequence
\[0 \to \gg_- \to \gg(\!(z)\!)^k \to \gg_+ \to 0.\]
\end{remark}

%\begin{remark}
%We can also discuss the multiplicative Hitchin system of Definition \ref{mult_Hitchin_system_def} in this derived context.  \chris{not sure what to say precisely.}
%\end{remark}

\section{Periodic Monopoles}
Moduli spaces of $q$-connections on a Riemann surface $C$ are closely related to moduli spaces of periodic monopoles, i.e. monopoles on 3-manifolds that fiber over the circle (more specifically, with fiber $C$ and monodromy determined by $q$).  Let $G_\RR$ be a compact Lie group whose complexification is $G$.  The discussion in this section will mostly follow that of \cite{CharbonneauHurtubise, Smith}.

Write $M = C\times_q S^1_R$ for the $C$-bundle over $S^1$ with monodromy given by the automorphism $q$.  More precisely, $M$ is the Riemannian 3-manifold obtained by gluing the ends of the product $C \times [0,2\pi R]$ of Riemannian manifolds by the isometry $(x,2\pi R) \sim (q(x), 0)$.

\begin{definition}
A \emph{monopole} on the Riemannian 3-manifold $M = C \times_q S^1_R$ is a smooth principal $G_\RR$-bundle $\bo P$ equipped with a connection $A$ and a section $\Phi$ of the associated bundle $\gg_{\bo P}$ satisfying the Bogomolny equation 
\[\ast F_A = \d_A \Phi.\]
\end{definition}

\begin{remark}
We should emphasise the difference between the Riemannian 3-manifold $M = C \times_q S^1_R$ appearing in this section and the derived stack $C \times_q S^1_B$ (the mapping torus) appearing in the previous section.  These should be thought of as smooth and algebraic realizations of the same object (justified by the comparison Theorem \ref{monopole_qconn_comparison_thm}) but they are a priori defined in different mathematical contexts.
\end{remark}

We can rephrase the data of a monopole on $M$ as follows.  Let $C_0 = C \times \{0\}$ be the fiber over $0$ in $S^1$, viewed as a Riemann surface.  Let $P$ be the restriction of the complexified bundle $\bo P_\CC$ to $C_0$.  Consider first the restriction of the complexification of $A$ to a connection $A_0$ on $P$ over $C_0$.  The $(0,1)$ part of $A_0$ automatically defines a holomorphic structure on $P$.  We can introduce an additional piece of structure on this holomorphic $G$-bundle.  In order to do so we can decompose the Bogomolny equation into one real and one complex equation as follows.
\begin{align*}
F_{A_0} - \nabla_t \Phi &= 0 \\
[\ol{\del}_{A_0}, \nabla_t - i\Phi \d t] &= 0 
\end{align*}
where $\nabla_t$ is the component of the covariant derivative $\d_A$ normal to $C_0$.  

\begin{definition} 
From now on we'll use the notation $\mc A$ for the combination $\nabla_t - i\Phi \d t$: an element of the space $\Omega^0(C_0, \gg_P)\d t$ of sections of the complex vector bundle $\gg_P$ on the complex curve $C_0$. 
\end{definition}

Let us now introduce singularities into the story.  We'll keep the description brief, referring the reader to \cite{CharbonneauHurtubise, Smith} for details.
\begin{definition}
  Let $D \sub M$ be a finite subset.  Let $\omega^\vee$ be a choice of coweight for $G$.  A monopole on $M \bs D$ has \emph{Dirac singularity} at $z \in D$ with charge $\omega^\vee$ if locally on a neighbourhood of $z$ in $M$ it is obtained by pulling back under $\omega^\vee$ the standard Dirac monopole solution to the Bogomolny equation where $\Phi$ is spherically symmetric with a simple pole at $z$ () and the restriction of a connection $A$ to a two-sphere $S^2$ enclosing the singularity defines a $U(1)$ bundle on this $S^2$ of degree $1$ so that
    \[\frac{1}{2\pi} \int_{S^2} F = 1 .\]
  See e.g. \cite[Section 2.2]{CharbonneauHurtubise} for a more detailed description.
\end{definition}

We can also introduce a framing (or a reduction of structure group as in the trigonometric example, though we won't consider the latter in this paper).  As usual let $c \in C$ be a point fixed by the automorphism $q$.
\begin{definition}
  A monopole on $M$ with \emph{framing} at the point $c \in C$ is a monopole $(\bo P,A,\Phi)$ on $M$ (possibly with Dirac singularities at $D$) along with a trivialization of the restriction of $\bo P$ to the circle $\{c\} \times S^1_R$, with the condition that the holonomy of $\mc A$ around this circle lies in a fixed conjugacy class $f \in G/G$.
\end{definition}

The moduli theory of monopoles on general compact 3-manifolds was described by Pauly \cite{Pauly}.  In this paper we'll be interested in moduli spaces of monopoles on 3-manifolds of the form $C \times S^1$, possibly with a fixed framing at a point in $C$.  

\begin{remark}
The moduli space of periodic monopoles on $\RR^2 \times S^1$ specifically has been studied in the mathematics literature by Foscolo \cite{FoscoloDef} , applying the analytic techniques of deformation theory to earlier work on periodic monopoles by Cherkis and Kapustin \cite{CherkisKapustin1, CherkisKapustin2}. This analysis considers a less restrictive boundary condition at infinity in $\RR^2$ than a framing, and therefore requires more sophisticated analysis than we'll need to consider in the present paper.
\end{remark}

In the cases of interest to us the moduli space of periodic monopoles can be obtained as a hyperk\"ahler quotient.  Let us focus initially on the rational case, meaning monopoles on $M = \bb{CP}^1 \times_\eps S^1_R$ with Dirac singularities at $D \times \{t_0\}$ and a framing at $\infty$.  Consider the infinite-dimensional vector space $\mc V$ consisting of pairs $(A,\Phi)$ where $A$ is a connection on a fixed principal $G_\RR$-bundle $\bo P$ on $M$, $\Phi$ is a section of $\gg_{\bo P}$, and $(A,\Phi)$ have a Dirac singularity with charge $\omega^\vee_{z_i}$ at each $(z_i,t_0)$ in $D \times \{t_0\}$.  Let $\mc G$ be the group of gauge transformations of the bundle $\bo P$.

The hyperk\"ahler moment map is given by the Bogomolny functional, namely
\begin{align*}
\mu \colon \mc V &\to \Omega^1(M; (\gg_\RR)_{\bo P}) \\
(A,\Phi) &\mapsto \ast F_A - \d_A \Phi.
\end{align*}

\begin{definition} \label{monopole_moduli_def}
Let $D$ be a finite subset $\{(z_1,t_1), \ldots, (z_k, t_k)\}$ of points in $M = \bb{CP}^1 \times_\eps S^1_R$, and let $\omega^\vee_{i}$ be a choice of coweight for each point in $D$. The moduli space $\mon_G(M, D, \omega^\vee)$ is the hyperk\"ahler quotient
\[\mon_G(M, D, \omega^\vee) = \mu^{-1}(0) / \mc G.\]
\end{definition}

Now let us address the relationship between periodic monopoles and $q$-connections.  Suppose from now on that $q$ is in the identity component of the group of automorphisms of $C$ (fixing the framing point $c$ if present).

\begin{theorem} \label{monopole_qconn_comparison_thm}
There is an analytic isomorphism between the moduli space of polystable monopoles on $C \times_q S^1$ with Dirac singularities at $D \times \{t_0\}$ (and a possible framing on $\{c\} \times S^1$) and the moduli space of $q$-connections on $C$ with singularities at $D$ and framing at $\{c\}$.  More precisely there is an analytic isomorphism
\[H \colon \mon^{(\fr)}_G(C \times_q S^1, D \times \{t_0\}, \omega^\vee) \to \qconn_G^{\text{ps,(fr)}}(C, D, \omega^\vee)\]
given by the holonomy map around $S^1$, i.e. sending a monopole $(\bo P, \mc A)$ to the holomorphic bundle $P = (\bo P_\CC)|_{C_0}$ with $q$-connection $g = \Hol_{S^1}(\mc A) \colon P \to q^*(P)$.
\end{theorem}

\begin{remark}
Note that in this statement we assumed that all the singularities occur in the same location in $S^1$, i.e. in the same slice $C \times {t_0}$.  This assumption is not necessary, but there is a constraint on the possible locations of the singularities as explained in \cite[Proposition 3.5]{CharbonneauHurtubise}.  
\end{remark}

\begin{proof}
\chris{check}
This follows by the same argument as that given by Charbonneau-Hurtubise \cite{CharbonneauHurtubise} and Smith \cite{Smith}.  More explicitly, first let us think about injectivity, so let $(\bo P, \mc A)$ and $(\bo P', \mc A')$ be a pair of periodic monopoles on $C \times_q S^1$ with images $(P,g)$ and $(P', g')$ respectively, and choose a bundle isomorphism $\tau \colon P \to P'$ intertwining the $q$-connections $g$ and $g'$.  One observes first that $\bo P$ and $\bo P'$ are also isomorphic $G$-bundles since, by intertwining with the $q$-connections, we have an isomorphism $\bo P|_{C \times \{t\}} \to \bo P'|_{C \times \{t\}}$ for every $t \in S^1$.  That the monopole structures also match up follows by the same argument as in \cite[Proposition 4.7]{CharbonneauHurtubise}.

For surjectivity, again we'll match the argument in the case where $q=\id$.  We begin by extending a holomorphic $G$-bundle $P$ on $C_0$ with $q$-connection $g$ to a $G$-bundle on $M \bs (D \times \{t_0\}) = (C \times_q S^1_R) \bs (D \times \{t_0\})$  Let $\gamma \colon [-2\pi R,2\pi R] \to \aut(C)$ be a geodesic with $\gamma(-2\pi R) = q^{-1}$, $\gamma(0)=1$ and $\gamma(2\pi R) = q$.  Let $\wt M$ be the 3-manifold
\[\wt M = ((-2\pi R, 2\pi R) \times C) \bs \bigcup_{j=1}^k (A^+_j \cup A^-_j)\]
where $A^+_j$ is the arc $\{(t+ t_0,\gamma(t)(z_j)) \colon t \in (0, 2\pi R - t_0]\}$ and $A^-_j$ is the arc $\{(t + t_0 - 4 \pi R,\gamma(t)(z_j)) \colon t \in [2\pi R-t_0, 2 \pi R)\}$.

Let $\pi \colon \wt M \to C$ be the projection sending $(t,z)$ to $\gamma(t)(z)$.  The bundle $P$ pulls back to a bundle $\pi^*(P)$ on $\wt M$.  We obtain a bundle on $M \bs (D \times t_0)$ by applying the identification $(t,z) \sim (t - 2 \pi R, q(z))$.  This bundle extends to an $S^1$-invariant holomorphic $G$-bundle on $M \times S^1$.  The remainder of the proof -- verifying the existence of the monopole structure associated to an appropriate choice of hermitian structure -- consists of local analysis which is independent of the value of the parameter $q$. 

It remains to remark on the compatibility of framing data on the two sides.  A trivialization of the bundle $\bo P$ along the circle $\{c\} \times S^1$ yields a trivialization of the fiber of the bundle $P$ at $c$.  The condition that the holonomy around the circle at $c$ is $f$ fixes the value of the $q$-connection at $c$ to be in the conjugacy class $f$.
\end{proof}

\begin{remark}
Mochizuki \cite{Mochizuki} proved a stronger result in the rational case for the group $G = \GL_n$.  He allows not just a framing at infinity in $\bb{CP}^1$ but also a singularity encoded in terms of a $B$-reduction of the bundle. 
\end{remark}

\subsection{Deformation Theory} \label{def_section}
In the next section we'll compare symplectic forms on these moduli spaces.  In order to do so it will be important to understand the tangent spaces at a point of the source and target.  There's a natural description of these tangent spaces in terms of the hypercohomology of certain cochain complexes.

From now on we'll focus on the example we're most interested in.  That is, we'll exclusively study the rational situation where $C = \bb{CP}^1$ and we fix a framing point $c = \infty$.  In this case we can use the description as a hyperk\"ahler reduction.  For more general deformation theory calculations we refer to Foscolo \cite{FoscoloDef}.  Recall that we can write
\[\mon^{(\fr)}_G(C \times_q S^1, D \times \{t_0\}, \omega^\vee) \iso \mu^{-1}(0)/ \mc G\]
where $\mc G$ is the group of gauge transformations of $\bo P$ and $\mu \colon \mc V \to \Omega^1(M; (\gg_R)_{\bo P})$ is the Bogomolny functional $(A,\Phi) \mapsto \ast F_A - \d_A \Phi$.  The tangent complex to this hyperk\"ahler quotient at a point $(\bo P, \mc A)$ can be written as $\Omega^0(M \!\bs\! D; (\gg_\RR)_{\bo P})[1] \to \bb T_{\mu^{-1}(0)}$ where $\bb T_{\mu^{-1}(0)}$ is the tangent complex to the zero locus of the moment map, concentrated in non-negative degrees. Roughly speaking $\bb T_{\mu^{-1}(0)} = \bb T_{\mc V} \overset {\d\mu} \to \Omega^1(M; (\gg_R)_{\bo P})[-1]$.  

More explicitly, following \cite{FoscoloDef} let 
\[\mc F^{\mr{mon}}_{\bo P, \mc A} = \left(\xymatrix{
\Omega^0(M \!\bs\! D; (\gg_\RR)_{\bo P}) \ar[r]^(.36){\d_1} &\Omega^1(M \!\bs\! D; (\gg_\RR)_{\bo P}) \oplus \Omega^0(M \!\bs\! D; (\gg_\RR)_{\bo P}) \ar[r]^(.64){\d_2} &\Omega^1(M \!\bs\! D; (\gg_\RR)_{\bo P})
}\right) \otimes_\RR \CC\]
placed in degrees $-1$, 0 and 1 where $\d_1(g) = -(\d_A(g),[\Phi, g])$ and $\d_2(a,\psi) = \ast \d_A(a) - \d_A(\psi) + [\Phi,a]$.  Write $\d_{\mr{mon}}$ for the total differential.

\begin{remark}
Here we've chosen a point in the twistor sphere, forgetting the hyperk\"ahler structure and retaining a holomorphic symplectic structure.  In other words we've identified the target of the hyperk\"ahler moment map -- the space of imaginary quaternions -- with $\RR \oplus \CC$, which is equivalent to choosing a point in the unit sphere of the imaginary quaternions: the twistor sphere.  
\end{remark}

\begin{remark} \label{monopole_holo_restriction_rmk}
If we restrict $\mc F^{\mr{mon}}_{\bo P, \mc A}$ to a slice $C_t = C \times \{t\}$ in the $t$-direction we can identify it with a complex of the form
\[\Omega^\bullet(C_t; \gg_P)[1] \overset {[\Phi,-]} \to \Omega^\bullet(C_t; \gg_P)\]
with total differential given by $\d_A$ on each of the two factors along with the differential $[\Phi,-]$ mixing the two factors.  These two summands each split up into the sum of a Dolbeault complex on $C$ with its dual.  That is, there's a natural subcomplex of the form
\[\Omega^{0,\bullet}(C_t; \gg_P)[1] \overset {[\Phi,-]} \to i \Omega^{0,\bullet}(C_t; \gg_P) \d t\]
where the internal differentials on the two factors are now given by $\ol \dd_{A_0}$.  This complex is in turn quasi-isomorphic to the complex
\[\Omega^\bullet(S^1; \Omega^{0,\bullet}(C_t;\gg_P))[1]\]
with total differential $\ol \dd_{A_0} + \d_{\mc A}$.
\end{remark}

\begin{remark}
If one introduces a framing at a point $c \in C$ then we must correspondingly twist the complex $\mc F^{\mr{mon}}$ above by the line bundle $\OO(c)$ on $C$.  So in that case we define \chris{check}
\[\mc F^{\text{mon,fr}}_{(\bo P,\mc A)} = \mc F^{\mr{mon}}_{\bo P, \mc A} \otimes (\CC_{S^1} \boxtimes \OO(c)).\]
\end{remark}

The following is proved in \cite{FoscoloDef}.
 
\begin{prop}
The tangent space of $\mon_G(S^1 \times C, D \times \{t_0\}, \omega^\vee)$ at the point $(\bo P,\mc A)$ is quasi-isomorphic to the hypercohomology $\bb H^0(C \times S^1; \mc F'_{(\bo P,\mc A)})$ of a subsheaf $\mc F' \sub \mc F^{\mr{mon}}$ where growth conditions are imposed on the degree 0 part of $\mc F^{\mr{mon}}$ near the singularities.
\end{prop}

Now let us consider the tangent complex to the moduli space of $q$-connections.  For the arguments in this article we'll only need to carefully consider the case $q=\id$ of multiplicative Higgs bundles, but we'll include some remarks regarding the more general case.  In this case the calculation was performed by Bottacin \cite{Bottacin}, see also \cite[Section 4]{HurtubiseMarkman}. Fix a multiplicative Higgs bundle $(P,g)$ on $C$.  We consider the sheaf of cochain complexes on $C$
\[\mc F_{(P,g)} = (\gg_P[1] \overset {\Ad_g} {\to} \gg_P(-D))\]
in degrees $-1$ and 0 with differential given by the adjoint action of $g$.  More precisely let $L_g$ and $R_g$ be the bundle maps $\gg_P \to \gg_P$ obtained as the derivative of left- and right-multiplication.  Then $\Ad_g = L_g - R_g$.  We can alternatively phrase this, as in \cite[Section 4]{HurtubiseMarkman}, as follows.  Define $\ad(g)$ to be the vector bundle
\[\ad(g) = (\gg_P \oplus \gg_P)/\{(X, -g X g^{-1}): X \in \gg_P\}.\]
Then we can write $\mc F$ as the sheaf of complexes
\[\mc F_{(P,g)} = (\gg_P[1] \overset {\Ad_g} {\to} \ad(g))\]
where now $\Ad_g$ is just the map $X \mapsto [(X,-X)]$.

\begin{remark}
  If one introduces a framing at a point $c \in C$ then we must correspondingly twist the complex $\mc F$ above by the line bundle $\OO(c)$ on $C$, i.e. we restrict to deformations that preserve the framing and therefore are zero at the point $c$.  So in that case we define
\[\mc F^\fr_{(P,g)} = (\gg_P[1] \overset {\Ad_g} {\to} \gg_P(-D)) \otimes \OO(c).\]
\end{remark}

\begin{remark}
For more general $q$ we should modify this description by replacing $g$ by a $q$-connection.  Note that one can still define the ($q$-twisted) adjoint action $X \mapsto g X g^{-1}$ using a $q$-connection, and so we can still define the complex
\[\mc F_{(P,g)} = (\gg_P[1] \overset {\Ad_g} {\to} \ad(g))\]
just as in the untwisted case \chris{check}.
\end{remark}

This complex defines the deformation theory of the moduli space of multiplicative Higgs bundles.

\begin{prop}[{\cite[Proposition 3.1.3]{Bottacin}}]
The tangent space of $\mhiggs_G(C, D, \omega^\vee)$ at the point $(P,g)$ is quasi-isomorphic to the hypercohomology $\bb H^0(C; \mc F_{(P,g)})$ of the sheaf $\mc F$.
\end{prop}

\begin{remark}
The remaining hypercohomology of the sheaf $\mc F_{(P,g)}$ generically has dimension $\dim \mf z_{\gg}$ (or 0 if we fix a framing at $c \in C$) in degree $-1$, and dimension $\mr{genus}(C) \cdot \dim \mf z_{\gg}$ in degree $1$.  However the moduli space $\mhiggs_G(C, D,\omega^\vee)$ is in fact a smooth algebraic variety.  This follows from a result of Hurtubise and Markman \cite[Theorem 4.13]{HurtubiseMarkman}, noting that their argument does not rely on the curve $C$ being of genus 1.
\end{remark}

\begin{corollary} \label{dim_of_moduli_space_cor}
In the rational case, the moduli space $\mhiggs^\fr_G(\bb{CP}^1, D, \omega^\vee)$ has dimension 
\[2 \sum_{z_i \in D} \langle \rho, \omega^\vee_{z_i} \rangle.\]
\end{corollary}

\begin{proof}
One can use the same argument as \cite[]{HurtubiseMarkman} (see also \cite[Proposition 5.6]{CharbonneauHurtubise}) with the additional observation that tensoring by the line bundle $\OO(\infty)$ kills the outer cohomology groups ($\bb H^{-1}$ and $\bb H^1$ with our degree conventions, which differ from the conventions of loc. cit. by one).  Indeed, $\bb H^{-1}$ consists of sections of $\gg_P$ that are annihilated by $\Ad_g$ (given for generic $g$ by constant sections valued in $\mf z_{\gg}$) while vanish at $\infty$, which are necessarily 0.  Likewise we can use the equivalence between the sheaf $\mc F_{(P,g)}$ and its Serre dual to see that $\bb H^1$ also vanishes.  Finally the Euler characteristic of the two step complex is unchanged by tensoring by $\OO(\infty)$. 
\end{proof}

In order to calculate with this hypercohomology group we'll use a \v Cech resolution.  This will be straightforward for the multiplicative Higgs moduli space, and we'll use the isomorphism of Theorem \ref{monopole_qconn_comparison_thm} to give an analogous description on the monopole side.    We define a cover $\mc U = \{U_0, U_1, \ldots, U_k, U_\infty\}$ of $C$ as follows.  Let $U_i$ be a contractible open neighbourhood of the point $z_i$ and let $U_\infty$ be a contractible analytic open neighbourhood of $c \in C$, all chosen to be pairwise disjoint.  Let $U_0$.  Finally let $U_0 = C \bs (D \cup \{c\})$.  Since the $U_i$ are contractible and the remaining subset $C \bs (D \cup \{c\})$ is an affine algebraic curve, which means that for any quasi-coherent sheaf of cochain complexes the higher cohomology groups vanish.  Likewise for the intersections: the punctured open sets $U_i^\times$ are analytic open sets of an affine curve.  

Specify a representative 0-cocycle $(\alpha_\infty, \{\alpha_i\}, \alpha_0, \beta_\infty, \{\beta_i\})$ for the \v Cech cohomology group with respect to our chosen cover $\mc U$.  Explicitly a 0-cochain is given by the following data:
\begin{align*}
 \alpha_\infty &\in \ad(g)(1)(U_\infty) \\
 \alpha_i &\in \ad(g)(-1)(U_i) \text{ for } i = 1,\ldots,k \\
 \alpha_0 &\in \ad(g)(C \bs (D \cup \{\infty\})) \\
 \beta_\infty &\in \gg_P(U^\times_\infty) \\
 \beta_i &\in \gg_P(U^\times_i) \text{ for } i=1,\ldots,k
\end{align*}
where the notation $(\pm 1)$ indicates tensoring by the line bundle $\OO(\pm 1)$.

Being a 0-cocycle means that $(\alpha_\infty - \alpha_0)|_{U^\times_\infty} = \mr{Ad}_g(\beta_\infty)$ and $(\alpha_i - \alpha_0)|_{U^\times_i} = \mr{Ad}_g(\beta_i)$ for each $i$.  We consider 0-cocycles modulo 0-coboundaries of the form $(\mr{Ad}_g(f_\infty), \{\mr{Ad}_g(f_i)\}, \mr{Ad}_g(f_0), (f_\infty -  f_0)|_{U_\infty^\times}, \{(f_i - f_0)|_{U_i^\times}\})$.  In fact $\mr{Ad}_g$ is an isomorphism on $U_0$ for the sections $\alpha_0$ of $\gg_P$ that occur: those with no poles or zeroes in $U_0$.  That means that we can add a coboundary to force $\alpha_0=0$.  %We can likewise use the freedom in $f_i$ to find a representative cocycle where the $\beta_i$ are all zero.  We now have a unique representative cocycle of the form $(\alpha_\infty, \{\alpha_i\}, 0,0,0\})$.

Now, rather than describing the tangent space to the moduli space of monopoles we'll define a complex that maps to it which we can define locally with respect to the cover $\mc U$.  Consider the open cover $\mc U \times S^1 = \{U_i \times S^1\}$ of $M = C \times S^1$.  For each element $U_i$ of the cover we can define a map
\begin{align*}
\mc F'(U_i \times S^1) &\to (\Omega^0(S^1; \Omega^{0,\bullet}(U_i; \gg_P)) \to \Omega^1(S^1; \Omega^{0,\bullet}(U_i; \gg_P)(D_{U_i})))[1] \\ %With singularity conditions to be fixed
&\iso \gg_P(U_i)[1] \to \gg_P(U_i)(D|_{U_i}) \\
&\iso \mc F(U_i)
\end{align*}
where we restrict the sheaf $\mc F'$ whose hypercohomology calculated the monopole tangent complex to the holomorphic part of the slice at $\{t\} \in S^1$.  Altogether this defines a map from the \v Cech cohomology with respect to this cover to the tangent complex of the moduli space of monopoles.  That is, we have an explicit map
\[{\mr {\check H}}^\bullet(M, \mc U \times S^1, \mc F') \to \bb H^\bullet(M; \mc F')\]
that factors through the (isomorphic) tangent complex of the moduli space of multiplicative Higgs bundles.  To verify this we need only note that these maps commute with the differentials in the \v Cech complex, i.e. with the restriction to the intersection of a pair of open sets. 

Explicitly a 0-cochain in this \v Cech complex is given by $(\alpha_\infty, \{\alpha_i\}, \alpha_0, \beta_\infty, \{\beta_i\})$ where now 
\begin{align*}
 \alpha_\infty &\in \mc F^{\mr{mon}}_0(U_\infty \times S^1) \\
 \alpha_i &\in \mc F^{\mr{mon}}_0(U_i\times S^1) \text{ for } i = 1,\ldots,k \\
 \alpha_0 &\in \mc F^{\mr{mon}}_0(U_0\times S^1) \\
 \beta_\infty &\in \Omega^0_\CC(U_\infty^\times \times S^1; \gg_{\bo P_\CC}) \\
 \beta_i &\in \Omega^0_\CC(U_i^\times \times S^1; \gg_{\bo P_\CC}) \text{ for } i=1,\ldots,k.
\end{align*}
Here we write $\mc F^{\mr{mon}}_0$ to indicate the degree 0 term in the cochain complex.  There's a similar condition for being a 0-cocycle involving the differential $\d_{\mr{mon}}$.  %Again we can add a 0-coboundary to find a representative cocycle of the form $(\alpha_\infty, \{\alpha_i\}, 0,0,0\})$.

To conclude this subsection it will also be important to have an explicit description of the derivative of the holonomy map $H$ as a map between tangent spaces.  We can describe this map using our \v Cech resolutions on each contractible open set $U_i$ individually.

\begin{prop} \label{local_derivative_description_prop}
The derivative $\d H \colon \bb H^\bullet(U_i \times S^1 ; \mc F'_{(\bo P,\mc A)}) \to \bb H^\bullet(\bb D_i; \mc F_{(P,g)})$ is given on an open patch $U_i \times (0,2\pi)$ by the formula
\[\d H(\alpha_i) = \d H(\d_{\mr{mon}} b_i) = b_i(2\pi)H(\mc A) - H(\mc A)b_i(0)\]
where $i = 1, \ldots, k$ or $\infty$.  More precisely by $b_i(2\pi)$ and $b_i(0)$ we mean the limit of $b_i(t)$ as $t \to 2\pi$ or 0 respectively.
\end{prop}

\begin{proof}
Note that the right-hand side is the derivative at $\mc A$ of the map $B_i \mapsto B_i(2\pi)H(\mc A)B_i(0)^{-1}$ where $B_i \in \Omega^0((U_i \times (0,2\pi)) \bs \{(z_i, t_0)\}; \gg_P)$.  This is the definition of the action of the group of gauge transformations on the holonomy $H(\mc A)$ from $t=0$ to $2\pi$. \chris{say more?}
\end{proof}

\section{Symplectic Structures} \label{symp_section}
\subsection{The Example of $\GL_2$}
\chris{Ultimately we might move this, but we should probably include a discussion of what happens for $\GL_2$ before or in parallel to discussing the general story.}

\vasily{Chris, is the idea to put here explicit expressions for Higgs field in 
terms of some chart
  Darboux coordinates $(p,q)$ that we have discussed in the past drafts
  draft?. If all singularities are miniscule, then for $GL_2$ the leaves could
  be described quite explicitly as products of regular semi-simple co-adoint orbits in $\mathfrak{gl}_2$. Rouven is finishing a draft with explicit parametrization by Darboux coordinates of a certain family of symplectic leaves for $GL_n$. Perhaps here we can put a reference on it. } 
  
  \chris{My idea was to include the discussion of, at least, the simple example that you worked out for $\GL_2$ with minimal singularities where you described the geometry of the moduli space and gave an expression for the symplectic form (in Section 2 of the file 2017\_11\_30\_ghiggs).  I really only had in mind for $\GL_2$ trying to give a more explicit argument for what will appear below, i.e. that the moduli spaces after fixing residues were naturally symplectic.  But maybe we could do what you suggest and described the symplectic leaves in some Darboux coordinates and refer to \cite{FrassekPestun} for proof?} 

\subsection{Symplectic Structures for General $G$} \label{general_symplectic_sec}
We begin the more abstract general analysis by briefly discussing the holomorphic symplectic structure on the moduli space of periodic monopoles on $\bb{CP}^1$ following the analysis of Cherkis and Kapustin \cite{CherkisKapustin1, CherkisKapustin3}.  This structure arises from the description we gave as a hyperk\"ahler quotient.  To describe it specifically, let $\delta^{(1)} \mc A$ and $\delta^{(2)} \mc A$ be two tangent vectors at $(\bo P, \mc A)$ to the moduli space of monopoles.  Recall that $\mc A$ denotes the combination $\nabla_t - i\Phi \d t$.  Choose representatives for these two tangent vectors of the form $\alpha_i$ and $\alpha'_i$ respectively  in the \v Cech resolution we described above.  Then we can write the holomorphic symplectic form coming from the hyperk\"ahler reduction in terms of the symplectic pairing on the infinite-dimensional vector space $\mc V$, which is given by the Killing form on $\gg$ along with the wedge pairing of differential forms.  So summing over the local patches in our \v Cech resolution we can write it as
\begin{align*}
\omega_{\mr{mon}}(\delta^{(1)} \mc A, \delta^{(2)} \mc A) &= \int_{M} \kappa(\delta^{(1)} \mc A, \delta^{(2)} \mc A) \\
&= \sum_{i=1}^k \int_{U_i \times S^1} \kappa(\alpha_i, \alpha'_i) \d z \d t
\end{align*}
where the contributions to the integral away from the $U_i$ vanish.

Our goal in this section will be describe a symplectic structure on the moduli space of multiplicative Higgs bundles -- the rational analogue of Hurtubise and Markman's symplectic structure -- and then prove that it's equivalent to this symplectic form under the equivalence between multiplicative Higgs bundles and periodic monopoles.

\begin{remark}
From now on we'll write $\langle - , - \rangle$ to denote the residue pairing between elements of $L\gg$.  That is, 
\[\langle g_1, g_2 \rangle = \oint_{\bb D^\times} \kappa(g_1, g_2).\]
\end{remark}

\begin{lemma}
There is a natural non-degenerate anti-symmetric pairing on the moduli space $\mhiggs_G^{\fr}(\bb{CP}^1,D,\omega^\vee)$.  In terms of our \v Cech description it is described by the formula 
\[\omega((\{\alpha_i\}, \{\beta_i\}), (\{\alpha'_i\},\{\beta_i'\})) = \frac 12 \sum_i \langle \rho_g^*(\alpha'_i + \alpha'_0)|_{U^\times_i}, \rho_g^*(\Ad_g^*)^{-1}(\beta_i) \rangle - \langle \rho_g^*(\alpha_i + \alpha_0)|_{U^\times_i}, \rho_g^*(\Ad_g^*)^{-1}(\beta'_i) \rangle,\]
where we use the Killing form to identify $\beta'_i$ with a $\gg^*$-valued form, and where $\rho_g^*$ is the pullback along the right multiplication by $g$.  In this expression the sum is over $i=1,\ldots,k$ and $i=\infty$.
\end{lemma}

\begin{remark}
How should we think about this structure?  There's an intuitive description just as in Hurtubise and Markman's elliptic moduli space.  The pairing is induced from the natural equivalence between the tangent and cotangent spaces of the moduli space of multiplicative Higgs bundles as described in Section \ref{def_section}.  That is, there's a map of complexes of sheaves
\[\xymatrix{
(\mc F^\fr_{(P,g)})^*[2] \ar@{=}[r] \ar[d] &\Big(\gg^*_P(D)[1] \otimes \OO(-c) \ar[d]^{\kappa \circ \Ad_g^*} \ar[r]^{\Ad_g^*} &\gg^*_P \OO(-c)\Big)\ar[d]^{\Ad_g \circ \kappa^{-1}} \\
\mc F^\fr_{(P,g)} \ar@{=}[r] &\Big(\gg_P[1] \OO(-c) \ar[r]^{\Ad_g} &\gg_P(-D) \OO(-c)\Big)
}\]
where here $\kappa$ denotes the isomorphism from $\gg_P \to \gg^*_P$ induced by the Killing form.  The top line is the Serre dual complex to the bottom line; note that the incorporation of the framing was necessary for this to be the case (that is, we're using the relative Calabi-Yau structure on the pair $(\bb{CP}^1, c)$).  Taking 0th hypercohomology we obtain a map from the cotangent space to the tangent space of our moduli space of multiplicative Higgs bundles.
\end{remark}

\begin{remark}
At this point we will not prove that the pairing is a symplectic structure: we will not verify the Jacobi identity.  While we expect that it is possible to prove this directly using the techniques of \cite[Section 5]{HurtubiseMarkman}, instead we'll see this below by proving that the pairing coincides with the symplectic pairing on the moduli space of periodic monopoles.
\end{remark}

\begin{proof}
We need only verify that the pairing described is non-degenerate.  It suffices to check non-degeneracy for each individual summand, i.e. that for fixed $i$, given $\alpha_i$ if the expression $\langle \rho_g^*(\alpha_i)|_{U^\times_i}, \rho_g^*(\Ad_g^*)^{-1}(\beta'_i) \rangle$ vanishes for all $\beta'_i$ then $\alpha_i=0$ (and the same with $\alpha$ and $\beta$ interchanged).  This is immediate because the residue pairing $\langle - , - \rangle$ is non-degenerate and the map $(\Ad_g^*)^{-1}$ is an isomorphism on the open set $U^\times_i$.
\end{proof}

\begin{remark}
If we do not fix residue data at the punctures, the full moduli space $\mhiggs_G^\fr(\bb{CP}^1,D)$, or indeed the moduli space $\mhiggs_G^{\text{fr,sing}}(\bb{CP}^1)$ with arbitrary singularities, will have an ind-Poisson structure, analogous to the ind-Poisson structure in the elliptic moduli space of Hurtubise and Markman.  One uses the filtration discussed in Remark \ref{ind_structure_remark} in order to define this structure: one needs to check that the moduli space  $\mhiggs_G^\fr(\bb{CP}^1,D, \preceq \omega^\vee)$ is Poisson, and that the inclusion of the symplectic moduli space $\mhiggs_G^\fr(\bb{CP}^1,D, \omega^\vee)$ is a Poisson map.  We won't discuss this in any more detail here, except to remark that this should follow using the techniques of \cite[Section 7]{HurtubiseMarkman}.  We will, however, discuss the close relationship between this Poisson structure and the rational Poisson Lie group, and the consequences for quantization, in Section \ref{quantization_section}.
\end{remark}

In fact the moduli space is not only symplectic, but the total space of a completely integrable system -- the multiplicative Hitchin system described in Definition \ref{mult_Hitchin_system_def}.  We already know that the generic fibers are half-dimensional tori, it remains to verify that they're isotropic.

\begin{prop}
The fibers of the multiplicative Hitchin fibration, in the rational case, are isotropic.
\end{prop}

\begin{proof}
In order to check this we need to describe the subspace of the tangent space to $\mhiggs^\fr_G(C,D,\omega^\vee)$ at a point $(P,g)$ tangent to the Hitchin fiber in our \v Cech description.
\chris{Todo: this needs further thought.  I think the tangent complex to the fiber has $\beta_i, \beta_\infty$ able to take arbitrary values in $\mf c_g$ the centralizer of $g$ (fix $g$ to be rss and this is a Cartan), and the $\alpha_i$ determined by the cocycle condition.  When the forms are valued in $\mf c_g$ the operators $\Ad_g$ and $(\Ad_g^*)^{-1}$ act trivially, which makes our pairing symmetric.  Since it's also anti-symmetric it vanishes, which implies isotropy.  So if we verify that those are indeed the required tangent vectors then we're done.  Note that if the tangent complex of the fiber is given by the hypercohomology of a subsheaf concentrated in degree $-1$ then for dimension reasons I think it has to be this.}
\end{proof}

We'll now begin the comparison between the pairing described above and the symplectic structure on the moduli space of periodic monopoles.

\begin{prop} \label{qconn_symp_description}
The non-degenerate pairing on the moduli space of multiplicative Higgs bundles can be written in the form
\begin{equation}
\label{eq:resid}
\omega(\delta g, \delta g') = \sum_{i=1}^d \langle b_i^L g^{-1}, b^{'R}_i g^{-1} \rangle - (b \leftrightarrow b')
\end{equation}
where $\delta g$ is represented on $U_i$ by a pair $(b^L, b^R) \in \Ad_g$.  Here and throughout the notation ``$- (b \leftrightarrow b')$'' denotes antisymmetrization.
\end{prop}

\begin{proof}
 Specify a representative cocycle $(\alpha_\infty, \{\alpha_i\}, \alpha_0, \beta_\infty, \{\beta_i\})$ for the first \v Cech cohomology group. By addition of a coboundary we can assume that $\alpha_0=0$, and to force $\alpha_\infty$ to land in $z\gg[[z]]$ and each $\alpha_i$ to land in $z^{-1}\gg[[z]]$.  We've now fixed a representative cocycle -- there's no further gauge freedom.  Ignoring the factor at infinity -- since after we add these coboundaries the residue pairing there vanishes -- the pairing we end up with looks like 
\begin{align*}
\omega(\delta g, \delta g') &= \frac 12 \sum_{i=1}^d \langle \rho_g^*(\alpha'_i + \alpha'_0)|_{U^\times_i},\rho_g^* (\Ad_g^*)^{-1}( \beta_i) \rangle - ((\alpha,\beta) \leftrightarrow (\alpha',\beta')\\ 
&= \frac 12 \sum_{i=1}^d \langle \rho_g^*(\alpha'_i + \alpha'_0)|_{U^\times_i}, \rho_g^*(\Ad_g^*)^{-1}(\Ad_g^{-1}(\alpha_i - \alpha_0)) \rangle - (\alpha \leftrightarrow \alpha')\\
&= \frac 12 \sum_{i=1}^d \langle \rho_g^*(\alpha'_i)|_{U^\times_i}, \rho_g^*(\Ad_g^*)^{-1}(\Ad_g^{-1}(\alpha_i)|_{U^\times_i}) \rangle - (\alpha \leftrightarrow \alpha').
\end{align*}

We can compute the composite operator $(\Ad_g^*)^{-1}\Ad_g^{-1}$ on $\mr{ad}(P)(U^\times_i)$.  It's given by the inverse of the operator $\Ad_g\Ad_g^*$ which sends a pair $(b^L, b^R) \in \mr{ad}(P)(U^\times_i)$ to $(b^L - b^R, b^R - b^L) \sim (b^L + g^{-1} b^L g, b^R - g b^R g^{-1})$.  Denote this by $((1+A_g)b^L,(1-A_{g^{-1}})b^R)$.  We can describe the inverse using the expansion $(1+A_g)^{-1} = 1 - A_g + A_g^2 + \cdots$.  After applying our gauge transformation above the remaining degree of freedom in $\alpha_i$ is its $z^{-1}$ term.  Each time we apply $A$ it raises the order in $z$ of this term by one, so only the linear summand $-A_g$ of $(1+A_g)^{-1}$ contributes to the residue pairing.  In the pairing we need to use the invariant pairing on $\gg((z)) \oplus \gg((z))$ that vanishes on the subalgebra spanned by $(X, A_g(X))$, so we take the difference of the residue / Killing form pairings on the two summands.
\begin{align*}
\omega(\delta g, \delta g') &= \frac 12 \sum_{i=1}^d  - \langle \rho_g^*A_g b^{'L}_i, \rho_g^*b^{R}_i \rangle - \langle \rho_g^*b^{'L}_i, \rho_g^*A_{g^{-1}}b^{R}_i \rangle- (b \leftrightarrow b') \\ 
&= \sum_{i=1}^d \langle b^L_ig^{-1}, b^{'R}_i g^{-1} \rangle  - (b \leftrightarrow b')
\end{align*}
using cyclic invariant to identify the two terms. 
\end{proof}

We'll use this description of the pairing to compare our symplectic structure with the Poisson structure of the rational Poisson Lie group in Section \ref{quantization_section}.

We can now establish our main result.
\begin{theorem} \label{symplectic_comparison_thm}
The symplectic structure on $\mon_G^\fr(\bb{CP}^1 \times S^1,D \times\{0\},\omega^\vee)$ and the pullback of the non-degenerate pairing on $\mhiggs_G^{\text{ps,fr}}(\bb{CP}^1,D,\omega^\vee)$ under $H$ coincide.
\end{theorem}

\begin{proof}
This is straightforward now, by combining the two local descriptions of the symplectic structure on monopoles and multiplicative Higgs bundles on a neighbourhood of a puncture with the description in Proposition \ref{local_derivative_description_prop}.  So, let us begin by taking the symplectic form $\omega_{\mr{mHiggs}}$ on the moduli space $\mhiggs_G^{\text{ps,fr}}(\bb{CP}^1,D,\omega^\vee)$ and evaluating it at the image under $\d H$ of two elements $(\alpha_i, \alpha'_i) = (\d_{\mr{mon}}(b_i), \d_{\mr{mon}}(b_i'))$.  Let us write $b_i(0) = b_i^L$ and $b_i(2\pi) = b_i^R$ for brevity.  Likewise for consistency with the calculations above let us denote the image $H(\mc A)$ under the holonomy map by $g$.

According to Proposition \ref{local_derivative_description_prop} we have
\begin{align*}
\omega_{\mr{mHiggs}}(\d H(\alpha_i), \d H(\alpha_i')) &= \omega_{\mr{mHiggs}}(\d H(\d_{\mr{mon}}(b_i)), \d H(\d_{\mr{mon}}(b_i'))) \\
&= \omega_{\mr{mHiggs}}(b_i^Rg - gb_i^L,b_i^{'R}g - gb_i^{'L}).
\end{align*}
We can write this in terms of a pairing on the bundle $\ad(g)$.  Represent the class $b_i^Rg - gb_i^L$ by the pair $(b_i^Rg, -gb_i^L)$.  Then applying the description of $\omega_{\mr{mHiggs}}$ provided by Proposition \ref{qconn_symp_description} we have
\begin{align*}
\omega_{\mr{mHiggs}}(\d H(\alpha_i), \d H(\alpha_i')) &= \omega_{\mr{mHiggs}}((b_i^Rg, -gb_i^L),(b_i^{'R}g, -gb_i^{'L})) \\ 
&= - \langle  b_i^R , g b_i'^L g^{-1}\rangle - (b \leftrightarrow b')
\end{align*}
where the remaining terms are killed by the antisymmetrization.

On the other hand, we can evaluate the pairing $\omega_{\mr{Mon}}(\alpha_i,\alpha'_i) = \omega_{\mr{Mon}}(\d_{\mr{mon}}(b_i),\d_{\mr{mon}}(b_i'))$ via integration by parts. We'll compute the symplectic pairing for the full \v Cech complex, but it will split into a sum over open sets $U_i$.  The result is that 
\begin{align*}
\omega_{\mr{Mon}}(\{\alpha_i\},\{\alpha'_i\}) &= \sum_{i=0,\ldots,k,\infty} \int_{U_i} \kappa(\d_{\mr{mon}}(b_i) \wedge \d_{\mr{mon}}(b_i')) \d z - (b \leftrightarrow b')\\
&= \sum_{i=1,\ldots,k,\infty} \int_{\dd \bb D_i \times [0,2\pi]} \kappa(b_i - b_0, \d_{\mr{mon}}(b'_i)) \d z - (b \leftrightarrow b')
\end{align*}
using here the fact that $\d_{\mr{mon}}b_i = \d_{\mr{mon}} b_0$ on $U_i \cap U_0$ and that $\d_{\mr{mon}} b_i(t) = 0$ when $t = 0$ or $2\pi$.  By Stokes' theorem we then have
\begin{align*}
 \omega_{\mr{Mon}}(\{\alpha_i\},\{\alpha'_i\})&= \sum_{i=1,\ldots,k,\infty} \oint_{\dd \bb D_i} \kappa(b_i^R - b_0^R, b_i^{'R}) - \kappa(b_i^L - b_0^L, b_i^{'L}) - (b \leftrightarrow b') \\
 &= \sum_{i=1,\ldots,k,\infty} \oint_{\dd \bb D_i} - \kappa(b_0^R, b_i^{'R}) + \kappa(b_0^L, b_i^{'L}) - (b \leftrightarrow b').
\end{align*}
Pick out the summand corresponding to $U_i$.  Choose our potentials so that on the boundary $\dd \bb D_i$ we have $b_0^L = b_0^{'L} = 0$.  We can do this by setting $b^{R} = \delta g g^{-1}$ since $g$ is non-singular on $U_0$.  This choice means that on $U_0 \cap U_i$, since $\delta g  = b_{i} ^R g - g b_i^L = b_0^{R} g $ we can make the identification $ b_0^{R} = b_{i}^{R} - g b_{i}^{L} g^{-1}$.  Therefore
\[\omega_{\mr{Mon}}(\{\alpha_i\},\{\alpha'_i\}) = \sum_{i=1,\ldots,k,\infty} \langle(g b_i^{L} g^{-1}, b_i^{'R}\rangle \quad - \quad  (b \leftrightarrow b')\]
agreeing with the expression coming from $\mr{mHiggs}$. 
\end{proof}

In particular the non-degenerate pairing on the multiplicative Higgs moduli space is determined by a closed 2-form, and the multiplicative Hitchin fibers are Lagrangian.

\begin{corollary}
The moduli space $\mhiggs^\fr_G(\bb{CP}^1,D,\omega^\vee)$ has the structure of a completely integrable system.
\end{corollary}

\section{Hyperk\"ahler Structures} \label{hyperkahler_section}
The results of the previous section imply that the symplectic structure on $\mhiggs_G^{\text{ps,fr}}(\bb{CP}^1,D,\omega^\vee)$ extends canonically to a hyperk\"ahler structure.  In this section we'll compare the twistor rotation with the deformation to the moduli space of $\eps$-connections. We'll show that in an appropriate limit passing to a point $\eps$ in the twistor sphere we obtain a complex manifold equivalent to the moduli space of $\eps$-connections as a deformation of the moduli space of multiplicative Higgs bundles.

Let us begin by describing the hyperk\"ahler structure on the moduli space of periodic monopoles and how the holomorphic symplectic structure varies under rotation in the twistor sphere.  We'll begin by recalling that the moduli space of periodic monopoles can be described as a hyperk\"ahler quotient as in the work of Atiyah and Hitchin \cite{AtiyahHitchin}.  In a context with more general boundary data at $\infty$ this was demonstrated by Cherkis and Kapustin \cite{CherkisKapustin3} for the group $\SU(2)$, see also Foscolo \cite{FoscoloDef}[Theorem 7.12].  In the case of $\bb{CP}^1$ with a fixed framing the analysis is much easier.

Recall from Definition \ref{monopole_moduli_def} that the moduli space of periodic monopoles on $\bb{CP}^1$ with a fixed framing at $\infty$ and fixed Dirac singularities can be described as the quotient 
\[\mon_G^{\fr}(\bb{CP}^1 \times_\eps S^1_R, D, \omega^\vee) = \mu^{-1}(0)/\mc G,\]
where 
\vspace{-10pt}
\begin{itemize}
\item The space $\mc V$ of fields is the space of pairs $(A,\Phi)$ consisting of framed connections $A$ on the trivial principal $G_\RR$-bundle $\bo P$ on $\bb{CP}^1 \times_\eps S^1_R$, $\Phi$ is a section of $\gg_{\bo P}$ vanishing at $\{\infty\} \times S^1_R$, and the pair $(A,\Phi)$ has a Dirac singularity with charge $\omega^\vee_{z_i}$ at each point $(z_i,t_0)$ in $D \times \{t_0\}$.
\item $\mu \colon \mc V \to \Omega^1(\bb{CP}^1 \times_\eps S^1_R; (\gg_\RR)_{\bo P})/\gg_\RR$ is the map sending a monopole $(A, \Phi)$ to the $\gg_\RR$-valued 1-form $\ast F_A - \d_A \Phi$ vanishing at $\{\infty\} \times S^1_R$.  
\item The group $\mc G$ of gauge transformations consists of automorphisms of the trivial bundle $\bo P$ fixing the framing.
\end{itemize}

This is an instance of hyperk\"ahler reduction.  Indeed the infinite-dimensional space $\mc V$ admits a hyperk\"ahler structure with symplectic forms defined by
\[\omega_i((A,\Phi),(A',\Phi')) = \int_{\bb{CP}^1 \times_\eps S^1_R} \left(\kappa(A_i,\Phi') - \kappa(\Phi, A_i') + \sum_{j,k=1,2,3} \eps_{ijk} \kappa(A_j,A'_k)\right) \d x \d y \d t, \]
where $i=1,2,3$, and where $A$ can be written as $A_1 \d t + A_2 \d x + A_3 \d y$ with respect to a choice $\d z = \d (x + iy)$ of meromorphic volume form on $\bb{CP}^1$ with a second order pole at $\infty$.  This descends to a hyperk\"ahler structure on the quotient since  \chris{... probably I want to find a good reference here.  Could refer to \cite[Section 1.4.2]{FoscoloThesis}.}

Let us restrict to the case where $\eps=0$ for now.  Consider the holomorphic symplectic structure $\omega_2 + i \omega_3$.  We can alternatively write the formula above for this choice of holomorphic symplectic in the form
\[\omega(\mc A, \mc A') = \int_{\bb{CP}^1 \times S^1_R} \kappa(\mc A, \mc A') \d z \d t,\]
as described at the beginning of Section \ref{general_symplectic_sec}. \chris{This needs justification.  What happens to the $xy$ term?  At face value you get something more like the antisymmetrization of $\int \kappa(\mc A, A'_2 + i A'_3)\d z$.}

This description for the hyperk\"ahler structure also tells us clearly what happens when we perform a rotation in the twistor sphere.  The following statement follows by identifying the holomorphic symplectic structure at $\eps$ in the twistor sphere by applying the corresponding rotation in $\SO(3)$ to the coordinates $x$, $y$ and $t$ on $\RR^3$ in the expression for the holomorphic symplectic structure at $0$.  Note that when we calculate the symplectic form as an integral as above, it's enough to take the integral over just $\RR^2 \times S^1_R$ instead of $\bb{CP}^1 \times S^1_R$.

\begin{prop}
Let $\zeta$ be a point in the twistor sphere.  The holomorphic symplectic form on the moduli space $\mon_G^{\fr}(\bb{CP}^1 \times S^1_R, D, \omega^\vee)$ of periodic monopoles is given by the formula 
\[\omega_\eps(\delta^{(1)}\mc A, \delta^{(2)}\mc A) = \int_{\RR^2 \times S^1_R} \kappa(\delta^{(1)} \mc A, \delta^{(2)} \mc A) \zeta(\d z \wedge \d t),\]
where we identify $\zeta$ with an element of $\SO(3)$ and apply this rotation to the volume form $\d z \wedge \d t$.  Equivalently we can identify $\RR^2 \times \bb{CP}^1$ with the quotient $\RR^3 \times L$ where $L$ is the rank one lattice $\{0\}^2 \times 2\pi R\ZZ$ and identify the rotated holomorphic symplectic form with 
\[\omega_\zeta(\delta^{(1)}\mc A, \delta^{(2)}\mc A) = \int_{\RR^3/(\zeta(L))} \kappa(\delta^{(1)} \mc A, \delta^{(2)} \mc A) \d z  \wedge \d t.\]
\end{prop}

Now, we'll perform our identification of the twistor deformation on the multiplicative Higgs side by rewriting this twistor rotation in a somewhat different way in the large $R$ limit.  In this limit we can identify the twistor rotation with the deformation obtained by replacing the product $\bb{CP}^1 \times S^1_R$ with the product twisted by an automorphism of $\RR^2$.  More precisely, the rotated symplectic structure can then be described in the following way.

\begin{theorem}
Let $\left(\mhiggs^{\text{ps,fr}}_G(\bb{CP}^1,D,\omega^\vee)\right)_{\zeta,R}$ be the family of complex manifolds obtained by pulling back the holomorphic symplectic structure $\omega_{\zeta,R}$ on $\mon_G^{\mr{ps},\fr}(\bb{CP}^1 \times S^1_R,D,\omega^\vee)$ at $\zeta$ in the twistor sphere.  In the limit where $R \to \infty$ with $2 \pi \zeta R = \eps$ fixed this complex structure coincides with the complex structure on $\epsconn^{\text{ps,fr}}_G(\bb{CP}^1,D,\omega^\vee)$, in the sense that the complex structures converge pointwise over the tangent bundle to the moduli space of monopoles. 
\end{theorem}

\begin{proof}
First let us note that the 3-manifolds $\bb{CP}^1 \times_\eps S^1_R$ for varying values of $R$ and $\eps$ are all diffeomorphic (via a diffeomorphism fixing the circle at $\infty$.  This means that we can identify the moduli spaces in question as smooth manifolds by choosing such a diffeomorphism.  Then, according to Theorem \ref{monopole_qconn_comparison_thm} it's enough to show that the complex structure on $\mon_G^{\text{ps,fr}}(\bb{CP}^1 \times S^1_R,D,\omega^\vee)$ at the point $\zeta$ converges to the complex structure on the $\eps$-deformation $\mon_G^{\mr{ps},\fr}(\bb{CP}^1 \times_\eps S^1_R,D,\omega^\vee)$.  The key observation that we'll use is that, fixing $2 \pi \zeta R = \eps$ we can identify the rotated lattice with a \emph{sheared} lattice.  Namely, when we rotate the rank one lattice $L_{R,0} = (0,0,2\pi R)\ZZ$ by $\zeta$ we obtain the lattice
\[\zeta(L_{R,0}) = \frac {1}{1+|\zeta|^2} (\mr{Re}(\eps), \mr{Im}(\eps), 2\pi R (1-|\zeta|^2) )\ZZ.\]
We can view the quotient of $\RR^3$ by the rotated lattice as a sheared lattice with different radius and shear.  Specifically, we identify
\[\RR^3 / \zeta(L_{R,0}) \iso \RR^2 \times_{\frac{\eps}{1+|\zeta|^2}} S^1_{R \frac {1-|\zeta|^2}{1+|\zeta|^2}}.\]

In order to identify the desired complex structures it's enough to identify the full family of symplectic structures on the $\eps$-deformation with the appropriately rotated family of symplectic structures without deforming.  So, if we apply the twistor rotation by $\zeta$ to the holomorphic symplectic structure $\omega_2 + i\omega_3$ on $\mon_G^{\text{ps,fr}}(\bb{CP}^1 \times S^1_R,D,\omega^\vee)$ (the argument will be identical for the other points in the $\bb{CP}^1$ family of holomorphic symplectic structures) we obtain the pairing
\[\omega_\zeta(\delta^{(1)}\mc A, \delta^{(2)}\mc A) = \int_{\RR^3/(\zeta(L_{R,0}))} \kappa(\delta^{(1)} \mc A, \delta^{(2)} \mc A) \d z  \wedge \d t.\]
On the other hand we can identify the holomorphic symplectic structure on $\mon_G^{\mr{ps},\fr}(\bb{CP}^1 \times_\eps S^1_R,D,\omega^\vee)$ as
\[\omega_\zeta(\delta^{(1)}\mc A, \delta^{(2)}\mc A) = \int_{\RR^3/(L_{R,\eps})} \kappa(\delta^{(1)} \mc A, \delta^{(2)} \mc A) \d z  \wedge \d t.\]
In particular, by tuning the radius and the shear parameter the above calculation means there's a hyperk\"ahler equivalence between the moduli spaces $\mon_G^{\text{ps,fr}}(\bb{CP}^1 \times S^1_R,D,\omega^\vee)$ -- after rotating the twistor sphere by $\zeta$ -- and $\mon_G^{\mr{ps},\fr}(\bb{CP}^1 \times_{\frac{\eps}{1+|\zeta|^2}} S^1_{R \frac {1-|\zeta|^2}{1+|\zeta|^2}},D,\omega^\vee)$.  

Now, let us consider the large $R$ limit.  We'd like to note that the following difference of symplectic pairings converges to zero pointwise as $R\to \infty$.
\[ \left\lvert \int_{\RR^3/(L_{R \frac {1-|\zeta|^2}{1+|\zeta|^2},\frac{\eps}{1+|\zeta|^2}})} \kappa(\delta^{(1)} \mc A, \delta^{(2)} \mc A) \d z  \wedge \d t - \int_{\RR^3/(L_{R,\eps})} \kappa(\delta^{(1)} \mc A, \delta^{(2)} \mc A) \d z  \wedge \d t \right \rvert.\]
Rescaling the radii of the circles only rescales the pairing by an overall constant, so we can rewrite this as 
\[ R\left\lvert \frac {1-|\zeta|^2}{1+|\zeta|^2} \int_{\RR^3/(L_{1,\frac{\eps}{1+|\zeta|^2}})} \kappa(\delta^{(1)} \mc A, \delta^{(2)} \mc A) \d z  \wedge \d t - \int_{\RR^3/(L_{1,\eps})} \kappa(\delta^{(1)} \mc A, \delta^{(2)} \mc A) \d z  \wedge \d t \right \rvert\]
in which we recall that $\zeta = \frac \eps {2\pi R}$.  Write $\delta^{(i)}\mc A_\zeta$ for the image of the deformation $\delta^{(i)}\mc A$ under the diffeomorphism induced by identifying the $\frac{\eps}{1+|\zeta|^2}$-twisted product with the $\eps$-twisted product, so $\delta^{(i)}\mc A_\zeta(z,t) = \delta^{(i)}\mc A(z-\frac{\eps}{1+|\zeta|^2},t)$.  Note that when we perform the integral the pairing between $\delta^{(1)}\mc A$ and $\delta^{(2)}\mc A$ and the pairing between $\delta^{(1)}\mc A_\zeta$ and $\delta^{(2)}\mc A_\zeta$ coincide.  In other words our difference of pairings becomes
\begin{align*} 
R\left\lvert \int_{\bb{CP}^1 \times_\eps S^1_1} \frac {1-|\zeta|^2}{1+|\zeta|^2} \kappa(\delta^{(1)} \mc A_\zeta, \delta^{(2)} \mc A_\zeta)  -  \kappa(\delta^{(1)} \mc A, \delta^{(2)} \mc A) \d z  \wedge \d t \right \rvert &= R \left\lvert \frac {1-|\zeta|^2}{1+|\zeta|^2} - 1 \right\rvert \left\lvert \int_{\bb{CP}^1 \times_\eps S^1_1} \kappa(\delta^{(1)} \mc A, \delta^{(2)} \mc A) \d z  \wedge \d t \right \rvert \\
&= \left\lvert\frac {2R |\eps|^2}{4 \pi^2 R^2 + |\eps|^2} \right \rvert \left\lvert \int_{\bb{CP}^1 \times_\eps S^1_1} \kappa(\delta^{(1)} \mc A, \delta^{(2)} \mc A) \d z  \wedge \d t \right \rvert.
\end{align*}
This converges to zero pointwise as $R \to \infty$.  The same calculation proves convergence for the other symplectic structures in the hyperk\"ahler family, and therefore convergence for the two complex structures.
\end{proof}

\begin{remark}
\chris{comment on the example of the Nahm transform.}
\end{remark}

\begin{remark}
\chris{comment on $q$-opers -- or maybe defer to the subsequent section}.
\end{remark}

\section{The Rational Poisson Lie Group and the Yangian} \label{quantization_section}
Let us proceed by discussing the connection between the symplectic structure on the moduli space of multiplicative Higgs bundles in the rational case and the rational Poisson Lie group.  This connection should be compared to the connection between the elliptic moduli space and the elliptic quantum group in the work of Hurtubise and Markman \cite[Theorem 9.1]{HurtubiseMarkman}.  

Consider the infinite-type moduli space $\mhiggs^{\text{fr,sing}}(\bb{CP}^1)$ of multiplicative Higgs bundles with arbitrary singularities.  There is a map
\[r_\infty \colon \mhiggs^{\text{fr,sing}}(\bb{CP}^1) \to G_1[[z^{-1}]],\]
defined by restricting a multiplicative Higgs bundle to a formal neighbourhood of infinity.  In particular we can restrict $r_\infty$ to any of the finite-dimensional subspaces $\mhiggs^\fr_G(\bb{CP}^1,D,\omega^\vee)$ (the symplectic leaves).  Here the notation $G_1[[z^{-1}]]$ denotes Taylor series in $G$ with constant term 1.  Recall that $G_1[[z^{-1}]]$ is a Poisson Lie group, with Poisson structure defined by the Manin triple $(G(\!(z)\!), G[[z]], G_1[[z^{-1}]])$.

The Poisson Lie group $G_1[[z^{-1}]]$ has the structure of an ind-scheme using the filtration where $G_1[[z^{-1}]]_n$ consists of $G$-valued polynomials in $z^{-1}$ with identity constant term and where the matrix elements in a fixed faithful representation have degree at most $n$.  When we talk about the Poisson algebra $\OO(G_1[[z^{-1}]])$ of regular functions on $G_1[[z^{-1}]]$, we are therefore referring to the limit over $n \in \ZZ_{\ge 0}$ of the finitely generated algebras $\OO(G_1[[z^{-1}]]_n)$.  Due to the condition on the constant term the group $G_1[[z^{-1}]]$ is nilpotent, which means we can identify its algebra of functions with the algebra of functions on the Lie algebra.  That is:
\begin{align*}
\OO(G_1[[z^{-1}]]) &= \underset n \lim \OO(G_1[[z^{-1}]]_n) \\
&\iso \underset n \lim \sym(\gg_1^*[[z^{-1}]]_n).
\end{align*}
The Poisson bracket is compatible with this filtration, in the sense that it defines a map $\{,\} \colon \OO(G_1[[z^{-1}]]_m) \otimes \OO(G_1[[z^{-1}]]_n) \to \OO(G_1[[z^{-1}]]_{m+n})$ for each $m$ and $n$.

\begin{theorem} \label{Poisson_Lie_Comparison_thm}
The map $r_\infty \colon \mhiggs^\fr_G(\bb{CP}^1,D,\omega^\vee) \to G_1[[z^{-1}]]$ is Poisson.  That is, the pullback of the Poisson structure on $\OO(G_1[[z^{-1}]])$ coincides with the Poisson bracket on $\OO(\mhiggs^\fr_G(\bb{CP}^1,D,\omega^\vee))$. 
\end{theorem}

\begin{proof}
We just need to compute the pullback of the Poisson bracket on $\OO(G_1[[z^{-1}]])$ under $r_\infty$.  This Poisson bracket is most easily described in terms of the rational $r$-matrix: that is, we have a linear map $r \colon \gg(\!(z)\!) \to \gg(\!(z)\!)$ defined by \chris{...}
\end{proof}

\begin{remark} \label{shapiro_leaves_remark}
It's instructive to compare this calculation with the work of Shapiro \cite{Shapiro} on symplectic leaves for the rational Poisson Lie group.  According to Shapiro, and for the group $G = \SL_n$, there are symplectic leaves in $G_1[[z^{-1}]]$ indexed by Smith normal forms, i.e. by a sequence $d_1, \ldots, d_n$ of polynomials where $d_i | d_{i+1}$ for each $i$.  This data is equivalent to a dominant coweight coloured divisor $(D,\omega^\vee)$ in the following way.  The data of the sequence of polynomials is equivalent to the data of an increasing sequence $(D_1, \ldots, D_n)$ of $n$ effective divisors in $\bb{CP}^1$ disjoint from $\infty$: the tuples of roots of the polynomials $d_i$.  Let $D = \{z_1, \ldots, z_k\}$ be the support of the largest divisor $D_n$, and for each $z_j$ let $\omega^\vee_{z_j} = (m_1, \ldots, m_n)$ be the dominant coweight where $m_i$ is the order of the root $z_j$ in the polynomial $d_i$.  The dominant coweight Shapiro refers to as the ``type'' of a leaf is, therefore, the sum of all these coweights.

To be a little more precise, Shapiro proves that for $G = \SL_n$ these symplectic leaves span the Poisson subgroup $\mc G \sub G_1[[z^{-1}]]$ of elements that can be factorized as the product of a polynomial element in $G_1[z^{-1}]$ and a monic $\CC$-valued power series.  In fact, the map $r_\infty$ (for any group, not necessarily $G = \SL_n$) factors through the subgroup $\mc G$: the Taylor expansion of all \emph{rational} $G$ valued functions can be factorized in this way.  As a consequence, Theorem \ref{Poisson_Lie_Comparison_thm} implies that our symplectic leaves for the group $\SL_n$ agree with Shapiro's symplectic leaves.
\end{remark}

This comparison has interesting consequences upon quantization.  The Poisson algebra $\OO(G_1[[z^{-1}]])$ has a well-studied quantization to the \emph{Yangian} $Y(\gg)$.  This is the unique Hopf algebra quantizing the algebra $U(\gg[z])$ with first order correction determined by the Lie bialgebra structure on $\gg[z]$.  Recall that $(\gg(\!(z)\!), \gg[z], \gg_1[[z^{-1}]])$ is a Manin triple, so the residue pairing on $\gg(\!(z)\!)$ induces an isomorphism between $\gg_1[[z^{-1}]]$ and the dual to $\gg[z]$, and therefore a Lie cobracket on $\gg[z]$.  The uniqueness of this quantization is a theorem due to Drinfeld {\cite[Theorem 2]{DrinfeldQuantum1}}.

\begin{definition}
The \emph{Yangian} of the Lie algebra $\gg$ is the unique graded topological Hopf algebra $Y(\gg)$ which is a topologically free module over the graded ring $\CC[[\hbar]]$ (where $\hbar$ has degree 1), so that when we set $\hbar=0$ we recover $Y(\gg) \otimes_{\CC[[\hbar]]} \CC \iso U(\gg[z])$ as graded rings (where $z$ has degree 1), and where the first order term in the comultiplication $\Delta$ is determined by the cobracket $\delta$ in the sense that
\[\hbar^{-1}(\Delta(f) - \sigma(\Delta(f))) \text{ mod } \hbar = \delta(f \text{ mod } \hbar),\]
for all elements $f \in Y(\gg)$.  Here $\sigma$ is the braiding automorphism of $Y(\gg)^{\otimes 2}$ (i.e. $\sigma(f \otimes g) = g \otimes f$).
\end{definition}

There is a very extensive literature on the Yangian and related quantum groups.  The general theory for quantization of Poisson Lie groups, including the Yangian, was developed by Etingof and Kazhdan \cite{EgingofKazhdanIII}.  For more information we refer the reader to Chari and Pressley \cite{ChariPressley}, or for the Yangian specifically to the concise introduction in \cite[Section 9]{CostelloYangian}.

Likewise, we can study deformation quantization for the algebra $\OO(\mhiggs^\fr_G(\bb{CP}^1,D,\omega^\vee))$ of functions on our symplectic moduli space.  This moduli space is, in particular, a smooth finite-dimensional Poisson manifold, so its algebra of functions can be quantized, for instance using Kontsevich's results on formality \cite{KontsevichQuantization}.

\begin{definition}
The quantum algebra of functions $\OO_\hbar(\mhiggs^\fr_G(\bb{CP}^1,D,\omega^\vee))$ is a choice of deformation quantization of the Poisson algebra $\OO(\mhiggs^\fr_G(\bb{CP}^1,D,\omega^\vee))$.  That is, an associative $\CC[[\hbar]]$-algebra where the antisymmetrization of the first order term in $\hbar$ of the product recovers the Poisson bracket. 
\end{definition}

While we have a Poisson morphism $\mhiggs^\fr_G(\bb{CP}^1,D,\omega^\vee) \to G_1[[z^{-1}]]$ by Theorem \ref{Poisson_Lie_Comparison_thm}, and therefore a map of Poisson algebras $\OO(G_1[[z^{-1}]]) \to \OO(\mhiggs^\fr_G(\bb{CP}^1,D,\omega^\vee))$ there's no automatic guarantee that we can choose a quantization of the target admitting an algebra map from the Yangian quantizing this Poisson map -- one would need to verify the absence of an anomaly obstructing quantization.  There is however a natural model for this quantization, to a $Y(\gg)$-module,  constructed by Gerasimov, Kharchev, Lebedev and Oblezin \cite{GKLO} (extending an earlier calculation \cite{GKL} for $\gg = \gl_n$).

\begin{theorem}
For any reductive group $G$, and any choice of residue data, the Poisson map $\OO(G_1[[z^{-1}]]) \to \OO(\mhiggs^\fr_G(\bb{CP}^1,D,\omega^\vee))$ quantizes to a $Y(\gg)$-module structure on a deformation quantization $\OO_\hbar(\mhiggs^\fr_G(\bb{CP}^1,D,\omega^\vee))$ of the algebra of functions on the multiplicative Higgs moduli space.  
\end{theorem}

\begin{proof}
\chris{todo, reorganize}
One can verify this claim by studying the classical limits of the $Y(\gg)$-modules from \cite{GKLO}.  Gerasimov, Kharchev, Lebedev and Oblezin -- henceforth referred to as GKLO -- constructed $Y(\gg)$-modules for any semisimple Lie algebra $\gg$ indexed by the data of a dominant coweight-coloured divisor $(D, \omega^\vee)$.  As in Shapiro, this data is encoded in the form of a set $\nu_{i,k}$ of complex numbers, where $i$ varies over simple roots of $\gg$: the coweight at a point $z \in D$ is encoded by the vector $(m_1, \ldots, m_r)$ where $m_i = \lvert\{\nu_{i,k} = z\}\rvert$.  Let us denote this module by $M_{D,\omega^\vee}$.

GKLO discuss the classical limits of these modules $M_{D,\omega^\vee}$ as symplectic subvarieties of the Poisson Lie group.  This is discussed with further detail in \cite[Section 4]{Shapiro}.  In particular, one can compare the Poisson Lie group with the Zastava space for the group $G$.  Recall that  moduli spaces of (non-periodic) monopoles on $\bb{CP}^1$ can be modelled by spaces of framed maps from $\bb{CP}^1$ to the flag variety $G/B$, sending $\infty$ to the point $[B]$ (this is Jarvis's description of monopoles via scattering data \cite{Jarvis}).  The space $\mr{Map}_{\alpha^\vee}^\fr(\bb{CP}^1, G/B)$ of monopoles of degree $\alpha^\vee$ (where $\alpha^\vee$ is a positive coroot, identified as a class in $\mr H_2(G/B;\ZZ)$) admits a compactification known as \emph{Zastava space}.  Concretely, this is a stratified algebraic variety, with stratification given by
\[Z_{\alpha^\vee}(G) = \coprod_{\beta^\vee \preceq \alpha^\vee} \mr{Map}_{\alpha^\vee}^\fr(\bb{CP}^1, G/B) \times \sym^{\alpha^\vee - \beta^\vee}(\CC),\]
where the factor $\sym^{\alpha^\vee - \beta^\vee}(\CC)$ represents the space of coloured divisors $(D, \omega^\vee)$ with $\sum_{z_i \in D} \omega^\vee_{z_i} = \alpha^\vee - \beta^\vee$.  We refer the reader to \chris{refs} for more information.

Using a birational map to this Zastava space, GKLO show that the union of symplectic leaves corresponding to representations of a given type (meaning a given sum $\sum \omega^\vee_{z_i}$) is equivalent to the union of the classical leaves of that given type.  It is, therefore, reasonable to conjecture that the decompositions of these two varieties into their individual symplectic leaves (i.e. into decompositions of $\sum \omega^\vee_{z_i}$ labelled by points in $\CC$) should coincide.  We should note that this doesn't seem to follow directly from the GKLO calculations.  They analyse the symplectic leaves corresponding to the classical limits of $Y(\gg)$-modules via a rational map, denoted by $e_i(z) = b_i(z)/a_i(z)$, defined on an open set of this symplectic leaf.  This rational map is, by construction, not sensitive to the positions of the singularities, only of the total type of a representation.

Nevertheless, the symplectic leaves obtained as classical limits of the $Y(\gg)$-modules $M_{D,\omega^\vee}$ \emph{are} symplectic subvarieties of the Poisson Lie group sweeping out the union of symplectic leaves of a given type.  In particular every symplectic leaf $\mhiggs^\fr_G(\bb{CP}^1,D,\omega^\vee)$ quantizes to some module within the GKLO classification as required.
\end{proof}

\begin{example}
One example is given by the case where the only pole lies at $0 \in \CC$, so the map to the Poisson Lie group $G_1[[z^{-1}]]$ factors through the polynomial group $G_1[z^{-1}]$.  This example is included in the work of Kamnitzer, Webster, Weekes and Yacobi \cite{KWWY} (though they also went much further), who show in particular that \chris{...}
\end{example}

We'll conclude this section with some additional discussion about 
\chris{Include some discussion on how we can think of $\OO_\hbar(\mhiggs^{\text{fr,sing}}(\bb{CP}^1))$ as an intermediate quantum group between the Yangian and something dual to it.  References and a remark on the comparison with the conjectures made in \cite[Chapter 8.1]{NekrasovPestun}.  Include some words about $q$-Opers and $q$-W algebras, citing Sevostyanov et al and Aganagic-Frenkel-Okounkov \cite{AFO}.  Remark on the comparison with our modules and those of \cite{GKLO} -- would need some kind of Nahm transform story to relate our periodic monopoles with the Zastava spaces that appear in that work.}

\section{Twisted Gauge Theory}
In this section we'll discuss yet another occurence of the moduli space of multiplicative Higgs bundles via gauge theory.  This story is distinct from the appearance of the moduli space in \cite{NekrasovPestun} as the Seiberg-Witten integral system of a 4d $\mc N=2$ theory.  Instead the moduli space will appear as the moduli space of solutions to the equations of motion in a twisted 5d $\mc N=2$ supersymmetric gauge theory, compactified on a circle.  This story will be directly analogous to the occurence of the ordinary moduli stack of Higgs bundles in a holomorphic twist of 4d $\mc N=4$ theory (see \cite{CostelloSH,ElliottYoo1} for a discussion of this story); we'll recover that example in the limit where the radius of the circle shrinks to zero.

We should begin by briefly recalling the idea behind twisting for supersymmetric field theories.  This idea goes back to Witten \chris{briefly describe in words, give some references, then say that what we'll do is to compute the twist of the action functional / BRST theory, and therefore of the classical BV complex.  The twisted theory actually tells us the moduli spaces of solutions to the equations of motion, with shifted tangent complex given by the classical BV complex.}

With the background in hand, we'll focus in on the example we're interested in.  Let us first describe the 5d $\mc N=2$ superalgebra, and the supercharge we'll be twisting by.  Recall that there is an exceptional isomorphism identifying the groups $\Spin(5)$ and $\mr{USp}(4)$.  The Dirac spinor representation $S$ of $\Spin(5)$ is four dimensional: under the above exceptional isomorphism it is identified with the defining representation of $\mr{USp}(4)$.

\begin{definition}
The complexified $\mc N=2$ supersymmetry algebra in dimension 5 is the super Lie algebra
\[(\sp(4;\CC) \oplus \sp(4;\CC)_R \oplus V) \oplus \Pi(S \otimes W),\]
where $V$ is the five-dimensional defining representation of $\so(5;\CC) \iso \sp(4;\CC)$, $W$ is the four-dimensional defining representation of $\sp(4;\CC)_R$, and where there's an additional bracket $\Gamma \colon \sym^2(S \otimes W) \to V$ defined in the following way.  We can project $\sym^2(S \otimes W)$ onto $\wedge^2(S) \otimes \wedge^2(W)$.  The first factor decomposes as an $\sp(4;\CC)$-module into irreducible summands $V \oplus \CC$.  The second factor likewise decomposes as an $\sp(4;\CC)_R$-module into $V \oplus \CC$.  There is, therefore, a $\sp(4;\CC) \oplus \sp(4;\CC)_R$-equivariant projection $\wedge^2(S) \otimes \wedge^2(W) \to V \otimes \CC$ as required.
\end{definition}

Now, let's describe the locus of elements $Q \in S \otimes W$ such that $\Gamma(Q,Q) = 0$.  We think of the $\sp(4;\CC) \oplus \sp(4;\CC)_R$-module $S \otimes W$ as the space of 4-by-4 matrices, where the two copies of $\sp(4;\CC)$ act by left and right matrix multiplication respectively.  Square-zero supercharges can be classified by the dimension of their image $\Gamma(Q,-) \sub \CC^5$, so for instance supercharges with 5-dimensional image are topological.  We'll be interested in twists which are not quite topological, but instead have 4-dimensional image.  These elements all have rank 2 in $S \otimes W$.  One can check that the locus of such square zero spinors consists of 4-by-4 matrices where both the rows and the columns span Lagrangian subspaces of $\CC^4$.

We can choose supercharges to be additionally compatible with partial twisting homomorphisms $\phi \colon \Spin(4) \to \USp(4)$.  We say $Q$ is \emph{compatible} with $\phi$ if $Q$ is invariant under the action of the subgroup $(1,\phi) \colon \Spin(4) \inj \Spin(4) \times \USp(4)$.  When we twist a supersymmetric field theory by $Q$, if $Q$ is compatible with $\phi$ then the action of this subgroup survives to the twisted theory. \chris{In particular can define on $C_1 \times C_2 \times S^1$ where $C_i$ are Calabi-Yau curves.  Do I actually want $\SU(2)$ not $\Spin(4)$ in order to allow Calabi-Yau surfaces but not more general surfaces?}

\chris{This section needs to include: 1) A description of the relevant twist and the 5-manifolds it's defined on (think about the connection to the theory studied in \cite{QiuZabzine}).  2) An easy calculation of the twisted BV complex / moduli space, with a remark on the algebraic structure. 3) A definition of the 't Hooft surface operators and a description of the effect on the calculation of the twist. 4) A comment on the $R \to 0$ limit and the ordinary holomorphic twist of 4d $\mc N=4$.}

\section{Langlands Duality}
\chris{For a first draft I'm inserting some notes I prepared for a talk.  Todo: talk about motivation and the content we discussed regarding non-simply-laced groups}

\begin{pseudoconj}[Multiplicative Geometric Langlands] \label{multLanglands}
Let $G$ be a Langlands self-dual group.  There is an equivalence of categories
\[\text{A-Branes}_{q^{-1}}(\mhiggs_G(C,D,\omega^\vee)) \iso \text{B-Branes}(\qconn_G(C, D, \omega^\vee))\]
where the category on the right-hand side depends on the value $q$.
\end{pseudoconj}

What does this mean, and are there situations in which we can make it precise?  We'll discuss three examples where we can say something more concrete.  In each case, by ``B-branes'' we'll just mean the category $\coh(\qconn_G(C, D, \omega^\vee))$ of coherent sheaves.  By ``A-branes'' we'll mean some version of \emph{$q^\vee$-difference connections} on the stack $\bun_G(C)$.

\begin{remark}
This equivalence is supposed to interchange objects corresponding to branes of opers on the two sides, and introduce an analogue of the Feigin-Frenkel isomorphism between deformed W-algebras (see \cite{FrenkelReshetikhinSTS, STSSevostyanov}.  This isomorphism only holds for self-dual groups, which motivates the restriction to the self-dual case here.
\end{remark}

\subsection{The Abelian Case}
Suppose $G = \GL(1)$ (more generally we could consider a higher rank abelian gauge group).  In general for an abelian group the moduli spaces we've defined are trivial -- for instance the rational and trigonometric spaces are always discrete.  However there is one interesting non-trivial example: the elliptic case.  For simplicity let us consider the abelian situation with $D = \emptyset$: the case with no punctures.

\begin{definition}
A \emph{$q$-difference module} on a variety $X$ with automorphism $q$ is a module for the sheaf $\Delta_{q,X}$ of non-commutative rings generated by $\OO_X$ and an invertible generator $\Phi$ with the relation $\Phi \cdot f = q^*(f) \cdot \Phi$.  Write $\diff_q(X)$ for the category of $q$-difference modules on $X$.
\end{definition}

In the abelian case the space $\qconn_{\GL(1)}(E)$ is actually a stack, but one can split off the stacky part to define difference modules on it.  Indeed, for any $q$ one can write
\[\bun_{\GL(1)}(E) = \iso B\GL(1) \times \ZZ \times E^\vee\]
and so
\[\qconn_{\GL(1)}(E) \iso B\GL(1) \times \ZZ \times (E^\vee \times_q \CC^\times)\]
which means one can define difference modules on these stacks associated to an automorphism of $E^\vee$ or $E^\vee \times_q \CC^\times$ respectively.

\begin{conjecture}
There is an equivalence of categories for any $q \in \bb{CP}^1$
\[\diff_q(\bun_{\GL(1)}(E)) \iso \coh(q^{-1}\conn_{\GL(1)}(E)).\]
\end{conjecture}

In this abelian case we can go even farther and make a more sensitive 2-parameter version of the conjecture.

\begin{conjecture}
There is an equivalence of categories for any $q_1, q_2 \in \bb{CP}^1$
\[\diff_{q_1}(q_2\conn_{\GL(1)}(E) \iso \diff_{q_2^{-1}}(q_1^{-1}\conn_{GL(1)}(E)\]
where $q_1$ is the automorphism of $E^\vee \times_{q_2} \CC^\times$ acting fiberwise over each point of $\CC^\times$.
\end{conjecture}

This conjecture should be provable using the same techniques as the ordinary geometric Langlands correspondence in the abelian case, i.e. by a (quantum) twisted Fourier-Mukai transform (as constructed by Polishchuk and Rothstein \cite{PolishchukRothstein}).

\subsection{The Classical Case}
Now, let us consider the limit $q \to 0$.  This will give a conjectural statement involving coherent sheaves on both sides analogous to the classical limit of the geometric Langlands conjecture as conjectured by Donagi and Pantev \cite{DonagiPantev}.  The existence of an equivalence isn't so interesting in the self-dual case (where both sides are the same), but the classical multiplicative Langlands functor should be an \emph{interesting} non-trivial equivalence.  For example we can make the following conjecture

\begin{conjecture}
Let $G$ be a Langlands self-dual group and let $E$ be an elliptic curve.  There is an automorphism of categories (for the rational, trigonometric and elliptic moduli spaces)
\[F \colon \coh(\mhiggs_G(E) \iso \coh(\mhiggs_{G}(E)\]
so that the following square commutes:
\[\xymatrix{
\coh(\mhiggs_T(E)) \ar[r]^{\mathrm{FM}} \ar[d]_{p_*q^!} &\coh(\mhiggs_T(E)) \ar[d]^{p_*q^!} \\
\coh(\mhiggs_G(E)) \ar[r]^{F} &\coh(\mhiggs_G(E)).
}\]
Here we're using the natural morphisms $p \colon \mhiggs_B(E) \to \mhiggs_G(E)$ and $q \colon \mhiggs_B(E) \to \mhiggs_T(E)$, and $\mr{FM}$ is the Fourier-Mukai transform.
\end{conjecture}

It ought to be possible to state something a bit stronger that includes singularities in this auto-duality.

\subsection{The Rational Type A Case}
There's one more example where we can say something precise, and even draw a connection to the ordinary geometric Langlands correspondence.  We already mentioned the Nahm transform in the previous section: in the case where $G = \GL(n)$ and $C$ is $\CC$ (where as usual we fix framing data at infinity) the Nahm transform identifies multiplicative Higgs bundles of degree $k$ with \emph{ordinary} Higgs bundles on $\bb{CP}^1$ for the group $\GL(k)$ with $n+2$ tame singularities (with appropriate fixed locations and residues).

\begin{claim}
Under the Nahm transform, Pseudo-Conjecture \ref{multLanglands} in the rational case for the group $\GL(n)$ becomes the ordinary geometric Langlands conjecture on $\bb{CP}^1$ with tame ramification.
\end{claim}

 
\pagestyle{bib}
\bibliographystyle{alpha}
\bibliography{Mult_Hitchin}
%\printbibliography

\textsc{Institut des Hautes \'Etudes Scientifiques}\\
\textsc{35 Route de Chartres, Bures-sur-Yvette, 91440, France}\\
\texttt{celliott@ihes.fr}\\ 
\texttt{pestun@ihes.fr}
 
\end{document}

