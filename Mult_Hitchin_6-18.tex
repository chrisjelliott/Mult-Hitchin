\documentclass[10pt, oneside]{article}
\input math_headers.sty

\title{Multiplicative Hitchin Systems and Supersymmetric Gauge Theory}
\author{Chris Elliott \and Vasily Pestun}
\date{\today}

\DeclareMathOperator{\mhiggs}{mHiggs}
\DeclareMathOperator{\mon}{Mon}
\newcommand{\map}{\ul{\mr{Map}}}
\newcommand{\qconn}{q\text{-Conn}}
\renewcommand{\conn}{\text{-Conn}}
\newcommand{\epsconn}{\varepsilon\text{-Conn}}
\DeclareMathOperator{\diff}{Diff}
\renewcommand{\d}{\mathrm{d}}
\newcommand{\fr}{\mathrm{fr}}
\DeclareMathOperator{\Hol}{Hol}
\renewcommand{\ad}{\mr{ad}}
\newcommand{\Ad}{\mr{Ad}}

\newtheorem{pseudoconj}[definition]{Pseudo-Conjecture}

\begin{document}

\maketitle 
\begin{abstract}
 
\end{abstract}

\section{Introduction}
\chris{There are a lot of notes in red: that just indicates that there are a lot of arguments that are only partially written.  They're notes for myself.}

\section{Multiplicative Higgs Bundles and $q$-Connections}
We'll begin with an abstract definition of the moduli spaces we'll investigate in this paper using the language of derived algebraic geometry.

Let $G$ be a reductive complex algebraic group, let $C$ be a smooth complex algebraic curve and fix a finite set $D = \{z_i, \ldots, z_k\}$ of closed points in $C$.  We write $\bun_G(C)$ for the moduli stack of $G$-bundles on $C$, which we view as a mapping stack $\map(C, BG)$ into the classifying stack of $G$.

\begin{definition}
The moduli stack of \emph{multiplicative $G$-Higgs fields} on $C$ with singularities at $D$ is the fiber product
\[\mhiggs_G(C,D) = \bun_G(C) \times_{\bun_G(C \! \bs \! D)} \map(C \! \bs \! D, G/G)\]
where $G/G$ is the adjoint quotient stack.
\end{definition}

\begin{remark}
A closed point of $\mhiggs_G(C,D)$ consists of a principal $G$-bundle $P$ on $C$ along with an automorphism of the restriction $P|_{C \! \bs \! D}$, i.e. a section of $\ad(P)$ away from $D$.
\end{remark}

The adjoint quotient stack can also be described as the derived loop space $\map(S^1_B, BG)$ of the classifying stack, where $S^1_B$ is the ``Betti stack'' of $S^1$, i.e. the constant derived stack at the simplicial set $S^1$.  We can therefore view $\map(C \! \bs \! D, G/G)$ instead as the mapping stack $\map((C \! \bs \! D) \times S^1_B, BG)$, and the moduli stack of multiplicative Higgs bundles instead as
\[\mhiggs_G(C,D) = \map((C \times S^1_B) \bs (D \times \{0\}), BG).\]  
The source of this mapping stack can be $q$-deformed.  Indeed, let $q$ denote an automorphism of the curve $C$.  Write $C \times_q S^1_B$ for the \emph{mapping torus} of $q$, i.e the derived fiber product
\[C \times_q S^1_B = C \times_{C \times C} C\]
where the two maps $C \to C \times C$ are given by the diagonal and the $q$-twisted diagonal $(1,q)$ respectively.

\begin{definition}
The moduli stack of \emph{$q$ difference connections} for the group $G$ on $C$ with singularities at $D$ is the mapping space
\[\qconn_G(C,D) = \map((C \times_q S^1_B) \bs (D \times \{0\}), BG).\] 
In particular when $q=1$ this recovers the moduli stack of multiplicative Higgs bundles.
\end{definition}

\begin{remark}
A closed point of $\qconn_G(C,D)$ consists of a principal $G$-bundle $P$ on $C$ along with a \emph{$q$ difference connection}: an isomorphism of $G$-bundles $P|_{C \! \bs \! D} \to q^*P|_{C \! \bs \! D}$ away from the divisor $D$.
\end{remark}

\begin{remark}
\chris{Expository remarks on $q$-difference connections / $q$-difference modules.  Cite Sauloy?  Others?}
\end{remark}

\subsection{Residue Conditions}
These moduli stacks are typically of infinite type.  In order to obtain finite type stacks, and later in order to define symplectic rather than only Poisson structure, we can fix the behaviour of our multiplicative Higgs fields and $q$ difference connections near the punctures $D \sub C$.

We'll write $\bb D$ to denote the \emph{formal disk} $\spec \CC[\![z]\!]$.  Likewise we'll write $\bb D^\times$ for the \emph{formal punctured disk} $\spec \CC(\!(z)\!)$.  We'll then write $\bb B$ for the derived pushout $\bb D \sqcup_{\bb D^\times} \bb D$.  Let $LG = \map(\bb D^\times, G)$ and let $L^+G = \map(\bb D, G)$.

There is a canonical inclusion $\bb B^{\sqcup k} \to (C \times_q S^1_B) \bs (D \times \{0\})$, the inclusion of the formal punctured neighbourhood of $D \times \{0\}$.  This induces a restriction map on mapping spaces
\[\res_D \colon \qconn_G(C, D) \to \bun_G(\bb B)^k.\]

The following is well-known (see e.g. the expository article \cite{Zhu}).

\begin{lemma}
The set of closed points of $\bun_G(\bb B)$ is in canonical bijection with the set of dominant coweights of $G$.
\end{lemma}

\begin{definition}
Choose a map from $D$ to the set of dominant coweights and denote it by $\omega^\vee \colon z_i \mapsto \omega^\vee_{z_i}$.  Write $\Lambda_i$ for the isotropy group of the point $\omega^\vee_{z_i}$ in $\bun_G(\bb B)$. The moduli stack of $q$ difference connections on $C$ with singularities at $D$ and fixed residues given by $\omega^\vee$ is defined to be the fiber product
\[\qconn_G(C,D, \omega^\vee) = \qconn_G(C,D) \times_{\bun_G(\bb B)^k} (B\Lambda_1 \times \cdots \times B\Lambda_k).\]
\end{definition}

\begin{examples}
The most important examples for our purposes are given by the following rational/trigonometric/elliptic trichotomy.
 \begin{itemize}
  \item \textbf{Rational:} We can enhance the definition of our moduli space by including a framing at a point $c \in C$ not contained in $D$.  We always assume that such framed points are fixed by the automorphism $q$.
    \begin{definition}
      \label{def:framing}
    The moduli space of $q$-difference connections on $C$ with a framing at $c$ is defined to be the relative mapping space 
    \[\qconn_G^\fr(C) = \map(C \times_q S^1_B, BG; f)\]
    where $f \colon \{c\} \times S^1_B \to BG$ (or equivalently $f \colon \{c\} \to G/G$) is a choice of adjoint orbit.  We can define the framed mapping space with singularities and fixed residues in exactly the same way as above.  
  \end{definition}
    
    In this paper we'll be most interested in the following example.  Let $C = \bb{CP}^1$ with framing point $c = \infty$, and consider automorphisms of the form $z \mapsto z + \eps$ for $\eps \in \CC$.  Choose a finite subset $D \sub \bb A^1$ and label the points $z_i \in D$ by dominant coweights $\omega^\vee_{z_i}$.  We can then study the moduli space $\epsconn^\fr_G(\bb{CP}^1,D, \omega^\vee)$.  The main object of study in this paper will be the holomorphic symplectic structure on this moduli space.  
    
    Note that the motivation for this definition comes in part from Spaide's formalism \cite{Spaide} of AKSZ symplectic structures on relative mapping spaces -- in this formalism $\bb{CP}^1$ with a single framing point is relatively 1-oriented, so mapping spaces out of it with 1-shifted symplectic targets have AKSZ 0-shifted symplectic structures.
  
  \item \textbf{Trigonometric:} Alternatively, we can enhance our definition by including a reduction of structure group at a point $c \in C$ not contained in $D$, again fixed by the automorphism $q$.
  \begin{definition}
   The moduli space of $q$-difference connections on $C$ with an $H$-reduction at $c$ for a subgroup $H \sub G$ is defined to be the fiber product
   \[\qconn_G^{H,c}(C) = \map(C \times_q S^1_B,BG) \times_{G/G} H/H\]
   associated to the evaluation at $c$ map $\map(C \times_q S^1_B,BG) \to G/G$.  We can define the moduli space with $H$-reduction with singularities and fixed residues in the same way as above.
  \end{definition}
  
  Again let $C = \bb{CP}^1$.  Fix a pair of opposite Borel subgroups $B_+$ and $B_- \sub G$ with unipotent radicals $N_\pm$ and consider the moduli space of $q$-connections with $B_+$-reduction at $0$ and $N_-$-reduction at $\infty$.  We'll now take $q$ to be an automorphism of the form $z \mapsto qz$ for $q \in \CC^\times$.  We'll defer in depth analysis of this example to future work.
  \item \textbf{Elliptic:} Finally, let $C = E$ be a smooth curve of genus one.  In this case we won't fix any additional boundary data, but just consider the moduli space $\qconn_G(E,D, \omega^\vee)$.  In the case $q = 1$ this space -- or rather its polystable locus -- was studied by Hurtubise and Markman \cite{HurtubiseMarkman}, who proved that it can be given the structure of an algebraic integrable system with symplectic structure related to the elliptic R-matrix of Etingof and Varchenko \cite{EtingofVarchenko}.
 \end{itemize}
\end{examples}

\vasily{What shall we say about higher genus? The definition of mHiggs as above makes sense. Will the resulting space be Poisson? I guess not because to define the Poisson structure we need to fix a section of canonical bundle on $C$, that is a one-form ``dz''. However, in higher genus such 1-form necessarily has zeroes. May be here is a good place to comment or restrict attention to the trichotomy with an explanation why higher genus is on different grounds} \chris{added a remark below.}

\begin{remark} \label{Elliptic_AKSZ_remark}
In the elliptic case it's natural to ask to what extent Hurtubise and Markman's integrable system structure can be extended from the variety of polystable multiplicative Higgs bundles to the full moduli stack.  If $D$ is empty then it's easy to see that we have a symplectic structure given by the AKSZ construction of Pantev-To\"en-Vaqui\'e-Vezzosi \cite{PTVV}.  Indeed, $E$ is compact 1-oriented and the quotient stack $G/G$ is 1-shifted symplectic, so the mapping stack $\map(E, G/G) = \mhiggs_G(E)$ is equipped with a 0-shifted symplectic structure by \cite[Theorem 2.5]{PTVV}.  The role of the Hitchin fibration is played by the Chevalley map $\chi \colon G/G \to T/W$, and therefore
\[\map(E,G/G) \to \map(E,T/W).\]
The fibers of this map over regular points in $T/W$ are given by moduli stacks of the form $\bun_T(\wt E)^W$ where $\wt E$ is a $W$-fold cover of $E$ (the cameral cover). \chris{Come back to this, at least by dimension counting.}

\chris{todo: remark on what the base looks like for general $D$.  Probably put it in the last subsection?  Include comparison to the story in \cite{DonagiGaitsgory}.}
\end{remark}

\begin{remark}
While the moduli space of multiplicative Higgs bundles makes sense on a general curve it's only after restricting identity to this trichotomy of examples that we'll expect the existence of a Poisson structure.  Such a structure arises by the AKSZ construction, i.e. by transgressing the 1-shifted symplectic structure on $G/G$ to the mapping space using a fixed section of the canonical bundle on $C$ (possibly with a boundary condition).  
\end{remark}

\subsection{Stability Conditions}
For comparison to results in the literature it is important that we briefly discuss the role of stability conditions for difference connections.  In our main example of interest -- the rational case -- these conditions won't play a role, but they do appear in the comparison results between $q$-connections and monopoles in the literature for more general curves.  For definitions for general $G$ we refer to \cite{Smith}, although see also \cite{AnchoucheBiswas} on polystable $G$-bundles.  In what follows we fix a choice of $0 < t_0 < 2\pi R$.

\begin{definition}
Let $(P,g)$ be a $q$-connection on a curve $C$, and let $\chi$ be a character of $G$.  The \emph{$\chi$-degree} of $(P,g)$ is defined to be 
\[\deg_\chi(P,g) = \deg(P \times_\chi \CC) - \frac {t_0}{2\pi R} \sum_{i=1}^k \deg(\chi \circ \omega^\vee_{z_i}).\]

\vasily{Sorry, what that last zero ``...$(0)$'' denotes above?} \chris{I changed the notation.  The map $\chi \circ \omega^\vee_{z_i}$ is a map $\bb G_m \to \bb G_m$ and we want its degree.}

A $q$-connection $(P,g)$ on $C$ is \emph{stable} if for every maximal parabolic subgroup $H \sub G$ with Levi decomposition $H = LN$ and every reduction of structure group $(P_H, g)$ to $H$, we have
\[\deg_\chi(P_H, g) < 0\]
for the character $\chi = \det(\mr{Ad}_L^{\mf n})$ defined to be the determinant of the adjoint representation of $L$ on $\mf n$.

The $q$-connection $(P,g)$ is \emph{polystable} if there exists a (not necessarily maximal) parabolic subgroup $H$ with Levi factor $L$ and a reduction of structure group $(P_L, g)$ to $L$ so that $(P_L,g)$ is a stable $q$-connection and so that the associated $H$-bundle is admissible, meaning that for every character $\chi$ of $H$ which is trivial on $Z(G)$ the associated line bundle $P_H \times_\chi \CC$ has degree zero. 
\end{definition}

Below we'll write $\qconn_G^{\text{ps}}(C, D, \omega^\vee)$ for the moduli space of polystable $q$-connections.  This moduli space is a smooth algebraic variety of finite type.  \chris{Reference for this other than for $q$ trivial?}

When $C = \bb{CP}^1$ every principal $G$-bundle on $C$ admits an essentially unique (up to the action of the Weyl group) holomorphic reduction of structure group to a maximal torus \cite{GrothendieckSphere}.  Since $q$-connections for an abelian group are automatically stable, polystability on the sphere is equivalent to admissibility of the torus reduction.  As a consequence, for our main example of interest -- the rational case -- the moduli space of polystable $q$-connections is equivalent to the moduli space of all $q$-connections of admissible degree.  For instance for $G=\SL_n$ the moduli space of polystable $q$-connection is equivalent to the moduli space of $q$-connections on the trivial bundle.

\subsection{Poisson Structures from Derived Geometry}
As we mentioned above in Remark \ref{Elliptic_AKSZ_remark}, in the case where $C$ is an elliptic curve and there are no punctures there's a symplectic structure on $\mhiggs_G(C)$ given by the AKSZ formalism.  More generally, when we do have punctures, we expect the moduli space $\mhiggs_G(C,D)$ to have a Poisson structure with a clear origin story coming from the theory of derived Poisson geometry.  In this section we'll explain what this story looks like.  However, we emphasise that there are technical obstructions to making this story precise with current technology: this section should be viewed as motivation for the structures we'll discuss in the rest of the paper.  We refer the reader to \cite{CPTVV} for the theory of derived Poisson structures and to \cite{MelaniSafronov1, MelaniSafronov2, Spaide} for that of derived coisotropic structures.

Here's the idea.  Recall that we can identify the moduli space of singular $q$-connections on a curve as a fiber product: $\qconn_G^{\fr}(\bb{CP}^1, D) \iso \bun_G^\fr(C) \times_{\bun_G(\bb{CP}^1 \! \bs \! D)^2} \bun_G(\bb{CP}^1 \! \bs \! D)$ where the map $g \colon \bun_G^\fr(\bb{CP}^1) \to \bun_G(\bb{CP}^1 \! \bs \! D)$ is given by $P \mapsto (P|_{\bb{CP}^1 \! \bs \! D}, q^*P|_{\bb{CP}^1 \! \bs \! D})$.  Consider the following commutative cube:

\[\xymatrix@C-30pt@R-8pt{
& \qconn_G^{\fr}(\bb{CP}^1, D) \ar[rr]^{f_1} \ar'[d][dd]^(.25){\mr{res}} & & \bun_G^{\fr}(\bb{CP}^1) \ar[dd]
\\
\bun_G^{\fr}(\bb{CP}^1 \bs D) \ar@{<-}[ur]^{f_2} \ar[rr]^(.6){g_2} \ar[dd] & & \bun_G^{\fr}(\bb{CP}^1 \bs D)^2 \ar@{<-}[ur]^{g_1} \ar[dd]^(.4)r
\\
& \bun_G(\bb B)^k \ar'[r][rr] & & BL^+G^{2k}
\\
BLG^k \ar[rr]\ar@{<-}[ur] & & BLG^{2k}. \ar@{<-}[ur]
}\]
Here the top and bottom faces are homotopy Cartesian.  What does this setup buy us?  We'll first answer informally.

\begin{claim}
\chris{first ingredient: $BLG$ being 2-shifted symplectic.  Second ingredient: that the restriction map is Lagrangian.  Together these give us what we want.  Can phrase these things formally as conjectures.}
\end{claim}

Now, let's make this claim more precise.  The main technical condition that makes this claim subtle comes from the fact that most of the derived stacks appearing in this cube, for instance the stack $BLG$m are not Artin.  As such we need to be careful when we try to, for instance, talk about the tangent complex to such stacks.  One can make careful statements using the formalism of ``Tate stacks'' developed by Hennion \cite{Hennion}.  We can therefore make our claim into a more formal conjecture.

\begin{conjecture}
The stack $BLG$ is Tate 2-shifted symplectic, and both $BL^+G \to BLG$ and $\bun_G(C \bs D) \to BLG^k$ are Tate 2-shifted Lagrangian.  Furthermore 
\end{conjecture}




\chris{all of the below is to be removed}

\begin{lemma} \label{restriction_Lagn_lemma}
The restriction map $r \colon \bun_G^\fr(\bb{CP}^1 \! \bs \! D) \to BLG^k$ has a 2-shifted Lagrangian structure with respect to the residue pairing on $BLG$.
\end{lemma}

\begin{remark}
More precisely, $BLG$ is a Tate stack in the sense of Hennion \cite{Hennion}.  Although we won't check that the residue pairing symplectic structure $\omega$ is compatible with the Tate structure, this problem doesn't arise when defining a Lagrangian structure as a potential for $r^*\omega$ in $\Omega^{2,\mr{cl}}(\bun_G^\fr(\bb{CP}^1 \! \bs \! D), 2)$ since the stack in question is of ind-finite type.
\end{remark}

\begin{proof}
Consider the inclusion $\d r \colon \gg_- = \bb T_{\bun_G^\fr(\bb{CP}^1 \! \bs \! D)}[-1] \to r^*\bb T_{BLG^k}[-1] = \gg(\!(z)\!)^k$: a map of ind-pro Lie algebras concentrated in degree zero.  The residue pairing vanishes after pulling back along $r$ since elements of $\gg_-$ are  $\gg$-valued functions on $\bb{CP}^1$ with at least a simple pole at every puncture in $D$.  So the map $r$ is isotropic with 0 isotropic structure; this structure is unique for degree reasons.  We must check that this structure is non-degenerate.  It suffices to check that the sequence
\[\bb T_{\bun_G^\fr(\bb{CP}^1 \! \bs \! D)}[-1] \to r^*\bb T_{BLG^k}[-1] \to (\bb T_{\bun_G^\fr(\bb{CP}^1 \! \bs \! D)}[-1])^\vee\]
is an exact sequence of ind-pro vector spaces, and therefore an exact sequence of quasi-coherent sheaves on the stack $\bun_G^\fr(\bb{CP}^1 \! \bs \! D)$.  To do this we identify the pair $(\gg_-, \gg(\!(z)\!)^k)$ as part of a Manin triple, where a complementary isotropic subalgebra to $\gg_-$ is given by $\gg_+ = \gg[[z]]^k$.  Using the residue pairing we can identify $\gg_+$ with $(\gg_-)^\vee$ and therefore identify our sequence with the split exact sequence
\[0 \to \gg_- \to \gg(\!(z)\!)^k \to \gg_+ \to 0.\]
\end{proof}

\begin{construction}
The front and the right faces are not homotopy Cartesian but there are canonical 1-shifted Lagrangian structures on the induced maps from the source of the face into the homotopy fiber product, i.e.
\begin{align*}
&\bun_G^{\fr}(\bb{CP}^1) \to \bun_G^{\fr}(\bb{CP}^1 \bs D)^2 \times_{BLG^{2k}} BL^+G^{2k} \iso \bun_G^{\fr}(\bb{CP}^1)^2 \\
\text{and } &\bun_G^{\fr}(\bb{CP}^1 \bs D) \to \bun_G^{\fr}(\bb{CP}^1 \bs D)^2 \times_{BLG^{2k}} BLG^{k}.
\end{align*}
Call these Lagrangian structures $\mu_1$ and $\mu_2$ respectively.  Denote the 2-shifted Lagrangian structure on the map $r$ from Lemma \ref{restriction_Lagn_lemma} by $\nu$.  The 1-shifted Lagrangian structure that we want on the map $\mr{res}$ is then given by 
\[\lambda = f_1^*(\mu_1) + f_2^*(\mu_2) + \eta
\]
where $\eta$ is a potential for the difference of the pullbacks $(f_1 \circ g)^*\nu - (f_2 \circ g_2)^*\nu$.  Here $\eta = 0$ since $\nu = 0$ and the potential is obtained as the pullback of $\nu$ along the homotopy making the top square commute (which in this case is the identity).  Furthermore, the Lagrangian structure $\mu_1$ on the diagonal is 0, and the Lagrangian structure $\mu_2$ is also 0 for degree reasons.  Indeed, the tangent complex of $\bun_G^\fr(\bb{CP}^1 \bs D)$ is concentrated in degree $-1$ so there are no 2-forms of degree 1.  Thus the 1-shifted Lagrangian structure occuring here is the trivial nullhomotopy for the zero closed 2-form \chris{check}.

Any 1-shifted Lagrangian structure $\lambda$ defines a quasi-isomorphism $\wt \lambda \colon \bb T_{\qconn_G^{\fr}(\bb{CP}^1, D)} \to \bb L_{\mr{res}}$ where I'm writing $\bb L_{\mr{res}}$ for the dual of $\mr{res}^*(\bb T_{\qconn_G^{\fr}(\bb{CP}^1, D)}) \to \bb T_{\bun_G(\bb B)^k)}[-1]$: the relative tangent complex to the map $\mr{res}$.  Write $\wt{\lambda}^{-1}$ for a quasi-inverse.  This defines a Poisson bivector on $\qconn_G^{\fr}(\bb{CP}^1, D)$ where the associated map from the cotangent complex to the tangent complex is given by
\[\pi = \wt{\lambda}^{-1} \circ \iota \colon \bb L_{\qconn_G^{\fr}(\bb{CP}^1, D)} \to \bb L_{\mr{res}} \to \bb T_{\qconn_G^{\fr}(\bb{CP}^1, D)}\]
where $\iota$ is the canonical inclusion of the cotangent complex into the relative cotangent complex.

We obtain the 0-shifted symplectic structure on a Poisson leaf as the Lagrangian intersection
\[\xymatrix{
\qconn_G^{\fr}(\bb{CP}^1, D, \omega^\vee) \ar[r] \ar[d] &\qconn_G^{\fr}(\bb{CP}^1, D) \ar[d] \\
\prod B\Gamma_i \ar[r] & \bun_G(\bb B)^k
}\]
where $B\Gamma_i$ is the classifying space of the group of symmetries of the $G$-bundle on $\bb B$ associated to the coweight $\omega^\vee_i$.  If we restrict the Poisson bivector to the leaf $\qconn_G^{\fr}(\bb{CP}^1, D, \omega^\vee)$ it becomes non-degenerate and so defines a 0-shifted symplectic structure.
\end{construction}

In order to work with this symplectic structure we'll need to give a local description as a pairing on the tangent complex.  Using the description of the tangent complex in terms of hypercohomology from Section \ref{def_section} it will suffice to describe the pairing on each summand in the \v Cech resolution.

\begin{remark}
From now on we'll write $\langle - , - \rangle$ to denote the residue pairing between elements of $L\gg$.  That is, 
\[\langle g_1, g_2 \rangle = \oint_{\bb D^\times} \kappa(g_1, g_2).\]
\end{remark}

\chris{remark on more general case of non-trivial $q$ and what we expect there.}

\begin{lemma}
As a pairing on the tangent complex at a point $(P,g)$ the symplectic form on the moduli space of multiplicative Higgs bundles pairs $\alpha_i$ with $\beta_i$ for $i=1,\ldots,k$.  It is given by the formula 
\[\omega((\{\alpha_i\}, \{\beta_i\}), (\{\alpha'_i\},\{\beta_i'\})) = \sum_i \langle \rho_g^*(\alpha_i + \alpha_0)|_{U^\times_i}, \rho_g^*(\Ad_g^*)^{-1}(\beta'_i) \rangle - \langle \rho_g^*(\alpha'_i + \alpha'_0)|_{U^\times_i}, \rho_g^*(\Ad_g^*)^{-1}(\beta_i) \rangle,\]
where we use the Killing form to identify $\beta'_i$ with a $\gg^*$-valued form, and where $\rho_g^*$ is the pullback along the right multiplication by $g$.  In this expression the sum is over $i=1,\ldots,k$ and $i=\infty$.
\end{lemma}

\begin{proof}
The formula given above is saying that the Poisson bivector -- viewed as a map $\pi \colon \bb L_{(P,g)} \to \bb T_{(P,g)}$ -- acts in degree $-1$ by identifying $\gg$ with $\gg^*$ then applying the map $\Ad_g^*$.  
\chris{Removed what was here because it was wrong.  I think maybe to fix it we want to say that if we find any pairing which is closed for the internal differential then it will agree with the pairing coming from the trivial Lagrangian structure because the pairing only depends on cocycles up to coboundaries, and on that level there's a unique map $\bb T_L \to \bb L_L$ which is null-homotopic to zero and therefore a unique Poisson structure?  Needs more thought.}

Consider the image of our candidate pairing under the internal differential on the space of 2-forms (here modelled by the \v Cech differential).  The result is the degree 1 pairing
\begin{align*}
\eta((\{\alpha_i\}, \{\beta_i\}), (\{\alpha'_i\},\{\beta_i'\})) &= \sum_i \langle \rho_g^* \Ad_g(\phi_i + \phi_0)|_{U^\times_i}, \rho_g^*(\Ad_g^*)^{-1}(\beta'_i) \rangle - \langle \rho_g^*\Ad_g(\phi'_i + \phi'_0)|_{U^\times_i}, \rho_g^*(\Ad_g^*)^{-1}(\beta_i) \rangle\\
&= \sum_i \langle \rho_g^*(\phi_i + \phi_0)|_{U^\times_i}, \rho_g^*(\beta'_i) \rangle - \langle \rho_g^*(\phi'_i + \phi'_0)|_{U^\times_i}, \rho_g^*(\beta_i) \rangle,
\end{align*}
where $\phi_i$ is an element of $\gg_P(U_i)$: a degree $-1$ element of the \v Cech complex.  This pairing vanishes on \v Cech cocycles, because if $\{\phi_i\}$ is a $-1$-cocycle then $(\phi'_i + \phi'_0)|_{U^\times_i}=0$.  Our candidate pairing is therefore closed for the internal differential.
\end{proof}

\section{Periodic Monopoles}
Moduli spaces of $q$-connections on a Riemann surface $C$ are closely related to moduli spaces of periodic monopoles, i.e. monopoles on 3-manifolds that fiber over the circle (more specifically, with fiber $C$ and monodromy determined by $q$).  Let $G_\RR$ be a compact Lie group whose complexification is $G$.  The discussion in this section will mostly follow that of \cite{CharbonneauHurtubise, Smith}.

Write $M = C\times_q S^1_R$ for the $C$-bundle over $S^1$ with monodromy given by the automorphism $q$.  More precisely, $M$ is the Riemannian 3-manifold obtained by gluing the ends of the product $C \times [0,2\pi R]$ of Riemannian manifolds by the isometry $(x,2\pi R) \sim (q(x), 0)$.

\begin{definition}
A \emph{monopole} on the Riemannian 3-manifold $M = C \times_q S^1_R$ is a smooth principal $G_\RR$-bundle $\bo P$ equipped with a connection $A$ and a section $\Phi$ of the associated bundle $\gg_{\bo P}$ satisfying the Bogomolny equation 
\[\ast F_A = \d_A \Phi.\]
\end{definition}

\begin{remark}
We should emphasise the difference between the Riemannian 3-manifold $M = C \times_q S^1_R$ appearing in this section and the derived stack $C \times_q S^1_B$ (the mapping torus) appearing in the previous section.  These should be thought of as smooth and algebraic realizations of the same object (justified by the comparison Theorem \ref{monopole_qconn_comparison_thm}) but they are a priori defined in different mathematical contexts.
\end{remark}

We can rephrase the data of a monopole on $M$ as follows.  Let $C_0 = C \times \{0\}$ be the fiber over $0$ in $S^1$, viewed as a Riemann surface.  Let $P$ be the restriction of the complexified bundle $\bo P_\CC$ to $C_0$.  Consider first the restriction of the complexification of $A$ to a connection $A_0$ on $P$ over $C_0$.  The $(0,1)$ part of $A_0$ automatically defines a holomorphic structure on $P$.  We can introduce an additional piece of structure on this holomorphic $G$-bundle.  In order to do so we can decompose the Bogomolny equation into one real and one complex equation as follows.
\begin{align*}
F_{A_0} - \nabla_t \Phi &= 0 \\
[\ol{\del}_{A_0}, \nabla_t - i\Phi \d t] &= 0 
\end{align*}
where $\nabla_t$ is the component of the covariant derivative $\d_A$ normal to $C_0$.  

\begin{definition}
From now on we'll use the notation $\mc A$ for the  \chris{...explain} $\nabla_t - i\Phi \d t$. 
\end{definition}

Let's now introduce singularities into the story.  We'll keep the description brief, referring the reader to \cite{CharbonneauHurtubise, Smith} for details.
\begin{definition}
  Let $D \sub M$ be a finite subset.  Let $\omega^\vee$ be a choice of coweight for $G$.  A monopole on $M \bs D$ has \emph{Dirac singularity} at $z \in D$ with charge $\omega^\vee$ if locally on a neighbourhood of $z$ in $M$ it is obtained by pulling back under $\omega^\vee$ the standard Dirac monopole solution to the Bogomolny equation where $\Phi$ is spherically symmetric with a simple pole at $z$ () and the restriction of a connection $A$ to a two-sphere $S^2$ enclosing the singularity defines a $U(1)$ bundle on this $S^2$ of degree $1$ so that
    \[\frac{1}{2\pi} \int_{S^2} F = 1 .\]
  See e.g. \cite[Section 2.2]{CharbonneauHurtubise} for a more detailed description.
\end{definition}

We can also introduce a framing (or a reduction of structure group as in the trigonometric example, though we won't consider the latter in this paper).  As usual let $c \in C$ be a point fixed by the automorphism $q$.
\begin{definition}
  A monopole on $M$ with \emph{framing} at the point $c \in C$ is a monopole $(\bo P,A,\Phi)$ on $M$ (possibly with Dirac singularities at $D$) along with a trivialization of the restriction of $\bo P$ to the circle $\{c\} \times S^1_R$, with the condition that the holonomy of $\mc A$ around this circle lies in a fixed conjugacy class $f \in G/G$.
\end{definition}

The moduli theory of monopoles on 3-manifolds of this form has been studied in the mathematics literature by Foscolo \cite{FoscoloDef}, applying the analytic techniques of deformation theory to earlier work on periodic monopoles by Cherkis and Kapustin \cite{CherkisKapustin1, CherkisKapustin2}.\vasily{need to add refs to Kapustin Cherkis and others before Foscolo} \chris{added something here.  Are there other earlier references regarding deformation / moduli theory you had in mind?}.  In the cases of interest to us it can be obtained as a hyperk\"ahler quotient.  Let's focus initially on the rational case, so monopoles on $M = \bb{CP}^1 \times_\eps S^1_R$ with Dirac singularities at $D \times \{t_0\}$ and a framing at $\infty$.  Consider the infinite-dimensional vector space $\mc V$ consisting of pairs $(A,\Phi)$ where $A$ is a connection on a fixed principal $G_\RR$-bundle $\bo P$ on $M$, $\Phi$ is a section of $\gg_{\bo P}$, and $(A,\Phi)$ have a Dirac singularity with charge $\omega^\vee_{z_i}$ at each $(z_i,t_0)$ in $D \times \{t_0\}$.  Let $\mc G$ be the group of gauge transformations of the bundle $\bo P$.

The hyperk\"ahler moment map is given by the Bogomolny functional, namely
\begin{align*}
\mu \colon \mc V &\to \Omega^1(M; (\gg_\RR)_{\bo P}) \\
(A,\Phi) &\mapsto \ast F_A - \d_A \Phi.
\end{align*}

\begin{definition}
Let $D$ be a finite subset $\{(z_1,t_1), \ldots, (z_k, t_k)\}$ of points in $M = \bb{CP}^1 \times_\eps S^1_R$, and let $\omega^\vee_{i}$ be a choice of coweight for each point in $D$. The moduli space $\mon_G(M, D, \omega^\vee)$ is the hyperk\"ahler quotient
\[\mon_G(M, D, \omega^\vee) = \mu^{-1}(0) / \mc G.\]
\end{definition}

Now let's address the relationship between periodic monopoles and $q$-connections.  Suppose from now on that $q$ is in the identity component of the group of automorphisms of $C$ (fixing the framing point $c$ if present).

\begin{theorem} \label{monopole_qconn_comparison_thm}
There is an analytic isomorphism between the moduli space of polystable monopoles on $C \times_q S^1$ with Dirac singularities at $D \times \{t_0\}$ (and a possible framing on $\{c\} \times S^1$) and the moduli space of $q$-connections on $C$ with singularities at $D$ and framing at $\{c\}$.  More precisely there is an analytic isomorphism
\[H \colon \mon^{(\fr)}_G(C \times_q S^1, D \times \{t_0\}, \omega^\vee) \to \qconn_G^{\text{ps,(fr)}}(C, D, \omega^\vee)\]
given by the holonomy map around $S^1$, i.e. sending a monopole $(\bo P, \mc A)$ to the holomorphic bundle $P = (\bo P_\CC)|_{C_0}$ with $q$-connection $g = \Hol_{S^1}(\mc A) \colon P \to q^*(P)$.
\end{theorem}

\begin{remark}
Note that in this statement we assumed that all the singularities occur in the same location in $S^1$, i.e. in the same slice $C \times {t_0}$.  This assumption is not necessary, as explained in \cite[]{CharbonneauHurtubise} \chris{add ref and explain.}
\end{remark}

\begin{proof}
\chris{check}
This follows by the same argument as that given by Charbonneau-Hurtubise \cite{CharbonneauHurtubise} and Smith \cite{Smith}.  More explicitly, first let's think about injectivity, so let $(\bo P, \mc A)$ and $(\bo P', \mc A')$ be a pair of periodic monopoles on $C \times_q S^1$ with images $(P,g)$ and $(P', g')$ respectively, and choose a bundle isomorphism $\tau \colon P \to P'$ intertwining the $q$-connections $g$ and $g'$.  One observes first that $\bo P$ and $\bo P'$ are also isomorphic $G$-bundles since, by intertwining with the $q$-connections, we have an isomorphism $\bo P|_{C \times \{t\}} \to \bo P'|_{C \times \{t\}}$ for every $t \in S^1$.  That the monopole structures also match up follows by the same argument as in \cite[Proposition 4.7]{CharbonneauHurtubise}.

For surjectivity, again we'll match the argument in the case where $q=\id$.  We begin by extending a holomorphic $G$-bundle $P$ on $C_0$ with $q$-connection $g$ to a $G$-bundle on $M \bs (D \times \{t_0\}) = (C \times_q S^1_R) \bs (D \times \{t_0\})$  Let $\gamma \colon [-2\pi R,2\pi R] \to \aut(C)$ be a geodesic with $\gamma(-2\pi R) = q^{-1}$, $\gamma(0)=1$ and $\gamma(2\pi R) = q$.  Let $\wt M$ be the 3-manifold
\[\wt M = ((-2\pi R, 2\pi R) \times C) \bs \bigcup_{j=1}^k (A^+_j \cup A^-_j)\]
where $A^+_j$ is the arc $\{(t+ t_0,\gamma(t)(z_j)) \colon t \in (0, 2\pi R - t_0]\}$ and $A^-_j$ is the arc $\{(t + t_0 - 4 \pi R,\gamma(t)(z_j)) \colon t \in [2\pi R-t_0, 2 \pi R)\}$.

Let $\pi \colon \wt M \to C$ be the projection sending $(t,z)$ to $\gamma(t)(z)$.  The bundle $P$ pulls back to a bundle $\pi^*(P)$ on $\wt M$.  We obtain a bundle on $M \bs (D \times t_0)$ by applying the identification $(t,z) \sim (t - 2 \pi R, q(z))$.  This bundle extends to an $S^1$-invariant holomorphic $G$-bundle on $M \times S^1$.  The remainder of the proof -- verifying the existence of the monopole structure associated to an appropriate choice of hermitian structure -- consists of local analysis which is independent of the value of the parameter $q$. 

It remains to remark on the compatibility of framing data on the two sides.  A trivialization of the bundle $\bo P$ along the circle $\{c\} \times S^1$ yields a trivialization of the fiber of the bundle $P$ at $c$.  The condition that the holonomy around the circle at $c$ is $f$ fixes the value of the $q$-connection at $c$ to be in the conjugacy class $f$.
\end{proof}

\begin{remark}
Mochizuki \cite{Mochizuki} proved a stronger result in the rational case for the group $G = \GL_n$.  He allows not just a framing at infinity in $\bb{CP}^1$ but also a singularity encoded in terms of a $B$-reduction of the bundle. 
\end{remark}

\subsection{Deformation Theory} \label{def_section}
In the next section we'll compare symplectic forms on these moduli spaces.  In order to do so it will be important to understand the tangent spaces at a point of the source and target.  There's a natural description of these tangent spaces in terms of the hypercohomology of certain cochain complexes.

Let's first describe the tangent complex to the moduli space of monopoles.  For simplicity we'll restrict attention to the rational case where we can use the description as a hyperk\"ahler reduction.  For more general deformation theory calculations we refer to Foscolo \cite{FoscoloDef}.  Recall that in this case we can write
\[\mon^{(\fr)}_G(C \times_q S^1, D \times \{t_0\}, \omega^\vee) \iso \mu^{-1}(0)/ \mc G\]
where $\mc G$ is the group of gauge transformations of $\bo P$ and $\mu \colon \mc V \to \Omega^1(M; (\gg_R)_{\bo P})$ is the Bogomolny functional $(A,\Phi) \mapsto \ast F_A - \d_A \Phi$.  The tangent complex to this hyperk\"ahler quotient at a point $(\bo P, \mc A)$ can be written as $\Omega^0(M \!\bs\! D; (\gg_\RR)_{\bo P})[1] \to \bb T_{\mu^{-1}(0)}$ where $\bb T_{\mu^{-1}(0)}$ is the tangent complex to the zero locus of the moment map, concentrated in non-negative degrees. Roughly speaking $\bb T_{\mu^{-1}(0)} = \bb T_{\mc V} \overset {\d\mu} \to \Omega^1(M; (\gg_R)_{\bo P})[-1]$.  

More explicitly, following \cite{FoscoloDef} let 
\[\mc F^{\mr{mon}}_{\bo P, \mc A} = \left(\xymatrix{
\Omega^0(M \!\bs\! D; (\gg_\RR)_{\bo P}) \ar[r]^(.36){\d_1} &\Omega^1(M \!\bs\! D; (\gg_\RR)_{\bo P}) \oplus \Omega^0(M \!\bs\! D; (\gg_\RR)_{\bo P}) \ar[r]^(.64){\d_2} &\Omega^1(M \!\bs\! D; (\gg_\RR)_{\bo P})
}\right) \otimes_\RR \CC\]
placed in degrees $-1$, 0 and 1 where $\d_1(g) = -(\d_A(g),[\Phi, g])$ and $\d_2(a,\psi) = \ast \d_A(a) - \d_A(\psi) + [\Phi,a]$.  Write $\d_{\mr{mon}}$ for the total differential.

\begin{remark}
Here we've chosen a point in the twistor sphere, forgetting the hyperk\"ahler structure and retaining a holomorphic symplectic structure.  In other words we've identified the target of the hyperk\"ahler moment map with $\RR \oplus \CC$.  One way of thinking about this is \chris{todo}

\vasily{Somewhere here we should clarify that the hyperKahler moment map takes value in the space imaginary quaternions which is isomorphic as
  real space to $\RR^3_{quat}$. After we pick a complex structure on the twistor sphere, there is natural split $\RR^{3}_{quat} = \CC \oplus \RR$.
  If we care only about complex structure and holomorphic symplectic form on the tangent space to the moduli space of monopoles with respect
  to a certain fixed
  point on a twistor sphere, we trade the real component of the equation in the split $\RR^{3}_{quat} = \CC \oplus \RR$ in the third term
with a complexification of the gauge group transformation in the first term in which we would replace $\gg_{\RR}$ by $\gg_{\CC}$}
\end{remark}

\begin{remark} \label{monopole_holo_restriction_rmk}
If we restrict $\mc F^{\mr{mon}}_{\bo P, \mc A}$ to a slice $C_t = C \times \{t\}$ in the $t$-direction we can identify it with a complex of the form
\[\Omega^\bullet(C_t; \gg_P)[1] \overset {[\Phi,-]} \to \Omega^\bullet(C_t; \gg_P)\]
with total differential given by $\d_A$ on each of the two factors along with the differential $[\Phi,-]$ mixing the two factors.  These two summands each split up into the sum of a Dolbeault complex on $C$ with its dual.  That is, there's a natural subcomplex of the form
\[\Omega^{0,\bullet}(C_t; \gg_P)[1] \overset {[\Phi,-]} \to i \Omega^{0,\bullet}(C_t; \gg_P) \d t\]
where the internal differentials on the two factors are now given by $\ol \dd_{A_0}$.  This complex is in turn quasi-isomorphic to the complex
\[\Omega^\bullet(S^1; \Omega^{0,\bullet}(C_t;\gg_P))[1]\]
with total differential $\ol \dd_{A_0} + \d_{\mc A}$.
\end{remark}

\begin{remark}
If one introduces a framing at a point $c \in C$ then we must correspondingly twist the complex $\mc F^{\mr{mon}}$ above by the line bundle $\OO(c)$ on $C$.  So in that case we define \chris{check}
\[\mc F^{\text{mon,fr}}_{(\bo P,\mc A)} = \mc F^{\mr{mon}}_{\bo P, \mc A} \otimes (\CC_{S^1} \boxtimes \OO(c)).\]
\end{remark}

The following is proved in \cite{FoscoloDef}.
 
\begin{prop}
The tangent complex of $\mon_G(S^1 \times C, D \times \{t_0\}, \omega^\vee)$ at the point $(\bo P,\mc A)$ is quasi-isomorphic to the hypercohomology $\bb H^\bullet(C \times S^1; \mc F'_{(\bo P,\mc A)})$ of a subsheaf $\mc F' \sub \mc F^{\mr{mon}}$ where growth conditions are imposed on the degree 0 part of $\mc F^{\mr{mon}}$ near the singularities.
\end{prop}

Now let's consider the tangent complex to the moduli space of $q$-connections.  For the arguments in this article we'll only need to carefully consider the case $q=\id$ of multiplicative Higgs bundles, but we'll include some remarks regarding the more general case.  In this case the calculation was performed by Bottacin \cite{Bottacin}, see also \cite[Section 4]{HurtubiseMarkman}. Fix a multiplicative Higgs bundle $(P,g)$ on $C$.  We consider the sheaf of cochain complexes on $C$
\[\mc F_{(P,g)} = (\gg_P[1] \overset {\Ad_g} {\to} \gg_P(-D))\]
in degrees $-1$ and 0 with differential given by the adjoint action of $g$.  More precisely let $L_g$ and $R_g$ be the bundle maps $\gg_P \to \gg_P$ obtained as the derivative of left- and right-multiplication.  Then $\Ad_g = L_g - R_g$.  We can alternatively phrase this, as in \cite[Section 4]{HurtubiseMarkman}, as follows.  Define $\ad(g)$ to be the vector bundle
\[\ad(g) = (\gg_P \oplus \gg_P)/\{(X, -g X g^{-1}): X \in \gg_P\}.\]
Then we can write $\mc F$ as the sheaf of complexes
\[\mc F_{(P,g)} = (\gg_P[1] \overset {\Ad_g} {\to} \ad(g))\]
where now $\Ad_g$ is just the map $X \mapsto [(X,-X)]$.

\begin{remark}
  If one introduces a framing at a point $c \in C$ then we must correspondingly twist the complex $\mc F$ above by the line bundle $\OO(c)$ on $C$, i.e. we restrict to deformations that preserve the framing and therefore are zero at the point $c$.  So in that case we define
\[\mc F^\fr_{(P,g)} = (\gg_P[1] \overset {\Ad_g} {\to} \gg_P(-D)) \otimes \OO(c).\]
\end{remark}

\begin{remark}
For more general $q$ we should modify this description by replacing $g$ by a $q$-connection.  Note that one can still define the ($q$-twisted) adjoint action $X \mapsto g X g^{-1}$ using a $q$-connection, and so we can still define the complex
\[\mc F_{(P,g)} = (\gg_P[1] \overset {\Ad_g} {\to} \ad(g))\]
just as in the untwisted case \chris{check}.
\vasily{I wonder what would replace the rational $r$-matrix that defines the Poisson structure on the Lie group of rational Higgs fields 
  given the explicit formula for the symplectic structure on the moduli space of $q$-connections  (\ref{eq:resid}) for non-trivial $q$ ? And if we quantize
the space of functions on $q$-deformed big mHiggs what do we get instead of completed Yangian when $q$ is trivial, say for the case of a single singularity at 0} \chris{If you turn on the $q$-deformation it's not a group anymore (because if you multiply a $q$-connection by a $q$-connection you get a $q^2$-connection).  So it's not clear to me whether there's any sense in which we can think of the Poisson structure as coming from an $r$-matrix anymore.  For instance probably the quantized algebra of functions won't be a Hopf algebra anymore.  However I guess there will still be a map, before quantization, from $\OO(G_0[[z^{-1}]])$ coming from restricting a $q$-connection to a formal punctured neighbourhood of $\infty$.  I'd guess however that that map is no longer Poisson after $q$-deformation.}
\end{remark}

\begin{prop}
The cohomology of the tangent complex of $\mhiggs_G(C, D)$ at the point $(P,g)$ is quasi-isomorphic to the hypercohomology $\bb H^\bullet(C; \mc F_{(P,g)})$ of the sheaf $\mc F$.
\end{prop}

\begin{proof}
We'll explain the proof in different language to that used in \cite{Bottacin}.  Recall that for a mapping stack $\map(X,Y)$ where $X$ and $Y$ are derived Artin stacks the cohomology of the tangent complex at a point $f$ can be computed as the cohomology $H^\bullet(\bb T_{\map(X,Y),f}) \iso \bb H^\bullet(X, f^*\bb T_Y)$.  Let $X = C \!\bs\! D$ and let $Y = G/G$, so $H^\bullet(\bb T_{(P,g), \map(C \!\bs\! D, G/G)}) \iso \bb H^\bullet(C \!\bs\! D; \gg_P[1]|_{C \!\bs\! D} \to \gg_P|_{C \!\bs\! D})$ where the map $\gg_P|_{C \!\bs\! D} \to \gg_P|_{C \!\bs\! D}$ is given by the adjoint action of $g$.  Alternatively one can view this complex as the cohomology of the operator
\[H^\bullet(C \! \bs \! D, \gg_P|_{C\!\bs\!D})[1] \to H^\bullet(C \! \bs \! D, \gg_P|_{C\!\bs\!D})\]
given by the adjoing action of $g$ since the relevant spectral sequence converges at the $E_2$-page.  From this point of view the first term corresponds to deformations of the $G$-bundle $P$ and the second to deformations of the multiplicative Higgs field $g$.

In order to compute the tangent complex to $\mhiggs_G(C,D)$ we must impose the conditions \chris{not really though since it's homotopical} that the deformation of the $G$-bundle $P$ extends across the divisor $D$ and that the deformation of the multiplicative Higgs field has a simple pole at $D$.  In other words we find
\[ H^\bullet(\bb T_{\mhiggs_G(C,D), (P,g)}) \iso \bb H^\bullet(C; \gg_P[1] \to \gg_P(-D)) \sub \bb H^\bullet(C \!\bs\! D; \gg_P[1]|_{C \!\bs\! D} \to \gg_P|_{C \!\bs\! D})\]
as required.
\chris{This last step isn't rigorous, needs improving.}
\end{proof}

\begin{remark}
The embedding of the moduli space $\mhiggs_G(C, D,\omega^\vee)$ into $\mhiggs_G(C, D)$ induces an isomorphism on $H^0$ of their respective tangent complexes: this is seen by comparing the discussion here with the arguments of Bottacin.  In other words, all infinitesimal deformations of $(P,g) \in \mhiggs_G(C, D,\omega^\vee)$ preserve the residue condition at $D$.  The remaining cohomology of the tangent complex $\bb T_{\mhiggs_G(C, D),(P,g)}$, i.e. the remaining hypercohomology of the sheaf $\mc F_{(P,g)}$ has dimension $\dim \mf z_{\gg}$ in degree $-1$ and $1$.  However the moduli space $\mhiggs_G(C, D,\omega^\vee)$ obtained by fixing residue conditions is in fact a smooth algebraic variety.  This  follows from a result of Hurtubise and Markman \cite[Theorem 4.13]{HurtubiseMarkman}, noting that their argument does not rely on the curve $C$ being of genus 1.
\end{remark}

In order to calculate with this hypercohomology space we'll use a \v Cech resolution.  This will be straightforward for the multiplicative Higgs moduli space, and we'll use the isomorphism of Theorem \ref{monopole_qconn_comparison_thm} to give an analogous description on the monopole side.    We define a cover $\mc U = \{U_0, U_1, \ldots, U_k, U_\infty\}$ of $C$ as follows.  Let $U_i$ be a contractible open neighbourhood of the point $z_i$ and let $U_\infty$ be a contractible analytic open neighbourhood of $c \in C$, all chosen to be pairwise disjoint.  Let $U_0$.  Finally let $U_0 = C \bs (D \cup \{c\})$.  Since the $U_i$ are contractible and the remaining subset $C \bs (D \cup \{c\})$ is an affine algebraic curve, which means that for any quasi-coherent sheaf of cochain complexes the higher cohomology groups vanish.  Likewise for the intersections: the punctured open sets $U_i^\times$ are analytic open sets of an affine curve.  

Specify a representative 0-cocycle $(\alpha_\infty, \{\alpha_i\}, \alpha_0, \beta_\infty, \{\beta_i\})$ for the \v Cech cohomology group with respect to our chosen cover $\mc U$.  Explicitly a 0-cochain is given by the following data:
\begin{align*}
 \alpha_\infty &\in \ad(g)(1)(U_\infty) \\
 \alpha_i &\in \ad(g)(-1)(U_i) \text{ for } i = 1,\ldots,k \\
 \alpha_0 &\in \ad(g)(C \bs (D \cup \{\infty\})) \\
 \beta_\infty &\in \gg_P(U^\times_\infty) \\
 \beta_i &\in \gg_P(U^\times_i) \text{ for } i=1,\ldots,k
\end{align*}
where the notation $(\pm 1)$ indicates tensoring by the line bundle $\OO(\pm 1)$.

Being a 0-cocycle means that $(\alpha_\infty - \alpha_0)|_{U^\times_\infty} = \mr{Ad}_g(\beta_\infty)$ and $(\alpha_i - \alpha_0)|_{U^\times_i} = \mr{Ad}_g(\beta_i)$ for each $i$.  We consider 0-cocycles modulo 0-coboundaries of the form $(\mr{Ad}_g(f_\infty), \{\mr{Ad}_g(f_i)\}, \mr{Ad}_g(f_0), (f_\infty -  f_0)|_{U_\infty^\times}, \{(f_i - f_0)|_{U_i^\times}\})$.  In fact $\mr{Ad}_g$ is an isomorphism on $U_0$ for the sections $\alpha_0$ of $\gg_P$ that occur: those with no poles or zeroes in $U_0$.  That means that we can add a coboundary to force $\alpha_0=0$.  %We can likewise use the freedom in $f_i$ to find a representative cocycle where the $\beta_i$ are all zero.  We now have a unique representative cocycle of the form $(\alpha_\infty, \{\alpha_i\}, 0,0,0\})$.

Now, rather than describing the tangent complex to the moduli space of monopoles we'll define a complex that maps to it which we can define locally with respect to the cover $\mc U$.  Consider the open cover $\mc U \times S^1 = \{U_i \times S^1\}$ of $M = C \times S^1$.  For each element $U_i$ of the cover we can define a map
\begin{align*}
\mc F'(U_i \times S^1) &\to (\Omega^0(S^1; \Omega^{0,\bullet}(U_i; \gg_P)) \to \Omega^1(S^1; \Omega^{0,\bullet}(U_i; \gg_P)(D_{U_i})))[1] \\ %With singularity conditions to be fixed
&\iso \gg_P(U_i)[1] \to \gg_P(U_i)(D|_{U_i}) \\
&\iso \mc F(U_i)
\end{align*}
where we restrict the sheaf $\mc F'$ whose hypercohomology calculated the monopole tangent complex to the holomorphic part of the slice at $\{t\} \in S^1$.  Altogether this defines a map from the \v Cech cohomology with respect to this cover to the tangent complex of the moduli space of monopoles.  That is, we have an explicit map
\[{\mr {\check H}}^\bullet(M, \mc U \times S^1, \mc F') \to \bb H^\bullet(M; \mc F')\]
that factors through the (isomorphic) tangent complex of the moduli space of multiplicative Higgs bundles.  To verify this we need only note that these maps commute with the differentials in the \v Cech complex, i.e. with the restriction to the intersection of a pair of open sets. 

Explicitly a 0-cochain in this \v Cech complex is given by $(\alpha_\infty, \{\alpha_i\}, \alpha_0, \beta_\infty, \{\beta_i\})$ where now 
\begin{align*}
 \alpha_\infty &\in \mc F^{\mr{mon}}_0(U_\infty \times S^1) \\
 \alpha_i &\in \mc F^{\mr{mon}}_0(U_i\times S^1) \text{ for } i = 1,\ldots,k \\
 \alpha_0 &\in \mc F^{\mr{mon}}_0(U_0\times S^1) \\
 \beta_\infty &\in \Omega^0_\CC(U_\infty^\times \times S^1; \gg_{\bo P_\CC}) \\
 \beta_i &\in \Omega^0_\CC(U_i^\times \times S^1; \gg_{\bo P_\CC}) \text{ for } i=1,\ldots,k.
\end{align*}
Here we write $\mc F^{\mr{mon}}_0$ to indicate the degree 0 term in the cochain complex.  There's a similar condition for being a 0-cocycle involving the differential $\d_{\mr{mon}}$.  %Again we can add a 0-coboundary to find a representative cocycle of the form $(\alpha_\infty, \{\alpha_i\}, 0,0,0\})$.

To conclude this subsection it will also be important to have an explicit description of the derivative of the holonomy map $H$ as a map between tangent spaces.  We can describe this map using our \v Cech resolutions on each contractible open set $U_i$ individually.

\begin{prop} \label{local_derivative_description_prop}
The derivative $\d H \colon \bb H^\bullet(U_i \times S^1 ; \mc F'_{(\bo P,\mc A)}) \to \bb H^\bullet(\bb D_i; \mc F_{(P,g)})$ is given on an open patch $U_i \times (0,2\pi)$ by the formula
\[\d H(\alpha_i) = \d H(\d_{\mr{mon}} b_i) = b_i(2\pi)H(\mc A) - H(\mc A)b_i(0)\]
where $i = 1, \ldots, k$ or $\infty$.  More precisely by $b_i(2\pi)$ and $b_i(0)$ we mean the limit of $b_i(t)$ as $t \to 2\pi$ or 0 respectively.
\end{prop}

\begin{proof}
Note that the right-hand side is the derivative at $\mc A$ of the map $B_i \mapsto B_i(2\pi)H(\mc A)B_i(0)^{-1}$ where $B_i \in \Omega^0((U_i \times (0,2\pi)) \bs \{(z_i, t_0)\}; \gg_P)$.  This is the definition of the action of the group of gauge transformations on the holonomy $H(\mc A)$ from $t=0$ to $2\pi$. \chris{say more?}
\end{proof}

\section{Symplectic Structures} \label{symp_section}
From now on we'll exclusively study the rational situation where $C = \bb{CP}^1$ and we fix a framing point $c = \infty$.  

\subsection{The Example of $\GL_2$}
\chris{Ultimately we might move this, but we should probably include a discussion of what happens for $\GL_2$ before or in parallel to discussing the general story.}

\vasily{Chris, is the idea to put here explicit expressions for Higgs field in 
terms of some chart
  Darboux coordinates $(p,q)$ that we have discussed in the past drafts
  draft?. If all singularities are miniscule, then for $GL_2$ the leaves could
  be described quite explicitly as products of regular semi-simple co-adoint orbits in $\mathfrak{gl}_2$. Rouven is finishing a draft with explicit parametrization by Darboux coordinates of a certain family of symplectic leaves for $GL_n$. Perhaps here we can put a reference on it. } 
  
  \chris{My idea was to include the discussion of, at least, the simple example that you worked out for $\GL_2$ with minimal singularities where you described the geometry of the moduli space and gave an expression for the symplectic form (in Section 2 of the file 2017\_11\_30\_ghiggs).  I really only had in mind for $\GL_2$ trying to give a more explicit argument for what will appear below, i.e. that the moduli spaces after fixing residues were naturally symplectic.  But maybe we could do what you suggest and described the symplectic leaves in some Darboux coordinates and refer to \cite{FrassekPestun} for proof?} 

\subsection{Symplectic Structures for General $G$}
We begin the more abstract general analysis by briefly discussing the holomorphic symplectic structure on the moduli space of periodic monopoles on $\bb{CP}^1$ following the analysis of Cherkis and Kapustin \cite{CherkisKapustin1, CherkisKapustin3}.  This structure arises from the description we gave as a hyperk\"ahler quotient.  To describe it specifically, let $\delta^{(1)} \mc A$ and $\delta^{(2)} \mc A$ be two tangent vectors at $(\bo P, \mc A)$ to the moduli space of monopoles.  Choose representatives for these two tangent vectors of the form $\alpha_i$ and $\alpha'_i$ respectively  in the \v Cech resolution we described above.  Then we can write the holomorphic symplectic form coming from the hyperk\"ahler reduction in terms of the symplectic pairing on the infinite-dimensional vector space $\mc V$, which is given by the Killing form on $\gg$ along with the wedge pairing of differential forms.  So summing over the local patches in our \v Cech resolution we can write it as
\begin{align*}
\omega_{\mr{mon}}(\delta^{(1)} \mc A, \delta^{(2)} \mc A) &= \int_{M} \kappa(\delta^{(1)} \mc A \wedge \delta^{(2)} \mc A) \d z \d t \\
&= \sum_{i=1}^k \int_{U_i \times S^1} \kappa(\alpha_i \wedge \alpha'_i) \d z \d t
\end{align*}
where the contributions to the integral away from the $U_i$ vanish.

\chris{From here on, replace what was here with a proof that the below formula does define a Poisson/symplectic structure following Hurtubise and Markman.}

\begin{lemma}
As a pairing on the tangent complex at a point $(P,g)$ the symplectic form on the moduli space of multiplicative Higgs bundles pairs $\alpha_i$ with $\beta_i$ for $i=1,\ldots,k$.  It is given by the formula 
\[\omega((\{\alpha_i\}, \{\beta_i\}), (\{\alpha'_i\},\{\beta_i'\})) = \sum_i \langle \rho_g^*(\alpha_i + \alpha_0)|_{U^\times_i}, \rho_g^*(\Ad_g^*)^{-1}(\beta'_i) \rangle - \langle \rho_g^*(\alpha'_i + \alpha'_0)|_{U^\times_i}, \rho_g^*(\Ad_g^*)^{-1}(\beta_i) \rangle,\]
where we use the Killing form to identify $\beta'_i$ with a $\gg^*$-valued form, and where $\rho_g^*$ is the pullback along the right multiplication by $g$.  In this expression the sum is over $i=1,\ldots,k$ and $i=\infty$.
\end{lemma}


\begin{prop} \label{qconn_symp_description}
The symplectic form on the moduli space of $q$-connections can be written in the form
\begin{equation}
\label{eq:resid}
\omega(\delta g, \delta g') = - \sum_{i=1}^d \langle b_i^L g^{-1}, b^{'R}_i g^{-1} \rangle - (b \leftrightarrow b')
\end{equation}
\vasily{I think some repairs are required here if we are to compare to the expression (\ref{eq:res-sum}) below: 1) we should expect sandwiches between $g$ and $g^{-1}$ and not $g^{-1}$ and $g^{-1}$ 2) the
  summation should also include  a contribution from $\infty$ 3) Pick either $\langle \rangle$ or $\kappa$ notation for the Killing form}
  \chris{other than making sure signs are correct I think the expression here works for what's below in Theorem \ref{symplectic_comparison_thm} though I may of course be making an error.  1) We evaluate this pairing on a pair of expressions of the form $(b^L g, - g b^R)$, yielding $(b^L, -g b^R g^{-1})$, 2) doesn't the term at $\infty$ vanish because it can be assumed to land in $z \gg[[z]]$ where the residue pairing vanishes?  3) I added a remark to clarify, but by $\langle - , - \rangle$ I mean the combination of the Killing form pairing with the residue, so $\langle a,b \rangle = \oint \kappa(a,b)$.}
where $\delta g$ is represented on $U_i$ by a pair $(b^L, b^R) \in \Ad_g$.  Here and throughout the notation ``$- (b \leftrightarrow b')$'' denotes antisymmetrization.
\end{prop}

\begin{proof}
 Specify a representative cocycle $(\alpha_\infty, \{\alpha_i\}, \alpha_0, \beta_\infty, \{\beta_i\})$ for the first \v Cech cohomology group. By addition of a coboundary we can assume that $\alpha_0=0$, and to force $\alpha_\infty$ to land in $z\gg[[z]]$ and each $\alpha_i$ to land in $z^{-1}\gg[[z]]$.  We've now fixed a representative cocycle -- there's no further gauge freedom.  Ignoring the factor at infinity -- since after we add these coboundaries the residue pairing there vanishes -- the pairing we end up with looks like 
\begin{align*}
\omega(\delta g, \delta g') &= \sum_{i=1}^d \langle \rho_g^*(\alpha_i + \alpha_0)|_{U^\times_i},\rho_g^* (\Ad_g^*)^{-1}( \beta'_i) \rangle - ((\alpha,\beta) \leftrightarrow (\alpha',\beta')\\ 
&= \sum_{i=1}^d \langle \rho_g^*(\alpha_i + \alpha_0)|_{U^\times_i}, \rho_g^*(\Ad_g^*)^{-1}(\Ad_g^{-1}(\alpha_i' - \alpha'_0)) \rangle - (\alpha \leftrightarrow \alpha')\\
&= \sum_{i=1}^d \langle \rho_g^*(\alpha_i)|_{U^\times_i}, \rho_g^*(\Ad_g^*)^{-1}(\Ad_g^{-1}(\alpha_i')|_{U^\times_i}) \rangle - (\alpha \leftrightarrow \alpha').
\end{align*}

We can compute the composite operator $(\Ad_g^*)^{-1}\Ad_g^{-1}$ on $\mr{ad}(P)(U^\times_i)$.  It's given by the inverse of the operator $\Ad_g\Ad_g^*$ which sends a pair $(b^L, b^R) \in \mr{ad}(P)(U^\times_i)$ to $(b^L - b^R, b^R - b^L) \sim (b^L + g^{-1} b^L g, b^R - g b^R g^{-1})$.  Denote this by $((1+A_g)b^L,(1-A_{g^{-1}})b^R)$.  We can describe the inverse using the expansion $(1+A_g)^{-1} = 1 - A_g + A_g^2 + \cdots$.  After applying our gauge transformation above the remaining degree of freedom in $\alpha_i$ is its $z^{-1}$ term.  Each time we apply $A$ it raises the order in $z$ of this term by one, so only the linear summand $-A_g$ of $(1+A_g)^{-1}$ contributes to the residue pairing.  In the pairing we need to use the invariant pairing on $\gg((z)) \oplus \gg((z))$ that vanishes on the subalgebra spanned by $(X, A_g(X))$, so we take the difference of the residue / Killing form pairings on the two summands \chris{scaled by 1/2: why this normalization?}.
\begin{align*}
\omega(\delta g, \delta g') &= \frac 12 \sum_{i=1}^d  - \langle \rho_g^*A_g b^L_i, \rho_g^*b^{'R}_i \rangle - \langle \rho_g^*b^L_i, \rho_g^*A_{g^{-1}}b^{'R}_i \rangle- (b \leftrightarrow b') \\ 
&= - \sum_{i=1}^d \langle b^L_ig^{-1}, b^{'R}_i g^{-1} \rangle  - (b \leftrightarrow b')
\end{align*}
using cyclic invariant to identify the two terms. 
\end{proof}

We can now establish our main result.
\begin{theorem} \label{symplectic_comparison_thm}
The symplectic structure on $\mon_G^\fr(\bb{CP}^1 \times S^1,D \times\{0\},\omega^\vee)$ and the pullback of the symplectic structure on $\mhiggs_G^{\text{ps,fr}}(\bb{CP}^1,D,\omega^\vee)$ under $H$ coincide.
\end{theorem}

\begin{proof}
This is straightforward now, by combining the two local descriptions of the symplectic structure on monopoles and multiplicative Higgs bundles on a neighbourhood of a puncture with the description in Proposition \ref{local_derivative_description_prop}.  So, let's begin by taking the symplectic form $\omega_{\mr{mHiggs}}$ on the moduli space $\mhiggs_G^{\text{ps,fr}}(\bb{CP}^1,D,\omega^\vee)$ and evaluating it at the image under $\d H$ of two elements $(\alpha_i, \alpha'_i) = (\d_{\mr{mon}}(b_i), \d_{\mr{mon}}(b_i'))$.  Let us write $b_i(0) = b_i^L$ and $b_i(2\pi) = b_i^R$ for brevity.  Likewise for consistency with the calculations above let us denote the image $H(\mc A)$ under the holonomy map by $g$.

According to Proposition \ref{local_derivative_description_prop} we have
\begin{align*}
\omega_{\mr{mHiggs}}(\d H(\alpha_i), \d H(\alpha_i')) &= \omega_{\mr{mHiggs}}(\d H(\d_{\mr{mon}}(b_i)), \d H(\d_{\mr{mon}}(b_i'))) \\
&= \omega_{\mr{mHiggs}}(b_i^Rg - gb_i^L,b_i^{'R}g - gb_i^{'L}).
\end{align*}
We can write this in terms of a pairing on the bundle $\ad(g)$.  Represent the class $b_i^Rg - gb_i^L$ by the pair $(b_i^Rg, -gb_i^L)$.  Then applying the description of $\omega_{\mr{mHiggs}}$ provided by Proposition \ref{qconn_symp_description} we have
\begin{align*}
\omega_{\mr{mHiggs}}(\d H(\alpha_i), \d H(\alpha_i')) &= \omega_{\mr{mHiggs}}((b_i^Rg, -gb_i^L),(b_i^{'R}g, -gb_i^{'L})) \\ 
&= \langle  b_i^R , g b_i'^L g^{-1}\rangle - (b \leftrightarrow b')
\end{align*}
where the remaining terms are killed by the antisymmetrization.

On the other hand, we can evaluate the pairing $\omega_{\mr{Mon}}(\alpha_i,\alpha'_i) = \omega_{\mr{Mon}}(\d_{\mr{mon}}(b_i),\d_{\mr{mon}}(b_i'))$ via integration by parts. We'll compute the symplectic pairing for the full \v Cech complex, but it will split into a sum over open sets $U_i$.  The result is that \vasily{the
  expression (\ref{eq:omega-mon}) expands as}
\vasily{ 1. let us unify the notations for the Higgs field in the equations above and the previous sections: somewhere holonomy or
  the Higgs field is called $H(\mc A)$, somewhere $\phi$, somewhere $g$. I would favour the notation $g$. 2. Let us introduce somewhere before the notation $b^{L} = b(0), b^{R} = b(2\pi)$ to make the equations more readable, and break the multline computation into separate lines and insert between
  the lines the explanations of what we are doing  } \chris{made some notational changes following your suggestions.}

\begin{align*}
\omega_{\mr{Mon}}(\{\alpha_i\},\{\alpha'_i\}) &= \sum_{i=0,\ldots,k,\infty} \int_{U_i} \kappa(\d_{\mr{mon}}(b_i) \wedge \d_{\mr{mon}}(b_i')) \d z - (b \leftrightarrow b')\\
&= \sum_{i=1,\ldots,k,\infty} \int_{\dd \bb D_i \times [0,2\pi]} \kappa(b_i - b_0, \d_{\mr{mon}}(b'_i)) \d z - (b \leftrightarrow b')
\end{align*}
using here the fact that $\d_{\mr{mon}}b_i = \d_{\mr{mon}} b_0$ on $U_i \cap U_0$ and that $\d_{\mr{mon}} b_i(t) = 0$ when $t = 0$ or $2\pi$.  By Stokes' theorem we then have
\begin{align*}
 \omega_{\mr{Mon}}(\{\alpha_i\},\{\alpha'_i\})&= \sum_{i=1,\ldots,k,\infty} \oint_{\dd \bb D_i} \kappa(b_i^R - b_0^R, b_i^{'R}) - \kappa(b_i^L - b_0^L, b_i^{'L}) - (b \leftrightarrow b') \\
 &= \sum_{i=1,\ldots,k,\infty} \oint_{\dd \bb D_i} - \kappa(b_0^R, b_i^{'R}) + \kappa(b_0^L, b_i^{'L}) - (b \leftrightarrow b').
\end{align*}
Pick out the summand corresponding to $U_i$.  Choose our potentials so that on the boundary $\dd \bb D_i$ we have $b_0^L = b_0^{'L} = 0$.  We can do this by setting $b^{R} = \delta g g^{-1}$ since $g$ is non-singular on $U_0$.  This choice means that on $U_0 \cap U_i$, since $\delta g  = b_{i} ^R g - g b_i^L = b_0^{R} g $ we can make the identification $ b_0^{R} = b_{i}^{R} - g b_{i}^{L} g^{-1}$.  Therefore
\[\omega_{\mr{Mon}}(\{\alpha_i\},\{\alpha'_i\}) = \sum_{i=1,\ldots,k,\infty} \langle(g b_i^{L} g^{-1}, b_i^{'R}\rangle \quad - \quad  (b \leftrightarrow b')\]
agreeing with the expression coming from $\mr{mHiggs}$. \chris{indeed there's a sign error somewhere which I haven't tracked down yet.  On the other hand maybe it isn't correct to say that the two symplectic forms are equal, maybe they're just proportional?}

\vasily{Instead, I would suggest that here we choose the potentials such
  that $b_0^L = 0$ and $b_0'^L = 0$ everywhere on $U_0$,
  and in particular on the boundary $\partial \bb D_i$. There is no
  obstruction to this choice because for 
  any  variation of the holonomy $\delta g  = b^R g - g b^L$
  with the choice $b^{L} = 0$ we set $b^{R} = \delta g g^{-1}$
  which is a holomorphic section on $U_0$ for the reason
  that $U_0$ does not contain any singularities and because of the Bogomolny
  equation. Given this choice the last line becomes
  \begin{equation}
     \sum_{i=1,\ldots,k,\infty} \oint_{\dd \bb D_i} - \kappa(b_0^R, b_i^R{}') \quad - \quad  (b \leftrightarrow b')
   \end{equation}
   Next, from $\delta g  = b_{i} ^R g - g b_i^L = b_0^{R} g $ on $\partial \bb D_i = U_0 \cap U_i$
   we find
   $ b_0^{R} = b_{i}^{R} - g b_{i}^{L} g^{-1}$ and plug in the above line. Because of the antisymmetrization $b \leftrightarrow b'$ the term
   containing 
$b_i^{R} b_{i}^{R}{}'$ vanishes and we are left with
    \begin{equation}
 \label{eq:res-sum}
      \sum_{i=1,\ldots,k,\infty} \oint_{\dd \bb D_i}  \kappa(g b_i^{L} g^{-1}, b_i^R{}') \quad - \quad  (b \leftrightarrow b')
    \end{equation}
} \chris{modified}

\vasily{I think the minus sign is missed if compared to (\ref{eq:res-sum})}
\end{proof}



\begin{remark}
\chris{connection to Shapiro's symplectic leaves in \cite{Shapiro}.  The claim is that taking the Taylor expansion at infinity defines a Poisson map to $\wt G \sub G_1[[z^{-1}]]$ sending symplectic leaves to (discrete) unions of symplectic leaves.  One can do the classification more completely in the type A case.}
\end{remark}


\section{Hyperk\"ahler Structures}
\vasily{As of July 19, 2018 I have not yet passed this and following sections}

The results of the previous section imply that the symplectic structure on $\mhiggs_G^{\text{ps,fr}}(\bb{CP}^1,D,\omega^\vee)$ extends canonically to a hyperk\"ahler structure.  In this section we'll compare the twistor rotation with the deformation to the moduli space of $\eps$-connections. \chris{todo: expand on this}

Let us begin by describing the hyperk\"ahler structure on the moduli space of periodic monopoles and how the holomorphic symplectic structure varies under rotation in the twistor sphere.  We'll begin by observing that the moduli space of periodic monopoles can be described as a hyperk\"ahler quotient as in the work of Atiyah and Hitchin \cite{AtiyahHitchin}.  This was demonstrated by Cherkis and Kapustin \cite{CherkisKapustin3} for the group $\SU(2)$, see also Foscolo \cite{FoscoloDef}[Theorem 7.12] \chris{is the Cherkis-Kapustin result sufficiently rigorous for our purposes?  Foscolo only talks about $\SO(3)$.  Reference for more general $G$?  Maybe \cite{NekrasovPestun}?}.

\chris{Key observation to include: the action of twistor rotation is by $\SO(3)$-action on a space of lattices.  More precisely the fibers in the twistor sphere are obtained, with their holomorphic symplectic structures by performing this rotation.}

\chris{define what I mean by $\eps$-deformation for monopoles.}

\begin{prop}
\chris{Monopoles in the large $R$ (Gaiotto) limit -- twistor rotation and $\eps$-deformation coincide.}
\end{prop}

\begin{proof}
 
\end{proof}

\begin{corollary}
If we equip $\mhiggs_G^{\text{ps,fr}}(\bb{CP}^1,D,\omega^\vee)$ with the complex structure at the twistor parameter $\eta \in \CC$, the resulting complex manifold is isomorphic to $\epsconn_G^{\text{ps,fr}}(\bb{CP}^1,D,\omega^\vee)$.
\end{corollary}

\begin{proof}
 
\end{proof}

\chris{comment on the holomorphic symplectic structure vs the one discussed above.}

\chris{comment on the example of the Nahm transform.}

\section{Quantization and Yangians}
\chris{We're still discussing how this story goes.}

\section{Duality}
\chris{For a first draft I'm inserting some notes I prepared for a talk.  Todo: talk about motivation and the content we discussed regarding non-simply-laced groups}

\begin{pseudoconj}[Multiplicative Geometric Langlands] \label{multLanglands}
Let $G$ be a Langlands self-dual group.  There is an equivalence of categories
\[\text{A-Branes}_{q^{-1}}(\mhiggs_G(C,D,\omega^\vee)) \iso \text{B-Branes}(\qconn_G(C, D, \omega^\vee))\]
where the category on the right-hand side depends on the value $q$.
\end{pseudoconj}

What does this mean, and are there situations in which we can make it precise?  We'll discuss three examples where we can say something more concrete.  In each case, by ``B-branes'' we'll just mean the category $\coh(\qconn_G(C, D, \omega^\vee))$ of coherent sheaves.  By ``A-branes'' we'll mean some version of \emph{$q^\vee$-difference connections} on the stack $\bun_G(C)$.

\begin{remark}
This equivalence is supposed to interchange objects corresponding to branes of opers on the two sides, and introduce an analogue of the Feigin-Frenkel isomorphism between deformed W-algebras (see \cite{FrenkelReshetikhinSTS, STSSevostyanov}.  This isomorphism only holds for self-dual groups, which motivates the restriction to the self-dual case here.
\end{remark}

\subsection{The Abelian Case}
Suppose $G = \GL(1)$ (more generally we could consider a higher rank abelian gauge group).  In general for an abelian group the moduli spaces we've defined are trivial -- for instance the rational and trigonometric spaces are always discrete.  However there is one interesting non-trivial example: the elliptic case.  For simplicity let's consider the abelian situation with $D = \emptyset$: the case with no punctures.

\begin{definition}
A \emph{$q$-difference module} on a variety $X$ with automorphism $q$ is a module for the sheaf $\Delta_{q,X}$ of non-commutative rings generated by $\OO_X$ and an invertible generator $\Phi$ with the relation $\Phi \cdot f = q^*(f) \cdot \Phi$.  Write $\diff_q(X)$ for the category of $q$-difference modules on $X$.
\end{definition}

In the abelian case the space $\qconn_{\GL(1)}(E)$ is actually a stack, but one can split off the stacky part to define difference modules on it.  Indeed, for any $q$ one can write
\[\bun_{\GL(1)}(E) = \iso B\GL(1) \times \ZZ \times E^\vee\]
and so
\[\qconn_{\GL(1)}(E) \iso B\GL(1) \times \ZZ \times (E^\vee \times_q \CC^\times)\]
which means one can define difference modules on these stacks associated to an automorphism of $E^\vee$ or $E^\vee \times_q \CC^\times$ respectively.

\begin{conjecture}
There is an equivalence of categories for any $q \in \bb{CP}^1$
\[\diff_q(\bun_{\GL(1)}(E)) \iso \coh(q^{-1}\conn_{\GL(1)}(E)).\]
\end{conjecture}

In this abelian case we can go even farther and make a more sensitive 2-parameter version of the conjecture.

\begin{conjecture}
There is an equivalence of categories for any $q_1, q_2 \in \bb{CP}^1$
\[\diff_{q_1}(q_2\conn_{\GL(1)}(E) \iso \diff_{q_2^{-1}}(q_1^{-1}\conn_{GL(1)}(E)\]
where $q_1$ is the automorphism of $E^\vee \times_{q_2} \CC^\times$ acting fiberwise over each point of $\CC^\times$.
\end{conjecture}

This conjecture should be provable using the same techniques as the ordinary geometric Langlands correspondence in the abelian case, i.e. by a (quantum) twisted Fourier-Mukai transform (as constructed by Polishchuk and Rothstein \cite{PolishchukRothstein}).

\subsection{The Classical Case}
Now, let's consider the limit $q \to 0$.  This will give a conjectural statement involving coherent sheaves on both sides analogous to the classical limit of the geometric Langlands conjecture as conjectured by Donagi and Pantev \cite{DonagiPantev}.  The existence of an equivalence isn't so interesting in the self-dual case (where both sides are the same), but the classical multiplicative Langlands functor should be an \emph{interesting} non-trivial equivalence.  For example we can make the following conjecture

\begin{conjecture}
Let $G$ be a Langlands self-dual group and let $E$ be an elliptic curve.  There is an automorphism of categories (for the rational, trigonometric and elliptic moduli spaces)
\[F \colon \coh(\mhiggs_G(E) \iso \coh(\mhiggs_{G}(E)\]
so that the following square commutes:
\[\xymatrix{
\coh(\mhiggs_T(E)) \ar[r]^{\mathrm{FM}} \ar[d]_{p_*q^!} &\coh(\mhiggs_T(E)) \ar[d]^{p_*q^!} \\
\coh(\mhiggs_G(E)) \ar[r]^{F} &\coh(\mhiggs_G(E)).
}\]
Here we're using the natural morphisms $p \colon \mhiggs_B(E) \to \mhiggs_G(E)$ and $q \colon \mhiggs_B(E) \to \mhiggs_T(E)$, and $\mr{FM}$ is the Fourier-Mukai transform.
\end{conjecture}

It ought to be possible to state something a bit stronger that includes singularities in this auto-duality.

\subsection{The Rational Type A Case}
There's one more example where we can say something precise, and even draw a connection to the ordinary geometric Langlands correspondence.  We already mentioned the Nahm transform in the previous section: in the case where $G = \GL(n)$ and $C$ is $\CC$ (where as usual we fix framing data at infinity) the Nahm transform identifies multiplicative Higgs bundles of degree $k$ with \emph{ordinary} Higgs bundles on $\bb{CP}^1$ for the group $\GL(k)$ with $n+2$ tame singularities (with appropriate fixed locations and residues).

\begin{claim}
Under the Nahm transform, Pseudo-Conjecture \ref{multLanglands} in the rational case for the group $\GL(n)$ becomes the ordinary geometric Langlands conjecture on $\bb{CP}^1$ with tame ramification.
\end{claim}

 
\pagestyle{bib}
\bibliographystyle{alpha}
\bibliography{Mult_Hitchin}
%\printbibliography

\textsc{Institut des Hautes \'Etudes Scientifiques}\\
\textsc{35 Route de Chartres, Bures-sur-Yvette, 91440, France}\\
\texttt{celliott@ihes.fr}\\ 
\texttt{pestun@ihes.fr}
 
\end{document}

